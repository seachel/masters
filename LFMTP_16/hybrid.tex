\section{Hybrid}
\label{sec:hybrid}

The Calculus of Inductive Constructions (CIC)
\cite{PaulinMohring93,BertotCasteran:2004} as implemented by Coq is
the \emph{reasoning logic} (RL) of Hybrid.  Both SLs and OLs are
implemented as inductive types in Coq, and Coq is used to carry out
all formal reasoning.
%
% To illusrate the specification layer in Hybrid, we will introduce
% HH and describe the basics of its encoding in Coq.  Details of its
% implementation are found in the next section.  For the object-level,
% we choose .. and introduce just enough details to illustrate the
% encoding of OLs in general.
%
Hybrid is implemented as a Coq library.  This library first introduces
a special type \mltm{expr} that encodes a de Bruijn representation of
$\lambda$-terms.  It is defined inductively and its definition appears
in Figure~\ref{fig:expr}.
\begin{figure}
\begin{lstlisting}
Inductive expr : Set :=
| CON : con -> expr
| VAR : var -> expr
| BND : bnd -> expr
| APP : expr -> expr -> expr
| ABS : expr -> expr.
\end{lstlisting}
\caption{Terms in Hybrid \label{fig:expr}}
\end{figure}
Here, \mltm{VAR} and
\mltm{BND} represent bound and free variables, respectively, and
\mltm{var} and \mltm{bnd} are defined to be the natural numbers.  The
type \mltm{con} is a parameter to be filled in when defining the
constants used to represent an OL.  The library then includes a series
of definitions used to define the operator \mltm{LAM} of type
$(\mltm{expr}\rightarrow\mltm{expr})\rightarrow\mltm{expr}$, which
provides the capability to express OL syntax using HOAS.  Expanding
its definition fully down to primitives gives the low-level de Bruijn
representation, which is hidden from the user when reasoning about
meta-theory. In fact, the user only needs \mltm{CON}, \mltm{VAR},
\mltm{APP}, and \mltm{LAM} to define operators for OL syntax.
One other predicate from the Hybrid library will appear in the proof
development: $\mltm{proper}:\mltm{expr}\rightarrow\coqtm{Prop}$.  This
predicate rules out terms that have occurrences of bound variables
that do not have a corresponding binder (\emph{dangling indices}).

To illustrate the encoding of OL syntax, we consider the example
mentioned earlier of reasoning about the correspondence between a HOAS
encoding and a de Bruijn representation of the terms of the untyped
$\lambda$-calculus.\footnote{In \cite{WCGN:PPDP13}, inference rules
  and proofs of properties of this OL are also discussed.}
%
We fill in the definition of \mltm{con}, and define operators
\sltm{hApp} and \sltm{hAbs} for the HOAS encoding and operators of the
untyped $\lambda$-calculus, and \sltm{dApp}, \sltm{dAbs}, and
\sltm{dVar} for object-level de Bruijn terms.
Figure~\ref{fig:olsyntax} defines this encoding.
%
\begin{figure}
\begin{lstlisting}
Inductive con : Set := 
| hAPP : con | hABS : con
| dAPP : con | dABS : con | dVAR : nat -> con.

Definition hApp : expr -> expr -> expr :=
 fun (t1 : expr) => fun (t2 : expr) =>
  APP (APP (CON hAPP) t1) t2. 
Definition hAbs : (expr -> expr) -> expr :=
 fun (f : expr -> expr) => 
  APP (CON hABS) (LAM f).

Definition dApp : expr -> expr -> expr :=
 fun (t1 : expr) => fun (t2 : expr) =>
  APP (APP (CON dAPP) t1) t2. 
Definition dAbs : expr -> expr :=
 fun (d : expr) => APP (CON cdABS) d.
Definition dVar : nat -> expr :=
 fun (n : nat) => (CON (dVAR n)).
\end{lstlisting}
\caption{Encoding OL Syntax in Hybrid \label{fig:olsyntax}}
\end{figure}
%
As mentioned, the type \mltm{con} is actually a parameter in
the Hybrid library. This will be explicit when discussing Coq proofs,
where we write $(\mltm{expr}~ \mltm{con})$ as the type used to express
OL terms.  Also, the keyword \coqtm{fun} is Coq's abstraction
operator.  In this paper, we will sometimes use the standard notation
from the $\lambda$-calculus for function abstraction.  For example,
the definition of \sltm{hApp} can also be written as $\lambda (\mltm{t1} :
\mltm{expr}~ \mltm{con}) (\mltm{t2} : \mltm{expr}~ \mltm{con}) \; . \;
\mltm{APP}~ (\mltm{APP}~ (\mltm{CON}~ \mltm{hAPP})~ \mltm{t1})~
\mltm{t2}$.


