\section{Introduction}

\emph{Logical frameworks} provide general languages in which it is possible to represent a wide variety of logics, programming languages,
and other formal systems.  They are designed to capture uniformities of the syntax and inference systems of these \emph{object logics} (OLs) and to provide support for implementing and reasoning about them.  Hybrid \cite{FeltyMomigliano:JAR10} is a logical framework that provides support for encoding OLs via \emph{higher-order abstract   syntax} (HOAS), also referred to as \emph{lambda-tree syntax}.
%\cite{MillerP99}.
Using HOAS, binding constructs in the OL are encoded using the binding constructs provided by an underlying $\lambda$-calculus or function space of the logical framework (the \emph{meta-language}).  Using such a representation allows us to delegate to the meta-language $\alpha$-conversion and capture-avoiding substitution.  Further, object logic substitution can be rendered as meta-level $\beta$-conversion.  HOAS encodings aim to relieve users from having to build up common (and often large) infrastructure implementing operations dealing with variables, such as capture-avoiding substitution, renaming, and fresh name generation.  In addition, in such logical frameworks, embedded implication and universal quantification are often used to represent \emph{hypothetical} and \emph{parametric} judgments, also called \emph{generic} judgments, %\cite{MillerTiu:TOCL05},
which allow elegant and succinct specifications of OL inference rules.

An important goal of Hybrid is to exploit the advantages of HOAS within the well-understood setting of higher-order logic as implemented by systems such as Isabelle and Coq.\footnote{Although Hybrid has been implemented in both Coq and Isabelle/HOL, we use the Coq version in this paper.}  Building on such a system allows us to easily experiment with new specification logics.  It also provides a high degree of trust; for instance proof terms in Coq serve as proof certificates, which can be independently checked.  In addition, Hybrid in Coq inherits Coq's full recursive function space as well as its extensive set of libraries.

Hybrid is implemented as a two-level system, an approach first introduced in the $\foldn$ logic \cite{McDowellMiller:TOCL01}, and now applied within a variety of logics and systems, such as the Abella interactive theorem prover \cite{Gacek:IJCAR08}.  In a two-level system, the \emph{specification} and (inductive) \emph{meta-reasoning} are done within a single system but at different \emph{levels}. An intermediate level is introduced by inductively defining a \emph{specification logic} (SL) in Coq, and OL judgments (including hypothetical and parametric judgments) are encoded in the SL.  Several meta-theoretic properties about the SL provide powerful tools for reasoning about OLs.  For example, the cut admissibility theorem provides a direct and convenient way to substitute a formula for an assumption in a context of assumptions.  Structural properties of the SL, such as weakening, contraction, and exchange, also provide tools that can be directly applied to reasoning in any OL.

%In this paper, we introduce an intuitionistic higher-order SL, namely
%the logic of higher-order hereditary Harrop formulas (HoHH)
%\cite{LProlog} with some restrictions, mainly on the types of terms
%in quantified formulas.  Previous SLs considered for Hybrid include a
%second-order fragment of HoHH and an ordered linear logic
%\cite{FeltyMomigliano:JAR10}.
%
In this paper, we introduce an intuitionistic higher-order SL, namely the logic of hereditary Harrop formulas (HH).  HH is a sublogic of the logic of higher-order hereditary Harrop formulas as presented in \cite{LProlog}.  Two kinds of ``order'' can be seen in this logic, the domain of quantification and the implicational complexity.  In terms of the former, HH allows quantification over second-order types, and in terms of the latter, HH is higher-order, allowing any level of nested implications.  Previous SLs considered for Hybrid include the fragment of HH with second-order implicational complexity and an ordered linear logic \cite{FeltyMomigliano:JAR10}.

We adopt a minor variation of the inference rules for HH used as an SL in recent versions of the Abella interactive theorem prover \cite{WCGN:PPDP13}.  We present our encoding in Coq, and discuss the proofs of meta-theoretic properties in some detail.  In particular, we develop an abstraction to capture uniformities across proofs of different meta-theoretic properties.  Cut admissibility in particular relies on a fairly complex inductive argument, involving mutual inductions and sub-inductions.  Our proof follows the tradition of many other syntactic proofs of cut admissibility for various logics that first induct on the formula depth and then on the proof structure, e.g. \cite{Girard89}.  Furthermore, our proof is \emph{structural} in the sense of \cite{Pfenning:IC00}, in our case using structural induction principles generated by Coq. %We discuss different strategies and even false starts, and then
We present the proofs via our abstraction, with the goal of providing a deeper insight into the proofs and the formalization process.  Variants of the properties we prove have also been proved in Abella. They are mentioned in \cite{WCGN:PPDP13}, but proofs are not presented there.  We briefly discuss some differences.

Our overall goal is to extend the reasoning power of Hybrid. Implementing HH as a new SL in Hybrid now allows us to directly encode, for example, the two OLs in the case studies considered in \cite{WCGN:PPDP13}.  The first involves reasoning about the correspondence between an HOAS encoding and a de Bruijn representation of the terms of the untyped $\lambda$-calculus, while the second involves reasoning about a structural characterization of reductions on untyped $\lambda$-terms, and is originally posed in \cite{LProlog}. Other examples we intend to study include the elegant algorithmic specification of bounded subtype polymorphism in System F in~\cite{Pientka:TPHOLs07}, which comes from the \textsl{PoplMark} challenge~\cite{Aydemir05TPHOLs}, as well as specifications of continuation-passing transformations in functional programs.  The specification of the main judgments of all of these OLs will benefit from the availability of embedded implication in HH (in particular, using two or three levels).

We also note that in addition to the advantages mentioned above with regard to implementing Hybrid inside a well-established theorem prover, Hybrid also provides an ideal setting in which to quickly prototype and experiment with new SLs.  Each one is developed as a library and a user can choose and import one that is best-suited to the task at hand and/or move between them easily.  For example, case studies that don't require the expressiveness of HH can use the second-order fragment, likely leading to simpler proofs.  Case studies that are better suited to a linear logic can directly import and use a linear SL, etc.  In contrast, in Abella, a slight extension of HH replaced the SL used in earlier versions of the system.  Fixing the SL allows developers to focus more on adding powerful automation for a particular SL, and thus proofs in Hybrid currently require more interaction.

In Section~\ref{sec:hybrid}, we give a brief introduction of Hybrid. In Section~\ref{sec:sl}, we introduce HH as an example specification layer and describe its implementation in Coq. Highlights of the the mutual structural induction used in later proofs is found in Section~\ref{sec:induction} followed by the presentation of a generalized SL in Section~\ref{sec:gsl} and a proof technique using this generalized SL in Section~\ref{sec:pfgsl}. Section~\ref{sec:structrules} outlines proofs of the structural rules of HH, while Section~\ref{sec:cutadmiss} details the proof of cut admissibility. Finally, Section~\ref{sec:concl} concludes and discusses related and future work. The files of the Coq formalization are available at \url{www.eecs.uottawa.ca/~afelty/lfmtp16/}.
