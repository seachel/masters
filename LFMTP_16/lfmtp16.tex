%-----------------------------------------------------------------------------
%
%               Template for sigplanconf LaTeX Class
%
% Name:         sigplanconf-template.tex
%
% Purpose:      A template for sigplanconf.cls, which is a LaTeX 2e class
%               file for SIGPLAN conference proceedings.
%
% Guide:        Refer to "Author's Guide to the ACM SIGPLAN Class,"
%               sigplanconf-guide.pdf
%
% Author:       Paul C. Anagnostopoulos
%               Windfall Software
%               978 371-2316
%               paul@windfall.com
%
% Created:      15 February 2005
%
%-----------------------------------------------------------------------------


%\documentclass[preprint]{sigplanconf}
\documentclass{sigplanconf}

% The following \documentclass options may be useful:

% preprint      Remove this option only once the paper is in final form.
% 10pt          To set in 10-point type instead of 9-point.
% 11pt          To set in 11-point type instead of 9-point.
% numbers       To obtain numeric citation style instead of author/year.


%%%%%%%%%%%%%%%%%%%%%%%%%
% PACKAGES
%%%%%%%%%%%%%%%%%%%%%%%%%

\usepackage{listings}
\usepackage{multicol}
%\usepackage{stmaryrd}
\usepackage{amsmath}
\usepackage{amssymb}
\usepackage{amsthm}
\usepackage{proof}
%\usepackage{lscape}
\usepackage{url}


%\mathchardef\mhyphen="2D % Define a "math hyphen"


\usepackage{framed}
\usepackage{enumitem}
\usepackage{mathtools}
\usepackage{float}

\usepackage{listings}
\usepackage{multicol}
\usepackage{stmaryrd}
\usepackage{proof}
\usepackage{bussproofs}
\usepackage{lscape}

%\usepackage{cite}
\usepackage{url}

%%%%%%%%%%%%%%%%%%%%%%%%%
% FORMATTING SEQUENTS
%%%%%%%%%%%%%%%%%%%%%%%%%

\newcommand{\oltm}[1]{\texttt{#1}}          % inline OL terms
\newcommand{\sltm}[1]{\texttt{#1}}          % inline SL terms
\newcommand{\mltm}[1]{\texttt{#1}}          % inline reasoning logic terms
\newcommand{\hybridtm}[1]{\texttt{#1}}      % inline Hybrid terms
\newcommand{\coqtm}[1]{\texttt{#1}}         % inline Coq terms

\newcommand{\rl}[1]{\textit{#1}}   % rule name

\newcommand{\signature}{\Sigma}             % default signature
\newcommand{\dyncon}{\Gamma}              % default dynamic context
\newcommand{\statcon}{\Pi}                % default static context
\newcommand{\inddyncon}{\Gamma'}

% standard sequents
\newcommand{\seq}[2][\signature{} \, ; \, \dyncon{}]{\ensuremath{#1 \vdash #2}}
\newcommand{\bch}[3][\signature{} \, ; \, \dyncon{}]{\ensuremath{#1 , {[#2]} \vdash #3}}

% SL sequents
\newcommand{\seqsl}[2][\dyncon{}]{\ensuremath{#1 \rhd #2}}
\newcommand{\bchsl}[3][\dyncon{}]{\ensuremath{#1 , {[#2]} \rhd #3}}

% specify signature
\newcommand{\seqslsig}[2]{\seqsl[\signature{} \, , \, #1 \, ; \, \dyncon{}]{#2}}


\newcommand{\atom}[1]{\ensuremath{\langle \; #1 \; \rangle}}  % atomic formula
\newcommand{\prog}[2]{\ensuremath{#1 \; {:} {-} \; #2}}     % static program clause


%%%%%%%%%%%%%%%%%%%%%%%%%
% FORMATTING RULES
%%%%%%%%%%%%%%%%%%%%%%%%%

\newcommand{\rlnmsbc}{\rl{g\_prog}}
\newcommand{\rlnmsinit}{\rl{g\_dyn}}
\newcommand{\rlnmst}{\rl{g\_tt}}
\newcommand{\rlnmsand}{\rl{g\_and}}
\newcommand{\rlnmsimp}{\rl{g\_imp}}
\newcommand{\rlnmsall}{\rl{g\_all}}
\newcommand{\rlnmsalls}{\rl{g\_allx}}
\newcommand{\rlnmssome}{\rl{g\_some}}
\newcommand{\rlnmbmatch}{\rl{b\_match}}
\newcommand{\rlnmbanda}{\rl{$b\_and_1$}}
\newcommand{\rlnmbandb}{\rl{$b\_and_2$}}
\newcommand{\rlnmbimp}{\rl{b\_imp}}
\newcommand{\rlnmball}{\rl{b\_all}}
\newcommand{\rlnmballs}{\rl{b\_allx}}
\newcommand{\rlnmbsome}{\rl{b\_some}}

\newcommand{\inferH}[3][]{\infer[\fontsize{8pt}{8pt}{#1}]{#2}{#3}}

\newcommand{\rlsbc}{\inferH[\rlnmsbc{}]{\seqsl{\atom{A}}}{\prog{A}{G} & \seqsl{G}}}
\newcommand{\rlsinit}{\inferH[\rlnmsinit{}]{\seqsl{\atom{A}}}{D \in \dyncon{} & \bchsl{D}{A}}}
\newcommand{\rlst}{\inferH[\rlnmst{}]{\seqsl{\sltm{T}}}{}}
\newcommand{\rlsand}{\inferH[\rlnmsand{}]{\seqsl{G_1 \, \& \, G_2}}{\seqsl{G_1} & \seqsl{G_2}}}
\newcommand{\rlsimp}{\inferH[\rlnmsimp{}]{\seqsl{D \longrightarrow G}}{\seqsl[\dyncon{} \, , \, D]{G}}}
\newcommand{\rlsall}{\inferH[\rlnmsall{}]{\seqsl{\sltm{All} \; G}}{\forall (E : \hybridtm{expr con}), (\sltm{proper} \; E \rightarrow \seqsl{G \, E})}}
\newcommand{\rlsalls}{\inferH[\rlnmsalls{}]{\seqsl{\sltm{Allx} \; G}}{\forall (E : \sltm{X}), (\seqsl{G \, E})}}
\newcommand{\rlssome}{\inferH[\rlnmssome{}]{\seqsl{\sltm{Some} \; G}}{\sltm{proper} \; E & \seqsl{G \, E}}}
\newcommand{\rlbmatch}{\inferH[\rlnmbmatch{}]{\bchsl{\atom{A}}{A}}{}}
\newcommand{\rlbanda}{\inferH[\rlnmbanda{}]{\bchsl{D_1 \, \& \, D_2}{A}}{\bchsl{D_1}{A}}}
\newcommand{\rlbandb}{\inferH[\rlnmbandb{}]{\bchsl{D_1 \, \& \, D_2}{A}}{\bchsl{D_2}{A}}}
\newcommand{\rlbimp}{\inferH[\rlnmbimp{}]{\bchsl{G \longrightarrow D}{A}}{\seqsl{G} & \bchsl{D}{A}}}
\newcommand{\rlball}{\inferH[\rlnmball{}]{\bchsl{\sltm{All} \; D}{A}}{\sltm{proper} \; E & \bchsl{D \, E}{A}}}
\newcommand{\rlballs}{\inferH[\rlnmballs{}]{\bchsl{\sltm{Allx} \; D}{A}}{\bchsl{D \, E}{A}}}
\newcommand{\rlbsome}{\inferH[\rlnmbsome{}]{\bchsl{\sltm{Some} \; D}{A}}{\forall (E : \hybridtm{expr con}), (\sltm{proper} \; E \rightarrow \bchsl{D \, x}{A})}}


%%%%%%%%%%%%%%%%%%%%%%%%%
% FORMATTING COQ
%%%%%%%%%%%%%%%%%%%%%%%%%

\lstdefinelanguage{Coq}
{
  % list of keywords
  morekeywords={
    fun,
    forall,
    exists,
    Set,
    Prop,
    Type,
    Definition,
    Inductive,
    Fixpoint,
    Theorem,
    Lemma,
    Example,
    Proof,
    Qed
  },
  emph={
    %hodb,
    %hAbs, hApp, dAbs, dApp, dVar,
    s_bc, s_init, s_imp, s_all, s_alls, s_some,
    b_match, b_and1, b_and2, b_imp, b_all, b_alls, b_some,
    APP, CON, ABS, VAR, lambda, abstr
    },
  %emphstyle={\color{Brightpurple}},
  sensitive=true, % keywords are case-sensitive
  morecomment=[l]{//}, % l is for line comment
  morecomment=[s]{/*}{*/}, % s is for start and end delimiter
  morestring=[b]" % defines that strings are enclosed in double quotes
}
 
% Set Language
\lstset{
  language={Coq},
  basicstyle=\small\ttfamily, % Global Code Style
  captionpos=b, % Position of the Caption (t for top, b for bottom)
  extendedchars=true, % Allows 256 instead of 128 ASCII characters
  tabsize=2, % number of spaces indented when discovering a tab 
  columns=fixed, % make all characters equal width
  keepspaces=true, % does not ignore spaces to fit width, convert tabs to spaces
  showstringspaces=false, % lets spaces in strings appear as real spaces
  breaklines=true, % wrap lines if they don't fit
  %frame=trbl, % draw a frame at the top, right, left and bottom of the listing
  %frameround=tttt, % make the frame round at all four corners
  framesep=4pt, % quarter circle size of the round corners
  %numbers=left, % show line numbers at the left
  %numberstyle=\tiny\ttfamily, % style of the line numbers
  %commentstyle=\color{eclipseGreen}, % style of comments
  %keywordstyle=\color{Forestgreenlt}, % style of keywords
  %stringstyle=\color{eclipseBlue}, % style of strings
}


%%%%%%%%%%%%%%%%%%%%%%%%%
% FORMATTING, OTHER
%%%%%%%%%%%%%%%%%%%%%%%%%

\newcommand{\type}[1]{\texttt{Type\ensuremath{_{#1}}}}
\newcommand{\N}{\ensuremath{\mathbb{N}}}
\newcommand{\Z}{\ensuremath{\mathbb{Z}}}
\newcommand{\cic}{\textsc{Cic}}
\newcommand{\coc}{\textsc{CoC}}
\newcommand{\reduce}{\; \triangleright_{\beta\delta\iota} \;}
%\newcommand{\seq}[2][\Gamma]{\ensuremath{#1 \vdash #2}}
\newcommand{\subtype}{\leq_{\beta\delta\iota}}

\newcommand{\pfshift}{-5pt}

\newcommand{\ol}[1]{\overline{#1}}

\newcommand{\args}[1]{(\hspace{-0.35em} \langle #1 \rangle \hspace{-0.35em})}

\newtheorem{theorem}{Theorem}
\newtheorem*{thm}{Theorem}
\newtheorem{lem}{Lemma}
\newtheorem{corollary}{Corollary}


\newcommand{\cL}{{\cal L}}
\newcommand{\foldn}{\ensuremath{FO\lambda^{\Delta I\!\!N}}}

\begin{document}

\special{papersize=8.5in,11in}
\setlength{\pdfpageheight}{\paperheight}
\setlength{\pdfpagewidth}{\paperwidth}

\conferenceinfo{LFMTP '16}{June 23, 2016, Porto, Portugal}
\copyrightyear{2016}
%\copyrightdata{978-1-nnnn-nnnn-n/yy/mm}
\copyrightdata{978-1-4503-4777-8/16/06}
\copyrightdoi{http://dx.doi.org/10.1145/2966268.2966271}

% Uncomment the publication rights you want to use.
\publicationrights{transferred}
%\publicationrights{licensed}     % this is the default
%\publicationrights{author-pays}

%\titlebanner{banner above paper title}        % These are ignored unless
%\preprintfooter{short description of paper}   % 'preprint' option specified.

\title{The Logic of Hereditary Harrop Formulas as a Specification
  Logic for Hybrid}
%\title{Two-Level Reasoning in Hybrid with Higher-Order Logic}
%\subtitle{}

\authorinfo{Chelsea Battell}
           {Department of Mathematics and Statistics\\
            University of Ottawa, Canada}
           {cbattell@uottawa.ca}
\authorinfo{Amy Felty}
           {School of Electrical Engineering and Computer Science\\
            and Department of Mathematics and Statistics\\
            University of Ottawa, Canada}
           {afelty@uottawa.ca}

\maketitle

\begin{abstract}
  Hybrid is a logical framework that supports the use of higher-order
  abstract syntax (HOAS) in representing formal systems or ``object
  logics'' (OLs).  It is implemented in Coq and follows a two-level
  approach, where a specification logic (SL) is implemented as an
  inductive type and used to concisely and elegantly encode the
  inference rules of the formal systems of interest.  In this paper,
  we develop a new higher-order specification logic for Hybrid.  By
  increasing the expressive power of the SL beyond what was considered
  previously, we increase the flexibility of encoding OLs and thus
  extend the class of formal systems for which we can reason about
  efficiently.  We focus on formalizing the meta-theory of the SL.
% new:
  We develop an abstract way in which to present an important class of
  meta-theorems.  This class includes
%
  properties such as weakening, contraction, exchange, and the
  admissibility of the cut rule.  The cut admissibility theorem
  establishes consistency and also provides justification for
  substituting a formula for an assumption in a context of
  assumptions.  It can greatly simplify reasoning about OLs in systems
  that provide HOAS.
%  We discuss the proof of this theorem in detail, illustrating the
%  challenges of its formalization.
% new:
  We present the abstraction and show how it is used to prove all of
  these theorems.
\end{abstract}

%\category{CR-number}{subcategory}{third-level}
\category{F.3.1}{Logics and Meanings of Programs}{Specifying and
Verifying and Reasoning about Programs}
\category{F.4.1}{Mathematical Logic and Formal Languages}{Mathematical
Logic}

% general terms are not compulsory anymore,
% you may leave them out
%\terms
%term1, term2

\keywords
logical frameworks, higher-order abstract syntax, cut admissibility,
structural rules, interactive theorem proving, Coq

\section{Introduction}

\emph{Logical frameworks} provide general languages in which it is possible to represent a wide variety of logics, programming languages,
and other formal systems.  They are designed to capture uniformities of the syntax and inference systems of these \emph{object logics} (OLs) and to provide support for implementing and reasoning about them.  Hybrid \cite{FeltyMomigliano:JAR10} is a logical framework that provides support for encoding OLs via \emph{higher-order abstract   syntax} (HOAS), also referred to as \emph{lambda-tree syntax}.
%\cite{MillerP99}.
Using HOAS, binding constructs in the OL are encoded using the binding constructs provided by an underlying $\lambda$-calculus or function space of the logical framework (the \emph{meta-language}).  Using such a representation allows us to delegate to the meta-language $\alpha$-conversion and capture-avoiding substitution.  Further, object logic substitution can be rendered as meta-level $\beta$-conversion.  HOAS encodings aim to relieve users from having to build up common (and often large) infrastructure implementing operations dealing with variables, such as capture-avoiding substitution, renaming, and fresh name generation.  In addition, in such logical frameworks, embedded implication and universal quantification are often used to represent \emph{hypothetical} and \emph{parametric} judgments, also called \emph{generic} judgments, %\cite{MillerTiu:TOCL05},
which allow elegant and succinct specifications of OL inference rules.

An important goal of Hybrid is to exploit the advantages of HOAS within the well-understood setting of higher-order logic as implemented by systems such as Isabelle and Coq.\footnote{Although Hybrid has been implemented in both Coq and Isabelle/HOL, we use the Coq version in this paper.}  Building on such a system allows us to easily experiment with new specification logics.  It also provides a high degree of trust; for instance proof terms in Coq serve as proof certificates, which can be independently checked.  In addition, Hybrid in Coq inherits Coq's full recursive function space as well as its extensive set of libraries.

Hybrid is implemented as a two-level system, an approach first introduced in the $\foldn$ logic \cite{McDowellMiller:TOCL01}, and now applied within a variety of logics and systems, such as the Abella interactive theorem prover \cite{Gacek:IJCAR08}.  In a two-level system, the \emph{specification} and (inductive) \emph{meta-reasoning} are done within a single system but at different \emph{levels}. An intermediate level is introduced by inductively defining a \emph{specification logic} (SL) in Coq, and OL judgments (including hypothetical and parametric judgments) are encoded in the SL.  Several meta-theoretic properties about the SL provide powerful tools for reasoning about OLs.  For example, the cut admissibility theorem provides a direct and convenient way to substitute a formula for an assumption in a context of assumptions.  Structural properties of the SL, such as weakening, contraction, and exchange, also provide tools that can be directly applied to reasoning in any OL.

%In this paper, we introduce an intuitionistic higher-order SL, namely
%the logic of higher-order hereditary Harrop formulas (HoHH)
%\cite{LProlog} with some restrictions, mainly on the types of terms
%in quantified formulas.  Previous SLs considered for Hybrid include a
%second-order fragment of HoHH and an ordered linear logic
%\cite{FeltyMomigliano:JAR10}.
%
In this paper, we introduce an intuitionistic higher-order SL, namely the logic of hereditary Harrop formulas (HH).  HH is a sublogic of the logic of higher-order hereditary Harrop formulas as presented in \cite{LProlog}.  Two kinds of ``order'' can be seen in this logic, the domain of quantification and the implicational complexity.  In terms of the former, HH allows quantification over second-order types, and in terms of the latter, HH is higher-order, allowing any level of nested implications.  Previous SLs considered for Hybrid include the fragment of HH with second-order implicational complexity and an ordered linear logic \cite{FeltyMomigliano:JAR10}.

We adopt a minor variation of the inference rules for HH used as an SL in recent versions of the Abella interactive theorem prover \cite{WCGN:PPDP13}.  We present our encoding in Coq, and discuss the proofs of meta-theoretic properties in some detail.  In particular, we develop an abstraction to capture uniformities across proofs of different meta-theoretic properties.  Cut admissibility in particular relies on a fairly complex inductive argument, involving mutual inductions and sub-inductions.  Our proof follows the tradition of many other syntactic proofs of cut admissibility for various logics that first induct on the formula depth and then on the proof structure, e.g. \cite{Girard89}.  Furthermore, our proof is \emph{structural} in the sense of \cite{Pfenning:IC00}, in our case using structural induction principles generated by Coq. %We discuss different strategies and even false starts, and then
We present the proofs via our abstraction, with the goal of providing a deeper insight into the proofs and the formalization process.  Variants of the properties we prove have also been proved in Abella. They are mentioned in \cite{WCGN:PPDP13}, but proofs are not presented there.  We briefly discuss some differences.

Our overall goal is to extend the reasoning power of Hybrid. Implementing HH as a new SL in Hybrid now allows us to directly encode, for example, the two OLs in the case studies considered in \cite{WCGN:PPDP13}.  The first involves reasoning about the correspondence between an HOAS encoding and a de Bruijn representation of the terms of the untyped $\lambda$-calculus, while the second involves reasoning about a structural characterization of reductions on untyped $\lambda$-terms, and is originally posed in \cite{LProlog}. Other examples we intend to study include the elegant algorithmic specification of bounded subtype polymorphism in System F in~\cite{Pientka:TPHOLs07}, which comes from the \textsl{PoplMark} challenge~\cite{Aydemir05TPHOLs}, as well as specifications of continuation-passing transformations in functional programs.  The specification of the main judgments of all of these OLs will benefit from the availability of embedded implication in HH (in particular, using two or three levels).

We also note that in addition to the advantages mentioned above with regard to implementing Hybrid inside a well-established theorem prover, Hybrid also provides an ideal setting in which to quickly prototype and experiment with new SLs.  Each one is developed as a library and a user can choose and import one that is best-suited to the task at hand and/or move between them easily.  For example, case studies that don't require the expressiveness of HH can use the second-order fragment, likely leading to simpler proofs.  Case studies that are better suited to a linear logic can directly import and use a linear SL, etc.  In contrast, in Abella, a slight extension of HH replaced the SL used in earlier versions of the system.  Fixing the SL allows developers to focus more on adding powerful automation for a particular SL, and thus proofs in Hybrid currently require more interaction.

In Section~\ref{sec:hybrid}, we give a brief introduction of Hybrid. In Section~\ref{sec:sl}, we introduce HH as an example specification layer and describe its implementation in Coq. Highlights of the the mutual structural induction used in later proofs is found in Section~\ref{sec:induction} followed by the presentation of a generalized SL in Section~\ref{sec:gsl} and a proof technique using this generalized SL in Section~\ref{sec:pfgsl}. Section~\ref{sec:structrules} outlines proofs of the structural rules of HH, while Section~\ref{sec:cutadmiss} details the proof of cut admissibility. Finally, Section~\ref{sec:concl} concludes and discusses related and future work. The files of the Coq formalization are available at \url{www.eecs.uottawa.ca/~afelty/lfmtp16/}.


\index{Hybrid}

%\newcommand{\elist}{\epsilon}
\newcommand{\hybrid}{Hybrid}
\newcommand{\llFun}[2]{\mathsf{fun}\,#1.\,#2}
\newcommand{\llRec}[2]{\mathsf{fix}\,#1.\,#2}
\newcommand{\llrec}[1]{\ikw{fix} \  x\, .\, #1}
\newcommand{\ikw}[1]{\ensuremath{\mathsf{#1}}}
\newcommand{\hastype}{\mathrel{:}}
\newcommand{\slvdn}[3]{{#1}\rhd_{{#2}} {#3}}

\begin{figure}    \setlength{\unitlength}{4144sp}  \begingroup\makeatletter\ifx\SetFigFont\undefined
    \def\x#1#2#3#4#5#6#7\relax{\def\x{#1#2#3#4#5#6}}  \expandafter\x\fmtname xxxxxx\relax \def\y{splain}  \ifx\x\y   \gdef\SetFigFont#1#2#3{    \ifnum #1<17\tiny\else \ifnum #1<20\small\else \ifnum
    #1<24\normalsize\else \ifnum #1<29\large\else \ifnum
    #1<34\Large\else \ifnum #1<41\LARGE\else \huge\fi\fi\fi\fi\fi\fi
    \csname #3\endcsname}  \else \gdef\SetFigFont#1#2#3{\begingroup \count@#1\relax \ifnum
    25<\count@\count@25\fi
    \def\x{\endgroup\@setsize\SetFigFont{#2pt}}    \expandafter\x \csname \romannumeral\the\count@
    pt\expandafter\endcsname \csname @\romannumeral\the\count@
    pt\endcsname \csname #3\endcsname}  \fi \fi\endgroup
  %\begin{picture}(4692,2010)(34,-1198) \thinlines
  \begin{picture}(0,2010)(34,-1198) \thinlines
    {\color[rgb]{0,0,0}\put(1456,269){\oval(210,210)[bl]}
      \put(1456,509){\oval(210,210)[tl]}
      \put(2821,269){\oval(210,210)[br]}
      \put(2821,509){\oval(210,210)[tr]} \put(1456,164){\line( 1,
        0){1365}} \put(1456,614){\line( 1, 0){1365}}
      \put(1351,269){\line( 0, 1){240}} \put(2926,269){\line( 0,
        1){240}} }    {\color[rgb]{0,0,0}\put(1006,-181){\oval(210,210)[bl]} \put(1006,
      59){\oval(210,210)[tl]} \put(3271,-181){\oval(210,210)[br]}
      \put(3271, 59){\oval(210,210)[tr]} \put(1006,-286){\line( 1,
        0){2265}} \put(1006,164){\line( 1, 0){2265}}
      \put(901,-181){\line( 0, 1){240}} \put(3376,-181){\line( 0,
        1){240}} }    {\color[rgb]{0,0,0}\put(556,-631){\oval(210,210)[bl]}
      \put(556,-391){\oval(210,210)[tl]}
      \put(3721,-631){\oval(210,210)[br]}
      \put(3721,-391){\oval(210,210)[tr]} \put(556,-736){\line( 1,
        0){3165}} \put(556,-286){\line( 1, 0){3165}}
      \put(451,-631){\line( 0, 1){240}} \put(3826,-631){\line( 0,
        1){240}} }    {\color[rgb]{0,0,0}\put(151,-1081){\oval(210,210)[bl]}
      \put(151,-841){\oval(210,210)[tl]}
      \put(4216,-1081){\oval(210,210)[br]}
      \put(4216,-841){\oval(210,210)[tr]} \put(151,-1186){\line( 1,
        0){4065}} \put(151,-736){\line( 1, 0){4065}} \put(
      46,-1081){\line( 0, 1){240}} \put(4321,-1081){\line( 0, 1){240}}
    %}    \put(1936,-556){\makebox(0,0)[lb]{\smash{\SetFigFont{10}{12.0}{rm}{\color[rgb]{0,0,0}\hybrid}        }}}
    }    \put(1736,-556){\makebox(0,0)[lb]{\smash{\SetFigFont{10}{12.0}{rm}{\color[rgb]{0,0,0}\hybrid}        }}}
    \put(1851,-1006){\makebox(0,0)[lb]{\smash{\SetFigFont{10}{12.0}{rm}{\color[rgb]{0,0,0}Coq}        }}}
    %\put(3376,504){\makebox(0,0)[lb]{\smash{\SetFigFont{10}{12.0}{rm}{\color[rgb]{0,0,0}Syntax:
    %        $\llFun{x}{E\ x}, \llrec{E\ x}\dots$ }        }}}
    %\put(3376,279){\makebox(0,0)[lb]{\smash{\SetFigFont{10}{12.0}{rm}{\color[rgb]{0,0,0}Semantics:
    %        typing $E\hastype t$,\dots}        }}}

    \put(3601,299){\makebox(0,0)[lb]{\smash{\SetFigFont{10}{12.0}{rm}{\color[rgb]{0,0,0}          } }}} 

    %\put(3626,10){\makebox(0,0)[lb]{\smash{\SetFigFont{10}{12.0}{rm}{\color[rgb]{0,0,0}Sequent
    %        calculus: $\slvdn{\Gamma}{n}{G}$}        }}}
    \put(4051,-151){\makebox(0,0)[lb]{\smash{\SetFigFont{10}{12.0}{rm}{\color[rgb]{0,0,0}}        }}}
    %\put(4000,-421){\makebox(0,0)[lb]{\smash{\SetFigFont{10}{12.0}{rm}{\color[rgb]{0,0,0}Meta-language:
    %        quasi}        }}}
    %\put(4000,-601){\makebox(0,0)[lb]{\smash{\SetFigFont{10}{12.0}{rm}{\color[rgb]{0,0,0}
    %        datatype for
    %        a $\lambda$-calculus}        }}}
    %\put(4500,-871){\makebox(0,0)[lb]{\smash{\SetFigFont{10}{12.0}{rm}{\color[rgb]{0,0,0}Ambient
    %        logic:}        }}}
    %\put(4500,-1071){\makebox(0,0)[lb]{\smash{\SetFigFont{10}{12.0}{rm}{\color[rgb]{0,0,0}
    %        tactics/simplifier}        }}}
    %\put(4500,-1271){\makebox(0,0)[lb]{\smash{\SetFigFont{10}{12.0}{rm}{\color[rgb]{0,0,0}
    %        (co)induction}        }}}
    %\put(1800,344){\makebox(0,0)[lb]{\smash{\SetFigFont{10}{12.0}{rm}{\color[rgb]{0,0,0}Object
    \put(1500,344){\makebox(0,0)[lb]{\smash{\SetFigFont{10}{12.0}{rm}{\color[rgb]{0,0,0}Object
            logic}        }}}
    \put(1400,-106){\makebox(0,0)[lb]{\smash{\SetFigFont{10}{12.0}{rm}{\color[rgb]{0,0,0}Specification
            logic}        }}}
  \end{picture}
  \caption{Architecture of the Hybrid system}
  \label{fig:arch}
\end{figure}


\begin{figure}
\begin{center}
\includegraphics[height=5cm]{HybridFig.png}
\caption{High-Level Hybrid Structure \label{fig:hybrid}}
\end{center}
\end{figure}

%Recall from Chapter~\ref{ch:intro}, Hybrid is a two-level logical framework implementing operators that allow higher-order abstract syntax (HOAS) encodings of object logics (OLs) to be expressed.
%A logic that we wish to study using these systems is called an object logic (OL).
%Hybrid is implemented as a library in the interactive theorem proving language Coq, thus making it relatively easy to modify and extend the reasoning power by the addition of new intermediate logics called specification logics. One can choose the simplest specification logic necessary for the present task, or possibly a combination of more than one depending on the OL to be encoded. 
In this chapter each layer of Hybrid will be explored to provide more intuition on how it is constructed and used. This explanation will be driven by an analogy, for use as an aid to both memory and understanding of the system.

The orientation of the layers is as in Figure~\ref{fig:hybrid}. We will first consider the top layer, the object logic, in Section~\ref{sec:hybridol} with an example to motivate what we are trying to accomplish. Next we will consider each layer bottom-up, beginning with the ambient logic in Section~\ref{sec:hybridcoq}, then the higher-order abstract syntax layer in Section~\ref{sec:hybridhoas}. Continuing up the stack we next come to the specification logic. Since much of the work presented later is on the implementation and metatheory of the specification logic required for our motivating example of Section~\ref{sec:hybridol}, we will not see details of the specification logic here. Rather, Section~\ref{sec:hybridsl} will illustrate the benefit a specification logic adds to Hybrid and reinforce its necessity. This will be followed by another look at the object logic in Section~\ref{sec:hybrid_ol_imp}, but this time we will be focusing on implementation details with the rest of the system in place. To conclude this chapter, Section~\ref{sec:hybridcompare} will compare Hybrid with alternative architectures for systems intended to reason about object logics using HOAS.

\section{Object Logics}
\label{sec:hybridol}
\index{object logic}

\begin{sidestory}
Suppose we wish to study flowers and create things with them. Then we need to be able to grow flowers.
\end{sidestory}

Suppose we wish to prove something about a programming language or logic, the OL. This language will have rules expressing syntax and semantics that we need to encode in some proof assistant so that we can reason about it. It is also necessary to define the judgments of this language so that we can make claims about the OL.

%\subsection{Example OL}
\begin{expl}[Object Logic: Equivalence of Named and Nameless \lambda-terms]

We consider one of the examples presented in~\cite{WCGN:PPDP13}. Following the presentation there, we can define a syntax and rules expressing direct and de Bruijn representations of untyped $\lambda$-terms. By direct we mean the standard notation for $\lambda$-terms where abstractions reference a named variable that may be used in the body of the abstraction. De Bruijn indices~\cite{debruijn}\index{de Bruijn indices} are a nameless representation of \lambda-terms where rather than using variable names, a natural number is used for occurrences of a variable.

Let $n$ represent a natural number, $x$ a variable, and $e$ and $d$ represent direct and de Bruijn representations, respectively. Then the following are grammars for these $\lambda$-terms:
\begin{align*}
e &::= x \;\; | \;\; \lambda x . e \;\; | \;\; e \; e \\
d &::= n \;\; | \;\; \lambda d \;\; | \;\; d \; d
\end{align*}
A natural number $n$ in the grammar for de Bruijn terms $d$ serves as a pointer to the abstraction bounding that variable. This representation of \lambda-terms is more efficient for computation as we can avoid issues surrounding bound variable names. The $\lambda$-term $\lambda x . \lambda y . x \; y$ can be written using de Bruijn indices as $\lambda \; (\lambda \; (2 \; 1))$. The number $2$ refers to the outer binder (it is contained in two abstractions) and $1$ refers to the inner binder.

An example property we might want to prove is that these two representations are equivalent (or seen another way, to construct equivalent $\lambda$-terms in these different forms). This logic has a judgment to say that \lambda-term $e$ is equivalent to de Bruijn term $d$ at depth $n$, written $e \equiv_n d$. There are three inference rules expressing equivalence of these two kinds of terms, one for each of application, abstraction, and variables, seen below.
$$
\inferH[\rl{hodb\_app}]{\dyncon{} \vdash e_1 \; e_2 \equiv_n d_1 \; d_2}{\dyncon{} \vdash e_1 \equiv_n d_1 & \dyncon{} \vdash e_2 \equiv_n d_2}
$$

$$
\inferH[\rl{hodb\_abs}]{\dyncon{} \vdash \lambda x . e \equiv_n \lambda d}{\dyncon{} , x \equiv_{n+k} k \vdash e \equiv_{n+1} d}
$$

$$
\inferH[\rl{hodb\_var}]{\dyncon{} \vdash x \equiv_{n+k} k}{x \equiv_{n+k} k \in \dyncon{}}
$$

Applications in the two notations are considered equivalent under $n$ abstractions if their corresponding components are. The rule \rl{hodb\_abs} is more complicated to understand due to an additional assumption in the context of the premise of the rule. Informally, this rule says if whenever assuming variable $x$ is equivalent to index $k$ at depth $n + k$ it can be shown that the bodies of the \lambda-terms $e$ and $d$ are equivalent at depth $n + 1$, then we can conclude that the abstractions $\lambda x . e$ and $\lambda d$ are equivalent at depth $n$. As an illustration of how to use this system, we will see how to prove $\vdash \lambda x . \lambda y . x \; y \equiv_0 \lambda \; (\lambda \; (2 \; 1))$ (i.e. these two \lambda-terms are equivalent under zero additional abstractions).

\paragraph{Claim:} $\vdash \lambda x . \lambda y . x \; y \equiv_0 \lambda \; (\lambda \; (2 \; 1))$

\begin{proof}

Observe that by the \rl{hodb\_var} rule, both sequents below are provable.
\begin{align}
x \equiv_{2} 2 , y \equiv_{2} 1 \vdash x \equiv_{2} 2 \label{eqn:olex1} \\
x \equiv_{2} 2 , y \equiv_{2} 1 \vdash y \equiv_{2} 1 \label{eqn:olex2}
\end{align}
This requires $n=0$ and $k=2$ in~\eqref{eqn:olex1} and $n = k = 1$ in~\eqref{eqn:olex2}. Using the rule \rl{hodb\_app} with~\eqref{eqn:olex1} and~\eqref{eqn:olex2} we derive the sequent
\begin{align}
x \equiv_{2} 2 , y \equiv_{2} 1 \vdash x \; y \equiv_{2} 2 \; 1 \label{eqn:olex3}
\end{align}
Reviewing our claim, we are proving an equivalence of abstractions. The \rl{hodb\_abs} rule is used on~\eqref{eqn:olex3} with $k = 1$.
\begin{align}
x \equiv_{2} 2 \vdash \lambda y . x \; y \equiv_{1} \lambda \; (2 \; 1)
\end{align}
We apply \rl{hodb\_abs} again, this time with $k = 2$.
\begin{align}
\vdash \lambda x . \lambda y . x \; y \equiv_0 \lambda \; (\lambda \; (2 \; 1))
\end{align}
We have derived the sequent claimed provable, so this proof is complete.

\end{proof}

Notice that the $\lambda$ on the left of the equivalence in the conclusion of the rule $\rl{hodb\_abs}$ is a binding operator. This observation will be important when we see how to represent untyped $\lambda$-terms using HOAS in Hybrid in Section~\ref{sec:hybridhoas} and then implement this OL in Section~\ref{sec:hybrid_ol_imp}.

\end{expl}

%A few observations should be made about the rule $\equiv_{\mathit{abs}}$ before moving on. First, there is a binding operator in the conclusion of this rule. A standard abstract syntax encoding of this rule would need to reason about variable naming issues such as \alpha-conversion, \beta-reduction and managing the names of free and bound variables. Second, this rule has a premise with a variable $k$ that is only used in the context of assumptions. In the encoding this will be a parametric judgment and we will have some local quantification in the context of assumptions of this sequent (*TODO: okay? trying to motivate this example). The first observation is justification for encoding in a system using HOAS and the second necessitates the specification logic presented later.

%We have presented a logic but have so far ignored the issue of where and how we will encode it to reason about it formally.

\section{Ambient Logic}
\label{sec:hybridcoq}
\index{ambient logic}
\index{reasoning logic}

\begin{sidestory}
We can plant seeds in the ground and use the natural resources around us to reach our goal. The sun will provide energy and rain will give water.
\end{sidestory}

The ambient logic (also known as the reasoning logic or the meta-meta-logic) is the layer of the system that everything else is defined in. It is an implementation of a logic and so has its own reasoning rules and allows us to define other reasoning systems within it. In our case, this is \cic{} and its implementation in Coq. This is the lowest reasoning level we consider carefully as part of our system; we will not be concerned with lower-level details of the implementation of \cic{} or its compilation. Chapter~\ref{ch:coq} covered all aspects of Coq, the ambient logic of Hybrid, that are necessary for understanding the contributions later in this thesis.

Existing theorem proving systems are an ideal tool to allow a language and its judgments to be encoded without building extra infrastructure. Hybrid is a Coq library (a collection of Coq files), so it is relatively easy to make modular updates to the system and to add new intermediate reasoning layers called specification logics, as will be explained in Section~\ref{sec:hybridsl}.
%Further benefits to using Coq include
Hybrid can also make use of the inductive and interactive reasoning tools of Coq as well as existing Coq libraries.

\section{Representing Higher-Order Abstract Syntax in Hybrid}
\label{sec:hybridhoas}
\index{higher-order abstract syntax}

\begin{sidestory}
As our aspirations continue to grow, we find it difficult to scale up our flower production. When the rain doesn't fall as we require, we manually make up for the shortfall. The task of watering every plant every day is tedious. A dedicated plot of land with an organized arrangement and an irrigation system is a solution to this problem.
\end{sidestory}

Many tedious computations are necessary for each encoding of an OL with binding structures. Examples include fresh name generation and capture-avoiding substitution. Since Hybrid is implemented in an ambient logic that is a typed \lambda-calculus, the technique of \emph{higher-order abstract syntax} (HOAS) can be used for representing OL expressions. The idea is to use the binder of \lambda-calculus, function abstraction, to represent all OL binding operators. Using HOAS one can avoid implementing logic to reason about variable naming concepts, thus inheriting the meta-level solutions to these challenges. In addition, OL renaming and substitution are handled as meta-level \alpha-conversion and \beta-reduction, respectively.

At this level we have a type \hybridtm{expr} (see Figure~\ref{fig:expr}) encoding a de Bruijn index version of the \lambda-calculus designed to be used to represent OL syntax. A parameter \hybridtm{con} is a placeholder for OL constants, to be defined for each OL. We define \hybridtm{var} and \hybridtm{bnd} to be the natural numbers. Hybrid expressions $(\hybridtm{VAR} \; i)$ and $(\hybridtm{BND} \; j)$ represent object-level free and bound variables, respectively. The constructor \hybridtm{APP} is used to build applications and \hybridtm{ABS} to build abstractions in de Bruijn notation.

\begin{figure}
\begin{lstlisting}
Inductive expr : Set :=
| CON : con -> expr
| VAR : var -> expr
| BND : bnd -> expr
| APP : expr -> expr -> expr
| ABS : expr -> expr.
\end{lstlisting}
\caption{Terms in Hybrid \label{fig:expr}}
\end{figure}

Note that \hybridtm{con} is an implicit parameter in the environment it is defined in; uses outside of this environment must explicitly state this parameter (e.g. \sltm{expr con} instead of \sltm{expr}). A type to be given to this placeholder is defined for each OL. For example, the OL in Section~\ref{sec:hybridol} will have constants for application and abstraction for each kind of \lambda-term and a constant for variables in de Bruijn terms. These will be defined as an inductive type that is then used to instantiate the type \hybridtm{expr} for this particular OL. This example is implemented in Section~\ref{sec:hybrid_ol_imp}.

Object-level binding operators are encoded in HOAS using the Hybrid operator $\hybridtm{lambda} : (\hybridtm{expr con} \rightarrow \hybridtm{expr con}) \rightarrow \hybridtm{expr con}$ which is the meta-level binder defined in the Hybrid library. When using it to encode HOAS, the expanded definition is the underlying de Bruijn notation using only the constructors of \hybridtm{expr}. Although a Hybrid user never sees the expanded form and only works at the HOAS level. As an example, consider the untyped \lambda-term $(\lambda x . \lambda y . x \; y)$. We can represent this in Hybrid as $(\hybridtm{lambda} \; (\lambda x . (\hybridtm{lambda} \; (\lambda y . x \; y))))$ which expands to $\hybridtm{ABS} \; (\hybridtm{ABS} \; (\hybridtm{APP} \; (\hybridtm{BND} \; 1) \; (\hybridtm{BND} \; 0)))$. The \hybridtm{lambda} operator and the constructors of \hybridtm{expr} are used to encode OL syntax.

%Our example of Section~\ref{sec:hybridol} is continued in Section~\ref{sec:hybrid_ol_imp} where we can see how \hybridtm{lambda} is used in defining the OL constants and syntax in Figure \ref{fig:hoasdb_con}.


\section{Specification Logic}
\label{sec:hybridsl}
\index{specification logic}

\begin{sidestory}
Not all flowers will grow in the same conditions. Given any plot of land, there are many plants that will not grow there because they need specific nutrients in their soil. We can create different soil mixes depending on the needs of different classes of flowers.
\end{sidestory}

There are OL judgments that we cannot encode as an inductive type in Coq. One example is a HOAS encoding of inference rules assigning simple types to \lambda-expressions.
%The HOAS  rule for typing abstractions contains negative occurrences of this judgment, which is not allowed by the Coq type system. This can be seen in the HOAS encoding of the rule for typing abstractions.
The standard rule for typing abstractions can be seen in Figure~\ref{fig:stlctp}. Building on the example of Section~\ref{sec:introhoas}, let \coqtm{typ} be the type of OL types in the encoding in Coq. Let \coqtm{arr} be a constant of type $\coqtm{typ} \rightarrow \coqtm{typ} \rightarrow \coqtm{typ}$ representing arrow types. Recall \coqtm{tm} is the type of OL terms.
%Suppose that we have constants expressing the higher-order syntax of terms, including \coqtm{lambda} of type $(\underline{\coqtm{tm}} \rightarrow \coqtm{tm}) \rightarrow \coqtm{tm}$.
We want to define a typing predicate $\coqtm{tp} : \coqtm{tm} \rightarrow \coqtm{typ} \rightarrow \coqtm{Prop}$. Then the HOAS encoding of the rule for typing abstractions would be expressed as
\begin{align*}
\forall (T \; T' : \coqtm{typ}) \; & (E : \coqtm{tm} \rightarrow \coqtm{tm}), \\
& (\forall (x : \coqtm{typ}), \underline{\coqtm{tp} \; x \; T} \rightarrow \coqtm{tp} \; (E \; x) \; T') \rightarrow \coqtm{tp} \; (\coqtm{lambda} \; E) \; (\coqtm{arr} \; T \; T').
\end{align*}
Note that the \coqtm{tp} predicate cannot be expressed inductively because of the (underlined) \emph{negative occurrence} of the \coqtm{tp} predicate in the above formula for the typing abstraction rule. Inductive types with negative recursive occurrences is not allowed by the Coq type system.

As a solution to the problem of needing to reason about judgments that violate this strict positivity requirement, Hybrid is a two-level system. By two-level we mean an intermediate specification level is introduced between the OL encoding and the meta-levels. The specification logic is less expressive than the ambient logic, the calculus of constructions, but it allows us to encode judgments with negative occurrences.

\begin{figure}
$$
%\inferH[tp\_abs]{(\lambda x \, . \, E \; x) : (T \rightarrow T')}{\forall x , \underline{x : T} \rightarrow (E \; x) : T'}
\inferH[tp\_abs]{\dyncon{} \vdash \lambda x \, . \, E : T \rightarrow T'}{\dyncon{} , x : T \vdash E : T'}
%\inferH[tp\_abs]{\dyncon{} \vdash \mathit{tp} \; (\lambda x \, . \, E) \; (T \rightarrow T')}{\dyncon{} , \underline{\mathit{tp} \; x \; T} \vdash \mathit{tp} \; E \; T'}
$$
\centering{\caption{Typing of \lambda-calculus Abstractions \label{fig:stlctp}}}
\end{figure}

Hybrid is a Coq library and as mentioned earlier, this architectural decision makes quick prototyping of SLs possible. Another important benefit is that one can choose the simplest specification logic necessary for the present task, or possibly a combination of more than one depending on the OL to be encoded. Judgments that can be defined inductively do not need to be defined in a SL. This may simplify proofs of OL properties as the user can avoid using a more complicated logic than necessary.

The two levels of the OL and SL interact through a parameter of the SL,
$$
\sltm{prog} : \sltm{atm} \rightarrow \sltm{oo} \rightarrow \coqtm{Prop},
$$
which is used to encode inference rules for OL judgments (and thus define provability at the OL level). There are two arguments to \sltm{prog}; the first is the (atomic) inference rule conclusion of type \sltm{atm} and the second a formula of type \sltm{oo} representing the premise(s) of the rule.

We use $a$ for atoms with type \coqtm{atm} and $o$ for formulas of type \coqtm{oo}, possibly with subscripts.

In this implementation, the type \sltm{atm} is a parameter of the SL and is instantiated with an inductive type whose constructors predicates expressing the judgments of a particular OL. For instance, the definition of \coqtm{atm} for our above example might include a predicate $\oltm{hodb} : (\mltm{expr}~ \mltm{con}) \rightarrow \coqtm{nat} \rightarrow (\mltm{expr}~ \mltm{con}) \rightarrow \coqtm{atm}$ relating the higher-order and de Bruijn encodings at a given depth.

The type \sltm{oo} is the type of goals and clauses in the SL. The definition of \sltm{oo} for the SL defined later is in Figure~\ref{fig:oofig}.
\begin{figure}
\begin{lstlisting}
Inductive oo : Type :=
| atom : atm -> oo
| T : oo
| Conj : oo -> oo -> oo
| Imp : oo -> oo -> oo
| All : (expr con -> oo) -> oo
| Allx : (X -> oo) -> oo
| Some : (expr con -> oo) -> oo.
\end{lstlisting}
\centering{\caption{Type of SL Formulas \label{fig:oofig}}}
\end{figure}
The constant \sltm{atom} coerces an atom (a predicate applied to its arguments) to an SL formula. For any $\alpha$ of type \sltm{atm}, we may refer to ($\sltm{atom} \; \alpha$) as an atomic formula. The constructor \sltm{Conj} represents conjunction and \sltm{Imp} is used to build implications. Also note that in this implementation, we restrict the type of universal quantification to two types, (\mltm{expr}~ \mltm{con}) and \mltm{X}, where \mltm{X} is a parameter that can be instantiated with any primitive type; in our running example, \mltm{X} would become \coqtm{nat} for the depth of binding in a de Bruijn term. We leave out disjunction. It is not difficult to extend our implementation to include disjunction and quantification (universal or existential) over other primitive types, but these have not been needed in reasoning about OLs.

We write \atom{a}, ($o_1$ \& $o_2$), and ($o_1 \longrightarrow o_2$) as notation for (\sltm{atom} $a$), (\sltm{Conj} $o_1$ $o_2$), and (\sltm{Imp} $o_1$ $o_2$), respectively. Formulas quantified by \sltm{All} are written $(\sltm{All}~ o)$ or $(\sltm{All}~ \lambda (x:\mltm{expr}~ \mltm{con}) \; . \; o \; x)$, where $o$ has type $\coqtm{expr con} \rightarrow \coqtm{oo}$. The latter is the $\eta$-long form with types included explicitly. The other quantifiers are treated similarly.

The type \sltm{oo} is an inductive type, so Coq will automatically generate the induction principle shown in Figure~\ref{fig:ooip} as discussed in Section~\ref{sec:coqinduction}. We can use this induction principle to prove a statement of the form $\forall (o : \sltm{oo}), P \; o$ for some $P : \sltm{oo} \rightarrow \coqtm{Prop}$. This proof will have one subcase for each constructor of \sltm{oo}.

\begin{figure}
\begin{align*}
\sltm{oo\_ind} &: \forall (P : \sltm{oo} \rightarrow \coqtm{Prop}), \\
(*\sltm{atom}*) & \qquad (\forall (a : \sltm{atm}), P (\atom{a})) \rightarrow \\
(*\sltm{T}*) & \qquad (P \; \sltm{T}) \rightarrow \\
(*\sltm{Conj}*) & \qquad (\forall (o_1 : \sltm{oo}), P \; o_1 \rightarrow \forall (o_2 : \sltm{oo}), P \; o_2 \rightarrow P \; (o_1 \& o_2)) \rightarrow \\
(*\sltm{Imp}*) & \qquad (\forall (o_1 : \sltm{oo}), P \; o_1 \rightarrow \forall (o_2 : \sltm{oo}), P \; o_2 \rightarrow P \; (o_1 \longrightarrow o_2)) \rightarrow \\
(*\sltm{All}*) & \qquad (\forall (o : \sltm{expr con} \rightarrow \sltm{oo}), (\forall (e : \sltm{expr con}), P \; (o \; e)) \rightarrow P \; (\sltm{All} \; o)) \rightarrow \\
(*\sltm{Allx}*) & \qquad (\forall (o : \sltm{X} \rightarrow \sltm{oo}), (\forall (x : \sltm{X}), P \; (o \; x)) \rightarrow P \; (\sltm{Allx} \; o)) \rightarrow \\
(*\sltm{Some}*) & \qquad (\forall (o : \sltm{expr con} \rightarrow \sltm{oo}), (\forall (e : \sltm{expr con}), P \; (o \; e)) \rightarrow P \; (\sltm{Some} \; o)) \rightarrow \\
&\forall (o : \sltm{oo}), P \; o
\end{align*}
\centering{\caption{Induction Principle for \sltm{oo} \label{fig:ooip}}}
\end{figure}

%A Hybrid SL is defined as an inductive type in Coq where each rule is represented by a constructor of the type. The constructor name is the rule name, and the type arrow is seen as implication.

A Hybrid SL is defined as an inductive type in Coq to encode a sequent calculus. Each rule of the sequent calculus is represented by a constructor of the inductive type. The constructor name is the rule name and the type arrow is used for implication from premises to conclusion. The context of the sequent is defined to behave as a set of elements of type \sltm{oo}. We write $\dyncon{}$ or $c$ for contexts.

Since we explore the SL and proofs of its structural properties in detail later when describing the contributions of this research, we cut short the discussion here. For continuity in this chapter, some notation and the meaning of provability judgments of the SL are all we need now. We write $\seqsl{o}$ to denote an SL, where $\dyncon{}$ has type \coqtm{context} and $o$ has type \coqtm{oo}. The symbol $\rhd$ is used as the SL sequent arrow.


\section{Example OL Implementation}
\label{sec:hybrid_ol_imp}

Now we can see how to encode our example syntax and judgments in Hybrid. Let \oltm{tm} represent the type of direct \lambda-terms and \oltm{dtm} represent the type of de Bruijn terms. Since these are used to form OL expressions, \oltm{tm} and \oltm{dtm} are aliases for \sltm{expr con}. Before stating the implementation of the rules of the logic, we have to define the OL constants. For direct application and abstraction we have $\oltm{hApp} : \oltm{tm} \rightarrow \oltm{tm} \rightarrow \oltm{tm}$ and $\oltm{hAbs} : (\oltm{tm} \rightarrow \oltm{tm}) \rightarrow \oltm{tm}$, respectively. Direct variables are encoded as meta-level variables. For de Bruijn application, abstraction, and variables we have $\oltm{dApp} : \oltm{dtm} \rightarrow \oltm{dtm} \rightarrow \oltm{dtm}$, $\oltm{dAbs} : \oltm{dtm} \rightarrow \oltm{dtm}$, and $\oltm{dVar} : \coqtm{nat} \rightarrow \oltm{dtm}$, respectively.

In Figure~\ref{fig:hoasdb_con} the constants of the OL are defined in the inductive type \oltm{con}. We also have the definitions of OL applications and abstractions for the direct and de Bruijn forms of \lambda-terms in terms of the OL constants and HOAS application and \hybridtm{lambda} operator. Note that in Coq, \coqtm{fun} is notation for abstractions. When we write Coq code we use this notation but when writing pretty-printed versions of the code we will use \lambda-calculus abstraction notation. For example, we often write Coq abstractions \coqtm{fun x => f x} as $\lambda x . f \; x$ because the latter is often more readable in our discussions. In Figure~\ref{fig:hoasdb_con} we can see the use of HOAS in the definition of \oltm{hAbs} where we use the Hybrid \hybridtm{lambda} operator.

\begin{figure}
\begin{lstlisting}
Inductive con : Set := 
| hAPP : con
| hABS : con
| dAPP : con
| dABS : con
| dVAR : nat -> con.

Definition hApp : tm -> tm -> tm :=
  fun (e1 : tm) =>
    fun (e2 : tm) =>
      APP (APP (CON hAPP) e1) e2. 
Definition hAbs : (tm -> tm) -> tm :=
  fun (f : tm -> tm) => 
    APP (CON hABS) (lambda f).

Definition dApp : dtm -> dtm -> dtm :=
  fun (d1 : dtm) =>
    fun (d2 : dtm) =>
      APP (APP (CON dAPP) d1) d2. 
Definition dAbs : dtm -> dtm :=
  fun (d : dtm) =>
    APP (CON cdABS) d.
Definition dVar : nat -> dtm :=
  fun (n : nat) =>
    (CON (dVAR n)).
\end{lstlisting}
\caption{Example OL: Encoding Syntax in Hybrid \label{fig:hoasdb_con}}
\end{figure}

The atomic judgment discussed for this example (equivalence between the two representations of lambda terms) is part of the inductive type \oltm{atm} defined below.
\begin{lstlisting}
Inductive atm : Set :=
| hodb : tm -> nat -> dtm -> atm.
\end{lstlisting}
The predicate \coqtm{hodb} corresponds to the infix $\equiv_n$ relation in the rules in Section~\ref{sec:hybridol} (i.e. $\coqtm{hodb} \; e \; n \; d$ is notation for $e \equiv_n d$). In the environment where the SL is defined, there are parameters \coqtm{atm} for atomic judgments of the OL, \coqtm{con} for OL constants, and \coqtm{X} for another type we wish to universally quantify over. Now the type of SL formulas with all parameters filled in is \oltm{oo atm con X}. This is the type of SL formulas at the OL level. %We define the SL parameter $X$ used for quantification over primitive types as $\coqtm{nat}$ for this OL.

The rules shown in Section~\ref{sec:hybridol} can now be defined in Hybrid using HOAS and a SL. More specifically, we can now define the inductive type \sltm{prog} as shown in Figure~\ref{fig:hoasdb_prog}, where we see the HOAS encoding of the rules in Section~\ref{sec:hybridol}. The inductive type \coqtm{prog} has a constructor for each of the inference rules \rl{hodb\_app} and \rl{hodb\_abs}. As we will see, the \rl{hodb\_var} rule is not represented explicitly because it is taken care of at the level of the SL. The first argument to \coqtm{prog} is an atomic OL inference rule conclusion and the second argument is a formula to encode the premises of the same OL inference rule. The Coq notation for \atom{a} is \coqtm{<<a>>}.

\begin{figure}
\begin{lstlisting}
Inductive prog : atm -> oo atm (expr con) X -> Prop :=
| hobd_app : forall (e1 e2 : tm) (n : nat) (d1 d2 : dtm),
   prog (hodb (hApp e1 e2) n (dApp d1 d2))
    (<<hodb e1 n d1>> & <<hodb e2 n d2>>)
| hodb_abs : forall (f : tm -> tm) (n : nat) (d : dtm),
   abstr f ->
   prog (hodb (hAbs f) n (dAbs d))
    (All (fun (x : tm) =>
     (Allx (fun (k : X) => <<hodb x (n + k) (dVar k)>>)) --->
       <<hodb (f x) (n + 1) d>>)).
\end{lstlisting}
\caption{Example OL: Encoding OL Inference Rules \rl{hodb\_app} and \rl{hodb\_abs} in Hybrid \label{fig:hoasdb_prog}}
\end{figure}

An example theorem for this OL is to prove that the judgment $\oltm{hodb}$ is deterministic in its first and third arguments (and thus the relational definition of the rules represents a function). To do this we want to prove the two theorems below (where $=$ is equality in the ambient logic).

\begin{prop}[\oltm{hodb\_det1}]
\label{thm:hodb_det1}
\begin{align*}
& \forall (\dyncon{} : \mathtt{context}) (e : \mathtt{tm}) (d_1 \; d_2 : \mathtt{dtm}) (n : \mathtt{nat}), \\
& \qquad \seqsl{\atom{\mathtt{hodb} \; e \; n \; d_1}} \rightarrow \seqsl{\atom{\mathtt{hodb} \; e \; n \; d_2}} \rightarrow d_1 = d_2.
\end{align*}
\end{prop}

\begin{prop}[\oltm{hodb\_det3}]
\label{thm:hodb_det3}
\begin{align*}
& \forall (\dyncon{} : \mathtt{context}) (e_1 \; e_2 : \mathtt{tm}) (d : \mathtt{dtm}) (n : \mathtt{nat}), \\
& \qquad \seqsl{\atom{\mathtt{hodb} \; e_1 \; n \; d}} \rightarrow \seqsl{\atom{\mathtt{hodb} \; e_2 \; n \; d}} \rightarrow e_1 = e_2.
\end{align*}
\end{prop}

To prove these in Hybrid, we must first define a SL that is able to reason about this OL. Once we have defined the SL, using it and the encoding of the OL just described, we will be ready to prove the above propositions. As of this writing, these theorems are not proven in Hybrid.

%\begin{theorem}[\oltm{hodb\_det3}]
%\end{theorem}

\section{Comparison to Other Architectures}
\label{sec:hybridcompare}

\begin{sidestory}
Our approach to growing flowers is not the only solution. One alternative is to build a factory specializing in the production of flowers. This would give us full control over lighting, water, and soil composition; but the startup costs are high and modifications can be prohibitively expensive.
\end{sidestory}

Other systems use HOAS for encoding and reasoning about OLs with binders but different choices are made in the implementation of these systems. We will briefly look at the features of the two most closely related systems, Abella~\cite{Gacek:IJCAR08} and Beluga~\cite{Pientka:IJCAR10}, and compare these systems to Hybrid. These three systems, along with Twelf~\cite{TwelfSP}, are compared in detail using benchmark problems in~\cite{FMP:JAR15}.

One feature that sets Hybrid apart from these systems is that Hybrid is a library in an existing theorem proving system while Abella, Beluga, and Twelf are special-purpose theorem proving systems built for reasoning about OLs using HOAS. Using Coq means we can trust the proofs without having to develop extra infrastructure. These proofs can be independently checked because a proof term is a \lambda-term; a proof check is a type check in the Calculus of Constructions, a trusted and well studied theoretical foundation for our work. The trade-off is less control over the reasoning logic of Hybrid and more levels of encoding.

\paragraph{Abella}
\index{Abella}

Abella is an interactive proof environment using the special-purpose $\mathcal{G}$ logic as its reasoning logic. $\mathcal{G}$ is intuitionistic, predicative, higher-order, and has fixed-point definitions for atomic predicates. It also allows mathematical induction (over natural numbers). Infrastructure for reasoning using HOAS is built-in to this logic. Like Hybrid, it is a two-level logical framework. In contrast, since it is a special-purpose system for reasoning about OLs, only one SL is used by the system at a time; to use a different SL the system must be updated. Hybrid is a Coq library so multiple SLs can can be available for use by any OL.

Abella is a tactic-based interactive theorem prover. This is the same style used when using the interactive proof environment in Coq, but the crucial difference is that on completion of a Coq proof the system generates a proof term. This is an object that can be checked independent of the implementation of \cic{} or Coq. This means that rather than trusting the implementation of a language and the tactics, we are provided evidence on completion of the proof. Since Hybrid is implemented in Coq, we have access to proof terms once a theorem is proven. This is not the case in Abella.

An advantage to Abella is that is has the \nabla-quantifier, a new specialized quantifier providing better direct reasoning about binding in OLs. This allows Abella to prove some properties about OLs that cannot be proven in Hybrid until we implement \nabla.

\paragraph{Beluga}
\index{Beluga}

Beluga is also a logical framework for reasoning about OLs using HOAS. The reasoning logic in this system is contextual LF; it supports reasoning over contexts. It is more specialized for reasoning with HOAS than Hybrid is. It implements a type theory instead of a logic.

In Beluga, some metatheory about contexts (e.g. the structural rules of weakening, contraction, and exchange in sequent calculi) is implicit. This means that it is built-in to the implementation rather than being axioms of a logic or proven to be admissible as rules. The benefit of this choice is it is not necessary to prove theses structural rules. The argument against this is that it requires more trust from the user. It is necessary to trust the implementation of the system rather than being able to see how the rules are defined to be axiomatic or proven to be admissible.                                                                 


%The main contribution to Hybrid presented here has been to add a new specification logic (SL) and prove the necessary structural rules. This SL is an inference system based on higher-order hereditary Harrop formulas. (*motivation, names of people whose research this builds on, etc)

%We follow the presentation in~\cite{WCGN:PPDP13} to define this SL in a similar manner as the SL for Abella that is described there. That is, we distinguish between goal-reduction rules and backchaining rules (see figures~\ref{fig:grseq} and~\ref{fig:bcseq}). Goal-reduction rules are the right-introduction rules and reduce a formula to atomic (bottom-up). Backchaining rules are the left-introduction rules and allow backchaining over a formula focused from the context of assumptions at this reasoning level (bottom-up).

%The rules of this logic are encoded in inductive types for goal-reduction and backchaining sequents. Goal-reduction sequents have signature \sltm{grseq} $:$ \sltm{context} $\rightarrow$ \sltm{oo} $\rightarrow$ \coqtm{Prop} and we write \seqsl{\beta} as notation for \sltm{grseq} $\dyncon{}$ $\beta$. Backchaining sequents have signature \sltm{bcseq} $:$ \sltm{context} $\rightarrow$ \sltm{oo} $\rightarrow$ \sltm{atm} $\rightarrow$ \coqtm{Prop} and we write \bchsl{\beta}{\alpha} as notation for \sltm{bcseq} $\dyncon{}$ $\beta$ $\alpha$, understanding $\beta$ to be a formula from \dyncon{} that we focus.

%Something has been ignored in our classification of the rules so far. The rules \rlnmsbc{} and \rlnmsinit{} are presented with the goal-reduction rules in figure~\ref{fig:grseq}, even though they are not used to reduce a goal any further (the conclusion is atomic in these cases). Also, the rule \rlnmbmatch{} is considered a backchaining rule in figure~\ref{fig:bcseq}. The reason for this comes from how these rules are defined in Coq; a rule whose conclusion is a goal-reduction sequent must be defined in \sltm{grseq} and a rule whose conclusion is a backchaining sequent must be defined in \sltm{bcseq}.

%Since \sltm{grseq} references \sltm{bcseq} in the rule \rlnmsinit{} and \sltm{bcseq} references \sltm{grseq} in the rule \rlnmbimp{}, these are defined as mutually inductive types. Both types have a constructor for each rule that has conclusion of their type. The identifier (*right word?) of the constructor is the rule name, and the premises and conclusion of the rule are premises and conclusion of the constructor's type, respectively (with quantification over the necessary parameters in the type). For example, the definition of \sltm{grseq} has a constructor \rlnmsinit{} of type $\forall (L : \sltm{context}) (G : \sltm{oo}), G \in L \rightarrow \bchsl[L]{G}{A} \rightarrow \seqsl[L]{\atom{A}}$.

%The types \seqsl{\beta} and \bchsl{\beta}{\alpha} are dependent types, where $\dyncon{} : \sltm{context}$, $\beta : \sltm{oo}$, and $\alpha : \sltm{atm}$. Before considering the rules of the logic in detail, the types \sltm{atm}, \sltm{oo}, and \sltm{context} defined and used by the SL need to be explained.


The first stage of the contributions outlined in this thesis is defining a specification logic to increase the reasoning power of Hybrid. The new specification logic (SL) for Hybrid is based on hereditary Harrop formulas using an intuitionistic logic with focusing as described in Chapter~\ref{ch:hh}. We adopt a modified version of the rules very close to the style of the rules of the specification logic used in the higher-order version of Abella~\cite{WCGN:PPDP13}. We do not include any rules for disjunction here because they have not been necessary for object logics in case studies of interest. The SL could easily be extended to add these rules and the proofs of SL metatheory would have the same structure, as will be seen in Chapter~\ref{ch:gsl}.

%t the specification level, the terms of the language of hereditary Harrop formulas are the terms of the simply-typed $\lambda$-calculus. 
% When we refer to this logic as higher-order, we mean the
% implicational complexity.
We note that unlike in all previous SLs for Hybrid there is no restriction on the implicational complexity (see~\cite{FeltyMomigliano:JAR10}), because $G$-formulas in higher-order hereditary Harrop language allow $D$-formulas as the antecedent of implication as was seen in Section~\ref{sec:hohh}. In all previous SLs, only atomic formulas were allowed in place of the more general D-formulas allowed here.
%

The SL presented in this chapter is a sequent calculus implemented as an inductive type in Coq. Section~\ref{sec:context} describes how contexts are defined for this SL. Section~\ref{sec:hhsl} presents the Coq implementation of the SL based on hereditary Harrop formulas and we see how to prove properties of this SL by structural induction in Section~\ref{sec:induction}.

In Appendix~\ref{ch:notations}, we list notations that will be used in the rest of the thesis.

\section{Contexts in Coq}
\label{sec:context}

The type \sltm{context} represents contexts of assumptions in sequents and is defined using the Coq \coqtm{ensemble} library as \coqtm{ensemble} \sltm{oo} since we want contexts to behave as sets with elements of type \sltm{oo}. In proofs of some context lemmas stated below we use the \coqtm{ensemble} extensional equality axiom:
$$
\coqtm{Extensionality\_Ensembles} : \forall (E_1 \; E_2 : \coqtm{ensemble}),(\coqtm{Same\_set} \; E_1 \; E_2) \rightarrow E_1 = E_2
$$
where \coqtm{Same\_set} is defined in the \coqtm{Ensemble} library. We use $o \in c$ as notation for $\coqtm{elem} \; o \; c$ which means formula $o$ is an element of context $c$. Context subset, written $\dyncon{}_1 \subseteq \dyncon{}_2$, is defined as $\forall (o : \sltm{oo}), o \in \dyncon{}_1 \rightarrow o \in \dyncon{}_2$.

We write ($\dyncon{}, \beta$) as notation for ($\sltm{context\_cons} \; \dyncon{} \; \beta$). We write write $c$ or $\dyncon{}$ to denote contexts when discussing formalized proofs.

The context lemmas below are proven as part of this work and are used in later proofs in this thesis. See the accompanying source code for the proofs. Note that all variables are externally quantified and each occurrence of $\beta$ and $\dyncon{}$, possibly with subscripts, has type \sltm{oo} and \sltm{context}, respectively.
\begin{lemma}[\sltm{elem\_inv}] %$\forall (c : \mathtt{context}) (o_1 \; o_2 : \sltm{oo}), o_1 \in (c , o_2) \rightarrow (o_1 \in c \vee o_1 = o_2)$
\label{lem:elem_inv}
$$
\vcenter{\infer{(\beta_1 \in \dyncon{}) \vee (\beta_1 = \beta_2)}{\beta_1 \in (\dyncon{}, \beta_2)}}
$$
\end{lemma}

\begin{lemma}[\sltm{elem\_sub}] %$\forall (c : \mathtt{context}) (o_1 \; o_2 : \sltm{oo}),  \rightarrow $
$$
\vcenter{\infer{\beta_1 \in (\dyncon{} , \beta_2)}{\beta_1 \in \dyncon{}}}
$$
\end{lemma}

\begin{lemma}[\sltm{elem\_self}] %$\forall (c : \mathtt{context}) (o : \sltm{oo}), o \in (c , o)$
\label{lem:elem_self}
$$
\vcenter{\infer{\beta \in (\dyncon{} , \beta)}{}}
$$
\end{lemma}

\begin{lemma}[\sltm{elem\_rep}] %$\forall (c : \mathtt{context}) (o_1 \; o_2 : \sltm{oo}),  \rightarrow $
$$
\vcenter{\infer{\beta_1 \in (\dyncon{} , \beta_2)}{\beta_1 \in (\dyncon{} , \beta_2 , \beta_2)}}
$$
\end{lemma}

\begin{lemma}[\sltm{context\_swap}] %$\forall (c : \mathtt{context}) (o_1 \; o_2 : \sltm{oo}), (c , o_1 , o_2) = (c , o_2 , o_1)$
$$
\vcenter{\infer{(\dyncon{} , \beta_1 , \beta_2) = (\dyncon{} , \beta_2 , \beta_1)}{}}
$$
\end{lemma}

\begin{lemma}[\sltm{context\_sub\_sup}] %$\forall (c_1 \; c_2 : \mathtt{context}) (o : \mathtt{context}), c_1 \subseteq c_2 \rightarrow (c_1 , o) \subseteq (c_2 , o)$
\label{lem:context_sub_sup}
$$
\vcenter{\infer{(\dyncon{}_1 , \beta) \subseteq (\dyncon{}_2 , \beta)}{\dyncon{}_1 \subseteq \dyncon{}_2}}
$$
\end{lemma}

\section{Hereditary Harrop Specification Logic in Coq}
\label{sec:hhsl}

%We follow the description in~\cite{WCGN:PPDP13} of a specification logic for Abella based on hereditary Harrop formulas and as described in Chapter~\ref{ch:hh}.
The inference rules of the SL are implemented using two sequent judgments that distinguish between \emph{goal-reduction rules}\index{goal-reduction rule} and \emph{backchaining rules}\index{backchaining rule} which correspond to the right rules and left focused rules, respectively, of Figure~\ref{fig:hohhfoc} in Section~\ref{sec:focusing}. %To differentiate sequents of this implemented inference system, we use $\triangleright$ as the sequent arrow rather than $\vdash$. Goal-reduction sequents have the form \seqsl{G} and backchaining sequents are written \bchsl{D}{A}, where the latter is a left focusing judgment with $D$ the formula under (left) focus.
Figures~\ref{fig:grseq} and~\ref{fig:bcseq} implement the inference rules of the SL (except for the disjunction rules, as mentioned in the introduction to this chapter). They are encoded in Coq as two mutually inductive types, one each for goal-reduction and backchaining sequents. The syntax used in the figures is a pretty-printed version of the Coq inductive types \sltm{grseq} and \sltm{bcseq}. Goal-reduction sequents have type $\sltm{grseq} : \sltm{context} \rightarrow \sltm{oo} \rightarrow \coqtm{Prop}$, and we write $\seqsl{\beta}$ as notation for $\sltm{grseq}~\dyncon{}~ \beta$. Backchaining sequents have type $\sltm{bcseq} : \sltm{context} \rightarrow \sltm{oo} \rightarrow \sltm{atm} \rightarrow \coqtm{Prop}$ and we write \bchsl{\beta}{\alpha} as notation for $\sltm{bcseq}~ \dyncon{}~ \beta~ \alpha$, understanding $\beta$ to be the focused formula from \dyncon{}. The symbol $\forall$ is used for universal quantification in Coq, rather than universal quantification in SL formulas. When we see $\forall$ in the premises of rules, this is to make it clear that it is only over the premise of the rule. 

The rule names in the figures are the constructor names in the inductive definitions in the Coq files. The premises and conclusion of a rule are the argument types and the target type, respectively, of one clause in the definition. Quantification at the outer level is implicit and, as noted, inner quantification is written explicitly in the premises. For example, the linear format of the \rlnmsinit{} rule from Figure~\ref{fig:grseq} with all quantifiers explicit is
$$
\forall (\Gamma : \sltm{context}) (D : \sltm{oo}) (A :\sltm{atm}),D \in \Gamma \rightarrow \bchsl[\Gamma]{D}{A} \rightarrow \seqsl[\Gamma]{\atom{A}}
$$
This is the type of the \rlnmsinit{} constructor in the inductive definition of $\sltm{grseq}$ (see the definition of $\sltm{grseq}$ in the Coq files).

%A number of implementation details can be seen in the rules. 
The notation \atom{A} is to say that $A : \coqtm{atm}$ is coerced to have type \coqtm{oo}, where \coqtm{oo} is the implemented type of formulas (see Figure~\ref{fig:oofig}), referred to as $o$ in Chapter~\ref{ch:hh}. The constants \coqtm{Some} and \coqtm{All} are used for existential and universal quantification in SL formulas, respectively, over the type \coqtm{expr con} which is the type for OL expressions. \coqtm{Allx} is a constant for universal quantification over a type \coqtm{X} of type \coqtm{Set}. We juxtapose appropriate terms to denote application since Coq will reduce the expression, rather than explicitly writing the substitution (for example, compare rule \rl{$\forall_R$} in Figure~\ref{fig:hohhfoc} to rule \rlnmsalls{} in Figure~\ref{fig:grseq}). A final implementation byproduct is the predicate \coqtm{proper} appearing in the premise of some rules. This is used in the Hybrid library to rule out terms of type \coqtm{expr} that have dangling indices (see~\cite{FeltyMomigliano:JAR10}).

In the sequents for this SL there is also an implicit fixed context $\Delta$, called the \emph{static program clauses}, containing closed clauses ($D$-formulas) of the form
$$
\forall_{\tau_1}\ldots\forall_{\tau_n}.G\longrightarrow A
$$
with $n\ge0$. Any set of $D$-formulas can be transformed to an equivalent one that all have this form. These clauses represent the inference rules of an OL. We do not explicitly mention $\Delta$ in the rules for this SL because no rules modify it.

%Our encoding of the formulas of the SL in Coq is shown in Figure~\ref{fig:oofig}. Since \sltm{oo} is an inductive type, Coq will automatically generate the induction principle shown in Figure (*TODO) as discussed in Section~\ref{sec:coqinduction}.

%
%\begin{figure}
%\begin{lstlisting}
%Inductive oo : Type :=
%| atom : atm -> oo
%| T : oo
%| Conj : oo -> oo -> oo
%| Imp : oo -> oo -> oo
%| All : (expr con -> oo) -> oo
%| Allx : (X -> oo) -> oo
%| Some : (expr con -> oo) -> oo.
%\end{lstlisting}
%\centering{\caption{Type of SL Formulas \label{fig:oofig}}}
%\end{figure}
%
%In this implementation, the type \sltm{atm} is a parameter to the definition of \sltm{oo} and is used to define the predicates needed for reasoning about a particular OL. For instance, our above example might include a predicate $\oltm{hodb} : (\mltm{expr}~ \mltm{con}) \rightarrow \coqtm{nat} \rightarrow (\mltm{expr}~ \mltm{con})$ relating the higher-order and de Bruijn encodings at a given depth. The constant \sltm{atom} coerces an atomic formula (a predicate applied to its arguments) to an SL formula. Also, note that in this implementation, we restrict the type of universal quantification to two types, (\mltm{expr}~ \mltm{con}) and \mltm{X}, where \mltm{X} is a parameter that can be instantiated with any primitive type; in our running example, \mltm{X} would become \coqtm{nat} for the depth of binding in a de Bruijn term.  We also leave out disjunction. It is not difficult to extend our implementation to include disjunction and quantification (universal or existential) over other primitive types, but these have not been needed in reasoning about OLs.

%We write \atom{\alpha}, ($\beta_1$ \& $\beta_2$), and ($\beta_1 \longrightarrow \beta_2$) as notation for (\sltm{atom} $\alpha$), (\sltm{Conj} $\beta_1$ $\beta_2$), and (\sltm{Imp} $\beta_1$ $\beta_2$), respectively. Note that we write $\beta$ or $\delta$ for formulas (type \sltm{oo}), and $\alpha$ for elements of type \sltm{atm}, possibly with subscripts. When discussing proofs, we also write $o$ for formulas and $a$ for atoms. When we want to make explicit when a formula is a goal or clause, we write $G$ or $D$, respectively. Formulas quantified by \sltm{All} are written $(\sltm{All}~ \beta)$ or $(\sltm{All}~ \lambda (x:\mltm{expr}~ \mltm{con}) \; . \; \beta x)$. The latter is the $\eta$-long form with types included explicitly. The other quantifiers are treated similarly.

{
\renewcommand{\arraystretch}{3.5}
\newcommand{\GRrlsbc}{\inferH[\rlnmsbc{}]{\seqsl[\Gamma]{\atom{A}}}{\prog{A}{G} & \seqsl[\Gamma]{G}}}
\newcommand{\GRrlsinit}{\inferH[\rlnmsinit{}]{\seqsl[\Gamma]{\atom{A}}}{D \in \Gamma & \bchsl[\Gamma]{D}{A}}}
\newcommand{\GRrlst}{\inferH[\rlnmst{}]{\seqsl[\Gamma]{\sltm{T}}}{}}
\newcommand{\GRrlsand}{\inferH[\rlnmsand{}]{\seqsl[\Gamma]{G_1 \, \& \, G_2}}{\seqsl[\Gamma]{G_1} & \seqsl[\Gamma]{G_2}}}
\newcommand{\GRrlsimp}{\inferH[\rlnmsimp{}]{\seqsl[\Gamma]{D \longrightarrow G}}{\seqsl[\Gamma \, , \, D]{G}}}
\newcommand{\GRrlsall}{\inferH[\rlnmsall{}]{\seqsl[\Gamma]{\sltm{All} \; G}}{\forall (E : \hybridtm{expr con}), (\sltm{proper} \; E \rightarrow \seqsl[\Gamma]{G \, E})}}
\newcommand{\GRrlsalls}{\inferH[\rlnmsalls{}]{\seqsl[\Gamma]{\sltm{Allx} \; G}}{\forall (E : \sltm{X}), (\seqsl[\Gamma]{G \, E})}}
\newcommand{\GRrlssome}{\inferH[\rlnmssome{}]{\seqsl[\Gamma]{\sltm{Some} \; G}}{\sltm{proper} \; E & \seqsl[\Gamma]{G \, E}}}

\begin{figure}%[h]
$$
\begin{tabular}{c c c}
\GRrlsbc{}
&
\GRrlsinit{}
&
\GRrlst{} \\
%
\GRrlsand{}
&
\GRrlsimp{}
&
\GRrlssome{} \\
%
\multicolumn{3}{c}{
\GRrlsall{} \;\;\; \GRrlsalls{}
}
\end{tabular}
$$
\centering{\caption{Goal-Reduction Rules, $\sltm{grseq} : \sltm{context} \rightarrow \sltm{oo} \rightarrow \coqtm{Prop}$ \label{fig:grseq}}}

\end{figure}
}
%
{
\renewcommand{\arraystretch}{3.5}
\newcommand{\BCrlbmatch}{\inferH[\rlnmbmatch{}]{\bchsl[\Gamma]{\atom{A}}{A}}{}}
\newcommand{\BCrlbanda}{\inferH[\rlnmbanda{}]{\bchsl[\Gamma]{D_1 \, \& \, D_2}{A}}{\bchsl[\Gamma]{D_1}{A}}}
\newcommand{\BCrlbandb}{\inferH[\rlnmbandb{}]{\bchsl[\Gamma]{D_1 \, \& \, D_2}{A}}{\bchsl[\Gamma]{D_2}{A}}}
\newcommand{\BCrlbimp}{\inferH[\rlnmbimp{}]{\bchsl[\Gamma]{G \longrightarrow D}{A}}{\seqsl[\Gamma]{G} & \bchsl[\Gamma]{D}{A}}}
\newcommand{\BCrlball}{\inferH[\rlnmball{}]{\bchsl[\Gamma]{\sltm{All} \; D}{A}}{\sltm{proper} \; E & \bchsl[\Gamma]{D \, E}{A}}}
\newcommand{\BCrlballs}{\inferH[\rlnmballs{}]{\bchsl[\Gamma]{\sltm{Allx} \; D}{A}}{\bchsl[\Gamma]{D \, E}{A}}}
\newcommand{\BCrlbsome}{\inferH[\rlnmbsome{}]{\bchsl[\Gamma]{\sltm{Some} \; D}{A}}{\forall (E : \hybridtm{expr con}), (\sltm{proper} \; E \rightarrow \bchsl[\Gamma]{D \, E}{A})}}

\begin{figure}%[h]

$$
\begin{tabular}{c c c}
\BCrlbmatch{}
&
\BCrlbanda{}
&
\BCrlbandb{} \\
%
\BCrlbimp{}
&
\BCrlball{}
&
\BCrlballs{} \\
\multicolumn{3}{c}{
\BCrlbsome{}
}
\end{tabular}
$$
\centering{\caption{Backchaining Rules, $\sltm{bcseq} : \sltm{context} \rightarrow \sltm{oo} \rightarrow \sltm{atm} \rightarrow \coqtm{Prop}$ \label{fig:bcseq}}}

\end{figure}
}
%
The goal-reduction rules of Figure~\ref{fig:grseq} are implemented version of the right introduction rules of this sequent calculus as seen in Figure~\ref{fig:hohhfoc}. The rules \rlnmsbc{} and \rlnmsinit{} are the only goal-reduction rules with an atomic principal formula.

The rule \rlnmsbc{} is used to backchain over the static program clauses $\Delta$, which are defined for each new OL as an inductive type called \sltm{prog} of type $\sltm{atm} \rightarrow \sltm{oo} \rightarrow \coqtm{Prop}$, and represent the inference rules of the OL (this is discussed further in Section~\ref{sec:hybridsl}). The rule \rlnmsbc{} is the interface between the SL and OL layers and we say that the SL is parametric in OL provability. We write \prog{A}{G} for $(\sltm{prog} \; A \; G)$ to suggest backward implication. Recall that clauses in $\Delta$ may have outermost universal quantification. The premise \prog{A}{G} actually represents an instance of a clause in $\Delta$.

The rule \rlnmsinit{} allows backchaining over dynamic assumptions (i.e. a formula from \dyncon{}) and is the implemented version of the \rl{focus} rule of Figure~\ref{fig:hohhfoc}. To use this rule to prove \seqsl{\atom{A}}, we need to show $D \in \dyncon{}$ and \bchsl{D}{A}. Formula $D$ is chosen from, or shown to be in, the dynamic context \dyncon{} and we use the backchaining rules of Figure~\ref{fig:bcseq} to show \bchsl{D}{A} (where $D$ is the focused formula).

The backchaining rules of Figure~\ref{fig:bcseq} are the implemented version of the standard focused left rules for conjunction, implication, and universal and existential quantification seen in Figure~\ref{fig:hohhfoc}. Considered bottom up, they provide backchaining over the focused formula. In using the backchaining rules, each branch is either completed by \rlnmbmatch{} where the focused formula is an atomic formula identical to the goal of the sequent, or \rlnmbimp{} is used resulting in one branch switching back to using goal-reduction rules.

%We mention several Coq tactics when presenting proofs. The main one is the \coqtm{constructor} tactic, which applies a clause of an inductive definition in a backward direction (a step of meta-level backchaining), determining automatically which clause to apply.
%The \coqtm{apply} tactic also does a step of backchaining and takes as
%argument the name of a definition clause, hypothesis, or lemma.  The
%\coqtm{assumption} tactic is the ``base case'' for \coqtm{apply},
%closing the proof when there are no further subgoals.


%(*cut?) Note that the rules \rlnmsbc{} and \rlnmsinit{} are presented with the goal-reduction rules in Figure~\ref{fig:grseq}, even though they are not used to reduce a goal any further (the conclusion is atomic in these cases). Also, the rule \rlnmbmatch{} is considered a backchaining rule in Figure~\ref{fig:bcseq}. The reason for this comes from how these rules are defined in Coq; a rule whose conclusion is a goal-reduction sequent must be defined in \sltm{grseq} and a rule whose conclusion is a backchaining sequent must be defined in \sltm{bcseq}.


%Recall that Coq's dependent products are written $\forall(x_1:t_1)\cdots(x_n:t_n),M$, where $n\ge0$ and for $i=1,\ldots, n$, $x_i$ may appear free in $x_{i+1},\ldots,x_n,M$.  If it doesn't, implication can be used as an abbreviation, e.g., the premise of the \rlnmsall{} rule is an abbreviation for $\forall (E : \hybridtm{expr con})(H:\sltm{proper} \; E), (\seqsl[\Gamma]{G \, E})$.

\section{Mutual Structural Induction}
\label{sec:induction}

%
%This means that there will be a subcase for each rule/constructor, and every rule with a goal-reduction premise will have an induction hypothesis and every rule with a backchaining premise will have an induction hypothesis (two when both are present). \\
%
%Suppose the goal is to prove
Our theorem statements will often have the form
\begin{align*}
(\forall \; (c : \sltm{context}) & \; (o : \sltm{oo}), (\seqsl[c]{o}) \rightarrow (P_1 \; c \; o)) \;\; \wedge \\
(\forall \; (c : \sltm{context}) & \; (o : \sltm{oo}) \; (a : \sltm{atm}), (\bchsl[c]{o}{a}) \rightarrow (P_2 \; c \; o \; a))
\end{align*}
where we extract predicates $P_1 :$ \sltm{context} $\rightarrow$ \sltm{oo} $\rightarrow$ \coqtm{Prop} and $P_2 :$ \sltm{context} $\rightarrow$ \sltm{oo} $\rightarrow$ \sltm{atm} $\rightarrow$ \coqtm{Prop} from the statement to be proven. We can generate an induction principle over the mutually inductive sequent types to allow proof by mutual structural induction. This is done using the Coq \coqtm{Scheme} command.

To prove a statement of the above form by mutual structural induction over \seqsl[c]{o} and \bchsl[c]{o}{a}, 15 subcases must be proven, one corresponding to each inference rule of the SL. The proof state of each subcase of this induction is constructed from an inference rule of the system.
%as follows (where there is fresh quantification over all parameters):
%
We can see a snippet of the sequent mutual induction principle in Figure~\ref{fig:seqind}, where each antecedent (clause of the induction principle defining the cases) corresponds to a rule of the SL and a subcase for an induction using this technique. 
After applying the induction principle, the subcases are generated and
externally quantified variables in each antecedent are introduced to the context of assumptions of the proof state and are then considered \emph{signature variables}.
\begin{figure}%[h]
%\vspace{-20pt}
\begin{align*}
& \sltm{seq\_mutind} : \forall (P_1 : \sltm{context} \rightarrow \sltm{oo} \rightarrow \coqtm{Prop}) \\
& \qquad\qquad\qquad (P_2 : \sltm{context} \rightarrow \sltm{oo} \rightarrow \sltm{atm} \rightarrow \coqtm{Prop}), \\
% g_dyn:
& (* \rlnmsinit{} *) \quad (\forall (c : \sltm{context}) (o : \sltm{oo}) (a : \sltm{atm}), \\
& \qquad\qquad\qquad o \in c \rightarrow \bchsl[c]{o}{a} \rightarrow P_2 \; c \; o \; a \rightarrow P_1 \; c \; \atom{a}) \rightarrow \\
% g_all:
& (* \rlnmsall{} *) \quad (\forall (c : \sltm{context}) (o : \sltm{expr con} \rightarrow \sltm{oo}), \\
& \qquad\qquad\qquad (\forall (e : \sltm{expr con}), \hybridtm{proper} \; e \rightarrow \seqsl[c]{o \; e} \rightarrow \\
& \qquad\qquad\qquad (\forall (e : \sltm{expr con}), \hybridtm{proper} \; e \rightarrow P_1 \; c \; (o \; e) \rightarrow \\
& \qquad\qquad\qquad P_1 \; c \; (\sltm{All} \; o)) \rightarrow \\
% b_imp:
& (* \rlnmbimp{} *) \quad (\forall (c : \sltm{context}) (o_1 \; o_2 : \sltm{oo}) (a : \sltm{atm}), \\
& \qquad\qquad\qquad \seqsl[c]{o_1} \rightarrow P_1 \; c \; o_1 \rightarrow \bchsl[c]{o_2}{a} \rightarrow P_2 \; c \; o_2 \; a \rightarrow \\
& \qquad\qquad\qquad P_2 \; c \; (o_1 \longrightarrow o_2) \; a)   \rightarrow \\
& \quad \cdots\\
& \quad (\forall (c : \; \sltm{context}) (o : \sltm{oo}), \seqsl[c]{o} \rightarrow P_1 \; c \; o) \; \wedge \\
& \quad (\forall (c : \; \sltm{context}) (o : \sltm{oo}) (a : \sltm{atm}), \bchsl[c]{o}{a} \rightarrow P_2 \; c \; o \; a)
\end{align*}
\centering{\caption{Sequent Mutual Induction Principle Snippet \label{fig:seqind}}}
%\vspace{-20pt}
\end{figure}

This induction principle is automatically generated following the description shown below, with examples from the figure given in each point.
\begin{itemize}
 \item Non-sequent premises are assumptions of the induction subcase (e.g. $o \in c$ from the \rlnmsinit{} rule).
 \item For every rule premise that is a goal-reduction sequent (with possible local quantifiers) of the form $\forall (x_1 : T_1) \cdots (x_n : T_n), \seqsl{\beta}$ where $n \geq 0$, the induction subcase has assumptions ($\forall (x_1 : T_1) \cdots (x_n : T_n), \seqsl{\beta}$) and ($\forall (x_1 : T_1) \cdots (x_n : T_n), P_1 \; \dyncon{} \; \beta$) (e.g. $\forall (e : \sltm{expr con}), \sltm{proper} \; e \rightarrow \seqsl[c]{o \; e}$ and $\forall (e : \sltm{expr con}), \sltm{proper} \; e \rightarrow P_1 \; c \; (o \; e)$ from the \rlnmsall{} rule with $n = 2$ and unabbreviated prefix $\forall (e : \sltm{expr con}) (H : \sltm{proper} \; e)$).
 \item For every rule premise that is a backchaining sequent (with possible local quantifiers) of the form $\forall (x_1 : T_1) \cdots (x_n : T_n), \bchsl{\beta}{\alpha}$ where $n \geq 0$, the induction subcase has assumptions ($\forall (x_1 : T_1) \cdots (x_n : T_n), \bchsl{\beta}{\alpha}$) and ($\forall (x_1 : T_1) \cdots (x_n : T_n), P_2 \; \dyncon{} \; \beta \; \alpha$) (e.g. $\bchsl[c]{o_2}{a}$ and $(P_2 \; c \; o_2 \; a)$ from the \rlnmbimp{} rule).
 \item If the rule conclusion is a goal-reduction sequent of the form \seqsl{\beta}, then the subcase goal is $P_1 \; \dyncon{} \; \beta$ (e.g. $(P_1 \; c \; \atom{a})$ from the \rlnmsinit{} rule). 
 \item If the rule conclusion is a backchaining sequent of the form \bchsl{\beta}{\alpha}, then the subcase goal is $P_2 \; \dyncon{} \; \beta \; \alpha$ (e.g. $(P_2 \; c \; (o_1 \longrightarrow o_2) \; a)$ from the \rlnmbimp{} rule).
\end{itemize}
Implicit in these last two points is the possible introduction of more assumptions, in the case when $P_1$ and $P_2$ are dependent products themselves (i.e. 
contain quantification and/or implication).
We will refer to assumptions introduced this way as \emph{induction assumptions} in future proofs, since they are from a predicate that is used to construct induction hypotheses. That is, assumptions of the form $(P_1 \; \dyncon{} \; \beta)$ or $(P_2 \; \dyncon{} \; \beta \; \alpha)$ are induction hypotheses for any proof subcase for a rule with premises \seqsl{\beta} or \bchsl{\beta}{\alpha}. In this SL, exactly two cases of this induction principle have more than one induction hypothesis (\rlnmbimp{} and \rlnmsand{}).

%Given specific $P_1$ and $P_2$, if we are trying to prove $(\forall (c
%: \sltm{context}) (o : \sltm{oo}), \seqsl[c]{o} \rightarrow P_1 \; c
%\; o) \; \wedge \; (\forall (c : \sltm{context}) (o : \sltm{oo}) (a :
%\sltm{atm}), \bchsl[c]{o}{a} \rightarrow P_2 \; c \; o \; a)$,
In describing proofs, we will follow the Coq style and write the proof state in a vertical format with the assumptions above a horizontal line and the goal below it. For example, the \rlnmsinit{} subcase will have the following form:
\begin{align*}
H_1 &: o \in c \\
H_2 &: \bchsl[c]{o}{a} \\
\mathit{IH} &: P_2 \; c \; o \; a \\[\pfshift{}]
\cline{1-2}
&P_1 \; c \; \atom{a}
\end{align*}
As in Coq, we provide hypothesis names (so that we can refer to them as needed). Also, we often omit the type declarations of signature variables, in this case $c : \sltm{context}, o:\sltm{oo},$ and $a : \sltm{atm}$, when they can be easily inferred from context. Unlike in Coq, when we have multiple subcases to prove with the same context of assumptions we will write them all under the horizontal line in the same proof state, separated by commas.


%At a higher-level of abstraction, we note that all rules of the SL
%have one of the two following forms:


%\section{Proofs by Mutual Structural Induction over Sequents}

%We have seen that the SL is a collection of rules for proving goal-reduction and backchaining sequents. These rules may have premises that are either kind of sequent, or not a sequent at all. To use this mutual induction technique to prove something about goal-reduction and backchaining sequents, we have to prove 15 subcases; one for each rule of this SL. There are a few approaches that could be taken to present such a proof:

%1. state the higher-level induction structure and leave it to the reader to work out the details from the code

%2. present the reasoning for all subcases

%3. present the reasoning for a select few subcases to illustrate the reasoning

\section{Generalized SL Part I: Abstract Rules}
\label{sec:gsl}

Here we present generalized specification logic rules to reduce the number of induction cases and allow us to partition cases of the original SL based on rule structure. Our goal is to gain insight into the high-level structure of such inductive proofs, providing the proof writer and reader with the ability to understand where the difficult cases are and how similar cases can be handled in a general way.

All rules of the SL have some number of premises that are either non-sequent predicates, goal-reduction sequents, or backchaining sequents. Also, all rule conclusions are sequents; this is necessary to encode these rules in inductive types \sltm{grseq} and \sltm{bcseq}. With this observation, we can generalize the rules of the SL inference system and say that all rules have one of the following forms:

\begin{prooftree}
\Axiom$\fCenter \ol{Q_m} \args{c , o}$
\noLine
  \UnaryInf$\forall \ol{(x_{n,s_n} : R_{n,s_n})}, \fCenter (\seqsl[c \cup \ol{\gamma_n} \args{o}]{\ol{F_n} \args{o , \ol{x_{n, s_n}}}})$
  \noLine
  \UnaryInf$\forall \ol{(y_{p,t_p} : S_{p,t_p})}, \fCenter (\bchsl[c \cup \ol{\gamma'_p} \args{o}]{\ol{F'_p} \args{o , \ol{y_{p,t_p}}}}{\ol{a_p}})$
    \RightLabel{\rl{gr\_rule}}
    \UnaryInf$\fCenter \seqsl[c]{o}$
\end{prooftree}
\begin{prooftree}
\Axiom$\fCenter \ol{Q_m} \args{c , o}$
\noLine
  \UnaryInf$\forall \ol{(x_{n,s_n} : R_{n,s_n})}, \fCenter (\seqsl[c \cup \ol{\gamma_n} \args{o}]{\ol{F_n} \args{o , \ol{x_{n, s_n}}}})$
  \noLine
  \UnaryInf$\forall \ol{(y_{p,t_p} : S_{p,t_p})}, \fCenter (\bchsl[c \cup \ol{\gamma'_p} \args{o}]{\ol{F'_p} \args{o , \ol{y_{p,t_p}}}}{\ol{a_p}})$
    \RightLabel{\rl{bc\_rule}}
    \UnaryInf$\fCenter \bchsl[c]{o}{a}$
\end{prooftree}
%$$
%\inferH[\rl{gr\_rule}]{\seqsl[c]{o}}{\ol{Q_m} \args{c , o} & \forall \ol{(x_{n,s_n} : R_{n,s_n})}, (\seqsl[c \cup \ol{\gamma_n} \args{o}]{\ol{G_n} \args{o , \ol{x_{n, s_n}}}}) & \forall \ol{(y_{p,t_p} : S_{p,t_p})}, (\bchsl[c \cup \ol{\gamma'_p} \args{o}]{\ol{D_p} \args{o , \ol{y_{p,t_p}}}}{\ol{a_p}})}
%$$
%or
%$$
%\inferH[\rl{bc\_rule}]{\bchsl[c]{o}{a}}{\ol{Q_m} \args{c , o} & \forall \ol{(x_{n,s_n} : R_{n,s_n})}, (\seqsl[c \cup \ol{\gamma_n} \args{o}]{\ol{G_n} \args{o , \ol{x_{n, s_n}}}}) & \forall \ol{(y_{p,t_p} : S_{p,t_p})}, (\bchsl[c \cup \ol{\gamma'_p} \args{o}]{\ol{D_p} \args{o , \ol{y_{p,t_p}}}}{\ol{a_p}})}
%$$
where $m, n, p$ represent the (possibly zero) number of non-sequent
premises, goal-reduction sequent premises, and backchaining sequent
premises, respectively. Note that for all rules in our implemented SL,
$0\le m\le 1$, $0\le n\le 2$, and $0\le p\le 1$.

We call this collection of inference rules consisting of \rl{gr\_rule} and \rl{bc\_rule} the generalized specification logic (GSL). This is \emph{not} implemented in Coq as the previously described SL is; but rather all rules of the SL can be instantiated from the two rules of the GSL (see Subsection \ref{subsec:sltogsl}). The GSL allows us to investigate the SL without needing to consider each of the 15 rules of the SL separately. This makes it possible to more efficiently study and explain the metatheory of the SL.

Much of the notation used in these rules requires further explanation. A horizontal bar above an element with some subscript index, say $z$, means we have a collection of such items indexed from 1 to $z$. For example, the ``premise'' $\ol{Q_m} \args{c,o}$ represents the $m$ premises $Q_1 \args{c,o} , \ldots , Q_m \args{c,o}$. The premises with sequents can possibly have local quantification. For $i=1 , \ldots , n$, $\ol{(x_{i,s_i} : R_{i, s_i})}$ represents the prefix $(x_{i,1} : R_{i,1})\cdots(x_{i,s_i} : R_{i,s_i})$.

The notation $\args{\cdot}$ is used to list arguments from the conclusion that may be used by a function or predicate. We wish to show how elements of the rule conclusion propagate through a proof.

Given types $T_0, T_1 , \ldots , T_z$, when we write $F \args{a_1 : T_1 , \ldots , a_z : T_z} : T_0$, we mean a term of type $T_0$ that may contain any (sub)terms appearing in conclusion terms $a_1, \ldots , a_z$. For example, given $\gamma_1 \args{D \longrightarrow G : \sltm{oo}} : \sltm{context}$, we may ``instantiate'' this expression to $\{ D \}$. We often omit types and use definitional notation, e.g., in this case we may write $\gamma_1 \args{D \longrightarrow G} \coloneqq \{ D \}$.

We infer the following typing judgments from the GSL rules:
\begin{itemize}
 \item For $i = 1 ,\ldots , m$, the definition of $Q_i$ may use the context and formula of the conclusion, so with full typing information, $Q_i \args{c : \sltm{context} , o : \sltm{oo}} : \coqtm{Prop}$
 \item For $j = 1 , \ldots , n$, SL context $\gamma_j$ may use the formula of the conclusion and SL formula $F_j$ may use the formula of the conclusion and locally quantified variables. So with full typing information, $\gamma_j \args{o : \sltm{oo}} : \sltm{context}$ and $F_j \args{o : \sltm{oo} , x_{j,1} : R_{j,1} , \ldots , x_{j,s_j} : R_{j,s_j}} : \sltm{oo}$
 \item For $k = 1 , \ldots , p$, SL context $\gamma'_k$ may use the formula of the conclusion and SL formula $F'_k$ may use the formula of the conclusion and locally quantified variables. So with full typing information $\gamma'_k \args{o : \sltm{oo}} : \sltm{context}$ and $F'_k \args{o : \sltm{oo} , y_{k,1} : S_{k,1} , \ldots , y_{k,t_k} : S_{k,t_k}} : \sltm{oo}$
\end{itemize}

%In the GSL we have made the rules general enough to capture the rules of the SL, but it could be generalized further to explore other specification logics that do not fit the restrictions here.


\subsection{SL Rules from GSL Rules}
\label{subsec:sltogsl}

The rules of the GSL can be instantiated to obtain the SL by
specifying the values of the variables in the GSL rules. We first fill
in $m$, $n$, and $p$.  Then for $i = 1 , \ldots , m$, we specify
$Q_i$.  For $j = 1 , \ldots , n$, we specify $s_j, \gamma_j$, $F_j$,
$x_{j,s_j}$, and $R_{j,s_j}$. For $k = 1 , \ldots , p$, we specify
$\gamma'_k$, $F'_k$, $y_{k,t_k}$, and $S_{k,t_k}$. Below are examples
for SL rules \rlnmsinit{}, \rlnmsalls{} and \rlnmbimp{}.

\noindent
\begin{tabular}{c c c c c c}
\\
\hline
Rule & $m$ & $n$ & $p$ & $c$ & $o$ \\
\hline \hline \noalign{\smallskip}
$\vcenter{\rlsinit{}}$ & 1 & 0 & 1 & \dyncon{} & $\atom{A}$ \\
\noalign{\smallskip} \hline \noalign{\smallskip}
\multicolumn{6}{c}{$t_1 \coloneqq 0$} \\
\multicolumn{6}{c}{$Q_1 \args{\dyncon{} , \atom{A}} \coloneqq D \in \dyncon{} \;\;\;\; \gamma'_1 \args{\atom{A}} \coloneqq \emptyset \;\;\;\; F'_1 \args{\atom{A}} \coloneqq D$} \\
\noalign{\smallskip} \hline \hline \noalign{\smallskip}

$\vcenter{\rlsalls{}}$ & 0 & 1 & 0 & \dyncon{} & $\sltm{Allx} \; G$ \\
\noalign{\smallskip} \hline \noalign{\smallskip}
\multicolumn{6}{c}{$s_1 \coloneqq 1 \;\;\;\; x_{1,1} \coloneqq E \;\;\;\; R_{1,1} \coloneqq \sltm{X}$} \\
\multicolumn{6}{c}{$\gamma_1 \args{\sltm{Allx} \; G} \coloneqq \emptyset \;\;\;\; F_1 \args{\sltm{Allx} \; G , E} \coloneqq G \; E$} \\
\noalign{\smallskip} \hline \hline \noalign{\smallskip}

$\vcenter{\rlbimp{}}$ & 0 & 1 & 1 & \dyncon{} & $G \longrightarrow D$ \\
\noalign{\smallskip} \hline
\multicolumn{6}{c}{$s_1 \coloneqq 0 \;\;\;\; t_1 \coloneqq 0$} \\
\multicolumn{6}{c}{$\gamma_1 \args{G \longrightarrow D} \coloneqq \emptyset \;\;\;\; F_1 \args{G \longrightarrow D} \coloneqq G$} \\
\multicolumn{6}{c}{$\gamma'_1 \args{G \longrightarrow D} \coloneqq \emptyset \;\;\;\; F'_1 \args{G \longrightarrow D} \coloneqq D$} \\
\hline
\\
\end{tabular}

\noindent
%Notice that for the \rlnmsinit{} rule, $D$ is used in the definition
%of $Q_i$, even though it is not in the argument list from the rule
%conclusion. In instantiations from the GSL rules, we allow signature
%variables that have been introduced in the course of a proof to be
%used in these definitions.
Notice that for the \rlnmsinit{} rule, $D$ appears in $Q_1$, even
though it is not in the argument list of $Q_1$.  The notation
$\args{\cdot}$ only specifies arguments from the rule conclusion.  Any
variables that only appear in the premises of a rule of the SL are
also permitted to appear in the propositions, formulas, and contexts
when specializing the premises of a GSL rule to obtain the premises of
a specific SL rule.

%\paragraph{\rlsbc{}} ~\\

%Uses partial function $\mathit{extr} : \sltm{oo} \rightarrow \sltm{atm}$ to extract the atom from atomic formulas and we refer to some externally quantified $G : \sltm{oo}$. We have $m = 1, n = 1, p = 0, s_n = 0$ so we define $Q_i \args{c , o} \coloneqq \prog{(\mathit{extr} \; o)}{G}$, $\gamma_1 \args{o} \coloneqq \emptyset$, and $G_1 \args{o} \coloneqq G$. \\

%Then we can write \rlnmsbc{} as $\vcenter{\infer[\rlnmsbc{}]{\seqsl{\atom{A}}}{Q_1 \args{\dyncon{} , \atom{A}} & \seqsl[\dyncon{} \cup \gamma_1 \args{\atom{A}}]{G_1 \args{\atom{A}}}}}$.

%\paragraph{\rlsand{}} ~\\

%Uses partial functions $\mathit{fst} : \sltm{oo} \rightarrow \sltm{oo}$ and $\mathit{snd} : \sltm{oo} \rightarrow \sltm{oo}$ to extract the first and second conjunts, respectively, of a formula that is a conjunction. We have $m = 0, n = 2, p = 0, s_n = 0$ so we define $\gamma_1 \args{o} \coloneqq \emptyset$, $\gamma_2 \args{o} \coloneqq \emptyset$, $G_1 \args{o} \coloneqq (\mathit{fst} \; o)$, and $G_2 \args{o} \coloneqq (\mathit{snd} \; o)$. \\

%...eeks... rule naming issue where formulas $G_1$ and $G_2$ conflict with these in generalize rule. will need to rename something. Use $\ol{F_n}$ and $\ol{F'_p}$ in generalized rule? \\

%Then we can write \rlnmsand{} as $\vcenter{\infer[\rlnmsand{}]{\seqsl{o_1 \& o_2}}{\seqsl[\dyncon{} \cup \gamma_1 \args{o_1 \& o_2}]{G_1 \args{o_1 \& o_2}} & \seqsl[\dyncon{} \cup \gamma_2 \args{o_1 \& o_2}]{G_2 \args{o_1 \& o_2}}}}$.


%where $i$ (possibly zero) is the number of non-sequent premises, and
%$\overline{Q_i}$ denotes the $i$ hypotheses $Q_1,\ldots,Q_i$.
%Similarly for $j$ and $k$ and the corresponding goal-reduction sequent
%premises and backchaining sequent premises, respectively.  Note that
%for all the rules in our SL, $0\le i,k\le 1$ and $0\le j\le 2$.

%With these rule forms, we can begin to explore how proofs using mutual structural induction over sequents may be constructed without appealing to individual rules of the SL. We let the details of the statement to be proven dictate the constraints on the rule. The motivation for this approach is to encapsulate multiple proof subcases into (at most) two arguments so that such a proof explanation can be both comprehensive and brief. A benefit of this approach is that we build the proof gradually from these rules, using the minimum number of constraints on them to prove the desired metatheorems about this logic. We can then apply these proofs to any specification logic that satisfies the same constraints.

%Suppose the non-sequent rule premise depends on the contexts, formulas, and atoms that occur elsewhere in the rule. For brevity we will simply write $Q_i$ until we need further consideration of its use in hypotheses or goals, at which time we will write $Q_i$ followed by a list of arguments that may be part of its definition. This list will vary according to the rule constraints necessary for each proof.


%The only rule variables that cannot be instantiated as we choose are those occuring in the goal. So we will allow $Q_i$ to depend on the context and formula from the conclusion of the rule, writing hypothesis $H_i$ as $Q_i \; c \; o$. One of the current subgoals of the proof is $Q_i \; \dyncon{} \; \beta$. To prove this subgoal, we rely on some induction assumptions introduced after unfolding $P_1$ to allow us to use $H_i$



\section{Proof by Induction over the Generalized Rules}
\label{sec:pfgsl}

The induction subcase corresponding to \rl{gr\_rule}
(resp. \rl{bc\_rule}) requires a proof of:
\pagebreak[0]
\begin{align*}
\ol{H_m} &: \ol{Q_m} \args{c , o} \\
\ol{\mathit{Hg}_n} &: \forall \ol{(x_{n,s_n} : R_{n,s_n})}, (\seqsl[c \cup \ol{\gamma_n} \args{o}]{\ol{F_n} \args{o , \ol{x_{n,s_n}}})} \\
\ol{\mathit{IHg}_n} &: \forall \ol{(x_{n,s_n} : R_{n,s_n})}, P_1 \; (c \cup \ol{\gamma_n} \args{o}) \; (\ol{F_n} \args{o , \ol{x_{n,s_n}}}) \\
\ol{\mathit{Hb}_p} &: \forall \ol{(y_{p,t_p} : S_{p,t_p})}, (\bchsl[c \cup \ol{\gamma'_p} \args{o}]{\ol{F'_p} \args{o , \ol{y_{p,t_p}}}}{\ol{a_p}}) \\
\ol{\mathit{IHb}_p} &: \forall \ol{(y_{p,t_p} : S_{p,t_p})}, P_2 \; (c \cup \ol{\gamma'_p} \args{o}) \; (\ol{F'_p} \args{o , \ol{y_{p,t_p}}}) \; \ol{a_p} \\[\pfshift{}]
\cline{1-2}
& P_1 \; c \; o \; (\mathit{resp.} \; P_2 \; c \; o \; a)
\end{align*}

Given specific $P_1$ and $P_2$, we could unfold uses of these predicates and continue the proof.
%We will consider a restricted version of this abstraction that is sufficient for the SL and its metatheory presented here. For $j = 1 \ldots n$ and $k = 1 \ldots p$, we restrict $C_j \args{c,o}$ and $C'_k \args{c,o}$ to have the form $c \cup (\gamma_j \args{c,o})$ and $c \cup (\gamma'_k \args{c,o})$, respectively. So we require the conclusion context to be a subset of the premise contexts. This is satisfied by the rules of our implemented SL. In fact, we will have $\gamma_j \args{c,o} = \gamma'_k \args{c,o} = \emptyset$ for all rules other than \rlnmsimp{} where $j = 1$ and $\gamma_1 \args{\dyncon{} , (D \longrightarrow G)} = \{ D \}$. (*idea: build this restriction into the GSL rules, comment that could be generalized further, remove this paragraph? OR might just remove this restriction since it doesn't add much here)
Suppose
\begin{align*}
P_1 &:= \lambda c \; o . \forall (\inddyncon{} : \sltm{context}), \\
& \mathit{IA}_1 \args{c,o,\inddyncon{}} \rightarrow \dots \rightarrow \mathit{IA}_w \args{c,o,\inddyncon{}} \rightarrow \underline{\seqsl[\inddyncon{}]{o}} \qquad \mathrm{and}\\
P_2 &:= \lambda c \; o \; a . \forall (\inddyncon{} : \sltm{context}), \\
& \mathit{IA}_1 \args{c,o,\inddyncon{}} \rightarrow \dots \rightarrow \mathit{IA}_w \args{c,o,\inddyncon{}} \rightarrow \underline{\bchsl[\inddyncon{}]{o}{a}}
\end{align*}
The underlining of sequents in the definitions of $P_1$ and $P_2$ is
to highlight that these are the sequents we apply the generalized
rules to (following introductions). In particular, we unfold uses of
$P_1$ and $P_2$ in the proof state and introduce the variables and
induction assumptions.  Then the goal is either
\seqsl[\inddyncon{}]{o} or \bchsl[\inddyncon{}]{o}{a}. Apply
\rl{gr\_rule} or \rl{bc\_rule} as appropriate, and either will give
($m + n + p$) new subgoals which come from the three premise forms in
these rules, with appropriate instantiations for the externally
quantified variables. Now the proof state is
\begin{align*}
\ol{H_m} &: \ol{Q_m} \args{c , o} \\
\ol{\mathit{Hg}_n} &: \forall \ol{(x_{n,s_n} : R_{n,s_n})}, (\seqsl[c \cup \ol{\gamma_n} \args{o}]{\ol{F_n} \args{o , \ol{x_{n,s_n}}})} \\
\ol{\mathit{IHg}_n} &: \forall \ol{(x_{n,s_n} : R_{n,s_n})} (\inddyncon{} : \sltm{context}), \\
& \mathit{IA}_1 \args{c \cup \ol{\gamma_n} \args{o} , \ol{F_n} \args{o , \ol{x_{n,s_n}}} , \inddyncon{}} \rightarrow \dots \rightarrow \\
& \mathit{IA}_w \args{c \cup \ol{\gamma_n} \args{o} , \ol{F_n} \args{o , \ol{x_{n,s_n}}} , \inddyncon{}} \rightarrow \seqsl[\inddyncon{}]{\ol{F_n} \args{o , \ol{x_{n,s_n}}}} \\
\ol{\mathit{Hb}_p} &: \forall \ol{(y_{p,t_p} : S_{p,t_p})}, (\bchsl[c \cup \ol{\gamma'_p} \args{o}]{\ol{F'_p} \args{o , \ol{y_{p,t_p}}}}{\ol{a_p}}) \\
\ol{\mathit{IHb}_p} &: \forall \ol{(y_{p,t_p} : S_{p,t_p})} (\inddyncon{} : \sltm{context}), \\
& \mathit{IA}_1 \args{c \cup \ol{\gamma'_p} \args{o} , \ol{F'_p} \args{o , \ol{y_{p,t_p}}} , \inddyncon{}} \rightarrow \dots \rightarrow \\
& \mathit{IA}_w \args{c \cup \ol{\gamma'_p} \args{o} , \ol{F'_p} \args{o , \ol{y_{p,t_p}}} , \inddyncon{}} \rightarrow \bchsl[\inddyncon{}]{\ol{F'_p} \args{o , \ol{y_{p,t_p}}}}{\ol{a_p}} \\
\ol{\mathit{IP}_w} &: \ol{\mathit{IA}_w} \args{c , o , \inddyncon{}} \\[\pfshift{}]
\cline{1-2}
& \ol{Q_m} \args{\inddyncon{} , o}, \\
& \forall \ol{(x_{n,s_n} : R_{n,s_n})}, (\seqsl[\inddyncon{} \cup \ol{\gamma_n} \args{o}]{\ol{F_n} \args{o , \ol{x_{n,s_n}}}}), \\
& \forall \ol{(y_{p,t_p} : S_{p,t_p})}, (\bchsl[\inddyncon{} \cup \ol{\gamma'_p} \args{o}]{\ol{F'_p} \args{o , \ol{y_{p,t_p}}}}{\ol{a_p}})
\end{align*}
where $\inddyncon{}$ is a new signature variable.

\subsection{Subproofs for Sequent Premises}

\begin{figure}
\begin{align*}
\ol{H_m} &: \ol{Q_m} \args{c , o} \\
\ol{\mathit{Hg}_n} &: \forall \ol{(x_{n,s_n} : R_{n,s_n})}, (\seqsl[c \cup \ol{\gamma_n} \args{o}]{\ol{F_n} \args{o , \ol{x_{n,s_n}}})} \\
\ol{\mathit{IHg}_n} &: \forall \ol{(x_{n,s_n} : R_{n,s_n})} (\inddyncon{} : \sltm{context}), \\
& \mathit{IA}_1 \args{c \cup \ol{\gamma_n} \args{o} , \ol{F_n} \args{o , \ol{x_{n,s_n}}} , \inddyncon{}} \rightarrow \dots \rightarrow \\
& \mathit{IA}_w \args{c \cup \ol{\gamma_n} \args{o} , \ol{F_n} \args{o , \ol{x_{n,s_n}}} , \inddyncon{}} \rightarrow \seqsl[\inddyncon{}]{\ol{F_n} \args{o , \ol{x_{n,s_n}}}} \\
\ol{\mathit{Hb}_p} &: \forall \ol{(y_{p,t_p} : S_{p,t_p})}, (\bchsl[c \cup \ol{\gamma'_p} \args{o}]{\ol{F'_p} \args{o , \ol{y_{p,t_p}}}}{\ol{a_p}}) \\
\ol{\mathit{IHb}_p} &: \forall \ol{(y_{p,t_p} : S_{p,t_p})} (\inddyncon{} : \sltm{context}), \\
& \mathit{IA}_1 \args{c \cup \ol{\gamma'_p} \args{o} , \ol{F'_p} \args{o , \ol{y_{p,t_p}}} , \inddyncon{}} \rightarrow \dots \rightarrow \\
& \mathit{IA}_w \args{c \cup \ol{\gamma'_p} \args{o} , \ol{F'_p} \args{o , \ol{y_{p,t_p}}} , \inddyncon{}} \rightarrow \bchsl[\inddyncon{}]{\ol{F'_p} \args{o , \ol{y_{p,t_p}}}}{\ol{a_p}} \\
\ol{\mathit{IP}_w} &: \ol{\mathit{IA}_w} \args{c , o , \inddyncon{}} \\[\pfshift{}]
\cline{1-2}
& \ol{\mathit{IA}_w} \args{c \cup \ol{\gamma_n} \args{o} , \ol{F_n} \args{o , \ol{x_{n,s_n}}} , \inddyncon{} \cup \ol{\gamma_n} \args{o}} \\
& (\mathit{resp.} \; \ol{\mathit{IA}_w} \args{c \cup \ol{\gamma'_p} \args{o} , \ol{F'_p} \args{o , \ol{y_{p,t_p}}} , \inddyncon{} \cup \ol{\gamma'_p} \args{o}})
\end{align*}
\centering{\caption{Incomplete proof branches for sequent premises \label{fig:premgrseq}}}
\end{figure}

To prove the last ($n + p$) subgoals (the ``second'' and ``third''
subgoals above) we first introduce any locally quantified variables as
signature variables. For the goal-reduction (resp. backchaining)
subgoals, for $j = 1 , \ldots , n$ (resp. $k = 1 , \ldots , p$), we
apply induction hypothesis $\mathit{IHg}_j$ (resp. $\mathit{IHb}_k$),
instantiating $\inddyncon{}$ in the induction hypothesis with
$\inddyncon{} \cup \gamma_j \args{o}$ (resp. $\inddyncon{} \cup
\gamma'_k \args{o}$). This yields the proof state in Figure
\ref{fig:premgrseq} for goal-reduction premises (resp. backchaining
premises).


\subsection{Subproofs for Non-Sequent Premises}
\label{subsec:subpfnonseq}

The proof of the first $m$ subgoals depends on the definition of $Q_i$ for $i = 1 \ldots m$. If the first argument (a \sltm{context}) is not used in its definition, then $Q_i \args{\inddyncon{} , o}$ is provable by assumption $H_i$, since we will have $Q_i \args{\inddyncon{} , o} = Q_i \args{c , o}$. Any other dependencies on signature variables can be ignored since we can assign the variables as we choose when applying the generalized rule. We will illustrate this by considering each rule with non-sequent premises, starting from the second proof state in Section \ref{sec:pfgsl} and, for $(i = 1 , \ldots , m)$, $(j = 1 , \ldots , n)$, $(k = 1 , \ldots , p)$, show how to define $Q_i$, $\gamma_j$, $F_j$, $\gamma'_k$, and $F'_k$ and finish the subproofs where possible.

\paragraph{Case \rlnmsbc{} :} This rule has one non-sequent premise and one goal-reduction sequent premise with no local quantification, so $m = n = 1$, $p = 0$, $o = \atom{A}$, and $c = \dyncon{}$. Define $Q_1 \args{\dyncon{} , \atom{A}} \coloneqq \prog{A}{G}$, $\gamma_1 \args{\atom{A}} \coloneqq \emptyset$, and $F_1 \args{\atom{A}} \coloneqq G$, where $G : \sltm{oo}$ is a signature variable. Then we are proving
the following:
\begin{align*}
H_1 &: \prog{A}{G} \\
\mathit{Hg}_1 &: \seqsl{G} \\
\mathit{IHg}_1 &: \forall (\inddyncon{} : \sltm{context}), \mathit{IA}_1 \args{\dyncon{} , G , \inddyncon{}} \rightarrow \dots \rightarrow \\
& \qquad \mathit{IA}_w \args{\dyncon{} , G , \inddyncon{}} \rightarrow \seqsl[\inddyncon{}]{G} \\
\ol{\mathit{IP}_w} &: \ol{\mathit{IA}_w} \args{\dyncon{} , \atom{A} , \inddyncon{}} \\[\pfshift{}]
\cline{1-2}
& \prog{A}{G}
\end{align*}
which is completed by assumption $H_1$.

\paragraph{Case \rlnmsinit{} :} This rule has one non-sequent premise and one backchaining sequent premise with no local quantification, so $m = p = 1$, $n = 0$, $c = \dyncon{}$, and $o = \atom{A}$. Define $Q_1 \args{\dyncon{} , \atom{A}} \coloneqq D \in \dyncon{}$, $\gamma'_1 \args{\atom{A}} \coloneqq \emptyset$, and $F'_1 \args{\atom{A}} \coloneqq D$, where $D : \sltm{oo}$ is a signature variable. Then we need to prove what is displayed in Figure \ref{fig:incpfdyn}.
\begin{figure}
\begin{align*}
H_1 &: D \in \dyncon{} \\
\mathit{Hb}_1 &: \bchsl{D}{a_1} \\
\mathit{IHb}_1 &: \forall (\inddyncon{} : \sltm{context}), \mathit{IA}_1 \args{\dyncon{} , D , \inddyncon{}} \rightarrow \dots \rightarrow \\
& \qquad \mathit{IA}_w \args{\dyncon{} , D , \inddyncon{}} \rightarrow \bchsl[\inddyncon{}]{D}{a_1} \\
\ol{\mathit{IP}_w} &: \ol{\mathit{IA}_w} \args{\dyncon{} , \atom{A} , \inddyncon{}} \\[\pfshift{}]
\cline{1-2}
& D \in \inddyncon{}
\end{align*}
\centering{\caption{Incomplete proof branch (\rlnmsinit{} case) \label{fig:incpfdyn}}}
\end{figure}
Here we do not have enough information to finish this branch of the proof. An induction assumption may be of use, but we will need specific $P_1$ and $P_2$.

\paragraph{Case \rlnmssome{} :} This rule has one non-sequent premise and one goal-reduction sequent premise with no local quantification, so $m = n = 1$, $p = 0$, $c = \dyncon{}$, and $o = \sltm{Some} \; G$. Define $Q_1 \args{\dyncon{} , \sltm{Some} \; G} \coloneqq \sltm{proper} \; E$, $\gamma_1 \args{\sltm{Some} \; G} \coloneqq \emptyset$, and $F_1 \args{\sltm{Some} \; G} \coloneqq G \; E$ where $E : \sltm{expr con}$ is a signature variable. Then we are proving
the following:
\begin{align*}
H_1 &: \sltm{proper} \; E \\
\mathit{Hg}_1 &: \seqsl{G \; E} \\
\mathit{IHg}_1 &: \forall (\inddyncon{} : \sltm{context}), \mathit{IA}_1 \args{\dyncon{} , G \; E , \inddyncon{}} \rightarrow \dots \rightarrow \\
& \qquad \mathit{IA}_w \args{\dyncon{} , G \; E , \inddyncon{}} \rightarrow \seqsl[\inddyncon{}]{G \; E} \\
\ol{\mathit{IP}_w} &: \ol{\mathit{IA}_w} \args{\dyncon{} , \sltm{Some} \; G , \inddyncon{}} \\[\pfshift{}]
\cline{1-2}
& \sltm{proper} \; E
\end{align*}
which is completed by assumption $H_1$.

\paragraph{Case \rlnmball{} :} This case is proven as above but with $m = p = 1$, $n = 0$, $c = \dyncon{}$, and $o = \sltm{All} \; D$. Define $Q_1 \args{\dyncon{} , \sltm{All} \; D} \coloneqq \sltm{proper} \; E$, $\gamma'_1 \args{\sltm{All} \; D} \coloneqq \emptyset$, and $F'_1 \args{\sltm{All} \; D} \coloneqq D \; E$ where $E : \sltm{expr con}$ is a signature variable. The goal $\sltm{proper} \; E$ is provable by the assumption of the same form
as in the previous case.
%from the definition of the rule.

In the next two sections we will return to this idea of proofs about a specification logic from a generalized form of SL rule to prove properties of the SL once we have fully defined $P_1$ and $P_2$. The proof states in Figures \ref{fig:premgrseq} and \ref{fig:incpfdyn} (the incomplete branches) will be roots of these explanations.


\section{GSL Induction Part II: The Structural Rules Hold}
\label{sec:structrules}

%\begin{align*}
%P_1 :=& \lambda \; (\dyncon{}_1 : \sltm{context}) \; . \; \lambda \; (\beta : \sltm{oo}) \; . \\
%& \forall \; (\dyncon{}_2 : \sltm{context}), \dyncon{}_1 \subseteq \dyncon{}_2 \rightarrow \seqsl[\dyncon{}_2]{\beta} \\
%P_2 :=& \lambda \; (\dyncon{}_1 : \sltm{context}) \; . \; \lambda \; (\beta : \sltm{oo}) \; . \; \lambda \; (\alpha : \sltm{atm}) \; . \\
%& \forall \; (\dyncon{}_2 : \sltm{context}), \dyncon{}_1 \subseteq \dyncon{}_2 \rightarrow \bchsl[\dyncon{}_2]{\beta}{\alpha}
%\end{align*}
%we are proving
%\begin{align*}
%&(\forall \; (\dyncon{}_1 : \sltm{context}) \; (\beta : \sltm{oo}), \\
%&\;\;\;\;\; (\seqsl[\dyncon{}_1]{\beta}) \rightarrow (P_1 \; \dyncon{}_1 \; \beta)) \\
%\wedge \; & (\forall \; (\dyncon{}_1 : \sltm{context}) \; (\beta : \sltm{oo}) \; (\alpha : \sltm{atm}), \\
%&\;\;\;\;\; (\bchsl[\dyncon{_1}]{\beta}{\alpha}) \rightarrow (P_2 \; \dyncon{} \; \beta \; \alpha))
%\end{align*}

% section started here...

Recall from Section \ref{sec:structsl} we prove the standard rules of weakening\index{weakening}, contraction\index{contraction} and exchange\index{exchange} for both the goal-reduction and backchaining sequents as corollaries of~\nameref{thm:monotone} (Theorem~\ref{thm:monotone}) which states
\begin{align*}
(\forall \; (c : \sltm{context}) & \; (o : \sltm{oo}), \\
& (\seqsl[c]{o}) \rightarrow (P_1 \; c \; o)) \;\; \wedge \\
(\forall \; (c : \sltm{context}) & \; (o : \sltm{oo}) \; (a : \sltm{atm}), \\
& (\bchsl[c]{o}{a}) \rightarrow (P_2 \; c \; o \; a))
\end{align*}
where $P_1$ and $P_2$ are defined as
\begin{align*}
P_1 :=& \lambda \; (c : \sltm{context}) (o : \sltm{oo}) \; . \\
& \qquad \forall \; (\inddyncon{} : \sltm{context}), c \subseteq \inddyncon{} \rightarrow \seqsl[\inddyncon{}]{o} \\
P_2 :=& \lambda \; (c : \sltm{context}) (o : \sltm{oo}) (a : \sltm{atm}) \; . \\
& \qquad \forall \; (\inddyncon{} : \sltm{context}), c \subseteq \inddyncon{} \rightarrow \bchsl[\inddyncon{}]{o}{a}
\end{align*}

We build on the inductive proof in Section \ref{sec:gsl} over the GSL to prove~\nameref{thm:monotone} for this new logic. Recall that when we took the proof as far as we could we had three remaining groups of branches to finish ($m + n + p$ subgoals), one group for rules with non-sequent premises depending on the context of the rule conclusion, and one for each kind of sequent premise (see Figures \ref{fig:premgrseq} and \ref{fig:incpfdyn}). We will continue this effort below, using the $P_1$ and $P_2$ defined for this theorem. This means we will have one induction assumption (i.e., $w = 1$) which is $\mathit{IA}_1 \args{c , \inddyncon{}} \coloneqq c \subseteq \inddyncon{}$.

\subsection{Sequent Subgoals}

First we will prove the subgoals coming from the sequent premises, building on Figure \ref{fig:premgrseq} and using $\mathit{IA}_1$ as defined above. The proof state for goal-reduction (resp. backchaining) premises is

\begin{align*}
\ol{H_m} &: \ol{Q_m} \args{c , o} \\
\ol{\mathit{Hg}_n} &: \forall \ol{(x_{n,s_n} : R_{n,s_n})}, (\seqsl[c \cup \ol{\gamma_n} \args{o}]{\ol{F_n} \args{o , \ol{x_{n,s_n}}})} \\
\ol{\mathit{IHg}_n} &: \forall \ol{(x_{n,s_n} : R_{n,s_n})} (\inddyncon{} : \sltm{context}), (c \cup \ol{\gamma_n} \args{o}) \subseteq \inddyncon{} \rightarrow \seqsl[\inddyncon{}]{\ol{F_n} \args{o , \ol{x_{n,s_n}}}} \\
\ol{\mathit{Hb}_p} &: \forall \ol{(y_{p,t_p} : S_{p,t_p})}, (\bchsl[c \cup \ol{\gamma'_p} \args{o}]{\ol{F'_p} \args{o , \ol{y_{p,t_p}}}}{\ol{a_p}}) \\
\ol{\mathit{IHb}_p} &: \forall \ol{(y_{p,t_p} : S_{p,t_p})} (\inddyncon{} : \sltm{context}), (c \cup \ol{\gamma'_p} \args{o}) \subseteq \inddyncon{} \rightarrow \bchsl[\inddyncon{}]{\ol{F'_p} \args{o , \ol{y_{p,t_p}}}}{\ol{a_p}} \\
\inddyncon{} &: \sltm{context} \\
\mathit{IP}_1 &: c \subseteq \inddyncon{} \\
\ol{x_{n,s_n}} &: \ol{R_{n,s_n}} \; (\mathit{resp.} \; \ol{y_{p,t_p}} : \ol{S_{p,t_p}}) \\[\pfshift{}]
\cline{1-2}
& (c \cup \ol{\gamma_n} \args{o}) \subseteq (\inddyncon{} \cup \ol{\gamma_n} \args{o}) \; (\mathit{resp.} \; (c \cup \ol{\gamma'_p} \args{o}) \subseteq (\inddyncon{} \cup \ol{\gamma'_p} \args{o}))
\end{align*}
The goal is provable by~\nameref{lem:context_sub_sup} (Lemma~\ref{lem:context_sub_sup}) and assumption $\mathit{IP}_1$.


\subsection{Non-Sequent Subgoals}

Still to be proven are the subgoals for non-sequent premises. As seen in Section \ref{subsec:subpfnonseq}, the only rule of the SL whose corresponding subcase still needs to be proven is \rlnmsinit{}. From Figure \ref{fig:incpfdyn} and using $P_1$ and $P_2$ as defined here, we are proving
\begin{align*}
H_1 &: D \in \dyncon{} \\
\mathit{Hb}_1 &: \bchsl{D}{a_1} \\
\mathit{IHb}_1 &: \forall (\inddyncon{} : \sltm{context}), \dyncon{} \subseteq \inddyncon{} \rightarrow \bchsl[\inddyncon{}]{D}{a_1} \\
\inddyncon{} &: \sltm{context} \\
\mathit{IP}_1 &: \dyncon{} \subseteq \inddyncon{} \\[\pfshift{}]
\cline{1-2}
& D \in \inddyncon{}
\end{align*}
Unfolding the definition of context subset in $\mathit{IP}_1$ it becomes $\forall (o : \sltm{oo}), o \in \dyncon{} \rightarrow o \in \inddyncon{}$. 
%Applying this to the goal
Backchaining on this form of the goal gives subgoal $D \in \dyncon{}$, provable by assumption $H_1$.


In Section \ref{sec:gsl}, we explored how to prove statements about the GSL for a restricted form of theorem statement. There were three classes of incomplete proof branches that had a final form shown in Figures \ref{fig:premgrseq} and \ref{fig:incpfdyn}. In Section \ref{sec:sltogsl} we saw how to derive the SL from the GSL. So here we have proven a structural theorem for the rules of the GSL in a general way that can be followed for any SL rule.
\end{proof}

\section{GSL Induction Part III: Cut Rule Proven Admissible}
\label{sec:cutadmiss}
\index{cut admissibility}

% section started here...

Recall from Section \ref{sec:cutadmisssl} we are proving $\forall (\delta : \sltm{oo}), P \; \delta$ with $P$ defined as
\begin{align*}
& P : \sltm{oo} \rightarrow \hybridtm{Prop} := \lambda (\delta : \sltm{oo}) \; . \\
& \qquad\qquad (\forall (c : \sltm{context}) (o : \sltm{oo}), \\
& \qquad\qquad\qquad\qquad\qquad\quad \seqsl[c]{o} \rightarrow P_1 \; c \; o) \; \wedge \\
& \qquad\qquad (\forall (c : \sltm{context}) (o : \sltm{oo}) (a : \hybridtm{atm}), \\
& \qquad\qquad\qquad\qquad\qquad\quad \bchsl[c]{o}{a} \rightarrow P_2 \; c \; o \; a),
\end{align*}
where
\begin{align*}
P_1 &: \sltm{context} \rightarrow \sltm{oo} \rightarrow \coqtm{Prop} := \\
& \qquad \lambda (c : \sltm{context}) (o : \sltm{oo}) \; . \\
& \qquad\qquad \forall (\inddyncon{} : \sltm{context}), c = (\inddyncon{}, \delta) \rightarrow \seqsl[\inddyncon{}]{\delta} \rightarrow \underline{\seqsl[\inddyncon{}]{o}} \\
P_2 &: \sltm{context} \rightarrow \sltm{oo} \rightarrow \sltm{atm} \rightarrow \coqtm{Prop} := \\
& \qquad \lambda (c : \sltm{context}) (o : \sltm{oo}) (a : \sltm{atm}) \; . \\
& \qquad\qquad \forall (\inddyncon{} : \sltm{context}), c = (\inddyncon{}, \delta) \rightarrow \seqsl[\inddyncon{}]{\delta} \rightarrow \underline{\bchsl[\inddyncon{}]{o}{a}}
\end{align*}

As in the GSL proof of~\nameref{thm:monotone} (Theorem~\ref{thm:monotone}), we build on the inductive proof in Chapter \ref{ch:gslind}, unfolding $P_1$ and $P_2$ as defined here.
% Redundant:
%We will also make use of \sltm{weakening}, a corollary of the structural rule \sltm{monotone}.
Recall that we have now introduced assumptions and applied the appropriate generalized SL rule to the underlined sequents in the definition of $P_1$ and $P_2$. For the proof of cut admissibility, there are two induction assumptions from $P_1$ and $P_2$ (so $w = 2$). Define $\mathit{IA}_1 \args{c , \inddyncon{}} \coloneqq (c = (\inddyncon{} , \delta))$ and $\mathit{IA}_2 \args{c , \inddyncon{}} \coloneqq \seqsl[\inddyncon{}]{\delta}$, where $\delta$ is the cut formula in the cut rule.

\subsection{Sequent Subgoals}
\label{subsec:cutpfseqprem}

First we will prove the subgoals coming from the sequent premises, building on Figure \ref{fig:premgrseq} and using $\mathit{IA}_1$ and $\mathit{IA}_2$ as defined above. For a moment we will ignore the outer induction over the cut formula $\delta$. By ignore we mean let $\delta \coloneqq \eta$ where $\eta : \sltm{oo}$, and we will not display the induction hypothesis for this induction. The proof state for goal-reduction premises (resp. backchaining premises) is

\newpage

\begin{align*}
\ol{H_m} &: \ol{Q_m} \args{c , o} \\
\ol{\mathit{Hg}_n} &: \forall \ol{(x_{n,s_n} : R_{n,s_n})}, (\seqsl[c \cup \ol{\gamma_n} \args{o}]{\ol{F_n} \args{o , \ol{x_{n,s_n}}})} \\
\ol{\mathit{IHg}_n} &: \forall \ol{(x_{n,s_n} : R_{n,s_n})} (\inddyncon{} : \sltm{context}), \\
& \; (c \cup \ol{\gamma_n} \args{o}) = (\inddyncon{} , \eta) \rightarrow \seqsl[\inddyncon{}]{\eta} \rightarrow \seqsl[\inddyncon{}]{\ol{F_n} \args{o , \ol{x_{n,s_n}}}} \\
\ol{\mathit{Hb}_p} &: \forall \ol{(y_{p,t_p} : S_{p,t_p})}, (\bchsl[c \cup \ol{\gamma'_p} \args{o}]{\ol{F'_p} \args{o , \ol{y_{p,t_p}}}}{\ol{a_p}}) \\
\ol{\mathit{IHb}_p} &: \forall \ol{(y_{p,t_p} : S_{p,t_p})} (\inddyncon{} : \sltm{context}), \\
& \; (c \cup \ol{\gamma'_p} \args{o}) = (\inddyncon{} , \eta) \rightarrow \seqsl[\inddyncon{}]{\eta} \rightarrow \bchsl[\inddyncon{}]{\ol{F'_p} \args{o , \ol{y_{p,t_p}}}}{\ol{a_p}} \\
\inddyncon{} &: \sltm{context} \\
\mathit{IP}_1 &: c = (\inddyncon{} , \eta) \\
\mathit{IP}_2 &: \seqsl[\inddyncon{}]{\eta} \\
\ol{x_{n,s_n}} &: \ol{R_{n,s_n}} \; (\mathit{resp.} \; \ol{y_{p,t_p}} : \ol{S_{p,t_p}}) \\[\pfshift{}]
\cline{1-2}
& (c \cup \ol{\gamma_n} \args{o} = ((\inddyncon{} \cup \ol{\gamma_n} \args{o}) , \eta)) , (\seqsl[\inddyncon{} \cup \ol{\gamma_n} \args{o}]{\eta}) \\
& (\mathit{resp.} \; (c \cup \ol{\gamma'_p} \args{o} = ((\inddyncon{} \cup \ol{\gamma'_p} \args{o}) , \eta)) , (\seqsl[\inddyncon{} \cup \ol{\gamma'_p} \args{o}]{\eta}))
\end{align*}

To prove the sequent subgoal \seqsl[\inddyncon{} \cup \ol{\gamma_n} \args{o}]{\eta} (resp. \seqsl[\inddyncon{} \cup \ol{\gamma'_p} \args{o}]{\eta}), first apply weakening and the new subgoal is \seqsl[\inddyncon{}]{\eta} (resp. \seqsl[\inddyncon{}]{\eta}), provable by assumption $\mathit{IP}_2$.

The subgoals concerning context equality are proven by context lemmas and assumption $\mathit{IP}_1$. That is, we rewrite $((\inddyncon{} \cup \ol{\gamma_n} \args{o}) , \eta)$ to $(\inddyncon{} , \eta) \cup \ol{\gamma_n} \args{o}$ (resp. $(\inddyncon{} \cup \ol{\gamma'_p} \args{o}) , \eta$ to $(\inddyncon{} , \eta) \cup \ol{\gamma'_p} \args{o}$). The new subgoal is $c \cup \ol{\gamma_n} \args{o} = (\inddyncon{} , \eta) \cup \ol{\gamma_n} \args{o}$ (resp. $(c \cup \ol{\gamma'_p} \args{o} = (\inddyncon{} , \eta) \cup \ol{\gamma'_p} \args{o}$). Apply~\nameref{lem:context_sub_sup} (Lemma~\ref{lem:context_sub_sup}) to get assumption $\mathit{IP}_1$.

%(*maybe move?) A few comments are in order to review what is proven so far. We have considered a generalized form of rules of the SL and attempted to prove cut admissibility. We made constraints on the rule that allow us to work through the proof as described in the proof outline and the diagrams presented there. With these constraints we were able to prove the subcases of the sequent mutual induction for all rules other than \rlnmsinit{}. Since induction was first over the cut formula $\delta$ and there are seven formula construction rules and 14 sequent rule subcases just proven, the above argument can be applied to 98 of the 105 subcases (see Figure~\ref{fig:cutpf} for Coq code to automate proofs of these 98 subcases).


\subsection{Non-Sequent Subgoals}%

In Section \ref{subsec:subpfnonseq} we saw that the only rule of the SL whose corresponding subcase still needs to be proven is \rlnmsinit{}. For the non-sequent subgoals we were able to complete the proof while the cut formula $\delta$ was represented as a parameter (and thus could have any formula structure). In the remaining non-sequent proof branch we need to make use of the nested structure of this induction. The proof of this subcase is shown in detail in Section~\ref{subsec:cutadmissnonseq}.

\bigskip

In summary, the outer induction over $\delta$ gave seven cases for seven \sltm{oo} constructors. For each of these, an inner induction over sequents gave 15 new subgoals for 15 rules. We saw that for 14 of 15 rules, each rule has the same proof structure for every form of $\delta$. The remaining subgoals were all for the rule \rlnmsinit{} and were more challenging due to the presence of a non-sequent premise that depends on the context of the conclusion.

\end{proof}

\bigskip

Using the generalized proof presented in this chapter and instantiating the GSL to the SL as in Section~\ref{sec:sltogsl}, we have found condensed proofs of~\nameref{thm:monotone} (Theorem~\ref{thm:monotone}) and~\nameref{thm:cut_admissible} (Theorem~\ref{thm:cut_admissible}).

In this thesis we have seen how the Coq implementation of Hybrid has been extended by the addition of a new specification logic (SL) based on hereditary Harrop formulas. This extension increases the class of object logics that Hybrid can reason about efficiently. The metatheory of this SL is formalized in Coq with proofs by mutual structural induction over the structure of sequent types. We saw the proofs of some specific subcases and the later insight that many of the cases are proven in a similar way. This led to the development of a generalized SL and form of metatheory statement that we could use to better understand the proofs of the SL metatheory.

\section{Related Work}

Throughout this thesis we have seen some mention of related work. Hybrid is a system implementing HOAS and as seen in Section~\ref{sec:hybridcompare} there are other systems with the same goal that also use this technique. As previously discussed, Hybrid is the only known system implementing HOAS in an existing trusted general-purpose theorem prover. See~\cite{FMP:CoRR15} and~\cite{FMP:JAR15} for a more in-depth comparison of these systems on benchmarks defined there.

Although this work is contributing to the area of mechanizing programming language metatheory, the majority of the research presented here is applicable to the more general field of proof theory. We have seen proofs of the admissibility of structural rules of a specific sequent calculus, as well as a generalized sequent calculus which we tried to make only as general as necessary to encapsulate the specification logic presented earlier. Typically these kinds of proofs are by an induction on the height of derivations, but here we have proofs by mutual structural induction over dependent sequent types; the structural proofs in this thesis follow the style of Pfenning in~\cite{Pfenning:IC00}. The sequents in our logic do not have a natural number to represent the height of the derivation. So our presentation of this sequent calculus is perhaps more ``pure'' in some sense, but we may have lost a way to reason about some object logics. It is not yet clear if building proof height into the definition sequents is necessary for studying some object logics. Overall, a better understanding of the relationship between proofs of the metatheory of sequent calculi by induction on the height of derivations versus over the structure of sequents is desirable.

\section{Future Work}

The highest priority future task is to show the utility of the new specification logic in Hybrid. This will be done by presenting an object logic that makes use of the higher-order nature (in the sense of unrestricted implicational complexity) of the new specification logic. Object logics that we plan to represent include:
\begin{itemize}
 \item correspondence between HOAS and de Bruijn encodings of untyped $\lambda$-terms; this is our example OL of Chapter~\ref{ch:hybrid} but we have not yet proven Theorems~\ref{thm:hodb_det1} and~\ref{thm:hodb_det3} of Section~\ref{sec:hybridol} (see~\cite{WCGN:PPDP13})
 \item structural characterization of reductions on untyped $\lambda$-terms (see~\cite{WCGN:PPDP13})
 \item algorithmic specification of bounded subtype polymorphism in System F (see~\cite{Pientka:TPHOLs07}); this comes from the \poplmark{} challenge~\cite{Aydemir05TPHOLs}
\end{itemize}
We would also like to add automation to proofs containing object logic judgments so that the user of Hybrid will not need to be an expert user of proof assistants to be able to use the system.

The encoding of the new Hybrid SL follows the development of the specification logic of Abella as presented in~\cite{WCGN:PPDP13}, but it seems that the proofs of the admissibility of the structural rules differ between these systems. These proofs in Abella are not fully explained in~\cite{WCGN:PPDP13} so some work will need to be done to compare the different proofs. Also, the proof of cut admissibility for this specification logic in Abella requires a third conjunct that we did not need for our proof:
$$
\forall (c : \sltm{context}) (o : \sltm{oo}) (a: \hybridtm{atm}),\\
\seqsl[c]{o} \rightarrow \bchsl[c]{o}{a} \rightarrow \seqsl[c]{\atom{a}}
$$
Our understanding so far is that these proofs in Abella are over the height of derivations, which is an implicit parameter; it is not by structural induction over sequents in the fashion of the proofs founding this thesis.

%\appendix
%\section{Appendix Title}
%
%This is the text of the appendix, if you need one.

\acks

The authors acknowledge the support of the Natural Sciences and
Engineering Research Council of Canada.  In addition, special thanks
go to Alberto Momigliano for his involvement in the initial coding of
the data structures of the specification logic and for his insights
and discussions over the course of this work.


% We recommend abbrvnat bibliography style.

\bibliographystyle{abbrvnat}
\bibliography{lfmtp16}   

% The bibliography should be embedded for final submission.

\end{document}
