\subsubsection{Subproof for Non-Sequent Premises : Alternate Proof Attempt}

In the proof of subgoals from sequent premises, it seems that the outer induction over the cut formula $\delta$ was unnecessary. Here we will consider an alternate proof structure where induction is first over the sequent premises with the cut formula $\delta$ in the context, and wait until it is necessary to have an induction over the cut formula.

Still to be proven are the subgoals for non-sequent premises (see Figure \ref{fig:incpfdyn}). As seen in Ssection \ref{subsec:subpfnonseq}, the only rule of the SL whose corresponding subcase still needs to be proven is \rlnmsinit{}. Since $\in$ is a primitive definition (*better terminology), and we have no other hypotheses about it here, we will eventually reach a dead end. We will step back in Figure \ref{fig:incpfdyn} to before the rule \rlnmsinit{} was applied and use $P_1$ and $P_2$ as defined here. We are proving
\begin{align*}
H_1 &: D \in \inddyncon{} , \delta \\
\mathit{Hb}_1 &: \bchsl[\inddyncon{} , \delta]{D}{a_1} \\
\mathit{IHb}_1 &: \forall ({\dyncon{}_0} : \sltm{context}), (\inddyncon{} , \delta) = (\dyncon{}_0 , \delta) \rightarrow \seqsl[\dyncon{}_0]{\delta} \rightarrow \\
& \qquad \bchsl[\dyncon{}_0]{D}{\atom{a_1}} \\
\mathit{IP}_1 &: \dyncon{} = \inddyncon{} , \delta \\
\mathit{IP}_2 &: \seqsl[\inddyncon{}]{\delta} \\[\pfshift{}]
\cline{1-2}
& \seqsl[\inddyncon{}]{\atom{a_1}}
\end{align*}
with $(\inddyncon , \delta)$ substituted for $\dyncon{}$ in the assumptions by $\mathit{IP}_1$ and renaming in $\mathit{IHb}_1$ to avoid variable capture. We can get a new premise $P_3 : \bchsl[\inddyncon{}]{\delta}{a_1}$ by applying $\mathit{IHb}_1$ to a reflexivity lemma and $\mathit{IP}_2$ and now ignore $\mathit{IHb}_1$ and $\mathit{Hb}_1$ (which we can get from \sltm{weakening} and $P_3$).

Next apply the context lemma \sltm{elem\_inv} to $H_1$ to get the premise $(D \in \inddyncon{}) \vee (D = \delta)$. Applying \coqtm{inversion} to this, we have two new subgoals with diverging sets of assumptions. In the second we substitute $\delta$ for $D$ by $H_1$.
\begin{align*}
H_1 &: D \in \inddyncon{}  &H_1 &: D = \delta \\
\mathit{IP}_2 &: \seqsl[\inddyncon{}]{\delta}  &\mathit{IP}_2 &: \seqsl[\inddyncon{}]{\delta} \\
P_3 &: \bchsl[\inddyncon{}]{D}{a_1} &P_3 &: \bchsl[\inddyncon{}]{\delta}{a_1} \\[\pfshift{}]
\cline{1-5}
& \seqsl[\inddyncon{}]{\atom{a_1}}
&& \seqsl[\inddyncon{}]{\atom{a_1}}
\end{align*}
The left subgoal is provable by first applying \rlnmsinit{} to get subgoals $D \in \inddyncon{}$ and \bchsl[\inddyncon{}]{D}{a_1}, both proven by assumption.

%Notice that $\mathit{IHb}_1$ is the induction hypothesis corresponding to the portion of the cut rule for backchaining sequents. We get this because of the backchaining sequent premise of the \rlnmsinit{} rule. If we had a hypothesis about the goal-reduction portion of this rule, then we could finish this proof as in Figure \ref{fig:hyppf}.

%\begin{figure}
%$$
%\infer[\mathit{gr \; cut \; IH}]{\seqsl[\inddyncon{}]{\atom{a_1}}}{
%	\infer[\coqtm{apply \rlnmsinit{}}]{\seqsl[\inddyncon{} , \delta]{\atom{a_1}}}{
%		\infer[\coqtm{apply elem\_self}]{\delta \in \inddyncon{} , \delta}{}
%		&
%		\infer[\coqtm{assumption}]{\bchsl[\inddyncon{} , \delta]{\delta}{a_1}}{}
%	}
%	&
%	\infer[\coqtm{assumption}]{\seqsl[\inddyncon{}]{\delta}}{}
%}
%$$
%\centering{\caption{Cut Admissibility \rlnmsinit{} Branch with Goal-Reduction Hypothesis} \label{fig:hyppf}}
%\end{figure}

The proof on the right will be continued with an induction over $\delta$. The property to prove is
\begin{align*}
P_0 &: \sltm{oo} \rightarrow \coqtm{Prop} := \lambda (\delta : \sltm{oo}) \; . \; \forall (\inddyncon{} : \sltm{context}) (a_1 : \sltm{atm}), \\
& \qquad\quad \seqsl[\inddyncon{}]{\delta} \rightarrow \bchsl[\inddyncon{}]{\delta}{a_1} \rightarrow \seqsl[\inddyncon{}]{\atom{a_1}}
\end{align*}
We will now look at a specific subcase of this induction.

\paragraph{Subcase $\delta = o_1 \longrightarrow o_2$ :}
%\noindent\textbf{Subcase} $\delta = o_1 \longrightarrow o_2$ \textbf{:} ~\\

In this case we prove the appropriate antecedent of the induction principle for induction over $\delta$, shown below.
$$
\forall (o_1 \; o_2 : \sltm{oo}), P_0 \; o_1 \rightarrow P_0 \; o_2 \rightarrow P_0 \; (o_1 \longrightarrow o_2)
$$
The expanded proof state after premise introductions is:
\begin{align*}
\mathit{IH}_1 &: \forall (\inddyncon{} : \sltm{context}) (a_1 : \sltm{atm}), \\
& \qquad \seqsl[\inddyncon{}]{o_1} \rightarrow \bchsl[\inddyncon{}]{o_1}{a_1} \rightarrow \seqsl[\inddyncon{}]{\atom{a_1}} \\
\mathit{IH}_2 &: \forall (\inddyncon{} : \sltm{context}) (a_1 : \sltm{atm}), \\
& \qquad \seqsl[\inddyncon{}]{o_2} \rightarrow \bchsl[\inddyncon{}]{o_2}{a_1} \rightarrow \seqsl[\inddyncon{}]{\atom{a_1}} \\
\mathit{IP}_2 &: \seqsl[\inddyncon{}]{(o_1 \longrightarrow o_2)} \\
P_3 &: \bchsl[\inddyncon{}]{o_1 \longrightarrow o_2}{a_1} \\[\pfshift{}]
\cline{1-2}
& \seqsl[\inddyncon{}]{\atom{a_1}}
\end{align*}


We can apply \coqtm{inversion} to the premises $\mathit{IP}_2$ and $P_3$ to get new assumptions in the context.
\begin{align*}
\mathit{IH}_1 &: \forall (\inddyncon{} : \sltm{context}) (a_1 : \sltm{atm}), \bchsl[\inddyncon{} , o_1]{o_1}{a_1} \rightarrow \\
& \qquad \seqsl[\inddyncon{}]{o_1} \rightarrow \bchsl[\inddyncon{}]{o_1}{a_1} \rightarrow \seqsl[\inddyncon{}]{\atom{a_1}} \\
\mathit{IH}_2 &: \forall (\inddyncon{} : \sltm{context}) (a_1 : \sltm{atm}), \bchsl[\inddyncon{}, o_2]{o_2}{a_1} \rightarrow \\
& \qquad \seqsl[\inddyncon{}]{o_2} \rightarrow \bchsl[\inddyncon{}]{o_2}{a_1} \rightarrow \seqsl[\inddyncon{}]{\atom{a_1}} \\
\mathit{IP}_2 &: \seqsl[\inddyncon{}, o_1]{o_2} \\
P_{3_1} &: \bchsl[\inddyncon{}]{o_2}{a_1} \\
P_{3_2} &: \seqsl[\inddyncon{}]{o_1} \\[\pfshift{}]
\cline{1-2}
& \seqsl[\inddyncon{}]{\atom{a_1}}
\end{align*}
$\mathit{IH}_1$ is not useful here, since we have no way to prove sequents with $o_1$ focused. Applying $\mathit{IH}_2$ and ignoring induction hypotheses, we have:
\begin{align*}
\mathit{IP}_2 &: \seqsl[\inddyncon{}, o_1]{o_2} \\
P_{3_1} &: \bchsl[\inddyncon{}]{o_2}{a_1} \\
P_{3_2} &: \seqsl[\inddyncon{}]{o_1} \\[\pfshift{}]
\cline{1-2}
& (\bchsl[\inddyncon{}, o_2]{o_2}{a_1}), (\seqsl[\inddyncon{}]{o_2}), (\bchsl[\inddyncon{}]{o_2}{a_1}) 
\end{align*}
The first subgoal is proven using \sltm{weakening} and assumption $P_{3_1}$, and the third subgoal by $P_{3_1}$.

On trying to prove the second subgoal, we should reflect on two things; first, proving \seqsl[\inddyncon{}]{o_2} from the assumptions $\mathit{IP}_2$ and $P_{3_2}$ would be a use of the goal-reduction cut rule and second, since the \rlnmsinit{} rule only has a premise that is a backchaining sequent, we only get this part of the cut rule in the induction hypothesis. This branch cannot be continued any further. \\
