\section{Structural Rules}
\label{sec:structrules}

For our intuitionistic SL we prove the standard structural rules of weakening, contraction, and exchange:
\begin{thm}[\sltm{weakening}]
$$
\vcenter{\infer{\seqsl[\dyncon{} \, , \beta_1]{\beta_2}}{\seqsl{\beta_2}}} \; \wedge \; \vcenter{\infer{\bchsl[\dyncon{} \, , \beta_1]{\beta_2}{\alpha}}{\bchsl{\beta_2}{\alpha}}}
$$
\end{thm}

\begin{thm}[\sltm{contraction}]
$$
\vcenter{\infer{\seqsl[\dyncon{} \, , \beta_1]{\beta_2}}{\seqsl[\dyncon{} \, , \, \beta_1 \, , \beta_1]{\beta_2}}} \; \wedge \; \vcenter{\infer{\bchsl[\dyncon{} \, , \, \beta_1]{\beta_2}{\alpha}}{\bchsl[\dyncon{} \, , \, \beta_1 \, , \, \beta_1]{\beta_2}{\alpha}}}
$$
\end{thm}

\begin{thm}[\sltm{exchange}]
$$
\vcenter{\infer{\seqsl[\dyncon{} \, , \beta_1 \, , \, \beta_2]{\beta_3}}{\seqsl[\dyncon{} \, , \, \beta_2 \, , \, \beta_1]{\beta_3}}} \; \wedge \; \vcenter{\infer{\bchsl[\dyncon{} \, , \, \beta_1 \, , \, \beta_2]{\beta_3}{\alpha}}{\bchsl[\dyncon{} \, , \, \beta_2 \, , \, \beta_1]{\beta_3}{\alpha}}}
$$
\end{thm}

\noindent These are all corollaries of a general theorem:

\begin{theorem}[\sltm{monotone}]
$$
\vcenter{\infer{\seqsl[\inddyncon{}]{\beta}}{\dyncon{} \subseteq \inddyncon{} & \seqsl[\dyncon{}]{\beta}}} \; \wedge \; \vcenter{\infer{\bchsl[\inddyncon{}]{\beta}{\alpha}}{\dyncon{} \subseteq \inddyncon{} & \bchsl[\dyncon{}]{\beta}{\alpha}}}
$$
\label{thm:monotone}
\end{theorem}

The proofs of all of these theorems are automated in the Coq implementation.
Here, we continue our explanation using our generalized proof states.

\begin{proof}

Theorem~\ref{thm:monotone} is proven by mutual structural induction over the premises \seqsl{\beta} and \bchsl{\beta}{\alpha}. Defining $P_1$ and $P_2$ as
\begin{align*}
P_1 :=& \lambda \; (c : \sltm{context}) (o : \sltm{oo}) \; . \\
& \forall \; (\inddyncon{} : \sltm{context}), c \subseteq \inddyncon{} \rightarrow \underline{\seqsl[\inddyncon{}]{o}} \\
P_2 :=& \lambda \; (c : \sltm{context}) (o : \sltm{oo}) (a : \sltm{atm}) \; . \\
& \forall \; (\inddyncon{} : \sltm{context}), c \subseteq \inddyncon{} \rightarrow \underline{\bchsl[\inddyncon{}]{o}{a}}
\end{align*}
we are proving
\begin{align*}
& (\forall \; (c : \sltm{context}) \; (o : \sltm{oo}), (\seqsl[c]{o}) \rightarrow (P_1 \; c \; o)) \;\; \wedge \\
& (\forall \; (c : \sltm{context}) \; (o : \sltm{oo}) \; (a : \sltm{atm}), (\bchsl[c]{o}{a}) \rightarrow (P_2 \; c \; o \; a))
\end{align*}
which has the form discussed in Section~\ref{sec:induction},
so the mutual structural induction principle may be used.


%\begin{align*}
%P_1 :=& \lambda \; (\dyncon{}_1 : \sltm{context}) \; . \; \lambda \; (\beta : \sltm{oo}) \; . \\
%& \forall \; (\dyncon{}_2 : \sltm{context}), \dyncon{}_1 \subseteq \dyncon{}_2 \rightarrow \seqsl[\dyncon{}_2]{\beta} \\
%P_2 :=& \lambda \; (\dyncon{}_1 : \sltm{context}) \; . \; \lambda \; (\beta : \sltm{oo}) \; . \; \lambda \; (\alpha : \sltm{atm}) \; . \\
%& \forall \; (\dyncon{}_2 : \sltm{context}), \dyncon{}_1 \subseteq \dyncon{}_2 \rightarrow \bchsl[\dyncon{}_2]{\beta}{\alpha}
%\end{align*}
%we are proving
%\begin{align*}
%&(\forall \; (\dyncon{}_1 : \sltm{context}) \; (\beta : \sltm{oo}), \\
%&\;\;\;\;\; (\seqsl[\dyncon{}_1]{\beta}) \rightarrow (P_1 \; \dyncon{}_1 \; \beta)) \\
%\wedge \; & (\forall \; (\dyncon{}_1 : \sltm{context}) \; (\beta : \sltm{oo}) \; (\alpha : \sltm{atm}), \\
%&\;\;\;\;\; (\bchsl[\dyncon{_1}]{\beta}{\alpha}) \rightarrow (P_2 \; \dyncon{} \; \beta \; \alpha))
%\end{align*}

\subsection{Generalized SL Part II: The Structural Rules Hold}
\label{subsec:structpf}

We build on the inductive proof in Section \ref{sec:pfgsl} over the GSL. Recall that when we took the proof as far as we could; we had three remaining groups of branches to finish ($m + n + p$ subgoals), one group for rules with non-sequent premises depending on the context of the rule conclusion, and one for each kind of sequent premise (see Figures \ref{fig:premgrseq} and \ref{fig:incpfdyn}). We will continue this effort below, using the $P_1$ and $P_2$ defined for this theorem. This means we will have one induction assumption (i.e., $w = 1$) which is $\mathit{IA}_1 \args{c , o , \inddyncon{}} \coloneqq c \subseteq \inddyncon{}$.


\subsubsection{Subproofs for Sequent Premises}

First we will prove the subgoals coming from the sequent premises, building on Figure \ref{fig:premgrseq} and using $\mathit{IA}_1$ as defined above. The proof state for goal-reduction (resp. backchaining) premises is
\begin{align*}
\ol{H_m} &: \ol{Q_m} \args{c , o} \\
\ol{\mathit{Hg}_n} &: \forall \ol{(x_{n,s_n} : R_{n,s_n})}, (\seqsl[c \cup \ol{\gamma_n} \args{o}]{\ol{F_n} \args{o , \ol{x_{n,s_n}}})} \\
\ol{\mathit{IHg}_n} &: \forall \ol{(x_{n,s_n} : R_{n,s_n})} (\inddyncon{} : \sltm{context}), \\
& \qquad (c \cup \ol{\gamma_n} \args{o}) \subseteq \inddyncon{} \rightarrow \seqsl[\inddyncon{}]{\ol{F_n} \args{o , \ol{x_{n,s_n}}}} \\
\ol{\mathit{Hb}_p} &: \forall \ol{(y_{p,t_p} : S_{p,t_p})}, (\bchsl[c \cup \ol{\gamma'_p} \args{o}]{\ol{F'_p} \args{o , \ol{y_{p,t_p}}}}{\ol{a_p}}) \\
\ol{\mathit{IHb}_p} &: \forall \ol{(y_{p,t_p} : S_{p,t_p})} (\inddyncon{} : \sltm{context}), \\
& \qquad (c \cup \ol{\gamma'_p} \args{o}) \subseteq \inddyncon{} \rightarrow \bchsl[\inddyncon{}]{\ol{F'_p} \args{o , \ol{y_{p,t_p}}}}{\ol{a_p}} \\
\mathit{IP}_1 &: c \subseteq \inddyncon{} \\[\pfshift{}]
\cline{1-2}
(c \cup \ol{\gamma_n} \args{o}) & \subseteq (\inddyncon{} \cup \ol{\gamma_n} \args{o}) \; (\mathit{resp.} \; (c \cup \ol{\gamma'_p} \args{o}) \subseteq (\inddyncon{} \cup \ol{\gamma'_p} \args{o}))
\end{align*}
The goal is provable by context lemma \sltm{context\_sub\_sup} and assumption $\mathit{IP}_1$.


\subsubsection{Subproofs for Non-Sequent Premises}

Still to be proven are the subgoals for non-sequent premises. As seen in Subsection \ref{subsec:subpfnonseq}, the only rule of the SL whose corresponding subcase still needs to be proven is \rlnmsinit{}. From Figure \ref{fig:incpfdyn} and using $P_1$ and $P_2$ as defined here, we are proving
\begin{align*}
H_1 &: D \in \dyncon{} \\
\mathit{Hb}_1 &: \bchsl{D}{a_1} \\
\mathit{IHb}_1 &: \forall (\inddyncon{} : \sltm{context}), \dyncon{} \subseteq \inddyncon{} \rightarrow \bchsl[\inddyncon{}]{D}{a_1} \\
\mathit{IP}_1 &: \dyncon{} \subseteq \inddyncon{} \\[\pfshift{}]
\cline{1-2}
& D \in \inddyncon{}
\end{align*}
Unfolding the definition of context subset in $\mathit{IP}_1$ it becomes $\forall (o : \sltm{oo}), o \in \dyncon{} \rightarrow o \in \inddyncon{}$. 
%Applying this to the goal
Backchaining on this form of the goal
gives $D \in \dyncon{}$, provable by assumption $H_1$.


In Section \ref{sec:pfgsl}, we explored how to prove statements about the GSL for a restricted form of theorem statement. There were three classes of incomplete proof branches that had a final form shown in Figures \ref{fig:premgrseq} and \ref{fig:incpfdyn}. In Section \ref{subsec:sltogsl} we saw how to derive the SL from the GSL. So here we have proven a structural theorem for the rules of the GSL in a general way that can be followed for any SL rule.
\end{proof}