\section{Proof by Induction over the Generalized Rules}
\label{sec:pfgsl}

The induction subcase corresponding to \rl{gr\_rule}
(resp. \rl{bc\_rule}) requires a proof of:
\pagebreak[0]
\begin{align*}
\ol{H_m} &: \ol{Q_m} \args{c , o} \\
\ol{\mathit{Hg}_n} &: \forall \ol{(x_{n,s_n} : R_{n,s_n})}, (\seqsl[c \cup \ol{\gamma_n} \args{o}]{\ol{F_n} \args{o , \ol{x_{n,s_n}}})} \\
\ol{\mathit{IHg}_n} &: \forall \ol{(x_{n,s_n} : R_{n,s_n})}, P_1 \; (c \cup \ol{\gamma_n} \args{o}) \; (\ol{F_n} \args{o , \ol{x_{n,s_n}}}) \\
\ol{\mathit{Hb}_p} &: \forall \ol{(y_{p,t_p} : S_{p,t_p})}, (\bchsl[c \cup \ol{\gamma'_p} \args{o}]{\ol{F'_p} \args{o , \ol{y_{p,t_p}}}}{\ol{a_p}}) \\
\ol{\mathit{IHb}_p} &: \forall \ol{(y_{p,t_p} : S_{p,t_p})}, P_2 \; (c \cup \ol{\gamma'_p} \args{o}) \; (\ol{F'_p} \args{o , \ol{y_{p,t_p}}}) \; \ol{a_p} \\[\pfshift{}]
\cline{1-2}
& P_1 \; c \; o \; (\mathit{resp.} \; P_2 \; c \; o \; a)
\end{align*}

Given specific $P_1$ and $P_2$, we could unfold uses of these predicates and continue the proof.
%We will consider a restricted version of this abstraction that is sufficient for the SL and its metatheory presented here. For $j = 1 \ldots n$ and $k = 1 \ldots p$, we restrict $C_j \args{c,o}$ and $C'_k \args{c,o}$ to have the form $c \cup (\gamma_j \args{c,o})$ and $c \cup (\gamma'_k \args{c,o})$, respectively. So we require the conclusion context to be a subset of the premise contexts. This is satisfied by the rules of our implemented SL. In fact, we will have $\gamma_j \args{c,o} = \gamma'_k \args{c,o} = \emptyset$ for all rules other than \rlnmsimp{} where $j = 1$ and $\gamma_1 \args{\dyncon{} , (D \longrightarrow G)} = \{ D \}$. (*idea: build this restriction into the GSL rules, comment that could be generalized further, remove this paragraph? OR might just remove this restriction since it doesn't add much here)
Suppose
\begin{align*}
P_1 &:= \lambda c \; o . \forall (\inddyncon{} : \sltm{context}), \\
& \mathit{IA}_1 \args{c,o,\inddyncon{}} \rightarrow \dots \rightarrow \mathit{IA}_w \args{c,o,\inddyncon{}} \rightarrow \underline{\seqsl[\inddyncon{}]{o}} \qquad \mathrm{and}\\
P_2 &:= \lambda c \; o \; a . \forall (\inddyncon{} : \sltm{context}), \\
& \mathit{IA}_1 \args{c,o,\inddyncon{}} \rightarrow \dots \rightarrow \mathit{IA}_w \args{c,o,\inddyncon{}} \rightarrow \underline{\bchsl[\inddyncon{}]{o}{a}}
\end{align*}
The underlining of sequents in the definitions of $P_1$ and $P_2$ is
to highlight that these are the sequents we apply the generalized
rules to (following introductions). In particular, we unfold uses of
$P_1$ and $P_2$ in the proof state and introduce the variables and
induction assumptions.  Then the goal is either
\seqsl[\inddyncon{}]{o} or \bchsl[\inddyncon{}]{o}{a}. Apply
\rl{gr\_rule} or \rl{bc\_rule} as appropriate, and either will give
($m + n + p$) new subgoals which come from the three premise forms in
these rules, with appropriate instantiations for the externally
quantified variables. Now the proof state is
\begin{align*}
\ol{H_m} &: \ol{Q_m} \args{c , o} \\
\ol{\mathit{Hg}_n} &: \forall \ol{(x_{n,s_n} : R_{n,s_n})}, (\seqsl[c \cup \ol{\gamma_n} \args{o}]{\ol{F_n} \args{o , \ol{x_{n,s_n}}})} \\
\ol{\mathit{IHg}_n} &: \forall \ol{(x_{n,s_n} : R_{n,s_n})} (\inddyncon{} : \sltm{context}), \\
& \mathit{IA}_1 \args{c \cup \ol{\gamma_n} \args{o} , \ol{F_n} \args{o , \ol{x_{n,s_n}}} , \inddyncon{}} \rightarrow \dots \rightarrow \\
& \mathit{IA}_w \args{c \cup \ol{\gamma_n} \args{o} , \ol{F_n} \args{o , \ol{x_{n,s_n}}} , \inddyncon{}} \rightarrow \seqsl[\inddyncon{}]{\ol{F_n} \args{o , \ol{x_{n,s_n}}}} \\
\ol{\mathit{Hb}_p} &: \forall \ol{(y_{p,t_p} : S_{p,t_p})}, (\bchsl[c \cup \ol{\gamma'_p} \args{o}]{\ol{F'_p} \args{o , \ol{y_{p,t_p}}}}{\ol{a_p}}) \\
\ol{\mathit{IHb}_p} &: \forall \ol{(y_{p,t_p} : S_{p,t_p})} (\inddyncon{} : \sltm{context}), \\
& \mathit{IA}_1 \args{c \cup \ol{\gamma'_p} \args{o} , \ol{F'_p} \args{o , \ol{y_{p,t_p}}} , \inddyncon{}} \rightarrow \dots \rightarrow \\
& \mathit{IA}_w \args{c \cup \ol{\gamma'_p} \args{o} , \ol{F'_p} \args{o , \ol{y_{p,t_p}}} , \inddyncon{}} \rightarrow \bchsl[\inddyncon{}]{\ol{F'_p} \args{o , \ol{y_{p,t_p}}}}{\ol{a_p}} \\
\ol{\mathit{IP}_w} &: \ol{\mathit{IA}_w} \args{c , o , \inddyncon{}} \\[\pfshift{}]
\cline{1-2}
& \ol{Q_m} \args{\inddyncon{} , o}, \\
& \forall \ol{(x_{n,s_n} : R_{n,s_n})}, (\seqsl[\inddyncon{} \cup \ol{\gamma_n} \args{o}]{\ol{F_n} \args{o , \ol{x_{n,s_n}}}}), \\
& \forall \ol{(y_{p,t_p} : S_{p,t_p})}, (\bchsl[\inddyncon{} \cup \ol{\gamma'_p} \args{o}]{\ol{F'_p} \args{o , \ol{y_{p,t_p}}}}{\ol{a_p}})
\end{align*}
where $\inddyncon{}$ is a new signature variable.

\subsection{Subproofs for Sequent Premises}

\begin{figure}
\begin{align*}
\ol{H_m} &: \ol{Q_m} \args{c , o} \\
\ol{\mathit{Hg}_n} &: \forall \ol{(x_{n,s_n} : R_{n,s_n})}, (\seqsl[c \cup \ol{\gamma_n} \args{o}]{\ol{F_n} \args{o , \ol{x_{n,s_n}}})} \\
\ol{\mathit{IHg}_n} &: \forall \ol{(x_{n,s_n} : R_{n,s_n})} (\inddyncon{} : \sltm{context}), \\
& \mathit{IA}_1 \args{c \cup \ol{\gamma_n} \args{o} , \ol{F_n} \args{o , \ol{x_{n,s_n}}} , \inddyncon{}} \rightarrow \dots \rightarrow \\
& \mathit{IA}_w \args{c \cup \ol{\gamma_n} \args{o} , \ol{F_n} \args{o , \ol{x_{n,s_n}}} , \inddyncon{}} \rightarrow \seqsl[\inddyncon{}]{\ol{F_n} \args{o , \ol{x_{n,s_n}}}} \\
\ol{\mathit{Hb}_p} &: \forall \ol{(y_{p,t_p} : S_{p,t_p})}, (\bchsl[c \cup \ol{\gamma'_p} \args{o}]{\ol{F'_p} \args{o , \ol{y_{p,t_p}}}}{\ol{a_p}}) \\
\ol{\mathit{IHb}_p} &: \forall \ol{(y_{p,t_p} : S_{p,t_p})} (\inddyncon{} : \sltm{context}), \\
& \mathit{IA}_1 \args{c \cup \ol{\gamma'_p} \args{o} , \ol{F'_p} \args{o , \ol{y_{p,t_p}}} , \inddyncon{}} \rightarrow \dots \rightarrow \\
& \mathit{IA}_w \args{c \cup \ol{\gamma'_p} \args{o} , \ol{F'_p} \args{o , \ol{y_{p,t_p}}} , \inddyncon{}} \rightarrow \bchsl[\inddyncon{}]{\ol{F'_p} \args{o , \ol{y_{p,t_p}}}}{\ol{a_p}} \\
\ol{\mathit{IP}_w} &: \ol{\mathit{IA}_w} \args{c , o , \inddyncon{}} \\[\pfshift{}]
\cline{1-2}
& \ol{\mathit{IA}_w} \args{c \cup \ol{\gamma_n} \args{o} , \ol{F_n} \args{o , \ol{x_{n,s_n}}} , \inddyncon{} \cup \ol{\gamma_n} \args{o}} \\
& (\mathit{resp.} \; \ol{\mathit{IA}_w} \args{c \cup \ol{\gamma'_p} \args{o} , \ol{F'_p} \args{o , \ol{y_{p,t_p}}} , \inddyncon{} \cup \ol{\gamma'_p} \args{o}})
\end{align*}
\centering{\caption{Incomplete proof branches for sequent premises \label{fig:premgrseq}}}
\end{figure}

To prove the last ($n + p$) subgoals (the ``second'' and ``third''
subgoals above) we first introduce any locally quantified variables as
signature variables. For the goal-reduction (resp. backchaining)
subgoals, for $j = 1 , \ldots , n$ (resp. $k = 1 , \ldots , p$), we
apply induction hypothesis $\mathit{IHg}_j$ (resp. $\mathit{IHb}_k$),
instantiating $\inddyncon{}$ in the induction hypothesis with
$\inddyncon{} \cup \gamma_j \args{o}$ (resp. $\inddyncon{} \cup
\gamma'_k \args{o}$). This yields the proof state in Figure
\ref{fig:premgrseq} for goal-reduction premises (resp. backchaining
premises).


\subsection{Subproofs for Non-Sequent Premises}
\label{subsec:subpfnonseq}

The proof of the first $m$ subgoals depends on the definition of $Q_i$ for $i = 1 \ldots m$. If the first argument (a \sltm{context}) is not used in its definition, then $Q_i \args{\inddyncon{} , o}$ is provable by assumption $H_i$, since we will have $Q_i \args{\inddyncon{} , o} = Q_i \args{c , o}$. Any other dependencies on signature variables can be ignored since we can assign the variables as we choose when applying the generalized rule. We will illustrate this by considering each rule with non-sequent premises, starting from the second proof state in Section \ref{sec:pfgsl} and, for $(i = 1 , \ldots , m)$, $(j = 1 , \ldots , n)$, $(k = 1 , \ldots , p)$, show how to define $Q_i$, $\gamma_j$, $F_j$, $\gamma'_k$, and $F'_k$ and finish the subproofs where possible.

\paragraph{Case \rlnmsbc{} :} This rule has one non-sequent premise and one goal-reduction sequent premise with no local quantification, so $m = n = 1$, $p = 0$, $o = \atom{A}$, and $c = \dyncon{}$. Define $Q_1 \args{\dyncon{} , \atom{A}} \coloneqq \prog{A}{G}$, $\gamma_1 \args{\atom{A}} \coloneqq \emptyset$, and $F_1 \args{\atom{A}} \coloneqq G$, where $G : \sltm{oo}$ is a signature variable. Then we are proving
the following:
\begin{align*}
H_1 &: \prog{A}{G} \\
\mathit{Hg}_1 &: \seqsl{G} \\
\mathit{IHg}_1 &: \forall (\inddyncon{} : \sltm{context}), \mathit{IA}_1 \args{\dyncon{} , G , \inddyncon{}} \rightarrow \dots \rightarrow \\
& \qquad \mathit{IA}_w \args{\dyncon{} , G , \inddyncon{}} \rightarrow \seqsl[\inddyncon{}]{G} \\
\ol{\mathit{IP}_w} &: \ol{\mathit{IA}_w} \args{\dyncon{} , \atom{A} , \inddyncon{}} \\[\pfshift{}]
\cline{1-2}
& \prog{A}{G}
\end{align*}
which is completed by assumption $H_1$.

\paragraph{Case \rlnmsinit{} :} This rule has one non-sequent premise and one backchaining sequent premise with no local quantification, so $m = p = 1$, $n = 0$, $c = \dyncon{}$, and $o = \atom{A}$. Define $Q_1 \args{\dyncon{} , \atom{A}} \coloneqq D \in \dyncon{}$, $\gamma'_1 \args{\atom{A}} \coloneqq \emptyset$, and $F'_1 \args{\atom{A}} \coloneqq D$, where $D : \sltm{oo}$ is a signature variable. Then we need to prove what is displayed in Figure \ref{fig:incpfdyn}.
\begin{figure}
\begin{align*}
H_1 &: D \in \dyncon{} \\
\mathit{Hb}_1 &: \bchsl{D}{a_1} \\
\mathit{IHb}_1 &: \forall (\inddyncon{} : \sltm{context}), \mathit{IA}_1 \args{\dyncon{} , D , \inddyncon{}} \rightarrow \dots \rightarrow \\
& \qquad \mathit{IA}_w \args{\dyncon{} , D , \inddyncon{}} \rightarrow \bchsl[\inddyncon{}]{D}{a_1} \\
\ol{\mathit{IP}_w} &: \ol{\mathit{IA}_w} \args{\dyncon{} , \atom{A} , \inddyncon{}} \\[\pfshift{}]
\cline{1-2}
& D \in \inddyncon{}
\end{align*}
\centering{\caption{Incomplete proof branch (\rlnmsinit{} case) \label{fig:incpfdyn}}}
\end{figure}
Here we do not have enough information to finish this branch of the proof. An induction assumption may be of use, but we will need specific $P_1$ and $P_2$.

\paragraph{Case \rlnmssome{} :} This rule has one non-sequent premise and one goal-reduction sequent premise with no local quantification, so $m = n = 1$, $p = 0$, $c = \dyncon{}$, and $o = \sltm{Some} \; G$. Define $Q_1 \args{\dyncon{} , \sltm{Some} \; G} \coloneqq \sltm{proper} \; E$, $\gamma_1 \args{\sltm{Some} \; G} \coloneqq \emptyset$, and $F_1 \args{\sltm{Some} \; G} \coloneqq G \; E$ where $E : \sltm{expr con}$ is a signature variable. Then we are proving
the following:
\begin{align*}
H_1 &: \sltm{proper} \; E \\
\mathit{Hg}_1 &: \seqsl{G \; E} \\
\mathit{IHg}_1 &: \forall (\inddyncon{} : \sltm{context}), \mathit{IA}_1 \args{\dyncon{} , G \; E , \inddyncon{}} \rightarrow \dots \rightarrow \\
& \qquad \mathit{IA}_w \args{\dyncon{} , G \; E , \inddyncon{}} \rightarrow \seqsl[\inddyncon{}]{G \; E} \\
\ol{\mathit{IP}_w} &: \ol{\mathit{IA}_w} \args{\dyncon{} , \sltm{Some} \; G , \inddyncon{}} \\[\pfshift{}]
\cline{1-2}
& \sltm{proper} \; E
\end{align*}
which is completed by assumption $H_1$.

\paragraph{Case \rlnmball{} :} This case is proven as above but with $m = p = 1$, $n = 0$, $c = \dyncon{}$, and $o = \sltm{All} \; D$. Define $Q_1 \args{\dyncon{} , \sltm{All} \; D} \coloneqq \sltm{proper} \; E$, $\gamma'_1 \args{\sltm{All} \; D} \coloneqq \emptyset$, and $F'_1 \args{\sltm{All} \; D} \coloneqq D \; E$ where $E : \sltm{expr con}$ is a signature variable. The goal $\sltm{proper} \; E$ is provable by the assumption of the same form
as in the previous case.
%from the definition of the rule.

In the next two sections we will return to this idea of proofs about a specification logic from a generalized form of SL rule to prove properties of the SL once we have fully defined $P_1$ and $P_2$. The proof states in Figures \ref{fig:premgrseq} and \ref{fig:incpfdyn} (the incomplete branches) will be roots of these explanations.
