\section{Conclusion}
\label{sec:concl}

We have described the extension of Hybrid to a new more expressive SL,
and focused on presenting the proofs of important metatheoretic
properties in a general manner.
With the metatheory of this SL completed, the next step is to
illustrate the benefits of the extra expressive power via case
studies.

In the GSL we have made the rules general enough to capture the rules
of the SL, but it could be generalized further to explore other
specification logics that do not fit the restrictions here.  This is a
subject of future work, as we explore even more expressive SLs.

There are a variety of other systems supporting HOAS, such as Beluga
\cite{Pientka:IJCAR10} and Twelf \cite{TwelfSP}, to name just two.
These two systems, along with Abella and Hybrid were compared on
several benchmark problems in \cite{FMP:JAR15}.  Hybrid and Abella are
similar in the sense that they are based on a proof-theoretic
foundation and follow the two-level approach, implementing an SL
inside a logic or type theory, while Twelf and Beluga are built on
type-theoretic foundations.

Our formulation of sequent rules uses the style developed in
\cite{Pfenning:IC00}, where the cut rule incorporates aspects of
weakening and contraction, facilitating the kind of structural
induction argument used both there and here.  One main difference in
the approaches is the representation of inference rules.  In that
paper, the proofs are formalized in Elf (an early version of Twelf)
where rules are generally expressed in a natural deduction style with
implicit contexts.  In contrast, we represent and reason about
contexts directly, and consequently illustrate that doing so is not as
difficult as is argued in \cite{Pfenning:IC00}.\footnote{See
  \cite{FMP:JAR15} for a fuller comparison of these two approaches to
  reasoning.}

The proof of cut admissibility in Abella uses the same overall
structure with induction on the cut formula, with a sub-induction on
the structure of the proof of the left premise of the cut rule.  A
detailed analysis of the differences in the proofs is left as future
work.  We do note, however, that the statement of the two theorems
differs.  The Abella version requires the following additional
conjunct:
$$\forall (c : \sltm{context}) (o : \sltm{oo}) (a: \hybridtm{atm}),\\
\seqsl[c]{o} \rightarrow \bchsl[c]{o}{a} \rightarrow \seqsl[c]{\atom{a}}.$$

There are also some differences in the rules.  Our logic includes
existential quantification, while the Abella version does not.  Also,
without loss of generality, our $\rlnmsbc$ rule restricts static
program clauses to have the form
$\forall_{\tau_1}\ldots\forall_{\tau_n}.G\longrightarrow A$.  Finally,
we restrict universal quantification to second-order, while the Abella
version does not.  This does not affect the proofs of metatheorems,
and we don't expect it to significantly limit the kinds of OLs we can
consider.  For example, the two case studies in \cite{WCGN:PPDP13} do
not require more than second-order quantification in the SL.

As mentioned, implementing Hybrid in Coq gives us access to Coq's
extensive standard library and other facilities.  Having such access
allows us, for example, to simplify the encoding of the example
discussed in Section~\ref{sec:hybrid} in two ways.  First, in general,
when an OL's syntax can be directly encoded using a first-order
representation, we can define it directly as a Coq inductive type
instead of as a set of terms of type \oltm{expr}.  For example, we
could remove the three definitions for \oltm{dApp}, \oltm{dAbs}, and
\oltm{dVar} and the constants they depend on, and instead define an
type \oltm{dtm} with three constructors.
%the following inductive type:
%\begin{lstlisting}
%Inductive dtm : Set :=
%  dVar : nat -> dtm
%| dApp : dtm -> dtm -> dtm
%| dAbs : dtm -> dtm.
%\end{lstlisting}
Second, we can use Coq's library for natural numbers, which allows for
a simpler definition of the \oltm{hodb} inference rules (mentioned but
not shown earlier).

The development of the metatheory of the SL in this paper differs from
other Hybrid SLs, in particular, those that appear in
\cite{FeltyMomigliano:JAR10}.  In the definition of those SLs,
sequents had an additional natural number argument, and metatheoretic
properties were proved by induction over the height of a proof, rather
than by a direct structural induction on the definition of (the two
kinds of) sequents, as is done here.  Future work includes exploring
the connections between these two kinds of induction in the context of
the SL in this paper, as well as comparing their use in proving
properties of OLs.

%A detailed analysis of the different proofs of cut elimination within
%Hybrid and with others is another area of future work.  Others include
%the proof in Abella mentioned above.  In addition, the proof in
%\cite{Pfenning:IC00} uses an additional inner sub-induction on the
%premise not containing the cut formula.
