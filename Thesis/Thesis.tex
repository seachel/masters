%%%%%%%%%%%%%%%%%%%%%%%%%%%%%%%%%%%%%%%%%%%%%%%%%%%%%%%%%%%%%%%%%%%%%%
% All the option for the style report can be used (e.g. twoside, ....
% Moreover, the following options are also available: french, UOdraft,
% hyperref, ...
%
% The options french and UOdraft do not need any explanation.  You may want
% to add the option twoside to the option UOdraft to further reduce the
% number of pages.
%
% The option  hyperref is for those who would like to use the package
% hyperref.sty .
%
% Avec l'option  french , il est possible de compiler un texte qui
% possede des accents, ... sans avoir a recourir aux symboles \'e,
% \"u, etc.  En plus de permettre l'utilisation d'un editeur de texte
% normal, vous pouvez aussi utiliser votre "spell checker" prefere.
%
% With the option NoTofC, the section for the list of symbols isn't
% included in the table of Contents.  The default is that it is
% included in the list of contents.
%
% To compile the template, on has to do something like
%
% latex template.tex
% latex template.tex
% latex template.tex
% makeindex template.idx
% makeindex template.nlo -s nomencl.ist -o template.nls
% bibtex template
% latex template.tex
% makeindex template.idx
% makeindex template.nlo -s nomencl.ist -o template.nls
% bibtex template
% latex template.tex
%
% latex can be replaced by pdflatex if you have jpg or pdf figures. 
%%%%%%%%%%%%%%%%%%%%%%%%%%%%%%%%%%%%%%%%%%%%%%%%%%%%%%%%%%%%%%%%%%%%%%
\documentclass[12pt]{UOthesis}

%%%%%%%%%%%%%%%%%%%%%%%%%%%%%%%%%%%%%%%%%%%%%%%%%%%%%%%%%%%%%%%%%%%%%%
% Load the packages that you may need and that have not been loaded yet
%
% The following packages have already been loaded:
% ifthen, graphicx, color, amsmath, amssymb, amscd, amsthm, fancyhdr,
% nomencl, makeidx, letterpaper, setspace.
%
% In addition, for the French option, the following packages are also
% loaded: babel (french), ucs, inputenc (utf8x), and fontenc (T1).
% The last three packages let you write accents using the French
% keyboard.  No need to use \'e, \`e, ...
%
% The following packages are needed for the template but may not be
% needed for your thesis.  If they are not needed, don't hesitate to
% remove them.
%
% WARNING: The packages  babel  and  xy-pic  are incompatible.
%%%%%%%%%%%%%%%%%%%%%%%%%%%%%%%%%%%%%%%%%%%%%%%%%%%%%%%%%%%%%%%%%%%%%%
\usepackage{fancybox}
\usepackage{shadow}
\usepackage[all]{xy}
\usepackage{url}
\usepackage{rotating}

\usepackage{framed}
\usepackage{enumitem}
\usepackage{mathtools}
\usepackage{float}

\usepackage{listings}
\usepackage{multicol}
\usepackage{stmaryrd}
\usepackage{proof}
\usepackage{bussproofs}
\usepackage{lscape}

%\usepackage{cite}
\usepackage{url}

%%%%%%%%%%%%%%%%%%%%%%%%%
% FORMATTING SEQUENTS
%%%%%%%%%%%%%%%%%%%%%%%%%

\newcommand{\oltm}[1]{\texttt{#1}}          % inline OL terms
\newcommand{\sltm}[1]{\texttt{#1}}          % inline SL terms
\newcommand{\mltm}[1]{\texttt{#1}}          % inline reasoning logic terms
\newcommand{\hybridtm}[1]{\texttt{#1}}      % inline Hybrid terms
\newcommand{\coqtm}[1]{\texttt{#1}}         % inline Coq terms

\newcommand{\rl}[1]{\textit{#1}}   % rule name

\newcommand{\signature}{\Sigma}             % default signature
\newcommand{\dyncon}{\Gamma}              % default dynamic context
\newcommand{\statcon}{\Pi}                % default static context
\newcommand{\inddyncon}{\Gamma'}

% standard sequents
\newcommand{\seq}[2][\signature{} \, ; \, \dyncon{}]{\ensuremath{#1 \vdash #2}}
\newcommand{\bch}[3][\signature{} \, ; \, \dyncon{}]{\ensuremath{#1 , {[#2]} \vdash #3}}

% SL sequents
\newcommand{\seqsl}[2][\dyncon{}]{\ensuremath{#1 \rhd #2}}
\newcommand{\bchsl}[3][\dyncon{}]{\ensuremath{#1 , {[#2]} \rhd #3}}

% specify signature
\newcommand{\seqslsig}[2]{\seqsl[\signature{} \, , \, #1 \, ; \, \dyncon{}]{#2}}


\newcommand{\atom}[1]{\ensuremath{\langle \; #1 \; \rangle}}  % atomic formula
\newcommand{\prog}[2]{\ensuremath{#1 \; {:} {-} \; #2}}     % static program clause


%%%%%%%%%%%%%%%%%%%%%%%%%
% FORMATTING RULES
%%%%%%%%%%%%%%%%%%%%%%%%%

\newcommand{\rlnmsbc}{\rl{g\_prog}}
\newcommand{\rlnmsinit}{\rl{g\_dyn}}
\newcommand{\rlnmst}{\rl{g\_tt}}
\newcommand{\rlnmsand}{\rl{g\_and}}
\newcommand{\rlnmsimp}{\rl{g\_imp}}
\newcommand{\rlnmsall}{\rl{g\_all}}
\newcommand{\rlnmsalls}{\rl{g\_allx}}
\newcommand{\rlnmssome}{\rl{g\_some}}
\newcommand{\rlnmbmatch}{\rl{b\_match}}
\newcommand{\rlnmbanda}{\rl{$b\_and_1$}}
\newcommand{\rlnmbandb}{\rl{$b\_and_2$}}
\newcommand{\rlnmbimp}{\rl{b\_imp}}
\newcommand{\rlnmball}{\rl{b\_all}}
\newcommand{\rlnmballs}{\rl{b\_allx}}
\newcommand{\rlnmbsome}{\rl{b\_some}}

\newcommand{\inferH}[3][]{\infer[\fontsize{8pt}{8pt}{#1}]{#2}{#3}}

\newcommand{\rlsbc}{\inferH[\rlnmsbc{}]{\seqsl{\atom{A}}}{\prog{A}{G} & \seqsl{G}}}
\newcommand{\rlsinit}{\inferH[\rlnmsinit{}]{\seqsl{\atom{A}}}{D \in \dyncon{} & \bchsl{D}{A}}}
\newcommand{\rlst}{\inferH[\rlnmst{}]{\seqsl{\sltm{T}}}{}}
\newcommand{\rlsand}{\inferH[\rlnmsand{}]{\seqsl{G_1 \, \& \, G_2}}{\seqsl{G_1} & \seqsl{G_2}}}
\newcommand{\rlsimp}{\inferH[\rlnmsimp{}]{\seqsl{D \longrightarrow G}}{\seqsl[\dyncon{} \, , \, D]{G}}}
\newcommand{\rlsall}{\inferH[\rlnmsall{}]{\seqsl{\sltm{All} \; G}}{\forall (E : \hybridtm{expr con}), (\sltm{proper} \; E \rightarrow \seqsl{G \, E})}}
\newcommand{\rlsalls}{\inferH[\rlnmsalls{}]{\seqsl{\sltm{Allx} \; G}}{\forall (E : \sltm{X}), (\seqsl{G \, E})}}
\newcommand{\rlssome}{\inferH[\rlnmssome{}]{\seqsl{\sltm{Some} \; G}}{\sltm{proper} \; E & \seqsl{G \, E}}}
\newcommand{\rlbmatch}{\inferH[\rlnmbmatch{}]{\bchsl{\atom{A}}{A}}{}}
\newcommand{\rlbanda}{\inferH[\rlnmbanda{}]{\bchsl{D_1 \, \& \, D_2}{A}}{\bchsl{D_1}{A}}}
\newcommand{\rlbandb}{\inferH[\rlnmbandb{}]{\bchsl{D_1 \, \& \, D_2}{A}}{\bchsl{D_2}{A}}}
\newcommand{\rlbimp}{\inferH[\rlnmbimp{}]{\bchsl{G \longrightarrow D}{A}}{\seqsl{G} & \bchsl{D}{A}}}
\newcommand{\rlball}{\inferH[\rlnmball{}]{\bchsl{\sltm{All} \; D}{A}}{\sltm{proper} \; E & \bchsl{D \, E}{A}}}
\newcommand{\rlballs}{\inferH[\rlnmballs{}]{\bchsl{\sltm{Allx} \; D}{A}}{\bchsl{D \, E}{A}}}
\newcommand{\rlbsome}{\inferH[\rlnmbsome{}]{\bchsl{\sltm{Some} \; D}{A}}{\forall (E : \hybridtm{expr con}), (\sltm{proper} \; E \rightarrow \bchsl{D \, x}{A})}}


%%%%%%%%%%%%%%%%%%%%%%%%%
% FORMATTING COQ
%%%%%%%%%%%%%%%%%%%%%%%%%

\lstdefinelanguage{Coq}
{
  % list of keywords
  morekeywords={
    fun,
    forall,
    exists,
    Set,
    Prop,
    Type,
    Definition,
    Inductive,
    Fixpoint,
    Theorem,
    Lemma,
    Example,
    Proof,
    Qed
  },
  emph={
    %hodb,
    %hAbs, hApp, dAbs, dApp, dVar,
    s_bc, s_init, s_imp, s_all, s_alls, s_some,
    b_match, b_and1, b_and2, b_imp, b_all, b_alls, b_some,
    APP, CON, ABS, VAR, lambda, abstr
    },
  %emphstyle={\color{Brightpurple}},
  sensitive=true, % keywords are case-sensitive
  morecomment=[l]{//}, % l is for line comment
  morecomment=[s]{/*}{*/}, % s is for start and end delimiter
  morestring=[b]" % defines that strings are enclosed in double quotes
}
 
% Set Language
\lstset{
  language={Coq},
  basicstyle=\small\ttfamily, % Global Code Style
  captionpos=b, % Position of the Caption (t for top, b for bottom)
  extendedchars=true, % Allows 256 instead of 128 ASCII characters
  tabsize=2, % number of spaces indented when discovering a tab 
  columns=fixed, % make all characters equal width
  keepspaces=true, % does not ignore spaces to fit width, convert tabs to spaces
  showstringspaces=false, % lets spaces in strings appear as real spaces
  breaklines=true, % wrap lines if they don't fit
  %frame=trbl, % draw a frame at the top, right, left and bottom of the listing
  %frameround=tttt, % make the frame round at all four corners
  framesep=4pt, % quarter circle size of the round corners
  %numbers=left, % show line numbers at the left
  %numberstyle=\tiny\ttfamily, % style of the line numbers
  %commentstyle=\color{eclipseGreen}, % style of comments
  %keywordstyle=\color{Forestgreenlt}, % style of keywords
  %stringstyle=\color{eclipseBlue}, % style of strings
}


%%%%%%%%%%%%%%%%%%%%%%%%%
% FORMATTING, OTHER
%%%%%%%%%%%%%%%%%%%%%%%%%

\newcommand{\type}[1]{\texttt{Type\ensuremath{_{#1}}}}
\newcommand{\N}{\ensuremath{\mathbb{N}}}
\newcommand{\Z}{\ensuremath{\mathbb{Z}}}
\newcommand{\cic}{\textsc{Cic}}
\newcommand{\coc}{\textsc{CoC}}
\newcommand{\reduce}{\; \triangleright_{\beta\delta\iota} \;}
%\newcommand{\seq}[2][\Gamma]{\ensuremath{#1 \vdash #2}}
\newcommand{\subtype}{\leq_{\beta\delta\iota}}

\newcommand{\pfshift}{-5pt}

\newcommand{\ol}[1]{\overline{#1}}

\newcommand{\args}[1]{(\hspace{-0.35em} \langle #1 \rangle \hspace{-0.35em})}

\newtheorem{theorem}{Theorem}
\newtheorem*{thm}{Theorem}
\newtheorem{lem}{Lemma}
\newtheorem{corollary}{Corollary}

%%%%%%%%%%%%%%%%%%%%%%%%%%%%%%%%%%%%%%%%%%%%%%%%%%%%%%%%%%%%%%%%%%%%%%
% Table of contents, list of figures, list of tables, and list of
% symbols.
% If your thesis has section numbers larger than 9, many tables or
% many figures, a lot of pages, ... then some of the items on a line
% of the table of contents, the list of tables, the list of figures or
% the list of symbols may overlap.  The following commands should
% solve these problems.
%%%%%%%%%%%%%%%%%%%%%%%%%%%%%%%%%%%%%%%%%%%%%%%%%%%%%%%%%%%%%%%%%%%%%%
%\addSpaceContents{1em}         % add space for the section number
%\addSpaceFigure{1em}           % add space for the figure number
%\addSpaceTable{1em}            % add space for the table number
%\addSpacePageNumber{1em}       % add space for the page number
%\addSpaceRightMargin{1em}      % add space between the title, description,
                                % ... and the right margin
%\addSpaceLabelSymbols{1em}     % add space for the label
%\addSpacePageSymbols{1em}      % add space for the page number

%%%%%%%%%%%%%%%%%%%%%%%%%%%%%%%%%%%%%%%%%%%%%%%%%%%%%%%%%%%%%%%%%%%%%%
% If you want to allow page breaks in a multi-line equation,
% comment out the following line.  You may also use this option
% locally as follows.
% { \allowdisplaybreaks  multi lines equation  }
% You may also force a page break in a multi-line equation by adding
% \displaybreak  after a \\ (i.e. a newline) in your equation
%%%%%%%%%%%%%%%%%%%%%%%%%%%%%%%%%%%%%%%%%%%%%%%%%%%%%%%%%%%%%%%%%%%%%%
\allowdisplaybreaks

%%%%%%%%%%%%%%%%%%%%%%%%%%%%%%%%%%%%%%%%%%%%%%%%%%%%%%%%%%%%%%%%%%%%%%
% To save some trees when writing the thesis, it is possible
% to print copies of the thesis that do not include the page with the
% logo, the title page, and the page with the signature.  Comment out
% the appropriate line below.
%
% For the very final version of the thesis that you will submit to
% the library of the University of Ottawa, you MUST uncomment the
% \NoLogo command.  There must not be any logo on the very final
% version.
%
% If the option UOdraft has been selected, none of these pages are
% printed but the title page.  The thesis is als printed two sided.
%
% There is no page with the signatures of the supervisor(s) and
% examiners at the Univ. of Ottawa.  So \NoSignature must stay
% uncommented.
%%%%%%%%%%%%%%%%%%%%%%%%%%%%%%%%%%%%%%%%%%%%%%%%%%%%%%%%%%%%%%%%%%%%%%
%\NoLogo
\NoSignatures
%\NoTitlepage

%%%%%%%%%%%%%%%%%%%%%%%%%%%%%%%%%%%%%%%%%%%%%%%%%%%%%%%%%%%%%%%%%%%%%%
% This is the command to define the style of the bibliography if
% BibTeX is used.  The available style in the standard TeX/LaTeX
% installation are  plain, unsrt, alpha and abbrv
% plain: The items in the bibliography are in alphabetical order, and
%        refered to with a number.
% unsrt: The items in the bibliography are ordered according to their
%        first appearance in the text, and refered to with a number.
% alpha: The items in the bibliography are in alphabetical order, and
%        refered to with an abreviation of the author name plus the
%        year.
% abbrv: The items in the bibliography are in alphabetical order, and
%        refered to with a number.  Moreover, the items in the
%        bibliography are listed using an abreviation of first names,
%        months and journal names.
%
% The default is plain.
%%%%%%%%%%%%%%%%%%%%%%%%%%%%%%%%%%%%%%%%%%%%%%%%%%%%%%%%%%%%%%%%%%%%%%
\bibliographystyle{plain}

%%%%%%%%%%%%%%%%%%%%%%%%%%%%%%%%%%%%%%%%%%%%%%%%%%%%%%%%%%%%%%%%%%%%%%
% Some of the variables that you may or should define.  Comment out
% those variable that you want to define.  Most of these variables
% have a default value that is appropriate in most cases.
%%%%%%%%%%%%%%%%%%%%%%%%%%%%%%%%%%%%%%%%%%%%%%%%%%%%%%%%%%%%%%%%%%%%%%
\setUOname{Chelsea Battell}         % Name of the candidat
\setUOcpryear{2016}               % Year of graduation
\setUOtitle{The Logic of Hereditary Harrop Formulas as a Specification Logic for Hybrid}  % Title of the thesis

% The following variables have default values which are probably
% right for all students in mathematics and statistics at UO.
%\setUO{ }                    % Name of the university
%\setUOgradfac{ }             % Name of the faculty of graduate studies
%\setUOprogrfac{ }            % Name of the faculty giving the program
%\setUOdept{ }                % Name of the department giving the program
%\setUOfdegree{ }             % Full name of the degree (e.g. Doctorate
                              % in ...)  See \msc and \phd below.
%\setUOdegree{ }              % Abbreviation for the name of the degree
                              % (e.g. Ph.D. )  See \msc and \phd below.
%\setUOtitlefootnote{ }       % Abbreviation for the name of the degree
                              % for the footnote about the institute.
                              % See \msc and \phd below.

% The following 13 variables are NO USED at UO
%\setUOpdegree{ }             % Abbreviations of the degrees already
                              % owned by the candidate (e.g. B.Sc.,
                              % M.Sc., ...)
% \setUOdatedef{11/11/2015}           % Date of the defence of the thesis
% \setUOrefone{Bugs Bunny}            % First internal referee
% \setUOreftwo{Daffy Duck}            % Second internal referee
% \setUOrefthree{Elmer Fudd}          % Third internal referee
% \setUOreffour{ }                    % Fourth internal referee
% \setUOreffive{ }                    % Fifth internal referee
% \setUOextrefone{Marvin the Martian} % First external referee
% \setUOextreftwo{Yosemite Sam}       % Second external referee
% \setUOextrefthree{ }                % Third external referee
% \setUOsupone{Capitain Haddock}      % First supervisor
% \setUOsuptwo{Professeur Tournesol}  % Second supervisor
% \setUOsupthree{ }                   % Third supervisor

%%%%%%%%%%%%%%%%%%%%%%%%%%%%%%%%%%%%%%%%%%%%%%%%%%%%%%%%%%%%%%%%%%%%%%
% Comment out one of the following option
% This automatically defined \setUOfdegree{ }, \setUOdegree{ } and
% \setUOtitlefootnote{ } appropriately.  Thus, \setUOfdegree{ },
% \setUOdegree{ } and \setUOtitlefootnote{ } are only used for fine
% tuning when the default values are not appropriate.
%%%%%%%%%%%%%%%%%%%%%%%%%%%%%%%%%%%%%%%%%%%%%%%%%%%%%%%%%%%%%%%%%%%%%%
%\phd                % Comment out for the Ph.D. thesis
\msc               % Comment out for the M.Sc. thesis
%\project           % Comment out for the M.Sc. project
%\uproject          % Comment out for the B.Sc. project

%%%%%%%%%%%%%%%%%%%%%%%%%%%%%%%%%%%%%%%%%%%%%%%%%%%%%%%%%%%%%%%%%%%%%%
% Location of the logo of the university for the first page of the
% thesis, as well as the options for \includegraphics to produce the
% logo.  The default logo is given by the images  UOlogoBW.eps  or
% UOlogoBW.jpg  in the current directory.
%%%%%%%%%%%%%%%%%%%%%%%%%%%%%%%%%%%%%%%%%%%%%%%%%%%%%%%%%%%%%%%%%%%%%%
% \setUOlogoloc{UOlogoBW}{width=1.2in}

%%%%%%%%%%%%%%%%%%%%%%%%%%%%%%%%%%%%%%%%%%%%%%%%%%%%%%%%%%%%%%%%%%%%%%
% Uncomment the following two lines to switch to 3/2 interline spacing
% or another interline spacing, use the following command.  The
% default is single line spacing.  Ask you supervisor if he/she wants
% a larger interline spacing to write their comments.
%%%%%%%%%%%%%%%%%%%%%%%%%%%%%%%%%%%%%%%%%%%%%%%%%%%%%%%%%%%%%%%%%%%%%%
% \renewcommand{\baselinestretch}{1.5}

%%%%%%%%%%%%%%%%%%%%%%%%%%%%%%%%%%%%%%%%%%%%%%%%%%%%%%%%%%%%%%%%%%%%%%
%                   GENERAL INFORMATION
%
% Always use  \cleardoublepage  at the end of a chapter, appendix,
% , ...  This set up the next page on a odd numbered page if the
% document is in twoside mode.
% Use   \clearpage   and    \newpage   elsewhere.
%
%  The following environment are already defined:
%   theo    <-- Theorem
%   lem     <-- Lemma
%   defn    <-- Definition
%   cor     <-- Corollary
%   prop    <-- Proposition
%
% To add your own environment and have them properly numbered, use
% the command
% \newtheorem{ label }[theo]{ name of the environment }
%
% There is an environment "proof".  There is also a command \qed
% if you decide to write a proof outside the environment "proof".
%
% The commands \caption and \footnote have been redefined to use only
% a single line spacing as required.
%
% The tabular and array environment MUST also use single line
% spacing.  For this purpose, use the following environment.
%
% \begin{singlespace}
%  your table, array, ...
% \end{singlespace}
%
% You may (and probably should) use the following command for floating
% figures and tables.
%
% \figcap[ ]{ \includegraphics{...} or other figure commands }{Full text
% for the caption}{Short description for the list of figures at the
% beginning of the thesis}{label_name for referencing with \ref{} }
%
% \tabcap[ ]{ \begin{tabular}{ccc} ... \end{tabular} }{Full text for the
% caption}{Short description for the list of tables at the beginning
% of the thesis}{label_name for referencing with \ref{} }
%
% The [ ] is optional.  The possible values are:
% [h]  <- Put the figure or table here in the text.
% [t]  <- Put the figure or table at the top of the page
% [b]  <- Put the figure or table at the bottom of the page
% [p]  <- Put the figure or table on a special page with only figures
%         or tables.
% You may also combine them; for instance [tb] tells LaTeX to try to
% put the table on the top of the page and if this fail on the bottom
% of the page.  LaTeX is not forced to respect your choice.
% Some implementations of LaTeX have the additional option H to force
% LaTeX to put the figure or table where you want.  The standard LaTeX
% distribution does not have it.
%
% There are also two new commands:
%
% \nonumchapter{chapter_name}
% to produce the header of a chapter without a number;
% the chapter is still listed in the contents.
%
% \notachapter{chapter_name}
% to produce the header of a chapter without a number;
% the chapter is not listed in the contents.
%
% See the files included with the sample for examples.
%%%%%%%%%%%%%%%%%%%%%%%%%%%%%%%%%%%%%%%%%%%%%%%%%%%%%%%%%%%%%%%%%%%%%%

%%%%%%%%%%%%%%%%%%%%%%%%%%%%%%%%%%%%%%%%%%%%%%%%%%%%%%%%%%%%%%%%%%%%%%
% Your own macros, environments, ...
% Be careful not to over write the macros and environments of the
% University style.
%%%%%%%%%%%%%%%%%%%%%%%%%%%%%%%%%%%%%%%%%%%%%%%%%%%%%%%%%%%%%%%%%%%%%%

{
  \newtheoremstyle{remarkstyle}{\topsep}{\topsep}{\rm}{}{\bfseries}{.}{.5em}{}
  \theoremstyle{remarkstyle}
  \newtheorem{rmk}[theo]{Remark}
  \newtheorem{egg}[theo]{Example}
}

\newcommand{\RR}{\mathbb{R}}
\newcommand{\NN}{\mathbb{N}}
\newcommand{\OO}{\mathcal{O}}

\DeclareMathOperator{\Per}{Per}
\DeclareMathOperator{\IM}{Im}
\DeclareMathOperator{\KER}{Ker}
\DeclareMathOperator{\Fix}{Fix}

\begin{document}
%%%%%%%%%%%%%%%%%%%%%%%%%%%%%%%%%%%%%%%%%%%%%%%%%%%%%%%%%%%%%%%%%%%%%%
% We suggest to include your chapters, ... as illustrated below      %
% and leave this file with as Little of the contents of the thesis   %
% as it is possible.                                                 %
%%%%%%%%%%%%%%%%%%%%%%%%%%%%%%%%%%%%%%%%%%%%%%%%%%%%%%%%%%%%%%%%%%%%%%

%%%%%%%%%%%%%%%%%%%%%%%%%%%%%%%%%%%%%%%%%%%%%%%%%%%%%%%%%%%%%%%%%%%%%%
% Include the abstract of your thesis here.
%
% The abstract for a M.Sc. must have AT MOST 150 WORDS.
% The abstract for a Ph.D. must have AT MOST 350 WORDS.
%
%%%%%%%%%%%%%%%%%%%%%%%%%%%%%%%%%%%%%%%%%%%%%%%%%%%%%%%%%%%%%%%%%%%%%%
\notachapter{Abstract}

Hybrid is a two-level logical framework that supports higher-order abstract syntax (HOAS), where a specification logic (SL) extends the class of object logics (OLs) we can reason about. We develop a new Hybrid SL and formalize its metatheory, proving weakening, contraction, exchange, and cut admissibility; results that greatly simplify reasoning about OLs in systems providing HOAS. The SL is a sequent calculus defined as an inductive type in Coq and we prove properties by structural induction over SL sequents. We also present a generalized SL and metatheory statement, allowing us to prove many cases of such theorems in a general way and understand how to identify and prove the difficult cases. We make a concrete and measurable improvement to Hybrid with the new SL formalization and provide a technique for abstracting such proofs, leading to a condensed presentation, greater understanding, and a generalization that may be instantiated to other logics.


%1. Introduction. In one sentence, what's the topic? 47
%Hybrid is a two-level logical framework supporting higher-order abstract syntax (HOAS), where a specification logic (SL) extends the class of object logics (OLs) that we can reason about.
%Hybrid is a logical framework supporting higher-order abstract syntax (HOAS) in representing formal systems or ``object logics'' (OLs) and implemented following a two-level approach, where a specification logic (SL) extends the class of OLs that we can reason about.
%2. State the problem you tackle. 34
%We develop a new Hybrid SL and formalize its metatheory, proving weakening, contraction, exchange, and cut admissibility; results that greatly simplify reasoning about OLs in systems providing HOAS.
%3. Summarize (in one sentence) why nobody else has adequately answered the research question yet.
%Hybrid is the only known two-level logical framework supporting HOAS implemented in an existing general-purpose theorem prover.
%...so although related systems....?
%5. In one sentence, how did you go about doing the research that follows from your big idea?
%The SL is a sequent calculus defined as an inductive type in Coq and we prove properties by structural induction over SL sequents.
%4. Explain, in one sentence, how you tackled the research question.
%We also present a generalized SL and metatheory statement, allowing us to prove many cases of such theorems in a general way and understand how to identify and prove the difficult cases.
%6. As a single sentence, what's the key impact of your research?
%We make a concrete and measurable improvement to Hybrid with the new SL formalization and provide a technique for abstracting such proofs, leading to a condensed presentation, greater understanding, and a generalization that may be instantiated to other logics.

% Preface:
%Hybrid is a logical framework that supports the use of higher-order abstract syntax (HOAS) in representing formal systems or ``object logics'' (OLs). It is implemented in Coq and follows a two-level approach, where a specification logic (SL) is implemented as an inductive type and used to concisely and elegantly encode the inference rules of the formal systems of interest. In this thesis, we develop a new higher-order specification logic based on hereditary Harrop formulas for Hybrid. By increasing the expressive power of the SL beyond what was considered previously, we increase the flexibility of encoding OLs and thus extend the class of formal systems for which we can reason about efficiently. We focus on formalizing an important class of metatheorems for the SL. This class includes properties such as weakening, contraction, exchange, and the admissibility of the cut rule. The cut admissibility theorem establishes consistency and also provides justification for substituting a formula for an assumption in a context of assumptions. It can greatly simplify reasoning about OLs in systems that provide HOAS. The proofs of these theorems are discussed in detail, illustrating the challenges of their formalization. We develop an abstract way in which to present the specification logic and its metatheory and show how it can be used to prove the theorems mentioned above.


% Paper abstract:
%Hybrid is a logical framework that supports the use of higher-order abstract syntax (HOAS) in representing formal systems or ``object logics'' (OLs). It is implemented in Coq and follows a two-level approach, where a specification logic (SL) is implemented as an inductive type and used to concisely and elegantly encode the inference rules of the formal systems of interest. In this paper, we develop a new higher-order specification logic for Hybrid. By increasing the expressive power of the SL beyond what was considered previously, we increase the flexibility of encoding OLs and thus extend the class of formal systems for which we can reason about efficiently. We focus on formalizing the meta-theory of the SL. We develop an abstract way in which to present an important class of meta-theorems. This class includes properties such as weakening, contraction, exchange, and the admissibility of the cut rule. The cut admissibility theorem establishes consistency and also provides justification for substituting a formula for an assumption in a context of assumptions. It can greatly simplify reasoning about OLs in systems that provide HOAS. We discuss the proof of this theorem in detail, illustrating the challenges of its formalization. We present the abstraction and show how it is used to prove all of these theorems.

\cleardoublepage

%%%%%%%%%%%%%%%%%%%%%%%%%%%%%%%%%%%%%%%%%%%%%%%%%%%%%%%%%%%%%%%%%%%%%%
% You may include dedications here.
%%%%%%%%%%%%%%%%%%%%%%%%%%%%%%%%%%%%%%%%%%%%%%%%%%%%%%%%%%%%%%%%%%%%%%
\notachapter{Dedications}   % d\'edicace

To Amy Felty, my advisor and role model.

%%%%%%%%%%%%%%%%%%%%%%%%%%%%%%%%%%%%%%%%%%%%%%%%%%%%%%%%%%%%%%%%%%%%%%
% The command at the end of a chapter.  Don't remove
%%%%%%%%%%%%%%%%%%%%%%%%%%%%%%%%%%%%%%%%%%%%%%%%%%%%%%%%%%%%%%%%%%%%%%
\cleardoublepage

%%%%%%%%%%%%%%%%%%%%%%%%%%%%%%%%%%%%%%%%%%%%%%%%%%%%%%%%%%%%%%%%%%%%%%
% You may include you acknowledgement here.
%%%%%%%%%%%%%%%%%%%%%%%%%%%%%%%%%%%%%%%%%%%%%%%%%%%%%%%%%%%%%%%%%%%%%%
\notachapter{Acknowledgement}   % Remerciments

Thanks and acknowledgment go to the Natural Sciences and Engineering Research Council (NSERC) of Canada, Professor Amy Felty, and the University of Ottawa for the financial support provided while completing this research.

I am extremely grateful for the guidance provided by Professor Felty to help me through the many challenges faced while completing this degree and for her feedback and assistance in editing all of the presentations and written documents related to this research.

I would also like to thank Alberto Momigliano for the initial encoding of the data structures for the specification logic and discussions through the course of this work.

%%%%%%%%%%%%%%%%%%%%%%%%%%%%%%%%%%%%%%%%%%%%%%%%%%%%%%%%%%%%%%%%%%%%%%
% The command at the end of a chapter.  Don't remove
%%%%%%%%%%%%%%%%%%%%%%%%%%%%%%%%%%%%%%%%%%%%%%%%%%%%%%%%%%%%%%%%%%%%%%
\cleardoublepage

%%%%%%%%%%%%%%%%%%%%%%%%%%%%%%%%%%%%%%%%%%%%%%%%%%%%%%%%%%%%%%%%%%%%%%
% The table of contents
%%%%%%%%%%%%%%%%%%%%%%%%%%%%%%%%%%%%%%%%%%%%%%%%%%%%%%%%%%%%%%%%%%%%%%
\tableofcontents
\cleardoublepage

%%%%%%%%%%%%%%%%%%%%%%%%%%%%%%%%%%%%%%%%%%%%%%%%%%%%%%%%%%%%%%%%%%%%%%
% The list of figures
%%%%%%%%%%%%%%%%%%%%%%%%%%%%%%%%%%%%%%%%%%%%%%%%%%%%%%%%%%%%%%%%%%%%%%
\ListOfFigures
\cleardoublepage

%%%%%%%%%%%%%%%%%%%%%%%%%%%%%%%%%%%%%%%%%%%%%%%%%%%%%%%%%%%%%%%%%%%%%%
% The list of theorems
%%%%%%%%%%%%%%%%%%%%%%%%%%%%%%%%%%%%%%%%%%%%%%%%%%%%%%%%%%%%%%%%%%%%%%
\listoftheorems
\cleardoublepage

%%%%%%%%%%%%%%%%%%%%%%%%%%%%%%%%%%%%%%%%%%%%%%%%%%%%%%%%%%%%%%%%%%%%%%
% The list of tables
%%%%%%%%%%%%%%%%%%%%%%%%%%%%%%%%%%%%%%%%%%%%%%%%%%%%%%%%%%%%%%%%%%%%%%
%\ListOfTables
%\cleardoublepage

%%%%%%%%%%%%%%%%%%%%%%%%%%%%%%%%%%%%%%%%%%%%%%%%%%%%%%%%%%%%%%%%%%%%%%
% The list of symbols.  Some peoples prefer to have the list of
% symbols at the end of the thesis before the index.  You may do this
% by moving the following two lines before the index section at the
% end of the thesis.
%
% To define a symbol in the text, one uses the command \nomenclature
% as in the following examples.
% \nomenclature{$\alpha$}{the travelling speed of a wave}
% \nomenclature{$\Gamma$}{the Gamma function}
% \nomenclature{$\delta_{i,j}$}{The Kronecker's delta function}
%
% As we said in the preamble above, with the option NoTofC, the
% section for the list of symbols isn't included in the table of
% contents.  The default is that it is included in the table of
% contents.
%
% If you don't want to have a list of symbols, comment out the
% following two lines and don't run
% makeindex template.nlo -s nomencl.ist -o template.nls
%%%%%%%%%%%%%%%%%%%%%%%%%%%%%%%%%%%%%%%%%%%%%%%%%%%%%%%%%%%%%%%%%%%%%%
\PrintNomenclature
\cleardoublepage

%%%%%%%%%%%%%%%%%%%%%%%%%%%%%%%%%%%%%%%%%%%%%%%%%%%%%%%%%%%%%%%%%%%%%%
% For the following chapter, the pages will be numbered in roman
% numbers as the previous pages.  The size of this chapter should be
% very small.  If you have more then two pages, equations, and so on,
% you should instead have a normal chapter. 
%
% Note that the numbering of equations in a chapter with no chapter
% number will use the chapter number of the previous chapter.  This is
% certainly not what you want.
%%%%%%%%%%%%%%%%%%%%%%%%%%%%%%%%%%%%%%%%%%%%%%%%%%%%%%%%%%%%%%%%%%%%%%
%\nonumchapter{Preface}

%A really brief introduction to motivate your thesis.  Useful if you feel that the word limitation of the abstract didn't give you enough space to fully justify your thesis.  If you are going to have displayed equations with equation number, definitions, and so on, you should have a normal first chapter.

%%%%%%%%%%%%%%%%%%%%%%%%%%%%%%%%%%%%%%%%%%%%%%%%%%%%%%%%%%%%%%%%%%%%%%
% The command at the end of a chapter.  Don't remove
%%%%%%%%%%%%%%%%%%%%%%%%%%%%%%%%%%%%%%%%%%%%%%%%%%%%%%%%%%%%%%%%%%%%%%
%\cleardoublepage

%%%%%%%%%%%%%%%%%%%%%%%%%%%%%%%%%%%%%%%%%%%%%%%%%%%%%%%%%%%%%%%%%%%%%%
% From now on, you can let your genius out.
% Don't remove the next line
%%%%%%%%%%%%%%%%%%%%%%%%%%%%%%%%%%%%%%%%%%%%%%%%%%%%%%%%%%%%%%%%%%%%%%
\pagenumbering{arabic}

%\section{Introduction}

\emph{Logical frameworks} provide general languages in which it is possible to represent a wide variety of logics, programming languages,
and other formal systems.  They are designed to capture uniformities of the syntax and inference systems of these \emph{object logics} (OLs) and to provide support for implementing and reasoning about them.  Hybrid \cite{FeltyMomigliano:JAR10} is a logical framework that provides support for encoding OLs via \emph{higher-order abstract   syntax} (HOAS), also referred to as \emph{lambda-tree syntax}.
%\cite{MillerP99}.
Using HOAS, binding constructs in the OL are encoded using the binding constructs provided by an underlying $\lambda$-calculus or function space of the logical framework (the \emph{meta-language}).  Using such a representation allows us to delegate to the meta-language $\alpha$-conversion and capture-avoiding substitution.  Further, object logic substitution can be rendered as meta-level $\beta$-conversion.  HOAS encodings aim to relieve users from having to build up common (and often large) infrastructure implementing operations dealing with variables, such as capture-avoiding substitution, renaming, and fresh name generation.  In addition, in such logical frameworks, embedded implication and universal quantification are often used to represent \emph{hypothetical} and \emph{parametric} judgments, also called \emph{generic} judgments, %\cite{MillerTiu:TOCL05},
which allow elegant and succinct specifications of OL inference rules.

An important goal of Hybrid is to exploit the advantages of HOAS within the well-understood setting of higher-order logic as implemented by systems such as Isabelle and Coq.\footnote{Although Hybrid has been implemented in both Coq and Isabelle/HOL, we use the Coq version in this paper.}  Building on such a system allows us to easily experiment with new specification logics.  It also provides a high degree of trust; for instance proof terms in Coq serve as proof certificates, which can be independently checked.  In addition, Hybrid in Coq inherits Coq's full recursive function space as well as its extensive set of libraries.

Hybrid is implemented as a two-level system, an approach first introduced in the $\foldn$ logic \cite{McDowellMiller:TOCL01}, and now applied within a variety of logics and systems, such as the Abella interactive theorem prover \cite{Gacek:IJCAR08}.  In a two-level system, the \emph{specification} and (inductive) \emph{meta-reasoning} are done within a single system but at different \emph{levels}. An intermediate level is introduced by inductively defining a \emph{specification logic} (SL) in Coq, and OL judgments (including hypothetical and parametric judgments) are encoded in the SL.  Several meta-theoretic properties about the SL provide powerful tools for reasoning about OLs.  For example, the cut admissibility theorem provides a direct and convenient way to substitute a formula for an assumption in a context of assumptions.  Structural properties of the SL, such as weakening, contraction, and exchange, also provide tools that can be directly applied to reasoning in any OL.

%In this paper, we introduce an intuitionistic higher-order SL, namely
%the logic of higher-order hereditary Harrop formulas (HoHH)
%\cite{LProlog} with some restrictions, mainly on the types of terms
%in quantified formulas.  Previous SLs considered for Hybrid include a
%second-order fragment of HoHH and an ordered linear logic
%\cite{FeltyMomigliano:JAR10}.
%
In this paper, we introduce an intuitionistic higher-order SL, namely the logic of hereditary Harrop formulas (HH).  HH is a sublogic of the logic of higher-order hereditary Harrop formulas as presented in \cite{LProlog}.  Two kinds of ``order'' can be seen in this logic, the domain of quantification and the implicational complexity.  In terms of the former, HH allows quantification over second-order types, and in terms of the latter, HH is higher-order, allowing any level of nested implications.  Previous SLs considered for Hybrid include the fragment of HH with second-order implicational complexity and an ordered linear logic \cite{FeltyMomigliano:JAR10}.

We adopt a minor variation of the inference rules for HH used as an SL in recent versions of the Abella interactive theorem prover \cite{WCGN:PPDP13}.  We present our encoding in Coq, and discuss the proofs of meta-theoretic properties in some detail.  In particular, we develop an abstraction to capture uniformities across proofs of different meta-theoretic properties.  Cut admissibility in particular relies on a fairly complex inductive argument, involving mutual inductions and sub-inductions.  Our proof follows the tradition of many other syntactic proofs of cut admissibility for various logics that first induct on the formula depth and then on the proof structure, e.g. \cite{Girard89}.  Furthermore, our proof is \emph{structural} in the sense of \cite{Pfenning:IC00}, in our case using structural induction principles generated by Coq. %We discuss different strategies and even false starts, and then
We present the proofs via our abstraction, with the goal of providing a deeper insight into the proofs and the formalization process.  Variants of the properties we prove have also been proved in Abella. They are mentioned in \cite{WCGN:PPDP13}, but proofs are not presented there.  We briefly discuss some differences.

Our overall goal is to extend the reasoning power of Hybrid. Implementing HH as a new SL in Hybrid now allows us to directly encode, for example, the two OLs in the case studies considered in \cite{WCGN:PPDP13}.  The first involves reasoning about the correspondence between an HOAS encoding and a de Bruijn representation of the terms of the untyped $\lambda$-calculus, while the second involves reasoning about a structural characterization of reductions on untyped $\lambda$-terms, and is originally posed in \cite{LProlog}. Other examples we intend to study include the elegant algorithmic specification of bounded subtype polymorphism in System F in~\cite{Pientka:TPHOLs07}, which comes from the \textsl{PoplMark} challenge~\cite{Aydemir05TPHOLs}, as well as specifications of continuation-passing transformations in functional programs.  The specification of the main judgments of all of these OLs will benefit from the availability of embedded implication in HH (in particular, using two or three levels).

We also note that in addition to the advantages mentioned above with regard to implementing Hybrid inside a well-established theorem prover, Hybrid also provides an ideal setting in which to quickly prototype and experiment with new SLs.  Each one is developed as a library and a user can choose and import one that is best-suited to the task at hand and/or move between them easily.  For example, case studies that don't require the expressiveness of HH can use the second-order fragment, likely leading to simpler proofs.  Case studies that are better suited to a linear logic can directly import and use a linear SL, etc.  In contrast, in Abella, a slight extension of HH replaced the SL used in earlier versions of the system.  Fixing the SL allows developers to focus more on adding powerful automation for a particular SL, and thus proofs in Hybrid currently require more interaction.

In Section~\ref{sec:hybrid}, we give a brief introduction of Hybrid. In Section~\ref{sec:sl}, we introduce HH as an example specification layer and describe its implementation in Coq. Highlights of the the mutual structural induction used in later proofs is found in Section~\ref{sec:induction} followed by the presentation of a generalized SL in Section~\ref{sec:gsl} and a proof technique using this generalized SL in Section~\ref{sec:pfgsl}. Section~\ref{sec:structrules} outlines proofs of the structural rules of HH, while Section~\ref{sec:cutadmiss} details the proof of cut admissibility. Finally, Section~\ref{sec:concl} concludes and discusses related and future work. The files of the Coq formalization are available at \url{www.eecs.uottawa.ca/~afelty/lfmtp16/}.

%\cleardoublepage

%\include{discrete_dyn}
%\cleardoublepage

%\include{qualitative}
%\cleardoublepage

%\include{bifurcation}
%\cleardoublepage

%%%%%%%%%%%%%%%%%%%%%%%%%%%%%%%%%%%%%%%%%%%%%%%%%%
% OVERVIEW
%%%%%%%%%%%%%%%%%%%%%%%%%%%%%%%%%%%%%%%%%%%%%%%%%%

\chapter{Introduction}

The goal of this research is to increase the reasoning abilities of an existing system that is intended to help mechanize programming language metatheory. The system that this work contributes to is called Hybrid~\cite{FeltyMomigliano:JAR10}\index{Hybrid} and is part of the research program carried out by the Software Correctness and Safety Research Laboratory at the University of Ottawa under the supervision of Professor Amy Felty. Hybrid is implemented both in Coq~\cite{coq}\index{Coq} and Isabelle/HOL~\cite{NPW:2002}, interactive proof assistants used for applications such as formalizing mathematics, certifying compilers, and proving correctness of programs. Our application of Coq is proving metatheory of formal systems efficiently. The contributions to Hybrid described in this thesis are to the Coq implementation.

\section{Mechanized Reasoning}

Proof is essential in modern mathematics and in logic in particular. We trust and build on the work of others when we can see that they have presented a rigorous argument supporting their work. When writing proofs with many cases or details to manage, it is easy to make errors and these proofs are tedious to check. So we must trust the proof writer (and all proofs that their work builds on) or else spend an exorbitant amount of time checking proofs (and still possibly miss errors). Proof assistants such as Coq provide proof terms that can be independently checked. To trust all proof terms of theorems proven in Coq, one only needs to trust the underlying proof theory and the implementation of the system checking the proof.

Manually checking ``paper and pencil'' (non-formalized) proofs is not a scalable or practical technique for applications to software development in industry where financial concerns may be prioritized over software correctness and safety. The area of formal methods for software engineering is focused on (im)proving the correctness of software. Even if we develop techniques to allow software developers to prove correctness of software without needing expertise in the research area, we still need to be sure that the languages in which the programs are defined cause the expected behaviour. We need a mechanized solution to studying programming language metatheory.

Toward this goal, the \poplmark{}\index{POPLmark} challenge~\cite{Aydemir05TPHOLs} was introduced to merge the concerns of the proof theory and programming languages communities. It provides a set of challenge problems to explore how to mechanize programming language metatheory using a variety of systems and techniques and illustrates the importance of formalized reasoning in programming language research. In fact, it is standard for papers presented at programming language conferences to be accompanied by formal proofs of the metatheory, again for reasons related to confidence in the correctness of the work.

One approach to mechanize reasoning about programming languages involves higher-order abstract syntax. This technique is used by Hybrid, the system that the work in this thesis contributes to.

%Implementing Hybrid in an existing trusted general-purpose theorem prover makes it possible to easily make modular changes to the system.

\section{Higher-Order Abstract Syntax}
\label{sec:introhoas}

\emph{Higher-order abstract syntax}\index{higher-order abstract syntax} (HOAS)~\cite{PE:PLDI88}, also known as \lambda-tree syntax~\cite{MillerP99}\index{\lambda-tree syntax}, is a technique for representing formal systems, known as \emph{object logics} (OLs)\index{object logic}, that we wish to reason about. A variety of systems exist that implement a HOAS approach to reasoning about OLs such as programming languages and logics. An early example is the Twelf system~\cite{TwelfSP}. HOAS simplifies reasoning about OLs in these systems by allowing object-level name binding structures, also called \emph{binders}\index{binder}, to be encoded in the binding structures of the meta-language that the system is defined in.

As an example, we will illustrate a HOAS encoding of the untyped \lambda-calculus (the OL) in a type theory (the meta-level) so that we can reason about it formally. Function abstraction in \lambda-calculus is a \emph{binder} because the name of a variable is bound in the body of the abstraction. Let $\mathit{tm}$ be the meta-level type of terms of the OL encoding. Suppose that we have constants expressing the higher-order syntax of terms, including $\mathit{app}$ of type $\mathit{tm} \rightarrow \mathit{tm} \rightarrow \mathit{tm}$ and $\mathit{abs}$ of type $(\underline{\mathit{tm}} \rightarrow \mathit{tm}) \rightarrow \mathit{tm}$. Then $\mathit{abs}$ represents the function abstraction construct of the OL, an object-level binder, and it is encoded in a meta-level binder. For example, the \lambda-term $\lambda x . \lambda y . x \; y$ can be encoded as $\mathit{abs} (\lambda x \, . \, (\mathit{abs} (\lambda y \, . \, (\mathit{app} \; x \; y))))$. Note that $\mathit{tm}$ cannot be defined inductively because of the (underlined) negative occurrence of $\mathit{tm}$ in the type of $\mathit{abs}$.

When the metalanguage is an appropriate \lambda-calculus, we can encode object-level substitution and renaming as meta-level \beta-reduction and \alpha-conversion, respectively. This avoids the requirement of implementing infrastructure consisting of libraries of definitions and lemmas to deal with issues surrounding binders and variable naming when studying an OL.

\section{Hybrid}

Coq is an implementation of a typed \lambda-calculus called the calculus of constructions. Hybrid is implemented as a Coq library so the meta-language is a \lambda-calculus. A type $\mathit{expr}$ is defined for representing OL ``programs'' as meta-level terms. In Hybrid this is implemented so that terms of type $\mathit{expr}$ expand to a nameless representation of \lambda-terms called de Bruijn indices~\cite{debruijn}\index{de Bruijn indices}. Hybrid uses HOAS and object-level binders are encoded in a newly introduced abstraction binder. This binder has type $(\mathit{expr} \rightarrow \mathit{expr}) \rightarrow \mathit{expr}$. It can be used to directly express the syntax of OLs such as the untyped \lambda-calculus example in Section~\ref{sec:introhoas}.

An intermediate reasoning layer called a \emph{specification logic} (SL)\index{specification logic} is added to interface between the OL encoding and the layer implementing HOAS. Adding a SL extends the class of OLs that we can reason about efficiently using a system supporting HOAS. The relationship between these levels is the reason Hybrid is considered a \emph{two-level} logical framework\index{two-level logical framework}. This approach was introduced by McDowell and Miller in~\cite{McDowellMiller:TOCL01} with the $\mathit{FO\lambda^{\Delta \mathbb{N}}}$ logic. In such a system, the specification and (inductive) meta-reasoning are done within a single system but at different levels. In Hybrid the SL is defined as an inductive type in Coq, and OL judgments (including hypothetical and parametric judgments) are encoded in the SL.

\section{Specification Logic Metatheory}

There are many features of Hybrid that help work toward the goal of efficiently mechanizing programming language metatheory, but this thesis is focused on the SL layer. Here we makes two contributions: an extension to the reasoning power of Hybrid and new insight into proofs of properties of certain kinds of sequent calculi.

First, a new SL is implemented and structural properties of this logic are formalized and proven in the Coq proof assistant. The new SL presented here is a sequent calculus based on the logic of hereditary Harrop formulas as presented in~\cite{LProlog}. We prove that the standard structural rules of weakening, contraction, exchange and cut are admissible in this logic. In proving admissibility of these rules, we do not have to include them in the logic as axiomatic and we still get the benefits that they provide in reasoning about OLs. The structural rules can then be used in proofs of OL theorems. For example, if the OL is a typed functional programming language, then proving subject reduction for this OL (i.e. that evaluation of expressions preserves typing) requires the cut rule. See~\cite{FeltyMomigliano:JAR10} for a detailed explanation of this example and subject reduction proof. Implementing a new SL and proving the admissibility of structural rules is a concrete extension to an existing computing tool (namely Hybrid) since it improves the reasoning abilities of this system in a fully formalized way.

The second contribution is more theoretical and educational. We present a generalization of the specification logic and form of theorem statement to encapsulate the implemented SL and desired structural rules, respectively. We show how the implemented SL can be instantiated from this generalized SL, so the results presented for the generalized SL can be applied to, and provide guidance on, the proofs of SL metatheory. The structural proofs are by induction, sometimes requiring mutual inductions and nested inductions, so they can have many cases and details to manage. This presentation allows us to see the structural proofs in a more condensed but still comprehensive way. We are also able to gain a deeper understanding of these proofs; the generalized SL helped us to partition cases for the original SL into classes with the same proof structure and isolate the difficult cases. It is our hope that this presentation will give others insight into the kind of proofs we work through and that this general framework may find some use in other applications.

\section{Outline}

This thesis is broken up into two parts: background and contributions. To understand the research described here, it is necessary to first ensure that the reader understands the logical foundations of the ambient reasoning system for this work (namely Coq), the basics of Hybrid, and the development of the style of logic used for the new SL. In Chapter~\ref{ch:coq} we review the type theory implemented by Coq and introduce the reader to using it as a proof assistant. We present an overview of Hybrid in Chapter~\ref{ch:hybrid}. The final background chapter, Chapter~\ref{ch:hh}, is on the logic of higher-order hereditary Harrop formulas and will set the stage for the SL of this thesis. Next we move to the contributions of this research. As stated above, this is focused on a new intermediate reasoning logic for Hybrid. Chapter~\ref{ch:sl} presents this logic and its metatheory is studied in Chapter~\ref{ch:slind}. From here we abstract the specification logic of Chapter~\ref{ch:sl} with a generalized specification logic in Chapter~\ref{ch:gsl} and prove properties of this logic in a general way in Chapter~\ref{ch:gslind}. We conclude in Chapter~\ref{ch:concl} with a review of the results presented and look at related and future work.

The research presented in this thesis is published in~\cite{BF:LFMTP16}. The files of the Coq formalization are available at \url{www.eecs.uottawa.ca/~afelty/BattellThesis/}.

\cleardoublepage

%%%%%%%%%%%%%%%%%%%%%%%%%%%%%%%%%%%%%%%%%%%%%%%%%%
% BACKGROUND
%%%%%%%%%%%%%%%%%%%%%%%%%%%%%%%%%%%%%%%%%%%%%%%%%%

\part{Background}
\label{pt:background}

% HOAS BACKGROUND

%\input{background_hoas}

%%%%%%%%%%%%%%%%%%%%%%%%%%%%%%%%%%%%%%%%%%%%%%%%%%
% COQ BACKGROUND
%%%%%%%%%%%%%%%%%%%%%%%%%%%%%%%%%%%%%%%%%%%%%%%%%%

\chapter{Coq}
\label{ch:coq}

%The \coc{} has the strong normalization property, meaning all terms of \coc{} will be reduced to an irreducible form by any sequence of reductions. Other than the standard \beta-reduction we have \delta-reductions, which replace an identifier with its definition, and \iota-reductions, which handle computations in recursive programs.

\index{Coq}
Coq~\cite{coq,coqart} is an implementation of the Calculus of Inductive Constructions (\cic{})\index{calculus of inductive constructions}, an extension of the Calculus of Constructions (\coc{})~\cite{coc}\index{calculus of constructions}, a typed lambda calculus with dependent types, polymorphism, and type operators. \cic{} extends \coc{} by adding inductive types, which will be explored in Section~\ref{sec:coqinduction}. Originally created by Thierry Coquand, \coc{} and its extensions have spurred the development of a variety of proof assistants and interactive theorem proving systems currently used in the research areas of automated deduction and formal methods for software engineering.


We will explore the Calculus of Constructions in Section~\ref{sec:coctype}, then see how it captures the simply typed $\lambda$-calculus in Section~\ref{sec:stlc}. Next we look at reductions in Section~\ref{sec:reductions}, followed by an exploration of the expressive power of \coc{} in Sections~\ref{sec:deptype} and~\ref{sec:hot}, where we will examine how the rules presented earlier in the chapter allow dependent types, polymorphism, and type operators.
%We will also look at the various reductions in \cic{} to see how terms are evaluated and develop a notion of equivalence of terms, convertibility between types and a subtype relation.
We see how to use Coq as an interactive proof assistant in Section~\ref{sec:coqpf} followed by information on inductive types and writing inductive proofs in Section~\ref{sec:coqinduction}.
The notation and style used to illustrate the concepts follows the presentation in~\cite{coqart} and~\cite{coq}. The discussion is motivated by~\cite{coqart} and~\cite{coc}. %Specific examples drawn from other sources will be attributed where used.

%The base sorts of CIC are \coqtm{Prop} and \coqtm{Set}. \coqtm{Prop} is the type of propositions. If $P :$ \coqtm{Prop}, then $P$ is considered a proposition. Any $p : P$ is considered a proof of $P$. The \coqtm{Prop} branch is used for proving (*elaborate). If $S :$ \coqtm{Set}, then $S$ is considered a function specification. So if $f : S$, then $f$ is a function satisfying specification $S$. Both \coqtm{Set} and \coqtm{Prop} have type $\coqtm{Type}_0$. For any $n \in \mathbb{N}$, $\coqtm{Type}_n$ has type $\coqtm{Type}_{n+1}$ (*more general?). \\


%There are many inference rules of this system, but we will only look at a few important rules presently.

%(*parametric product rule? see recursion theory project)

%The research presented here is built mainly under the \coqtm{Prop} sort (*necessary to say here?). 

\section{The Calculus of Constructions}
\label{sec:coctype}

%We distinguish two special types, \coqtm{Set} and \coqtm{Prop} with \coqtm{Set} : \type{0} and \coqtm{Prop} : \type{0}. This yields an infinite type hierarchy $s \leq \type{0} \leq ... \leq \type{j} \leq \type{j+1} ...$ with $s \in \{ \coqtm{Set} , \coqtm{Prop} \}$ and for all $i : \mathbb{N}$, $\type{i} : \type{i+1}$. $\coqtm{Set}$ and $\coqtm{Prop}$ are the lowest types in the hierarchy that we typically consider to be types of types, so we call them ``base sorts''. Terms of type \coqtm{Prop} are meant to be logical formulas. Terms of type \coqtm{Set} are meant to be data types. For example, we may construct a term $\N \rightarrow \N$, representing the type of functions from natural numbers to natural numbers, which has type $\coqtm{Set}$. We can define a term representing the successor function with type $\N \rightarrow \N$. So $\N \rightarrow \N$ is simultaneously thought of as a term and a type. Hence $\coqtm{Set}$ is a type of a type.

\subsection{Terms}

The terms of \coc{} are defined by the following grammar:

\begin{align*}
t_1, t_2, t_3 \; ::=& \; \mathit{Type}_i \\
&| \; \mathit{Set} \\
&| \; \mathit{Prop} \\
&| \; x_i \\
&| \; t_1 \; t_2 \\
&| \; \lambda x : t_1 . t_2 \\
&| \; \forall x : t_1 , t_2 \\
&| \; \mathit{let} \; x := t_1 : t_2 \; \mathit{in} \; t_3 \\
\end{align*}

Terms of \coc{} include a collection of constants indexed by natural numbers where for $i \in \mathbb{N}$, $\mathit{Type}_i$ denotes the $i$th constant. Together with this collection, the constants $\mathit{Set}$ and $\mathit{Prop}$ are called \emph{sorts}, which can be viewed as types of types. In the grammar, $x_i$ for all $i \in \mathbb{N}$ denotes a countable collection of variables. Application is denoted by juxtaposition of terms. It is a binary operator that associates to the left (e.g. we write $(t_1 \; t_2) \; t_3$ as $t_1 \; t_2 \; t_3$). Terms can also be \lambda-abstractions $\lambda x : t_1 . t_2$, where $x$ is a variable that is considered \emph{bound} in $t_2$, and $t_1$ is considered the type of variable $x$. It is also possible for terms to be universal quantifications where $x$ is again a variable of type $t_1$ bound in $t_2$. If $x$ does not occur in $t_2$, then we can write this quantification as $t_1 \rightarrow t_2$. The final construction in the term grammar above is for terms denoting the definition of variable $x$ to be $t_1$ of type $t_2$ locally bound in $t_3$. We will sometimes use parentheses in the binders for abstractions and quantification, writing $\lambda (x  : t_1) . t_2$ and $\forall (x : t_1) , t_2$ to make it easier for the reader to parse these expressions.

Rules assigning types to terms will be discussed below. In \coc{}, there is no syntactic difference between terms and types. We will use ``term'' and ``type'' interchangeably according to what is most reasonable for the current discussion.

\coc{} can be used both as a theorem proving system and a functional programming language. Its type system allows for a correspondence to be observed between theorem statements and types, and between proofs and terms. This is called the Curry-Howard correspondence~\cite{Howard80} and allows us to view proofs as programs. By the Curry-Howard correspondence, the arrow notation can be understood simultaneously as implication or the function type arrow, depending on what is appropriate for the topic under consideration. It associates to the right, so we will usually write the type $t_1 \rightarrow (t_2 \rightarrow t_3)$ as $t_1 \rightarrow t_2 \rightarrow t_3$.

In the rest of this chapter, we will write $t, T, u,$ or $U$ for terms and types, possibly with subscripts. We write $x$ for variables, also possibly with subscripts.

%, the base sorts \coqtm{Set} and \coqtm{Prop}, and countable collections of variables and constants. They can also be applications of terms and abstractions over terms in the style of typed \lambda-calculus. Finally, terms can be universal quantification over terms. The last construction in the grammar has an alternative notation; for $x : t_1$, if the named variable $x$ does not occur in $t_2$, then we say $t_2$ does not depend on $x$ and we can write the term $\forall (x : t_1), t_2$ as $t_1 \rightarrow t_2$. By the Curry-Howard correspondence the arrow notation can be understood simultaneously as implication or the function type arrow, depending on what is appropriate for the topic under consideration. It associates to the right, so we will usually write the type $t_1 \rightarrow (t_2 \rightarrow t_3)$ as $t_1 \rightarrow t_2 \rightarrow t_3$.


\subsection{Judgments}

Once we are able to build terms of \coc{}, we want to reason about them, possibly within some context of assumptions.

\begin{defnc}[Context]
A \emph{context} in \coc{} is a list of \emph{variable declarations}, written $x : t$ to say variable $x$ has type $t$, and \emph{definitions}, written $x := t_1 : t_2$ to say variable $x$ has value $t_1$ of type $t_2$. The context may be written as $[d_1 ; d_2 ; \ldots]$ to list the elements. We write $[ \; ]$ for the empty context and $\dyncon{} :: (t_1 : t_2)$ for adding an element to the end of the list.
\end{defnc}

The description of contexts in \coc{} in~\cite{coq} includes a global environment, written $E$, along with the local context, written \dyncon{}. Both are lists of variable declarations and definitions. In the presentation here we do not need to distinguish between global and local assumptions, so we will have one context, usually written \dyncon{}.

\begin{defnc}[Sequent, Judgment]
A \emph{sequent} $\dyncon{} \vdash t : T$ is a \emph{judgment} where $\dyncon{}$ is the context and $t$ and $T$ are \coc{} terms. We call the elements of the context \emph{antecedents} (or \emph{assumptions} or \emph{hypotheses}) and $t : T$ is said to be a \emph{consequent}. We write $\vdash t : T$ as notation for $[ \, ] \vdash t : T$.
\end{defnc}

%The sequents used in this section are all typing judgments, so the consequent of our sequents will be type declarations.

A sequent is notation representing a conditional assertion which may be true or false. We want to be able to determine when such assertions hold. For this, we need a set of inference rules to determine when a sequent is \emph{provable}. The rules of \coc{} are in Figure~\ref{fig:cicrules}.

{
\renewcommand{\arraystretch}{3.5}
\newcommand{\cicrlaxprop}{\inferH[Ax-Prop]{\seq[\dyncon{}]{\mathit{Prop} : \mathit{Type}_0}}{}}
\newcommand{\cicrlaxset}{\inferH[Ax-Set]{\seq[\dyncon{}]{\mathit{Set} : \mathit{Type}_0}}{}}
\newcommand{\cicrlaxtype}{\inferH[Ax-Type]{\seq[\dyncon{}]{\mathit{Type}_i : \mathit{Type}_{i + 1}}}{}}
\newcommand{\cicrlvar}{\inferH[Var]{\seq[\dyncon{}]{x : T}}{x : T \in \dyncon{} & (\mathrm{or} \; x := t : T \in \dyncon{} \; \mathrm{for \; some} \; t)}}
\newcommand{\cicrlprodprop}{\inferH[Prod-Prop]{\seq[\dyncon{}]{\forall (x : T), U : \mathit{Prop}}}{\seq[\dyncon{}]{T : s} & s \in \{\mathit{Set} , \mathit{Prop}, \mathit{Type}_i\} & \seq[\dyncon{} :: (x : T)]{U : \mathit{Prop}}}}
\newcommand{\cicrlprodset}{\inferH[Prod-Set]{\seq[\dyncon{}]{\forall (x : T), U : \mathit{Set}}}{\seq[\dyncon{}]{T : s} & s \in \{ \mathit{Set} , \mathit{Prop} \} & \seq[\dyncon{} :: (x : T)]{U : \mathit{Set}}}}
\newcommand{\cicrlprodtype}{\inferH[Prod-Type]{\seq[\dyncon{}]{\forall (x : T), U : \mathit{Type}_k}}{\seq[\dyncon{}]{T : \mathit{Type}_i} & i \leq k & \seq[\dyncon{} :: (x : T)]{U : \mathit{Type}_j} & j \leq k}}
\newcommand{\cicrllam}{\inferH[Lam]{\seq[\dyncon{}]{\lambda (x : T) . t : \forall (x : T), U}}{\seq[\dyncon{}]{\forall (x : T), U : s} & \seq[\dyncon{} :: (x : T)]{t : U}}}
\newcommand{\cicrlapp}{\inferH[App]{\seq[\dyncon{}]{(t \; u) : T \{ x / u \}}}{\seq[\dyncon{}]{t : \forall (x : U), T} & \seq[\dyncon{}]{u : U}}}
\newcommand{\cicrllet}{\inferH[Let]{\seq[\dyncon{}]{\mathit{let} \; x := t : T \; \mathit{in} \; u : U\{x / t\}}}{\seq[\dyncon{}]{t : T} & \seq[\dyncon{} :: (x := t : T)]{u : U}}}
\newcommand{\cicrlconv}{\inferH[Conv]{\seq{t : U}}{\seq{U : s} & s \in \{ \mathit{Set} , \mathit{Prop} , \mathit{Type}_i \} & \seq{t : T} & \seq{T \subtype{} U}}}

\begin{figure}
$$
\begin{tabular}{c c}
\cicrlaxprop{}
&
\cicrlaxset{} \\
%
\cicrlaxtype{}
&
\cicrlvar{} \\
%
\multicolumn{2}{c}{
\cicrllam{}
} \\
\multicolumn{2}{c}{
\cicrlapp{}
} \\
\multicolumn{2}{c}{
\cicrllet{}
} \\
\multicolumn{2}{c}{
\cicrlprodprop{}
} \\
\multicolumn{2}{c}{
\cicrlprodset{}
} \\
\multicolumn{2}{c}{
\cicrlprodtype{}
} \\
\multicolumn{2}{c}{
\cicrlconv{}
}
\end{tabular}
$$
\caption{Rules of \coc{} \label{fig:cicrules}}
\end{figure}
}

\begin{defnc}[Derivation, Valid/Provable Sequent]
\label{def:derivvalid}
A tree built using the rules of Figure~\ref{fig:cicrules} with \seq{t : T} at the root and \rl{Ax-Prop}, \rl{Ax-Set}, \rl{Ax-Type}, or \rl{Var} at the leaves is a \emph{derivation} of \seq{t : T}. If there is a derivation of \seq{t : T}, we say this sequent is \emph{valid} or \emph{provable}. We also say that $t$ has type $T$ in $\dyncon{}$ or just $t$ has type $T$ when $\dyncon{}$ is empty.
\index{derivation}
\index{valid sequent}
\index{provable sequent}
\end{defnc}

For terms $t$ and $T$ and variable $x$, the notation $T\{x / t\}$ denotes \emph{substitution}\index{substitution}, meaning the operation that replaces occurrences of $x$ in $T$ with $t$, with the usual renaming of bound variables to avoid instances of free variables becoming bound. The rules \rl{Ax-Prop}, \rl{Ax-Set}, and \rl{Ax-Type} are axioms that build the hierarchy of the sorts into the logic. The \rl{Var} rule allows a branch of a derivation to be completed by showing the consequent of a sequent to be present in the context. Here we have omitted a premise requiring that the context is well-formed and the rules for building well-formed contexts (see~\cite{coq} for details); informally, we understand this to mean that additions to the context are well-typed according to the rules of Figure~\ref{fig:cicrules}. The \rl{Lam} and \rl{App} rules are the standard rules for building terms of the typed \lambda-calculus.

\begin{defnc}[Dependent Product]
\label{def:depprod}
A term of the form $\forall t : T, U$ is called a \emph{dependent product}.
\index{dependent product}
\end{defnc}

Notice the \rl{Prod} rules are all for building dependent products. The notation $t_1 \subtype{} t_2$ in rule \rl{Conv} is to say $t_1$ is a subtype of $t_2$ and will be described in Section~\ref{sec:reductions} when discussing convertibility. The various \rl{Prod} rules, together with the \rl{Conv} rule, are what allow \coc{} to have such an expressive type system; they allow simple types, dependent types, polymorphism, and higher-order types as we will see in the coming sections.

%Consider the following product type construction rule from \coc{}:

%$$
%\infer[\rl{Prod(s, $s'$, $s''$)}]{\seq{\forall (t : T), U : s''}}{\seq{T : s} & \seq[\Gamma :: (t : T)]{U : s'}}
%$$

%\bigskip

%The possible tuples $(s, s', s'')$ that we allow can give us simple types, dependent types, and higher-order types and cause \cic{} to have a very expressive type system. For example, if we require $s$ is $\mathit{Set}$ and $s'$ is $\mathit{Prop}$, then the rule $\rl{Prod(s,$s'$,$s'$)}$ is a rule to build propositions with universal quantification:

%$$
%\infer[\rl{Prod($\mathit{Set}$, $\mathit{Prop}$, $\mathit{Prop}$)}]{\seq{\forall t : T, U : \mathit{Prop}}}{\seq{T : \mathit{Set}} & \seq[\Gamma :: (t : T)]{U : \mathit{Prop}}}
%$$

\begin{defnc}[Inhabited-in-Context]
\label{def:inhabited}
Let $\dyncon{}$ be a context and $T$ a type. We say that $T$ is \emph{inhabited in context} $\dyncon{}$ if there exists $t$ such that \seq{t : T} is provable.
\index{inhabited}
\end{defnc}

%\coc{} may be used as a functional programming language and for verifying the correctness of programs written in this language. The rich type system allows one to write far more specific types for functions than is allowed in more common programming languages (and any without dependent types).

The type system of \coc{} allows for two different approaches to be taken when using the language; it can be used as a functional programming language or for formalized reasoning via the two sorts $\mathit{Set}$ and $\mathit{Prop}$, respectively.

\begin{defnc}[Specification, Realization]
\label{def:spec}
If $\Gamma \vdash T : \mathit{Set}$ is provable, then $T$ is a \emph{specification}. If $\Gamma \vdash t : T$ is provable, then $t$ is a \emph{realization} of the specification $T$.
\end{defnc}

We can think of a specification as the type of a function and a realization as its implementation. For example, the identity function on natural numbers has specification $\mathbb{N} \rightarrow \mathbb{N}$ and a realization of this specification is the function $\lambda (x : \mathbb{N}) \, . \, x$. Note that we have not yet defined $\mathbb{N}$ in \coc{} (see below).

\begin{defnc}[Formula, Proof Object]
\label{def:thmpf}
If $\Gamma \vdash T : \mathit{Prop}$ is provable, then $T$ is a \emph{formula}. If $\Gamma \vdash t : T$ is provable, then $T$ is a \emph{theorem} and $t$ is a \emph{proof object} representing a proof of theorem $T$.
\end{defnc}

In the next few sections, as we explore the expressive power of the \coc{} type system, we will encounter some examples that make use of the type $\mathbb{N}$ of natural numbers. This type is inductive and cannot be properly defined until Section~\ref{sec:coqinduction}, but it is useful in illustrating earlier concepts. For this reason an informal definition of this type is given here as well as some (later justified) results about it.

The type $\mathbb{N}$ is an inductive type whose ``elements'' are constructed from the following rules:
\begin{itemize}
 \item the number 0 has type $\mathbb{N}$
 \item if $n$ has type $\mathbb{N}$, than the successor of $n$ (written $\mathit{S} \; n$) has type $\mathbb{N}$
\end{itemize}
In addition, in the examples here we make use of the fact that for any context \dyncon{}, the sequent \seq{\mathbb{N} : \mathit{Set}} is provable.

\section{Simply Typed Lambda Calculus}
\label{sec:stlc}

\index{simply typed \lambda-calculus}

From the rules of Figure~\ref{fig:cicrules}, we can see that \coc{} encompasses the simply-typed \lambda-calculus. These rules allow the construction of simple types. This includes atomic types, referred to by their identifier (e.g. \N, \Z), and arrow types $A \rightarrow B$ where $A$ and $B$ are simple types and $\rightarrow$ associates to the right. Observe that by the Curry-Howard correspondence we may view $A \rightarrow B$ as either a specification (function type) or a theorem (implication), depending on the sort of the arrow type. In either case, if $t : A \rightarrow B$, then $t$ maps either data of type $A$, or proofs of $A$, to data of type $B$, or proofs of $B$, respectively.

Abstractions and applications for simple types are build using the rules \rl{Lam} and \rl{App} of Figure~\ref{fig:cicrules}, where the bound variable in any universal quantification does not occur in its body. The rule \rl{Lam} makes it possible to construct a term of type $A \rightarrow B$, but we also need to be able to build this type. This is accomplished via the product rule for simple types:

$$
\infer[\rl{Prod-ST}]{\seq{A \rightarrow B} : s}{\seq{A : s} & \seq{B : s}} 
$$
where $s \in \{ \mathit{Set}, \mathit{Prop} \}$.

Notice the rule \rl{Prod-ST} is an instance of rule \rl{Prod-Set} with $s = \mathit{Set}$ or \rl{Prod-Prop} with $s = \mathit{Prop}$. Again note that $A \rightarrow B$ is shorthand for $\forall (x : A), B$ (given $x$ does not occur in $B$).


\section{Reductions}
\label{sec:reductions}

The calculus of constructions has the strong normalization property so all terms of \coc{} will be reduced to an irreducible form by any sequence of reductions. We use the notation $t \reduce{} s$ to say that a term $t$ evaluates to a term $s$ by some sequence of $\delta$-, $\beta$-, $\iota$-, and $\zeta$-reductions (described below).

\paragraph{$\delta$-Reduction} replaces an identifier with its definition. For example, if we have defined $f := \lambda (x : \mathit{Type}_0) \, . \, x : \mathit{Type}_0 \rightarrow \mathit{Type}_0$, then $f \; \mathit{Set} \triangleright_{\delta} \mathit{Set}$.

\paragraph{$\beta$-Reduction} evaluates a term acquired by the \rl{App} rule by replacing all occurrences of the bound variable in the body of the abstraction with the term the abstraction is applied to using standard substitution rules; defined as $(\lambda (x : T) \, . \, t) \; u \triangleright_\beta t \{ x / u \}$.

\paragraph{$\iota$-Reduction} handles computations in recursive programs; it will not be used in the examples presented.

\paragraph{$\zeta$-Reduction} deals with converting local bindings; it will not be used in the examples presented.

\subsection{Convertibility}

We say two terms $t_1$ and $t_2$ are $\alpha$-equivalent, written $t_1 \cong_\alpha t_2$, if they are the same term up to renaming of bound variables.

\begin{defnc}[$\beta\delta\iota\zeta$-Convertible]
\label{def:convequiv}
Two terms are considered equivalent, or \emph{$\beta\delta\iota\zeta$-convertible}, if they can be reduced to $\alpha$-equivalent terms by the reductions given above. When terms $t_1$ and $t_2$ are $\beta\delta\iota\zeta$-convertible, we write $t_1 =_{\beta\delta\iota\zeta} t_2$.
\end{defnc}

Symbolically, definition~\ref{def:convequiv} says for terms $t_1, t_2, u_1,$ and $u_2$, if $t_1 \reduce{} u_1$ and $t_2 \reduce{} u_2$ and $u_1 \cong_\alpha u_2$, then $t_1 =_{\beta\delta\iota\zeta} t_2$.

From convertibility we can develop the notion of \emph{subtyping} in \coc; $\subtype{}$ is a preorder on the collection of types in the type hierarchy. This relation is defined inductively in Figure~\ref{fig:conv}. This gives us a conversion rule for terms, the rule \rl{Conv} of Figure~\ref{fig:cicrules}.

\begin{figure}
\begin{enumerate}
 \item if \seq{t =_{\beta\delta\iota\zeta} u} then \seq{t \subtype{} u}
 \item for all $i, j \in \mathbb{N}$, if $i \leq j$ then \seq{\mathit{Type}_i \subtype{} \mathit{Type}_j}
 \item for all $i \in \mathbb{N}$, \seq{\mathit{Set} \subtype{} \mathit{Type}_i}
 \item \seq{\mathit{Prop} \subtype{} \mathit{Set}} and for all $i \in \mathbb{N}$, \seq{\mathit{Prop} \subtype{} \mathit{Type}_i}
 \item if \seq{T =_{\beta\delta\iota\zeta} U} and \seq[\Gamma :: (x : T)]{T' \subtype{} U'} then \seq{\forall x : T, T' \subtype{} \forall x : U, U'}
\end{enumerate}
\caption{Subtyping in \coc{} \label{fig:conv}}
\end{figure}



\section{Dependent Types}
\label{sec:deptype}

\index{dependent type}

A dependent type is the result of applying a function to appropriate expressions; in particular, it is the reduced form of a function applied to an argument of type $\mathit{Set}$ or $\mathit{Prop}$. By definitions~\ref{def:spec} and~\ref{def:thmpf}, a dependent type is a term of \coc{} that depends on a choice of a \emph{realization} of a specification (for parametric types) or a choice of \emph{proof object} (in the logical case).

\begin{expl}[Tuple as a Dependent Type]
Let $n$ be a term of type $\mathbb{N}$. By definition~\ref{def:spec}, $n$ is a \emph{realization} of the \emph{specification} $\mathbb{N}$. The type of tuples of size $n$, call this $\mathit{tuple}$, depends on the value of $n$ and has type $\mathbb{N} \rightarrow \mathit{Set}$. $\mathit{tuple} \; 1$ is a dependent type of one-tuples. 
\end{expl}

%\begin{expl}
%The characteristic function $\chi_S$ has type $S \rightarrow \mathit{Prop}$. If $S$ has type $\mathit{Set}$ and, then $\chi_S \; S$ is a dependent type because it depends on the set $S$.
%\end{expl}

Note that the type of a dependent type is a dependent product (see definition~\ref{def:depprod}). We need to be able to build the dependent products that will allow us to define the above example. We can use a derived rule, which we will call \rl{Prod-Dep}, for building dependent products that can then be used to build dependent types. Let $s \in \{ \mathit{Set} , \mathit{Prop} \}$. This rule is:
$$
\infer[\rl{Prod-Dep}]{\seq{\forall (x : T), U : \mathit{Type}_i}}{\seq{T : s} & \seq[\dyncon{} :: (x : T)]{U : \mathit{Type}_i}}
$$
We get this rule from the following derivation tree:
{\scriptsize
$$
\infer[\rl{Prod-Type}]{\seq{\forall (x : T), U : \mathit{Type}_i}}{
  \infer[\rl{Conv}]{\seq{T : \mathit{Type}_0}}{\infer[\rl{Ax-Type}]{\seq{\mathit{Type_0} : \mathit{Type}_1}}{} & \seq{T : s} & \infer[\rl{*}]{\seq{s \subtype{} \mathit{Type}_0}}{}}
  &
  \seq[\dyncon{} :: (x : T)]{U : \mathit{Type}_i}}
$$
}
The leaf labeled with \rl{*} is proven by clauses $3$ and $4$ in the definition of the subtype relation in Figure~\ref{fig:conv}.

Using \rl{Prod-Dep} it is possible to build a term $\forall (x : T), U$ of type $\mathit{Type}_i$, where $x$ may occur freely in $U$ and, most importantly for the contents of this section, $T : s$ where $s \in \{ \mathit{Set}, \mathit{Prop} \}$.% If $x$ is not a free variable of $U$, then $\forall (x : T), U$ is a non-dependent type abbreviated as $T \rightarrow U$ (since for any $t$ of type $T$, $U \equiv U\{x/t\}$).

\begin{expl}[Parametrized Types]

If $s$ is $\mathit{Set}$ and $U$ is $\mathit{Set}$, then the rule \rl{Prod-Dep} can be used to construct the type of parametrized types. Consider the first example above, the type of tuples of size $n$, where $n$ is either a variable in the context or a concrete value. We will define the name of this type to be $\mathit{tuple}$. Then the type of $\mathit{tuple}$ is $\mathbb{N} \rightarrow \mathit{Set}$ and the dependent type is an instantiation of this type with some value $n$ of type $\mathbb{N}$. The sequent \seq{\mathbb{N} \rightarrow \mathit{Set} : \mathit{Type}_i} is proven with \rl{Prod-Type}.

\end{expl}

\begin{expl}[Predicates]

Let $s$ be $\mathit{Set}$ and be $U$ be $\mathit{Prop}$. Then the rule \rl{Prod-Dep} can be used to build the type of unary predicates $T \rightarrow \mathit{Prop}$ where (from the rule) we also know that $T$ has type $\mathit{Set}$. We can extend this to $n$-ary predicates by repeated uses of the rule.

\end{expl}

\section{Higher-Order Types}
\label{sec:hot}

For all $i \in \mathbb{N}$, types $T$ with type $\mathit{Type}_i$ are considered here to be higher-order types since elements $t$ of type $T$ are types. For example, $\mathbb{N}$ has type $\mathit{Set}$, which has type $\mathit{Type}_0$. Traditionally we can think of $\mathbb{N}$ as a type and elements that inhabit it as values of that type. Dependent products with quantification over higher-order types are built with the rule \rl{Prod-Type} of Figure~\ref{fig:cicrules}. In this section we will see how polymorphism and type operators are permitted in Coq using the \rl{Prod-Type} rule and higher-order types.

\subsection{Polymorphism}
\label{subsec:polymorphism}

\index{polymorphism}

Informally, a polymorphic function is a function with a type parameter. Working toward an example of such a function, consider a unary function $\mathit{double}$ on natural numbers that returns the argument multiplied by two. Note that we do not define this function here as this requires concepts explained later, and the focus of this discussion is on the type of such a function. $\mathit{double}$ has type $\mathbb{N} \rightarrow \mathbb{N}$. In fact, all unary functions on natural numbers can be specified with type $\mathbb{N} \rightarrow \mathbb{N}$. Then a function that iterates such unary functions on $\mathbb{N}$, call this $\mathit{iter\_nat}$, will have type $(\mathbb{N} \rightarrow \mathbb{N}) \rightarrow \underline{\mathbb{N}} \rightarrow \mathbb{N} \rightarrow \mathbb{N}$, where the first argument is the function to iterate and the second argument (which is underlined) is the number of times to iterate the function argument. The definition of $\mathit{iter\_nat}$ does not make use of the fact that the unary functions are over $\mathbb{N}$ because it simply repeatedly applies the function the number of times specified by the second (underlined) argument. So the logic of $\mathit{nat\_iter}$ should be reusable to iterate unary functions over any type $t$ of type $\mathit{Set}$. We want to define a function, say $\mathit{iter}$, that will accomplish this. $\mathit{iter}$ will have type $\forall (t : \mathit{Set}), (t \rightarrow t) \rightarrow \mathbb{N} \rightarrow t \rightarrow t$. We can show that this is a valid type in \coc{}.

%Perhaps a function is needed to iteratively apply unary functions on the type \coqtm{bool} (this type has values \coqtm{true} and \coqtm{false}).

Let $T$ be $\mathit{Set}$, then we can use the \rl{Prod-Type} rule with $i=j=k=0$ to build specifications of polymorphic functions: \\

$$
\infer[\rl{Prod-Type}]{\seq{\forall (t : \mathit{Set}), U : \mathit{Type}_0}}{\seq{\mathit{Set} : \mathit{Type}_0} & \seq[\Gamma :: (t : \mathit{Set})]{U : \mathit{Type}_0}}
$$

\bigskip

Using this rule, we prove below that $\forall t : \mathit{Set}, (t \rightarrow t) \rightarrow \N{} \rightarrow t \rightarrow t$ has type $\mathit{Type}_0$ in \coc{}. In the course of this proof we will also show that $t : \mathit{Set} \vdash (t \rightarrow t) \rightarrow \N{} \rightarrow t \rightarrow t : \mathit{Set}$ is a valid sequent. By the terminology of definition~\ref{def:spec}, this means we can \emph{specify} a function for the $n$-th iterate of a unary function on some type $t$ with sort $\mathit{Set}$, where $n$ is the natural number argument to the function $\mathit{iter}$.

\paragraph{Claim:} $\vdash \forall t : \mathit{Set}, (t \rightarrow t) \rightarrow \N{} \rightarrow t \rightarrow t : \mathit{Type}_0$ is provable.
% See the next page for a proof of this sequent. Note: weakening has been used without being marked to remove unnecessary elements of the context where possible and reduce the size of the proof. This specification is stated in~\cite{coqart} with identifier \texttt{iterate} (without its construction). \\

\begin{proof}

%To show that we can build specifications of functions iterating unary functions over some type $t : \mathit{Set}$, we need to show that $t : \mathit{Set} \vdash (t \rightarrow t) \rightarrow \N{} \rightarrow t \rightarrow t : \mathit{Type}_0$ is derivable.

This proof will be presented from axioms and work towards the goal where most steps use some version of a \rl{Prod} rule. We show the sequent in the above paragraph holds as justification that we can construct the type of specifications of polymorphic functions. To finish the proof and show the claim above we will need to use the \rl{Conv} rule toward the end, since we are building a term of type $\mathit{Type}_0$.

For any $\dyncon{}$ with $t : \mathit{Set} \in \dyncon{}$, the \rl{Var} rule is used to show that the sequent $\dyncon{} \vdash t : \mathit{Set}$ is provable. So the following sequents are valid:

\begin{align}
[t : \mathit{Set}] &\vdash t : \mathit{Set} \label{eqn:seq1} \\
[t : \mathit{Set} ; t : \mathit{Set}] &\vdash t : \mathit{Set} \label{eqn:seq2} \\
[t : \mathit{Set} ; H_1 : t \rightarrow t ; H_2 : \mathbb{N}] &\vdash t : \mathit{Set} \label{eqn:seq3} \\
[t : \mathit{Set} ; H_1 : t \rightarrow t ; H_2 : \mathbb{N} ; t : \mathit{Set}] &\vdash t : \mathit{Set} \label{eqn:seq4}
\end{align}
By \eqref{eqn:seq1} and \eqref{eqn:seq2} above and the rule \rl{Prod-Set}, we derive the sequent
\begin{align}
[t: \mathit{Set}] \vdash (t \rightarrow t) : \mathit{Set}. \label{eqn:seq5}
\end{align}
The type $\mathbb{N}$ is defined to have type $\mathit{Set}$, so
\begin{align}
[t : \mathit{Set} ; H_1 : t \rightarrow t] \vdash \mathbb{N} : \mathit{Set} \label{eqn:seq6}
\end{align}
is also valid.
By \eqref{eqn:seq3} and \eqref{eqn:seq4} above and the rule \rl{Prod-Set}, we derive the sequent
\begin{align}
[t : \mathit{Set} ; H_1 : t \rightarrow t ; H_2 : \mathbb{N}] \vdash t \rightarrow t : \mathit{Set} \label{eqn:seq7}.
\end{align}
So we can use the \rl{Prod-Set} rule with \eqref{eqn:seq6} and \eqref{eqn:seq7} to show that
\begin{align}
[t : \mathit{Set} ; H_1 : t \rightarrow t] \vdash \mathbb{N} \rightarrow t \rightarrow t : \mathit{Set} \label{eqn:seq8}
\end{align}
is provable.
Now \rl{Prod-Set} applied to \eqref{eqn:seq5} and \eqref{eqn:seq8} gives us
\begin{align}
[t : \mathit{Set}] \vdash (t \rightarrow t) \rightarrow \mathbb{N} \rightarrow t \rightarrow t : \mathit{Set}. \label{eqn:seq9}
\end{align}
We have shown in \eqref{eqn:seq9} that we can construct the specification of a polymorphic function for iterating unary functions of type $t : \mathit{Set}$.
Continuing the proof, the sequent
\begin{align}
[t : \mathit{Set}] \vdash \mathit{Type}_0 : \mathit{Type}_1 \label{eqn:seq10}
\end{align}
is valid by \rl{Ax-Type}. By the definition of $\subtype{}$, the sequent
\begin{align}
[t : \mathit{Set}] \vdash \mathit{Set} \subtype{} \mathit{Type}_0 \label{eqn:seq11}
\end{align}
is also valid.
Applying \rl{Conv} to \eqref{eqn:seq10}, \eqref{eqn:seq9}, and \eqref{eqn:seq11} gives
\begin{align}
[t : \mathit{Set}] \vdash (t \rightarrow t) \rightarrow \mathbb{N} \rightarrow t \rightarrow t : \mathit{Type}_0. \label{eqn:seq12}
\end{align}
By the axiom \rl{Ax-Set}, the sequent
\begin{align}
\vdash \mathit{Set} : \mathit{Type}_0 \label{eqn:seq13}
\end{align}
is valid.
Finally, we use the rule \rl{Prod-Type} a final time with \eqref{eqn:seq13} and \eqref{eqn:seq12} to show that
\begin{align}
\vdash \forall (t : \mathit{Set}), (t \rightarrow t) \rightarrow \mathbb{N} \rightarrow t \rightarrow t : \mathit{Type}_0
\end{align}
is derivable, as claimed.

\end{proof}

\subsection{Type Operators}
\label{subsec:tpop}

\index{type operator}
Informally, a \emph{type operator} is a type built from other types. The \rl{Prod-Type} rule is what also allows us to express type operators in \coc{} because its conclusion is a typing judgment for a dependent product with quantification over higher-order types.

\begin{expl}[Logical Connectives]

Let $T$ be $\mathit{Prop}$ and $i=j=k=0$, then we have a rule to build the type of logical connectives.

$$
\infer[\rl{Prod-Type}]{\seq{\forall t : \mathit{Prop}, U : \mathit{Type}_0}}{\seq{\mathit{Prop} : \mathit{Type}_0} & \seq[\Gamma :: (t : \mathit{Prop})]{U : \mathit{Type}_0}}
$$

The infix binary connectives representing ``or'' and ``and'' can be declared as $\vee : \mathit{Prop} \rightarrow \mathit{Prop} \rightarrow \mathit{Prop}$ and $\wedge : \mathit{Prop} \rightarrow \mathit{Prop} \rightarrow \mathit{Prop}$, respectively.
%This expands what can be expressed with dependent products and allows formulas corresponding to natural deduction rules using these types.
Note that we can only \emph{declare} these types at this time, meaning we see here how to construct the types of these operators. This is necessary before we see how to define (and derive the type of) definitions. Inductive types will be discussed in Section~\ref{sec:coqinduction}.

\paragraph{Claim:} \seq[]{\mathit{Prop} \rightarrow \mathit{Prop} \rightarrow \mathit{Prop} : \mathit{Type}_0} is provable.

\begin{proof}

This proof is illustrated by the derivation tree below. First, we rewrite the consequent of the sequent in the form that more easily visually matches the conclusion of the \rl{Prod-Type} rule. Recall that the arrow notation $A \rightarrow B$ is a simplified notation for $\forall (x : A), B$. So we can rewrite $\mathit{Prop} \rightarrow \mathit{Prop} \rightarrow \mathit{Prop}$ as $\forall (t_1 : \mathit{Prop}), \forall (t_2 : \mathit{Prop}), Prop$.

$$
{\footnotesize
\infer[\rl{Prod-Type}]{\seq[]{\forall (t_1 : \mathit{Prop}), \forall (t_2 : \mathit{Prop}), \mathit{Prop} : \mathit{Type}_0}}{
    \infer[\rl{*}]{\seq[]{\mathit{Prop} : \mathit{Type}_0}}{}
    &
    \infer[\rl{Prod-Type}]{\seq[{[t_1 : \mathit{Prop}]}]{\forall (t_2 : \mathit{Prop}), \mathit{Prop} : \mathit{Type}_0}}{
      \infer[\rl{*}]{\seq[{[t_1 : \mathit{Prop}]}]{\mathit{Prop} : \mathit{Type}_0}}{}
      &
      \infer[\rl{*}]{\seq[{[t_1 : \mathit{Prop} ; t_2 : \mathit{Prop}]}]{\mathit{Prop} : \mathit{Type}_0}}{}
    }
}
}
$$

All three leaves are marked with \rl{*} and proven by clause $4$ in the definition of the subtype relation in Figure~\ref{fig:conv}.

\end{proof}

\end{expl}


\section{Interactive Proving in Coq}
\label{sec:coqpf}

For the remainder of this chapter we are considering the Coq implementation of \cic{} and use of this system. Now when we talk about built-in language types, tactics, commands, or define types in the Coq syntax, we will use \texttt{teletype} font rather than \textit{italicized} math font.

As described in Section~\ref{sec:coctype}, to prove a statement $P$ where \seq{P : \coqtm{Prop}} is provable, we construct (or find) a proof object $t$ through a derivation of \seq{t : P} (i.e. according to definition~\ref{def:inhabited}, show that $P$ is inhabited by $t$ in \dyncon{}). By definition~\ref{def:thmpf}, proof object $t$ represents a proof of theorem $P$. As an alternative to ``defining'' the proof object $t$ and allowing the type checker to verify that $t$ is a proof term for $P$, Coq provides an interactive proof mode where \emph{tactics}\index{tactic} are used to interactively build $t$. These proofs start with the theorem statement $P$ as the goal and work backward reducing the goal to subgoals at each step and eventually to axioms. Once all goals have been discharged, the system builds the proof term $t$. The names of some of these tactics will be mentioned throughout this document, so we collect descriptions of the relevant tactics at the end of this section.

\subsection{Proof State}
\label{subsec:pfstate}
The interactive proof engine of Coq can be used to build a derivation in \coc{} in a bottom-up fashion, meaning we construct the proof tree for a sequent \seq{t : P} beginning at the root. In fact, we are constructing both the proof tree and $t$, showing that $P$ is inhabited.

\begin{defnc}[Proof State]
Let \dyncon{} be the context $[ H_1 : P_1 ; \ldots ; H_k : P_k]$ and let $P$ be a formula that we want to show is inhabited in \dyncon{} by a proof object. The pair $(\dyncon{} , P)$ is a \emph{proof state}. We call $P$ a \emph{goal}. We say that a proof state $(\dyncon{}, P)$ is \emph{complete} when there exists a $t$ such that $t : P \in \dyncon{}$ or $P \reduce{} \top$. A proof state that is not complete is \emph{incomplete}.
\end{defnc}

Visually we will write a proof state in a vertical form with the assumptions in the context above a horizontal line and the goal below. For example:

\begin{align*}
H_1 &: P_1 \\
&\large{\vdots} \\
H_k &: P_k \\[\pfshift{}]
\cline{1-2}
& P
\end{align*}
%where $H_1$, \dots{} , $H_k$ are variables names with types $P_1$, \dots{} , $P_k$, respectively. The goal $P$ is a formula.

%When working through a derivation, the proof state is changing. It is possible to have multiple incomplete proof states at one time. These correspond to the sequents at the leaves of an incomplete proof tree. 

Unlike in Coq, when we have multiple incomplete proof states corresponding to leaves in a partial derivation and goals $G_1, \ldots , G_j$ have the same context of assumptions, we will write them all below the horizontal line, separated by commas.
\begin{align*}
H_1 &: P_1 \\
&\large{\vdots} \\
H_k &: P_k \\[\pfshift{}]
\cline{1-2}
& G_1 , \ldots , G_j
\end{align*}

We also sometimes refer to the goal as a subgoal\index{subgoal} as a reminder that it was acquired from a previous goal and there may be other subgoals.


\begin{expl}[Interactive Proof]
\label{ex:interactivepf}

To illustrate the Coq interactive theorem proving system, we will prove the conjunction elimination rule $\vcenter{\inferH[\wedge_{e_1}]{P}{P \wedge Q}}$. Note that in Coq, conjunction is defined as an inductive type. Since we look at inductive types and inductive reasoning in Coq in Section~\ref{sec:coqinduction} and we do not have to use any inductive reasoning in this proof, we elide the details of inductive type definitions here. The single rule for constructing conjunctions is
$$
\forall (P \; Q : \coqtm{Prop}), P \rightarrow Q \rightarrow P \wedge Q
$$
which says for all propositions $P$ and $Q$, if we have a proof of $P$ (i.e. a term of type $P$) and if we have a proof of $Q$ (i.e. a term of type $Q$), then we have a proof of $P \wedge Q$.

\paragraph{Claim:} $\vdash \forall (P \; Q : \coqtm{Prop}), P \wedge Q \rightarrow P$ \\

\begin{proof}
We begin this proof at the root of the proof tree. Initially the context of assumptions is empty.
\begin{align*}
\cline{1-2}
& \forall (P \; Q : \coqtm{Prop}), P \wedge Q \rightarrow P
\end{align*}
We use the \coqtm{intros} tactic, which applies the \rl{Lam} rule of Figure~\ref{fig:cicrules} in a backward direction as many times as possible, effectively moving the quantified variable declarations (including anything to the left of $\rightarrow$) in the goal to the context of assumptions of the proof state. Recall that a goal corresponds to the type on the right of a colon in the consequent of a \coc{} sequent. The \coqtm{intros} tactic automatically solves the left premise of each application of the \rl{Lam} rule (it involves only simple type checking), and presents the type on the right of the colon in the consequent of the last right premise as the new subgoal. At each step of a proof in Coq, the terms on the left of the colon are constructed internally and not displayed. Once a proof is completed, these terms are used to build the proof object for the theorem we started with, which can then be displayed at the user's request. Each application of the \rl{Lam} rule in a backward direction introduces a new hypothesis into the context of assumptions of the proof state.

%We backchain with a meta-level use of implication introduction using the tactic \coqtm{intros} to move the quantified variable declarations (including anything to the left of $\rightarrow$) to the context of assumptions of the proof state.
\begin{align*}
P &: \coqtm{Prop} \\
Q &: \coqtm{Prop} \\
H &: P \wedge Q \\[\pfshift{}]
\cline{1-2}
& P
\end{align*}
Now we use the \coqtm{inversion} tactic on $H$. The inversion tactic exploits the properties of injectivity and disjointedness of the constructors of an inductive type. Since we have not yet explained inductive types, it suffices here to say by the definition of conjunction and given that assumption $H$ is a witness of the conjunction $P \wedge Q$, it must be the case that we can also assume both $P$ and $Q$.
\begin{align*}
P &: \coqtm{Prop} \\
Q &: \coqtm{Prop} \\
H &: P \wedge Q \\
H_1 &: P \\
H_2 &: Q \\[\pfshift{}]
\cline{1-2}
& P
\end{align*}
Now the goal matches $H_1$ and we finish this proof with \coqtm{assumption}, which is a tactic that simply applies the \rl{Var} rule of Figure~\ref{fig:cicrules}. %Another way this proof can be completed is using the \coqtm{apply} tactic to write 

\end{proof}
\end{expl}


\subsection{Some Coq Tactics, Tacticals, and Commands}
\index{tactic}
\index{tactical}

A tactical is an operator that takes tactic arguments to build a tactic. Below are Coq tactics, tacticals, and commands used in the proofs in this thesis. These have been described using some terminology introduced earlier as well as informal descriptions. More information can be found in the Coq Reference Manual~\cite{coq}.

\begin{description}
 \item[\coqtm{intros}] ~\\
  introduces variables and assumptions from the goal to the context of assumptions; a meta-level backward reasoning step of implication introduction; optional arguments assign names to the variables and assumptions introduced, otherwise default names are used
 \item[\coqtm{apply}] ~\\
  used either for backward reasoning, also known as \emph{backchaining}\index{backchaining}, on the goal, or forward reasoning, also known as \emph{forward chaining}\index{forward chaining}, on assumptions in the context; to backchain on the goal \coqtm{G} over some \coqtm{P : t1 -> t2} where \coqtm{G} matches \coqtm{t2} we write \coqtm{apply P} and the new goal is \coqtm{t1}; to forward chain with some \coqtm{H : t} with \coqtm{t} matching \coqtm{t1} in the context over the same \coqtm{P} we write \coqtm{apply P in H} and assumption \coqtm{H} is now \coqtm{t2} (see the proofs in Chapters~\ref{ch:slind} and~\ref{ch:gslind} for examples of use)
 \item[\coqtm{constructor}] ~\\
  a specialized form of \coqtm{apply} which applies an appropriate constructor for the type of the goal without naming the constructor; constructors are the names of the clauses of an inductive definition as will be described in Section~\ref{sec:coqinduction}; proofs using this tactic can be seen in Chapters~\ref{ch:slind} and~\ref{ch:gslind}
 \item[\coqtm{reflexivity}] ~\\
  solves a goal when it is an equality with both sides $\beta\delta\iota\zeta$-equivalent (see definition~\ref{def:convequiv})
 \item[\coqtm{simpl}] ~\\
  applies $\beta\iota$-reduction then expands constants from their definitions and again tries $\beta\iota$-reduction; by default this tactic is used on the goal but it can be used on an element \coqtm{H : t} in the context by writing \coqtm{simpl in H} to simplify \coqtm{t}
 \item[\coqtm{rewrite}] ~\\
  rewrites from an equality (replacing all occurrences in the proof state of one side of the equality with the other) that is either an assumption, a local definition, or a theorem; optionally use either \coqtm{<-} or \coqtm{->} to give the rewrite direction
 \item[\coqtm{inversion}] ~\\
  all conditions derived for each constructor of the type of the argument are new assumptions; for each constructor matched, the proof has one new subgoal with the premises of that clause as new assumptions in the context (see example~\ref{ex:interactivepf})
 \item[\coqtm{induction}] ~\\
  applies the appropriate induction principle for the type we induct over; see Section~\ref{sec:coqinduction} for a discussion on induction in Coq
 \item[\coqtm{assumption}] ~\\
  used when the goal matches an assumption to complete the proof of a goal
 \item[\coqtm{auto}] ~\\
  attempts to prove the goal automatically using results in a hints database
 \item[\coqtm{Hint}] ~\\
  a Coq command; using \coqtm{Hint Resolve} \emph{theorem\_name} adds \emph{theorem\_name} to a list of hints used by \coqtm{auto}
 \item[\coqtm{try}] ~\\
  a tactical that tries to apply the tactic given as an argument and if it fails does not cause an error
 \item[\coqtm{;}] ~\\
  applies tactics in sequence
\end{description}

Many of the tactics can be replaced with the same tactic name prefixed with the letter \emph{e} (e.g. \coqtm{eapply}). This provides placeholders of appropriate type that act as logical variables that can be filled in by unification. They are used where we would otherwise need to provide a witness in cases of application as backward reasoning on the goal.



\section{Induction in Coq}
\label{sec:coqinduction}

\cic{} extends \coc{} by adding inductive type\index{inductive type} definitions. An inductive type is a type with \emph{constructors}\index{constructor} that may take arguments of that type, so it is self-referential. For example, the natural number type $\mathbb{N}$ is defined inductively in Coq as:
\begin{center}
\begin{tabular}{c}
\begin{lstlisting}
Inductive nat : Set :=
| Z : nat
| S : nat -> nat.
\end{lstlisting}
\end{tabular}
\end{center}
In Coq, we declare that we are defining an inductive type with the keyword \coqtm{Inductive}. The name of the type is \coqtm{nat} and its type is \coqtm{Set}. This inductive type has two \emph{constructors}, \coqtm{Z} (to represent zero) and \coqtm{S} (to represent the successor function). We can understand this type as saying that any natural number can be constructed either as zero or the successor of some other natural number and these are the only two ways to construct natural numbers.

From an inductive type, Coq automatically generates an \emph{induction principle}\index{induction principle} whose target type is \coqtm{Prop}. To prove a property of all elements of a type, proofs using these induction principles have one subcase for each constructor of the type. The induction principle for \coqtm{nat} is in Figure~\ref{fig:natind}, where $P$ is the property to be proven of all natural numbers. We sometimes refer to $P$ as the \emph{induction property}\index{induction property}.

\begin{figure}
\begin{align*}
\coqtm{nat\_ind} &: \forall (P : \coqtm{nat} \rightarrow \coqtm{Prop}), \\
(*\coqtm{Z}*) & \qquad P \; \coqtm{Z} \rightarrow \\
(*\coqtm{S}*) & \qquad (\forall (m : \coqtm{nat}), P \; m \rightarrow P \; (\coqtm{S} \; m)) \rightarrow \\
& \forall (n : \coqtm{nat}), P \; n
\end{align*}
\caption{Induction principle for type \coqtm{nat} \label{fig:natind}}
\end{figure}

Constructors that have recursive occurrences of the type being defined will have corresponding induction subcases with \emph{induction hypotheses}\index{induction hypothesis}. An example of such a constructor is \coqtm{S} in the definition of \coqtm{nat} because it requires a \coqtm{nat} argument. Notice the corresponding induction subcase is to prove $\forall (m : \coqtm{nat}), P \; m \rightarrow P \; (\coqtm{S} \; m)$. The formula $P \; m$ is an induction hypothesis.

Now that we have an inductive type defined in Coq, namely \coqtm{nat}, we can define functions by primitive recursion over this inductive type. This is done using the $\coqtm{match} \ldots \coqtm{with} \ldots \coqtm{end}$ construction. To define a recursive function over $n$ of type \coqtm{nat}, we need to consider the possible constructions of $n$. From the definition of \coqtm{nat}, we see that either $n = \coqtm{Z}$ or $n = \coqtm{S} \, m$ where $m$ has type \coqtm{nat}. We need to decide what happens in either case. Using the Coq syntax, we can write a primitive recursive function over $n$ as:
\begin{center}
\begin{tabular}{c}
\begin{lstlisting}
Fixpoint recursive_nat (n : nat) :=
  match n with
  | Z => f1
  | S m => f2
  end.
\end{lstlisting}
\end{tabular}
\end{center}
where \coqtm{Fixpoint} is a keyword for defining recursive functions and \coqtm{recursive\_nat} is the function name. If \coqtm{n} evaluates to \coqtm{Z}, then the result is \coqtm{f1}. If \coqtm{n} evaluates to \coqtm{S m'} for some \coqtm{m'} of type \coqtm{nat}, then the result is \coqtm{f2} with \coqtm{m} replaced by \coqtm{m'} by \iota-reduction.

\begin{expl}[Proof by Induction]
\index{induction}

We will see how to prove the statement $\forall (n : \coqtm{nat}), n = n + \coqtm{Z}$ by induction in Coq, also pointing out the induction property used by the induction principle. This proof uses the definition of $+$, which is notation for \coqtm{plus}, in reductions to irreducible terms.

Using the $\coqtm{match} \ldots \coqtm{with} \ldots \coqtm{end}$ construction described above, \coqtm{plus n m} is defined recursively with cases on the structure of \coqtm{n} as:
\begin{center}
\begin{tabular}{c}
\begin{lstlisting}
Fixpoint plus (n m : nat) :=
  match n with
  | Z => m
  | S n' => S (plus n' m)
  end.
\end{lstlisting}
\end{tabular}
\end{center}
We write $\coqtm{n} + \coqtm{m}$ as infix notation for \coqtm{plus n m}

\paragraph{Claim:} $\vdash \forall (n : \coqtm{nat}), n = n + \coqtm{Z}$ \\

\begin{proof}

This proof is completed bottom-up, so the initial proof state is the node at the root of the proof tree. The context is empty and the goal is the statement that we wish to prove.
\begin{align*}
\cline{1-2}
& \forall (n : \coqtm{nat}), n = n + \coqtm{Z}
\end{align*}
The tactic \coqtm{induction} $n$ is used to backchain with the induction principle for natural numbers in Figure~\ref{fig:natind}. This proof has two subcases, one corresponding to each constructor of \coqtm{nat}. These are to prove $P \; \coqtm{Z}$ and $\forall (m : \coqtm{nat}), P \; m \rightarrow P \; (\coqtm{S} \; m)$ where
$$
P \coloneqq \lambda (n : \coqtm{nat}) \; . \; n = n + \coqtm{Z}
$$
is the induction property.

We will first prove $P \; \coqtm{Z}$ (usually called the ``base case''):
\begin{align*}
\cline{1-2}
& \coqtm{Z} = \coqtm{Z} + \coqtm{Z}
\end{align*}
This is done by reducing $\coqtm{Z} + \coqtm{Z}$ to \coqtm{Z} by the first branch in the definition of $+$ and then with \coqtm{reflexivity}.

To complete this proof we need to show $\forall (m : \coqtm{nat}), P \; m \rightarrow P \; (\coqtm{S} \; m)$ (the ``inductive step''):
\begin{align*}
\cline{1-2}
& \forall (m : \coqtm{nat}), m = m + \coqtm{Z} \rightarrow \coqtm{S} \; m = (\coqtm{S} \; m) + \coqtm{Z}
\end{align*}
We make introductions into the context with \coqtm{intros}.
\begin{align*}
m &: \coqtm{nat} \\
H &: m = m + \coqtm{Z} \\[\pfshift{}]
\cline{1-2}
& \coqtm{S} \; m = (\coqtm{S} \; m) + \coqtm{Z}
\end{align*}
The right side of the goal equality can be reduced by \coqtm{simpl}, using the second branch in the definition of $+$.
\begin{align*}
m &: \coqtm{nat} \\
H &: m = m + \coqtm{Z} \\[\pfshift{}]
\cline{1-2}
& \coqtm{S} \; m = \coqtm{S} \; (m + \coqtm{Z})
\end{align*}
Now we can use $\coqtm{rewrite <- H}$ to replace $m + \coqtm{Z}$ with $m$ on the right side of the goal equality.
\begin{align*}
m &: \coqtm{nat} \\
H &: m = m + \coqtm{Z} \\[\pfshift{}]
\cline{1-2}
& \coqtm{S} \; m = \coqtm{S} \; m
\end{align*}
The goal is an equality with both sides equal, so this proof is finished with \coqtm{reflexivity}.

\end{proof}

\end{expl}

\subsection{Mutually Inductive Types}
\index{mutual induction}

A type may be built using types that are already defined. When two types have dependencies on each other, they cannot both be defined before the other. In this case we define a mutually inductive type\index{mutually inductive type}. An example of where this is useful is in defining two types \coqtm{even} and \coqtm{odd} which are unary relations to identify even and odd natural numbers, respectively. In Coq these can be defined as:
\begin{center}
\begin{tabular}{c}
\begin{lstlisting}
Inductive even : nat -> Prop :=
| e_Z : even Z
| e_S : forall (n : nat), odd n -> even (S n)
with odd : nat -> Prop :=
| o_S : forall (n : nat), even n -> odd (S n).
\end{lstlisting}
\end{tabular}
\end{center}
Intuitively this says that Z (meaning zero) is even and the successor of any odd number is even. Also, the successor of any even number is odd.

Coq automatically generates an induction principle for each of these types. For \coqtm{even}, this is
\begin{align*}
\coqtm{even\_ind} &: \forall (P : \coqtm{nat} \rightarrow \coqtm{Prop}), \\
(*\coqtm{e\_Z}*) & \qquad (P \; \coqtm{Z}) \rightarrow \\
(*\coqtm{e\_S}*) & \qquad (\forall (n : \coqtm{nat}), \coqtm{odd} \; n \rightarrow P \; (S \; n)) \rightarrow \\
& \forall (n : \coqtm{nat}), \coqtm{even} \; n \rightarrow P \; n
\end{align*}
where $P$ is the induction property for even natural numbers. Notice that a proof using this induction principle will have one subcase for each constructor of \coqtm{even}. Also, in the case corresponding to the constructor \coqtm{e\_S}, there is no induction hypothesis about the premise $\coqtm{odd} \; n$. So for some types and some theorems to prove, the generated induction principle is insufficient.

The command \coqtm{Scheme} may be used to generate induction principles over mutually inductive types. These induction principles will have subcases for every constructor in every type in the mutually inductive type. Continuing the example above, we can get the following mutual induction principle over \coqtm{even}:
\begin{align*}
\coqtm{even\_mutind} &: \forall (P_1 \; P_2 : \coqtm{nat} \rightarrow \coqtm{Prop}), \\
(*\coqtm{e\_Z}*) & \qquad (P_1 \; \coqtm{Z}) \rightarrow \\
(*\coqtm{e\_S}*) & \qquad (\forall (n : \coqtm{nat}), \coqtm{odd} \; n \rightarrow P_2 \; n \rightarrow P_1 \; (S \; n)) \rightarrow \\
(*\coqtm{o\_S}*) & \qquad (\forall (n : \coqtm{nat}), \coqtm{even} \; n \rightarrow P_1 \; n \rightarrow P_2 \; (S \; n)) \rightarrow \\
& \forall (n : \coqtm{nat}), \coqtm{even} \; n \rightarrow P_1 \; n
\end{align*}
This induction principle has more cases but provides more powerful assumptions in each inductive case. Notice that now the subcase corresponding to the constructor \coqtm{e\_S} also has the induction hypothesis $P_2 \; n$ and we also have a subcase for the constructor \coqtm{o\_S}.


\section{Conclusion}

The \coc{} inference system implements a type checker, so it can be used both for proof verification and checking that a function satisfies its specification (i.e. it is a realization of the required type). It can also be used to construct a proof. By appropriately instantiating the \rl{Prod} rule, the system is made more expressive as a functional language and theorem proving system while still maintaining many desirable properties including strong normalization and consistency.

Coq can be used to prove formalized statements. A large library of tactics are available to assist in proof development. Since Coq is an implementation of \cic{}, it is possible to define inductive types and then prove statements by induction over these types using automatically generated induction principles.

The upcoming presentation will make use of all of the concepts just presented: the rich type system of \cic{}, interactive proofs, and inductive types, culminating in proofs by structural induction over mutually inductive dependent types.




\cleardoublepage

%%%%%%%%%%%%%%%%%%%%%%%%%%%%%%%%%%%%%%%%%%%%%%%%%%
% HYBRID BACKGROUND
%%%%%%%%%%%%%%%%%%%%%%%%%%%%%%%%%%%%%%%%%%%%%%%%%%

\chapter{Hybrid}
\label{ch:hybrid}

\index{Hybrid}

%\newcommand{\elist}{\epsilon}
\newcommand{\hybrid}{Hybrid}
\newcommand{\llFun}[2]{\mathsf{fun}\,#1.\,#2}
\newcommand{\llRec}[2]{\mathsf{fix}\,#1.\,#2}
\newcommand{\llrec}[1]{\ikw{fix} \  x\, .\, #1}
\newcommand{\ikw}[1]{\ensuremath{\mathsf{#1}}}
\newcommand{\hastype}{\mathrel{:}}
\newcommand{\slvdn}[3]{{#1}\rhd_{{#2}} {#3}}

\begin{figure}    \setlength{\unitlength}{4144sp}  \begingroup\makeatletter\ifx\SetFigFont\undefined
    \def\x#1#2#3#4#5#6#7\relax{\def\x{#1#2#3#4#5#6}}  \expandafter\x\fmtname xxxxxx\relax \def\y{splain}  \ifx\x\y   \gdef\SetFigFont#1#2#3{    \ifnum #1<17\tiny\else \ifnum #1<20\small\else \ifnum
    #1<24\normalsize\else \ifnum #1<29\large\else \ifnum
    #1<34\Large\else \ifnum #1<41\LARGE\else \huge\fi\fi\fi\fi\fi\fi
    \csname #3\endcsname}  \else \gdef\SetFigFont#1#2#3{\begingroup \count@#1\relax \ifnum
    25<\count@\count@25\fi
    \def\x{\endgroup\@setsize\SetFigFont{#2pt}}    \expandafter\x \csname \romannumeral\the\count@
    pt\expandafter\endcsname \csname @\romannumeral\the\count@
    pt\endcsname \csname #3\endcsname}  \fi \fi\endgroup
  %\begin{picture}(4692,2010)(34,-1198) \thinlines
  \begin{picture}(0,2010)(34,-1198) \thinlines
    {\color[rgb]{0,0,0}\put(1456,269){\oval(210,210)[bl]}
      \put(1456,509){\oval(210,210)[tl]}
      \put(2821,269){\oval(210,210)[br]}
      \put(2821,509){\oval(210,210)[tr]} \put(1456,164){\line( 1,
        0){1365}} \put(1456,614){\line( 1, 0){1365}}
      \put(1351,269){\line( 0, 1){240}} \put(2926,269){\line( 0,
        1){240}} }    {\color[rgb]{0,0,0}\put(1006,-181){\oval(210,210)[bl]} \put(1006,
      59){\oval(210,210)[tl]} \put(3271,-181){\oval(210,210)[br]}
      \put(3271, 59){\oval(210,210)[tr]} \put(1006,-286){\line( 1,
        0){2265}} \put(1006,164){\line( 1, 0){2265}}
      \put(901,-181){\line( 0, 1){240}} \put(3376,-181){\line( 0,
        1){240}} }    {\color[rgb]{0,0,0}\put(556,-631){\oval(210,210)[bl]}
      \put(556,-391){\oval(210,210)[tl]}
      \put(3721,-631){\oval(210,210)[br]}
      \put(3721,-391){\oval(210,210)[tr]} \put(556,-736){\line( 1,
        0){3165}} \put(556,-286){\line( 1, 0){3165}}
      \put(451,-631){\line( 0, 1){240}} \put(3826,-631){\line( 0,
        1){240}} }    {\color[rgb]{0,0,0}\put(151,-1081){\oval(210,210)[bl]}
      \put(151,-841){\oval(210,210)[tl]}
      \put(4216,-1081){\oval(210,210)[br]}
      \put(4216,-841){\oval(210,210)[tr]} \put(151,-1186){\line( 1,
        0){4065}} \put(151,-736){\line( 1, 0){4065}} \put(
      46,-1081){\line( 0, 1){240}} \put(4321,-1081){\line( 0, 1){240}}
    %}    \put(1936,-556){\makebox(0,0)[lb]{\smash{\SetFigFont{10}{12.0}{rm}{\color[rgb]{0,0,0}\hybrid}        }}}
    }    \put(1736,-556){\makebox(0,0)[lb]{\smash{\SetFigFont{10}{12.0}{rm}{\color[rgb]{0,0,0}\hybrid}        }}}
    \put(1851,-1006){\makebox(0,0)[lb]{\smash{\SetFigFont{10}{12.0}{rm}{\color[rgb]{0,0,0}Coq}        }}}
    %\put(3376,504){\makebox(0,0)[lb]{\smash{\SetFigFont{10}{12.0}{rm}{\color[rgb]{0,0,0}Syntax:
    %        $\llFun{x}{E\ x}, \llrec{E\ x}\dots$ }        }}}
    %\put(3376,279){\makebox(0,0)[lb]{\smash{\SetFigFont{10}{12.0}{rm}{\color[rgb]{0,0,0}Semantics:
    %        typing $E\hastype t$,\dots}        }}}

    \put(3601,299){\makebox(0,0)[lb]{\smash{\SetFigFont{10}{12.0}{rm}{\color[rgb]{0,0,0}          } }}} 

    %\put(3626,10){\makebox(0,0)[lb]{\smash{\SetFigFont{10}{12.0}{rm}{\color[rgb]{0,0,0}Sequent
    %        calculus: $\slvdn{\Gamma}{n}{G}$}        }}}
    \put(4051,-151){\makebox(0,0)[lb]{\smash{\SetFigFont{10}{12.0}{rm}{\color[rgb]{0,0,0}}        }}}
    %\put(4000,-421){\makebox(0,0)[lb]{\smash{\SetFigFont{10}{12.0}{rm}{\color[rgb]{0,0,0}Meta-language:
    %        quasi}        }}}
    %\put(4000,-601){\makebox(0,0)[lb]{\smash{\SetFigFont{10}{12.0}{rm}{\color[rgb]{0,0,0}
    %        datatype for
    %        a $\lambda$-calculus}        }}}
    %\put(4500,-871){\makebox(0,0)[lb]{\smash{\SetFigFont{10}{12.0}{rm}{\color[rgb]{0,0,0}Ambient
    %        logic:}        }}}
    %\put(4500,-1071){\makebox(0,0)[lb]{\smash{\SetFigFont{10}{12.0}{rm}{\color[rgb]{0,0,0}
    %        tactics/simplifier}        }}}
    %\put(4500,-1271){\makebox(0,0)[lb]{\smash{\SetFigFont{10}{12.0}{rm}{\color[rgb]{0,0,0}
    %        (co)induction}        }}}
    %\put(1800,344){\makebox(0,0)[lb]{\smash{\SetFigFont{10}{12.0}{rm}{\color[rgb]{0,0,0}Object
    \put(1500,344){\makebox(0,0)[lb]{\smash{\SetFigFont{10}{12.0}{rm}{\color[rgb]{0,0,0}Object
            logic}        }}}
    \put(1400,-106){\makebox(0,0)[lb]{\smash{\SetFigFont{10}{12.0}{rm}{\color[rgb]{0,0,0}Specification
            logic}        }}}
  \end{picture}
  \caption{Architecture of the Hybrid system}
  \label{fig:arch}
\end{figure}


\begin{figure}
\begin{center}
\includegraphics[height=5cm]{HybridFig.png}
\caption{High-Level Hybrid Structure \label{fig:hybrid}}
\end{center}
\end{figure}

%Recall from Chapter~\ref{ch:intro}, Hybrid is a two-level logical framework implementing operators that allow higher-order abstract syntax (HOAS) encodings of object logics (OLs) to be expressed.
%A logic that we wish to study using these systems is called an object logic (OL).
%Hybrid is implemented as a library in the interactive theorem proving language Coq, thus making it relatively easy to modify and extend the reasoning power by the addition of new intermediate logics called specification logics. One can choose the simplest specification logic necessary for the present task, or possibly a combination of more than one depending on the OL to be encoded. 
In this chapter each layer of Hybrid will be explored to provide more intuition on how it is constructed and used. This explanation will be driven by an analogy, for use as an aid to both memory and understanding of the system.

The orientation of the layers is as in Figure~\ref{fig:hybrid}. We will first consider the top layer, the object logic, in Section~\ref{sec:hybridol} with an example to motivate what we are trying to accomplish. Next we will consider each layer bottom-up, beginning with the ambient logic in Section~\ref{sec:hybridcoq}, then the higher-order abstract syntax layer in Section~\ref{sec:hybridhoas}. Continuing up the stack we next come to the specification logic. Since much of the work presented later is on the implementation and metatheory of the specification logic required for our motivating example of Section~\ref{sec:hybridol}, we will not see details of the specification logic here. Rather, Section~\ref{sec:hybridsl} will illustrate the benefit a specification logic adds to Hybrid and reinforce its necessity. This will be followed by another look at the object logic in Section~\ref{sec:hybrid_ol_imp}, but this time we will be focusing on implementation details with the rest of the system in place. To conclude this chapter, Section~\ref{sec:hybridcompare} will compare Hybrid with alternative architectures for systems intended to reason about object logics using HOAS.

\section{Object Logics}
\label{sec:hybridol}
\index{object logic}

\begin{sidestory}
Suppose we wish to study flowers and create things with them. Then we need to be able to grow flowers.
\end{sidestory}

Suppose we wish to prove something about a programming language or logic, the OL. This language will have rules expressing syntax and semantics that we need to encode in some proof assistant so that we can reason about it. It is also necessary to define the judgments of this language so that we can make claims about the OL.

%\subsection{Example OL}
\begin{expl}[Object Logic: Equivalence of Named and Nameless \lambda-terms]

We consider one of the examples presented in~\cite{WCGN:PPDP13}. Following the presentation there, we can define a syntax and rules expressing direct and de Bruijn representations of untyped $\lambda$-terms. By direct we mean the standard notation for $\lambda$-terms where abstractions reference a named variable that may be used in the body of the abstraction. De Bruijn indices~\cite{debruijn}\index{de Bruijn indices} are a nameless representation of \lambda-terms where rather than using variable names, a natural number is used for occurrences of a variable.

Let $n$ represent a natural number, $x$ a variable, and $e$ and $d$ represent direct and de Bruijn representations, respectively. Then the following are grammars for these $\lambda$-terms:
\begin{align*}
e &::= x \;\; | \;\; \lambda x . e \;\; | \;\; e \; e \\
d &::= n \;\; | \;\; \lambda d \;\; | \;\; d \; d
\end{align*}
A natural number $n$ in the grammar for de Bruijn terms $d$ serves as a pointer to the abstraction bounding that variable. This representation of \lambda-terms is more efficient for computation as we can avoid issues surrounding bound variable names. The $\lambda$-term $\lambda x . \lambda y . x \; y$ can be written using de Bruijn indices as $\lambda \; (\lambda \; (2 \; 1))$. The number $2$ refers to the outer binder (it is contained in two abstractions) and $1$ refers to the inner binder.

An example property we might want to prove is that these two representations are equivalent (or seen another way, to construct equivalent $\lambda$-terms in these different forms). This logic has a judgment to say that \lambda-term $e$ is equivalent to de Bruijn term $d$ at depth $n$, written $e \equiv_n d$. There are three inference rules expressing equivalence of these two kinds of terms, one for each of application, abstraction, and variables, seen below.
$$
\inferH[\rl{hodb\_app}]{\dyncon{} \vdash e_1 \; e_2 \equiv_n d_1 \; d_2}{\dyncon{} \vdash e_1 \equiv_n d_1 & \dyncon{} \vdash e_2 \equiv_n d_2}
$$

$$
\inferH[\rl{hodb\_abs}]{\dyncon{} \vdash \lambda x . e \equiv_n \lambda d}{\dyncon{} , x \equiv_{n+k} k \vdash e \equiv_{n+1} d}
$$

$$
\inferH[\rl{hodb\_var}]{\dyncon{} \vdash x \equiv_{n+k} k}{x \equiv_{n+k} k \in \dyncon{}}
$$

Applications in the two notations are considered equivalent under $n$ abstractions if their corresponding components are. The rule \rl{hodb\_abs} is more complicated to understand due to an additional assumption in the context of the premise of the rule. Informally, this rule says if whenever assuming variable $x$ is equivalent to index $k$ at depth $n + k$ it can be shown that the bodies of the \lambda-terms $e$ and $d$ are equivalent at depth $n + 1$, then we can conclude that the abstractions $\lambda x . e$ and $\lambda d$ are equivalent at depth $n$. As an illustration of how to use this system, we will see how to prove $\vdash \lambda x . \lambda y . x \; y \equiv_0 \lambda \; (\lambda \; (2 \; 1))$ (i.e. these two \lambda-terms are equivalent under zero additional abstractions).

\paragraph{Claim:} $\vdash \lambda x . \lambda y . x \; y \equiv_0 \lambda \; (\lambda \; (2 \; 1))$

\begin{proof}

Observe that by the \rl{hodb\_var} rule, both sequents below are provable.
\begin{align}
x \equiv_{2} 2 , y \equiv_{2} 1 \vdash x \equiv_{2} 2 \label{eqn:olex1} \\
x \equiv_{2} 2 , y \equiv_{2} 1 \vdash y \equiv_{2} 1 \label{eqn:olex2}
\end{align}
This requires $n=0$ and $k=2$ in~\eqref{eqn:olex1} and $n = k = 1$ in~\eqref{eqn:olex2}. Using the rule \rl{hodb\_app} with~\eqref{eqn:olex1} and~\eqref{eqn:olex2} we derive the sequent
\begin{align}
x \equiv_{2} 2 , y \equiv_{2} 1 \vdash x \; y \equiv_{2} 2 \; 1 \label{eqn:olex3}
\end{align}
Reviewing our claim, we are proving an equivalence of abstractions. The \rl{hodb\_abs} rule is used on~\eqref{eqn:olex3} with $k = 1$.
\begin{align}
x \equiv_{2} 2 \vdash \lambda y . x \; y \equiv_{1} \lambda \; (2 \; 1)
\end{align}
We apply \rl{hodb\_abs} again, this time with $k = 2$.
\begin{align}
\vdash \lambda x . \lambda y . x \; y \equiv_0 \lambda \; (\lambda \; (2 \; 1))
\end{align}
We have derived the sequent claimed provable, so this proof is complete.

\end{proof}

Notice that the $\lambda$ on the left of the equivalence in the conclusion of the rule $\rl{hodb\_abs}$ is a binding operator. This observation will be important when we see how to represent untyped $\lambda$-terms using HOAS in Hybrid in Section~\ref{sec:hybridhoas} and then implement this OL in Section~\ref{sec:hybrid_ol_imp}.

\end{expl}

%A few observations should be made about the rule $\equiv_{\mathit{abs}}$ before moving on. First, there is a binding operator in the conclusion of this rule. A standard abstract syntax encoding of this rule would need to reason about variable naming issues such as \alpha-conversion, \beta-reduction and managing the names of free and bound variables. Second, this rule has a premise with a variable $k$ that is only used in the context of assumptions. In the encoding this will be a parametric judgment and we will have some local quantification in the context of assumptions of this sequent (*TODO: okay? trying to motivate this example). The first observation is justification for encoding in a system using HOAS and the second necessitates the specification logic presented later.

%We have presented a logic but have so far ignored the issue of where and how we will encode it to reason about it formally.

\section{Ambient Logic}
\label{sec:hybridcoq}
\index{ambient logic}
\index{reasoning logic}

\begin{sidestory}
We can plant seeds in the ground and use the natural resources around us to reach our goal. The sun will provide energy and rain will give water.
\end{sidestory}

The ambient logic (also known as the reasoning logic or the meta-meta-logic) is the layer of the system that everything else is defined in. It is an implementation of a logic and so has its own reasoning rules and allows us to define other reasoning systems within it. In our case, this is \cic{} and its implementation in Coq. This is the lowest reasoning level we consider carefully as part of our system; we will not be concerned with lower-level details of the implementation of \cic{} or its compilation. Chapter~\ref{ch:coq} covered all aspects of Coq, the ambient logic of Hybrid, that are necessary for understanding the contributions later in this thesis.

Existing theorem proving systems are an ideal tool to allow a language and its judgments to be encoded without building extra infrastructure. Hybrid is a Coq library (a collection of Coq files), so it is relatively easy to make modular updates to the system and to add new intermediate reasoning layers called specification logics, as will be explained in Section~\ref{sec:hybridsl}.
%Further benefits to using Coq include
Hybrid can also make use of the inductive and interactive reasoning tools of Coq as well as existing Coq libraries.

\section{Representing Higher-Order Abstract Syntax in Hybrid}
\label{sec:hybridhoas}
\index{higher-order abstract syntax}

\begin{sidestory}
As our aspirations continue to grow, we find it difficult to scale up our flower production. When the rain doesn't fall as we require, we manually make up for the shortfall. The task of watering every plant every day is tedious. A dedicated plot of land with an organized arrangement and an irrigation system is a solution to this problem.
\end{sidestory}

Many tedious computations are necessary for each encoding of an OL with binding structures. Examples include fresh name generation and capture-avoiding substitution. Since Hybrid is implemented in an ambient logic that is a typed \lambda-calculus, the technique of \emph{higher-order abstract syntax} (HOAS) can be used for representing OL expressions. The idea is to use the binder of \lambda-calculus, function abstraction, to represent all OL binding operators. Using HOAS one can avoid implementing logic to reason about variable naming concepts, thus inheriting the meta-level solutions to these challenges. In addition, OL renaming and substitution are handled as meta-level \alpha-conversion and \beta-reduction, respectively.

At this level we have a type \hybridtm{expr} (see Figure~\ref{fig:expr}) encoding a de Bruijn index version of the \lambda-calculus designed to be used to represent OL syntax. A parameter \hybridtm{con} is a placeholder for OL constants, to be defined for each OL. We define \hybridtm{var} and \hybridtm{bnd} to be the natural numbers. Hybrid expressions $(\hybridtm{VAR} \; i)$ and $(\hybridtm{BND} \; j)$ represent object-level free and bound variables, respectively. The constructor \hybridtm{APP} is used to build applications and \hybridtm{ABS} to build abstractions in de Bruijn notation.

\begin{figure}
\begin{lstlisting}
Inductive expr : Set :=
| CON : con -> expr
| VAR : var -> expr
| BND : bnd -> expr
| APP : expr -> expr -> expr
| ABS : expr -> expr.
\end{lstlisting}
\caption{Terms in Hybrid \label{fig:expr}}
\end{figure}

Note that \hybridtm{con} is an implicit parameter in the environment it is defined in; uses outside of this environment must explicitly state this parameter (e.g. \sltm{expr con} instead of \sltm{expr}). A type to be given to this placeholder is defined for each OL. For example, the OL in Section~\ref{sec:hybridol} will have constants for application and abstraction for each kind of \lambda-term and a constant for variables in de Bruijn terms. These will be defined as an inductive type that is then used to instantiate the type \hybridtm{expr} for this particular OL. This example is implemented in Section~\ref{sec:hybrid_ol_imp}.

Object-level binding operators are encoded in HOAS using the Hybrid operator $\hybridtm{lambda} : (\hybridtm{expr con} \rightarrow \hybridtm{expr con}) \rightarrow \hybridtm{expr con}$ which is the meta-level binder defined in the Hybrid library. When using it to encode HOAS, the expanded definition is the underlying de Bruijn notation using only the constructors of \hybridtm{expr}. Although a Hybrid user never sees the expanded form and only works at the HOAS level. As an example, consider the untyped \lambda-term $(\lambda x . \lambda y . x \; y)$. We can represent this in Hybrid as $(\hybridtm{lambda} \; (\lambda x . (\hybridtm{lambda} \; (\lambda y . x \; y))))$ which expands to $\hybridtm{ABS} \; (\hybridtm{ABS} \; (\hybridtm{APP} \; (\hybridtm{BND} \; 1) \; (\hybridtm{BND} \; 0)))$. The \hybridtm{lambda} operator and the constructors of \hybridtm{expr} are used to encode OL syntax.

%Our example of Section~\ref{sec:hybridol} is continued in Section~\ref{sec:hybrid_ol_imp} where we can see how \hybridtm{lambda} is used in defining the OL constants and syntax in Figure \ref{fig:hoasdb_con}.


\section{Specification Logic}
\label{sec:hybridsl}
\index{specification logic}

\begin{sidestory}
Not all flowers will grow in the same conditions. Given any plot of land, there are many plants that will not grow there because they need specific nutrients in their soil. We can create different soil mixes depending on the needs of different classes of flowers.
\end{sidestory}

There are OL judgments that we cannot encode as an inductive type in Coq. One example is a HOAS encoding of inference rules assigning simple types to \lambda-expressions.
%The HOAS  rule for typing abstractions contains negative occurrences of this judgment, which is not allowed by the Coq type system. This can be seen in the HOAS encoding of the rule for typing abstractions.
The standard rule for typing abstractions can be seen in Figure~\ref{fig:stlctp}. Building on the example of Section~\ref{sec:introhoas}, let \coqtm{typ} be the type of OL types in the encoding in Coq. Let \coqtm{arr} be a constant of type $\coqtm{typ} \rightarrow \coqtm{typ} \rightarrow \coqtm{typ}$ representing arrow types. Recall \coqtm{tm} is the type of OL terms.
%Suppose that we have constants expressing the higher-order syntax of terms, including \coqtm{lambda} of type $(\underline{\coqtm{tm}} \rightarrow \coqtm{tm}) \rightarrow \coqtm{tm}$.
We want to define a typing predicate $\coqtm{tp} : \coqtm{tm} \rightarrow \coqtm{typ} \rightarrow \coqtm{Prop}$. Then the HOAS encoding of the rule for typing abstractions would be expressed as
\begin{align*}
\forall (T \; T' : \coqtm{typ}) \; & (E : \coqtm{tm} \rightarrow \coqtm{tm}), \\
& (\forall (x : \coqtm{typ}), \underline{\coqtm{tp} \; x \; T} \rightarrow \coqtm{tp} \; (E \; x) \; T') \rightarrow \coqtm{tp} \; (\coqtm{lambda} \; E) \; (\coqtm{arr} \; T \; T').
\end{align*}
Note that the \coqtm{tp} predicate cannot be expressed inductively because of the (underlined) \emph{negative occurrence} of the \coqtm{tp} predicate in the above formula for the typing abstraction rule. Inductive types with negative recursive occurrences is not allowed by the Coq type system.

As a solution to the problem of needing to reason about judgments that violate this strict positivity requirement, Hybrid is a two-level system. By two-level we mean an intermediate specification level is introduced between the OL encoding and the meta-levels. The specification logic is less expressive than the ambient logic, the calculus of constructions, but it allows us to encode judgments with negative occurrences.

\begin{figure}
$$
%\inferH[tp\_abs]{(\lambda x \, . \, E \; x) : (T \rightarrow T')}{\forall x , \underline{x : T} \rightarrow (E \; x) : T'}
\inferH[tp\_abs]{\dyncon{} \vdash \lambda x \, . \, E : T \rightarrow T'}{\dyncon{} , x : T \vdash E : T'}
%\inferH[tp\_abs]{\dyncon{} \vdash \mathit{tp} \; (\lambda x \, . \, E) \; (T \rightarrow T')}{\dyncon{} , \underline{\mathit{tp} \; x \; T} \vdash \mathit{tp} \; E \; T'}
$$
\centering{\caption{Typing of \lambda-calculus Abstractions \label{fig:stlctp}}}
\end{figure}

Hybrid is a Coq library and as mentioned earlier, this architectural decision makes quick prototyping of SLs possible. Another important benefit is that one can choose the simplest specification logic necessary for the present task, or possibly a combination of more than one depending on the OL to be encoded. Judgments that can be defined inductively do not need to be defined in a SL. This may simplify proofs of OL properties as the user can avoid using a more complicated logic than necessary.

The two levels of the OL and SL interact through a parameter of the SL,
$$
\sltm{prog} : \sltm{atm} \rightarrow \sltm{oo} \rightarrow \coqtm{Prop},
$$
which is used to encode inference rules for OL judgments (and thus define provability at the OL level). There are two arguments to \sltm{prog}; the first is the (atomic) inference rule conclusion of type \sltm{atm} and the second a formula of type \sltm{oo} representing the premise(s) of the rule.

We use $a$ for atoms with type \coqtm{atm} and $o$ for formulas of type \coqtm{oo}, possibly with subscripts.

In this implementation, the type \sltm{atm} is a parameter of the SL and is instantiated with an inductive type whose constructors predicates expressing the judgments of a particular OL. For instance, the definition of \coqtm{atm} for our above example might include a predicate $\oltm{hodb} : (\mltm{expr}~ \mltm{con}) \rightarrow \coqtm{nat} \rightarrow (\mltm{expr}~ \mltm{con}) \rightarrow \coqtm{atm}$ relating the higher-order and de Bruijn encodings at a given depth.

The type \sltm{oo} is the type of goals and clauses in the SL. The definition of \sltm{oo} for the SL defined later is in Figure~\ref{fig:oofig}.
\begin{figure}
\begin{lstlisting}
Inductive oo : Type :=
| atom : atm -> oo
| T : oo
| Conj : oo -> oo -> oo
| Imp : oo -> oo -> oo
| All : (expr con -> oo) -> oo
| Allx : (X -> oo) -> oo
| Some : (expr con -> oo) -> oo.
\end{lstlisting}
\centering{\caption{Type of SL Formulas \label{fig:oofig}}}
\end{figure}
The constant \sltm{atom} coerces an atom (a predicate applied to its arguments) to an SL formula. For any $\alpha$ of type \sltm{atm}, we may refer to ($\sltm{atom} \; \alpha$) as an atomic formula. The constructor \sltm{Conj} represents conjunction and \sltm{Imp} is used to build implications. Also note that in this implementation, we restrict the type of universal quantification to two types, (\mltm{expr}~ \mltm{con}) and \mltm{X}, where \mltm{X} is a parameter that can be instantiated with any primitive type; in our running example, \mltm{X} would become \coqtm{nat} for the depth of binding in a de Bruijn term. We leave out disjunction. It is not difficult to extend our implementation to include disjunction and quantification (universal or existential) over other primitive types, but these have not been needed in reasoning about OLs.

We write \atom{a}, ($o_1$ \& $o_2$), and ($o_1 \longrightarrow o_2$) as notation for (\sltm{atom} $a$), (\sltm{Conj} $o_1$ $o_2$), and (\sltm{Imp} $o_1$ $o_2$), respectively. Formulas quantified by \sltm{All} are written $(\sltm{All}~ o)$ or $(\sltm{All}~ \lambda (x:\mltm{expr}~ \mltm{con}) \; . \; o \; x)$, where $o$ has type $\coqtm{expr con} \rightarrow \coqtm{oo}$. The latter is the $\eta$-long form with types included explicitly. The other quantifiers are treated similarly.

The type \sltm{oo} is an inductive type, so Coq will automatically generate the induction principle shown in Figure~\ref{fig:ooip} as discussed in Section~\ref{sec:coqinduction}. We can use this induction principle to prove a statement of the form $\forall (o : \sltm{oo}), P \; o$ for some $P : \sltm{oo} \rightarrow \coqtm{Prop}$. This proof will have one subcase for each constructor of \sltm{oo}.

\begin{figure}
\begin{align*}
\sltm{oo\_ind} &: \forall (P : \sltm{oo} \rightarrow \coqtm{Prop}), \\
(*\sltm{atom}*) & \qquad (\forall (a : \sltm{atm}), P (\atom{a})) \rightarrow \\
(*\sltm{T}*) & \qquad (P \; \sltm{T}) \rightarrow \\
(*\sltm{Conj}*) & \qquad (\forall (o_1 : \sltm{oo}), P \; o_1 \rightarrow \forall (o_2 : \sltm{oo}), P \; o_2 \rightarrow P \; (o_1 \& o_2)) \rightarrow \\
(*\sltm{Imp}*) & \qquad (\forall (o_1 : \sltm{oo}), P \; o_1 \rightarrow \forall (o_2 : \sltm{oo}), P \; o_2 \rightarrow P \; (o_1 \longrightarrow o_2)) \rightarrow \\
(*\sltm{All}*) & \qquad (\forall (o : \sltm{expr con} \rightarrow \sltm{oo}), (\forall (e : \sltm{expr con}), P \; (o \; e)) \rightarrow P \; (\sltm{All} \; o)) \rightarrow \\
(*\sltm{Allx}*) & \qquad (\forall (o : \sltm{X} \rightarrow \sltm{oo}), (\forall (x : \sltm{X}), P \; (o \; x)) \rightarrow P \; (\sltm{Allx} \; o)) \rightarrow \\
(*\sltm{Some}*) & \qquad (\forall (o : \sltm{expr con} \rightarrow \sltm{oo}), (\forall (e : \sltm{expr con}), P \; (o \; e)) \rightarrow P \; (\sltm{Some} \; o)) \rightarrow \\
&\forall (o : \sltm{oo}), P \; o
\end{align*}
\centering{\caption{Induction Principle for \sltm{oo} \label{fig:ooip}}}
\end{figure}

%A Hybrid SL is defined as an inductive type in Coq where each rule is represented by a constructor of the type. The constructor name is the rule name, and the type arrow is seen as implication.

A Hybrid SL is defined as an inductive type in Coq to encode a sequent calculus. Each rule of the sequent calculus is represented by a constructor of the inductive type. The constructor name is the rule name and the type arrow is used for implication from premises to conclusion. The context of the sequent is defined to behave as a set of elements of type \sltm{oo}. We write $\dyncon{}$ or $c$ for contexts.

Since we explore the SL and proofs of its structural properties in detail later when describing the contributions of this research, we cut short the discussion here. For continuity in this chapter, some notation and the meaning of provability judgments of the SL are all we need now. We write $\seqsl{o}$ to denote an SL, where $\dyncon{}$ has type \coqtm{context} and $o$ has type \coqtm{oo}. The symbol $\rhd$ is used as the SL sequent arrow.


\section{Example OL Implementation}
\label{sec:hybrid_ol_imp}

Now we can see how to encode our example syntax and judgments in Hybrid. Let \oltm{tm} represent the type of direct \lambda-terms and \oltm{dtm} represent the type of de Bruijn terms. Since these are used to form OL expressions, \oltm{tm} and \oltm{dtm} are aliases for \sltm{expr con}. Before stating the implementation of the rules of the logic, we have to define the OL constants. For direct application and abstraction we have $\oltm{hApp} : \oltm{tm} \rightarrow \oltm{tm} \rightarrow \oltm{tm}$ and $\oltm{hAbs} : (\oltm{tm} \rightarrow \oltm{tm}) \rightarrow \oltm{tm}$, respectively. Direct variables are encoded as meta-level variables. For de Bruijn application, abstraction, and variables we have $\oltm{dApp} : \oltm{dtm} \rightarrow \oltm{dtm} \rightarrow \oltm{dtm}$, $\oltm{dAbs} : \oltm{dtm} \rightarrow \oltm{dtm}$, and $\oltm{dVar} : \coqtm{nat} \rightarrow \oltm{dtm}$, respectively.

In Figure~\ref{fig:hoasdb_con} the constants of the OL are defined in the inductive type \oltm{con}. We also have the definitions of OL applications and abstractions for the direct and de Bruijn forms of \lambda-terms in terms of the OL constants and HOAS application and \hybridtm{lambda} operator. Note that in Coq, \coqtm{fun} is notation for abstractions. When we write Coq code we use this notation but when writing pretty-printed versions of the code we will use \lambda-calculus abstraction notation. For example, we often write Coq abstractions \coqtm{fun x => f x} as $\lambda x . f \; x$ because the latter is often more readable in our discussions. In Figure~\ref{fig:hoasdb_con} we can see the use of HOAS in the definition of \oltm{hAbs} where we use the Hybrid \hybridtm{lambda} operator.

\begin{figure}
\begin{lstlisting}
Inductive con : Set := 
| hAPP : con
| hABS : con
| dAPP : con
| dABS : con
| dVAR : nat -> con.

Definition hApp : tm -> tm -> tm :=
  fun (e1 : tm) =>
    fun (e2 : tm) =>
      APP (APP (CON hAPP) e1) e2. 
Definition hAbs : (tm -> tm) -> tm :=
  fun (f : tm -> tm) => 
    APP (CON hABS) (lambda f).

Definition dApp : dtm -> dtm -> dtm :=
  fun (d1 : dtm) =>
    fun (d2 : dtm) =>
      APP (APP (CON dAPP) d1) d2. 
Definition dAbs : dtm -> dtm :=
  fun (d : dtm) =>
    APP (CON cdABS) d.
Definition dVar : nat -> dtm :=
  fun (n : nat) =>
    (CON (dVAR n)).
\end{lstlisting}
\caption{Example OL: Encoding Syntax in Hybrid \label{fig:hoasdb_con}}
\end{figure}

The atomic judgment discussed for this example (equivalence between the two representations of lambda terms) is part of the inductive type \oltm{atm} defined below.
\begin{lstlisting}
Inductive atm : Set :=
| hodb : tm -> nat -> dtm -> atm.
\end{lstlisting}
The predicate \coqtm{hodb} corresponds to the infix $\equiv_n$ relation in the rules in Section~\ref{sec:hybridol} (i.e. $\coqtm{hodb} \; e \; n \; d$ is notation for $e \equiv_n d$). In the environment where the SL is defined, there are parameters \coqtm{atm} for atomic judgments of the OL, \coqtm{con} for OL constants, and \coqtm{X} for another type we wish to universally quantify over. Now the type of SL formulas with all parameters filled in is \oltm{oo atm con X}. This is the type of SL formulas at the OL level. %We define the SL parameter $X$ used for quantification over primitive types as $\coqtm{nat}$ for this OL.

The rules shown in Section~\ref{sec:hybridol} can now be defined in Hybrid using HOAS and a SL. More specifically, we can now define the inductive type \sltm{prog} as shown in Figure~\ref{fig:hoasdb_prog}, where we see the HOAS encoding of the rules in Section~\ref{sec:hybridol}. The inductive type \coqtm{prog} has a constructor for each of the inference rules \rl{hodb\_app} and \rl{hodb\_abs}. As we will see, the \rl{hodb\_var} rule is not represented explicitly because it is taken care of at the level of the SL. The first argument to \coqtm{prog} is an atomic OL inference rule conclusion and the second argument is a formula to encode the premises of the same OL inference rule. The Coq notation for \atom{a} is \coqtm{<<a>>}.

\begin{figure}
\begin{lstlisting}
Inductive prog : atm -> oo atm (expr con) X -> Prop :=
| hobd_app : forall (e1 e2 : tm) (n : nat) (d1 d2 : dtm),
   prog (hodb (hApp e1 e2) n (dApp d1 d2))
    (<<hodb e1 n d1>> & <<hodb e2 n d2>>)
| hodb_abs : forall (f : tm -> tm) (n : nat) (d : dtm),
   abstr f ->
   prog (hodb (hAbs f) n (dAbs d))
    (All (fun (x : tm) =>
     (Allx (fun (k : X) => <<hodb x (n + k) (dVar k)>>)) --->
       <<hodb (f x) (n + 1) d>>)).
\end{lstlisting}
\caption{Example OL: Encoding OL Inference Rules \rl{hodb\_app} and \rl{hodb\_abs} in Hybrid \label{fig:hoasdb_prog}}
\end{figure}

An example theorem for this OL is to prove that the judgment $\oltm{hodb}$ is deterministic in its first and third arguments (and thus the relational definition of the rules represents a function). To do this we want to prove the two theorems below (where $=$ is equality in the ambient logic).

\begin{prop}[\oltm{hodb\_det1}]
\label{thm:hodb_det1}
\begin{align*}
& \forall (\dyncon{} : \mathtt{context}) (e : \mathtt{tm}) (d_1 \; d_2 : \mathtt{dtm}) (n : \mathtt{nat}), \\
& \qquad \seqsl{\atom{\mathtt{hodb} \; e \; n \; d_1}} \rightarrow \seqsl{\atom{\mathtt{hodb} \; e \; n \; d_2}} \rightarrow d_1 = d_2.
\end{align*}
\end{prop}

\begin{prop}[\oltm{hodb\_det3}]
\label{thm:hodb_det3}
\begin{align*}
& \forall (\dyncon{} : \mathtt{context}) (e_1 \; e_2 : \mathtt{tm}) (d : \mathtt{dtm}) (n : \mathtt{nat}), \\
& \qquad \seqsl{\atom{\mathtt{hodb} \; e_1 \; n \; d}} \rightarrow \seqsl{\atom{\mathtt{hodb} \; e_2 \; n \; d}} \rightarrow e_1 = e_2.
\end{align*}
\end{prop}

To prove these in Hybrid, we must first define a SL that is able to reason about this OL. Once we have defined the SL, using it and the encoding of the OL just described, we will be ready to prove the above propositions. As of this writing, these theorems are not proven in Hybrid.

%\begin{theorem}[\oltm{hodb\_det3}]
%\end{theorem}

\section{Comparison to Other Architectures}
\label{sec:hybridcompare}

\begin{sidestory}
Our approach to growing flowers is not the only solution. One alternative is to build a factory specializing in the production of flowers. This would give us full control over lighting, water, and soil composition; but the startup costs are high and modifications can be prohibitively expensive.
\end{sidestory}

Other systems use HOAS for encoding and reasoning about OLs with binders but different choices are made in the implementation of these systems. We will briefly look at the features of the two most closely related systems, Abella~\cite{Gacek:IJCAR08} and Beluga~\cite{Pientka:IJCAR10}, and compare these systems to Hybrid. These three systems, along with Twelf~\cite{TwelfSP}, are compared in detail using benchmark problems in~\cite{FMP:JAR15}.

One feature that sets Hybrid apart from these systems is that Hybrid is a library in an existing theorem proving system while Abella, Beluga, and Twelf are special-purpose theorem proving systems built for reasoning about OLs using HOAS. Using Coq means we can trust the proofs without having to develop extra infrastructure. These proofs can be independently checked because a proof term is a \lambda-term; a proof check is a type check in the Calculus of Constructions, a trusted and well studied theoretical foundation for our work. The trade-off is less control over the reasoning logic of Hybrid and more levels of encoding.

\paragraph{Abella}
\index{Abella}

Abella is an interactive proof environment using the special-purpose $\mathcal{G}$ logic as its reasoning logic. $\mathcal{G}$ is intuitionistic, predicative, higher-order, and has fixed-point definitions for atomic predicates. It also allows mathematical induction (over natural numbers). Infrastructure for reasoning using HOAS is built-in to this logic. Like Hybrid, it is a two-level logical framework. In contrast, since it is a special-purpose system for reasoning about OLs, only one SL is used by the system at a time; to use a different SL the system must be updated. Hybrid is a Coq library so multiple SLs can can be available for use by any OL.

Abella is a tactic-based interactive theorem prover. This is the same style used when using the interactive proof environment in Coq, but the crucial difference is that on completion of a Coq proof the system generates a proof term. This is an object that can be checked independent of the implementation of \cic{} or Coq. This means that rather than trusting the implementation of a language and the tactics, we are provided evidence on completion of the proof. Since Hybrid is implemented in Coq, we have access to proof terms once a theorem is proven. This is not the case in Abella.

An advantage to Abella is that is has the \nabla-quantifier, a new specialized quantifier providing better direct reasoning about binding in OLs. This allows Abella to prove some properties about OLs that cannot be proven in Hybrid until we implement \nabla.

\paragraph{Beluga}
\index{Beluga}

Beluga is also a logical framework for reasoning about OLs using HOAS. The reasoning logic in this system is contextual LF; it supports reasoning over contexts. It is more specialized for reasoning with HOAS than Hybrid is. It implements a type theory instead of a logic.

In Beluga, some metatheory about contexts (e.g. the structural rules of weakening, contraction, and exchange in sequent calculi) is implicit. This means that it is built-in to the implementation rather than being axioms of a logic or proven to be admissible as rules. The benefit of this choice is it is not necessary to prove theses structural rules. The argument against this is that it requires more trust from the user. It is necessary to trust the implementation of the system rather than being able to see how the rules are defined to be axiomatic or proven to be admissible.

\cleardoublepage


%%%%%%%%%%%%%%%%%%%%%%%%%%%%%%%%%%%%%%%%%%%%%%%%%%
% HEREDITARY HARROP BACKGROUND
%%%%%%%%%%%%%%%%%%%%%%%%%%%%%%%%%%%%%%%%%%%%%%%%%%

\chapter{Hereditary Harrop Formulas}
\label{ch:hh}

The logic of hereditary Harrop formulas is foundational in the theory of logic programming languages. Although these formulas have their origins in encoding search behaviour and extending the power of logic programming languages in a semantically clear way, we make use of them in this thesis for their role in restricting the structure of proofs. Using such a restricted logic as a specification logic in Hybrid simplifies SL metatheory proofs and proofs about object logics.

Section~\ref{sec:hohh} will introduce the language of higher-order hereditary Harrop formulas and an inference system for reasoning about them. Following this, in Section~\ref{sec:focusing} we will see how to modify this logic to one with \emph{focusing}, which helps to optimize proof search as will be described later.

\section{Higher-Order Hereditary Harrop Formulas}
\label{sec:hohh}

The terms of the logic defined here are the terms of the simply-typed $\lambda$-calculus. Types are built from the primitive types and the (right-associative) function arrow $\rightarrow$ as usual. We introduce a type $o$ for formulas. Logical connectives and quantifiers are introduced as constants with their corresponding types as in \cite{Church40}. For example, conjunction has type $o \rightarrow o \rightarrow o$ and the quantifiers have type $(\tau\rightarrow o)\rightarrow o$, with some restrictions on $\tau$ described below. Predicates are function symbols whose target type is $o$. Following \cite{LProlog}, the grammars below for $G$ (goals) and $D$ (clauses) define the formulas of the higher-order hereditary Harrop language.

\begin{defnc}[Higher-Order Hereditary Harrop Formulas]
Formulas built from the grammar for $G$ are called $G$-formulas and formulas build from the grammar for $D$ are called $D$-formulas or \emph{higher-order hereditary Harrop formulas}\index{higher-order hereditary Harrop formulas}.
\end{defnc}

\begin{align*}
G & ::=
 \top \mathrel{|}
 A \mathrel{|}
 G\mathrel{\&}G \mathrel{|}
 G \lor G \mathrel{|}
 D\longrightarrow G \mathrel{|}
 \forall_\tau x. G \mathrel{|}
 \exists_\tau x. G
\\
% \mbox{\textit{Clauses}} &
D & ::=
 A \mathrel{|}
 G\longrightarrow D \mathrel{|}
 D \mathrel{\&} D \mathrel{|}
 \forall_\tau x. D
\end{align*}

We use the metavariable $A$ (possibly with subscripts) for atoms and write $\&$ for conjunction, $\longrightarrow$ for (right-associative) implication, and $\vee$ for disjunction. For universal and existential quantification, written as usual with symbols $\forall$ and $\exists$, we include the subscript $\tau$ to explicitly state the domain of quantification. This may be left out when it can be inferred from context. In goal formulas, we restrict $\tau$ to be a primitive type not containing $o$. In clauses, $\tau$ also cannot contain $o$, and is either primitive or has the form $\tau_1\rightarrow\tau_2$ where both $\tau_1$ and $\tau_2$ are primitive types.

With the language of formulas defined, we can now consider an inference system for reasoning about these formulas. This is a sequent calculus with the same conventions as described for \coc{} in Section~\ref{sec:coctype}. A grammar for contexts of these sequents is below. Contexts here are lists of hereditary Harrop formulas.
$$
%\mbox{\textit{Context}} &
\Gamma ::= [ \, ] \mathrel{|} \Gamma,D
$$

The rules for this logic are in Figure~\ref{fig:hohhform}. We use the same naming conventions as in the grammars of this chapter and also use $x$ for bound variables, $c$ for fresh variables, and $t$ for terms. %If we don't know if a formula is a goal or a clause, then we will write $B$, possibly with subscripts.

{
\renewcommand{\arraystretch}{3.5}
\newcommand{\hhinit}{\infer[\rl{init}]{\seq{A}}{A \in \dyncon{}}}
\newcommand{\hhtop}{\infer[\rl{$\top_R$}]{\seq{\top}}{}}
\newcommand{\hhandl}{\infer[\rl{$\&_L$}]{\seq[\dyncon{} , D_1 \& D_2]{G}}{\seq[\dyncon{} , D_1 , D_2]{G}}}
\newcommand{\hhandr}{\infer[\rl{$\&_R$}]{\seq{G_1 \& G_2}}{\seq{G_1} & \seq{G_2}}}
\newcommand{\hhorl}{\infer[\rl{$\vee_L$}]{\seq[\dyncon{} , D_1 \vee D_2]{G}}{\seq[\dyncon{} , D_1]{G} & \seq[\dyncon{} , D_2]{G}}}
\newcommand{\hhorr}{\infer[\rl{$\vee_{R_i}$}]{\seq{G_1 \vee G_2}}{\seq{G_i}}}
\newcommand{\hhimpl}{\infer[\rl{$\longrightarrow_L$}]{\seq[\dyncon{} , G_1 \longrightarrow D]{G_2}}{\seq{G_1} & \seq[\dyncon{} , D]{G_2}}}
\newcommand{\hhimpr}{\infer[\rl{$\longrightarrow_R$}]{\seq{D \longrightarrow G}}{\seq[\dyncon{} , D]{G}}}
\newcommand{\hhalll}{\infer[\rl{$\forall_L$}]{\seq[\dyncon{} , \forall_\tau x, D]{G}}{\seq[\dyncon{} , D\{ x / t\}]{G}}}
\newcommand{\hhallr}{\infer[\rl{$\forall_R$}]{\seq{\forall_\tau x , G}}{\seq{G \{ x / c \}}}}
\newcommand{\hhexl}{\infer[\rl{$\exists_L$}]{\seq[\dyncon{} , \exists_\tau x , D]{G}}{\seq[\dyncon{} , D \{ x / c \}]{G}}}
\newcommand{\hhexr}{\infer[\rl{$\exists_R$}]{\seq{\exists_\tau x , G}}{\seq{G \{ x / t \}}}}

\begin{figure}
\begin{center}
\begin{tabular}{c c}
\hhinit{} & \hhtop{} \\
\hhandl{} & \hhandr{} \\
\hhorl{} & \hhorr{} \\
\hhimpl{} & \hhimpr{} \\
\hhalll{} & \hhallr{} \\
\hhexl{} & \hhexr{}
\end{tabular}
\end{center}
\caption{The logic of higher-order hereditary Harrop formulas \label{fig:hohhform}}
\end{figure}
}
The \rl{init} rule allows a branch of a proof to be completed when an atom $A$ on the right of the sequent is in the context \dyncon{}. The only other way to finish a branch of a proof is with the axiom \rl{$\top$} when the consequent of a sequent is $\top$. The remaining rules are standard left and right introduction rules for formulas built from the grammars for $G$ and $D$. In the rule \rl{$\vee_{R_i}$}, $i \in \{ 1 , 2\}$.

Notice that all sequents have only one formula on the right of the sequent (as was also the case in the rules of \coc{} in Chapter~\ref{ch:coq}). A derivation tree built using a set of rules with this property is called \textbf{M}-proofs to say that it is a proof in minimal logic (it is also an \textbf{I}-proof since it will also hold in intuitionistic logic, see~\cite{mnps:uniformproofs}).

This logic has both left and right rules for each logical connective. In the course of a proof, a proof writer may have multiple decisions on how to proceed. This nondeterminism is not desirable if our goal is to automate proof search, which is the case for logic programming.

\begin{expl}[Hereditary Harrop Derivations]
\label{expl:hh}
Consider the sequent \seq[\forall_{\tau} x , (A_1 \; x) \& (A_2 \; x)]{(A_1 \; t) \& (A_2 \; t)} where $t : \tau$. Below are two derivation trees for this sequent:

$$
\infer[\rl{$\forall_L$}]{\seq[\forall_\tau x , (A_1 \; x) \& (A_2 \; x)]{(A_1 \; t) \& (A_2 \; t)}}{
	\infer[\rl{$\&_R$}]{\seq[(A_1 \; t) \& (A_2 \; t)]{(A_1 \; t) \& (A_2 \; t)}}{
		\infer[\rl{$\&_L$}]{\seq[(A_1 \; t) \& (A_2 \; t)]{A_1 \; t}}{
			\infer[\rl{init}]{\seq[A_1 \; t , A_2 \; t]{A_1 \; t}}{A_1 \; t \in A_1 \; t , A_2 \; t}
		}
		&
		\infer[\rl{$\&_L$}]{\seq[(A_1 \; t) \& (A_2 \; t)]{A_2 \; t}}{
			\infer[\rl{init}]{\seq[A_1 \; t , A_2 \; t]{A_2 \; t}}{A_2 \; t \in A_1 \; t , A_2 \; t}
		}
	}
}
$$
In this first derivation, we alternate between uses of left and right rules until all leaves that are sequents with the goal on the right contained in the context.

$$
\infer[\rl{$\&_R$}]{\seq[\forall_\tau x , (A_1 \; x) \& (A_2 \; x)]{(A_1 \; t) \& (A_2 \; t)}}{
	\infer[\rl{$\forall_L$}]{\seq[\forall_\tau x , (A_1 \; x) \& (A_2 \; x)]{A_1 \; t}}{
		\infer[\rl{$\&_L$}]{\seq[(A_1 \; t) \& (A_2 \; t)]{A_1 \; t}}{
			\infer[\rl{init}]{\seq[A_1 \; t , A_2 \; t]{A_1 \; t}}{A_1 \; t \in A_1 \; t , A_2 \; t}
		}
	}
	&
	\infer[\rl{$\forall_L$}]{\seq[\forall_\tau x , (A_1 \; x) \& (A_2 \; x)]{A_2 \; t}}{
		\infer[\rl{$\&_L$}]{\seq[(A_1 \; t) \& (A_2 \; t)]{A_2 \; t}}{
			\infer[\rl{init}]{\seq[A_1 \; t , A_2 \; t]{A_2 \; t}}{A_2 \; t \in A_1 \; t , A_2 \; t}
		}
	}
}
$$
In this second derivation, beginning at the root we first use as many right rules as necessary to only have atoms on the right side of sequents. Then we apply left rules until we finish the proof as above. This derivation is an example of a \emph{uniform proof}.
\end{expl}

\begin{defnc}[Uniform Proof]
A \emph{uniform proof}\index{uniform proof} is an \textbf{I}-proof where every sequent in the derivation tree that is non-atomic on the right is derived from the right introduction rule (e.g. $\&_R$, $\forall_R$, etc.) of its top-level connective.
\end{defnc}

We can see that the first derivation above is \emph{not} a uniform proof, because the rule \rl{$\forall_L$} is used to derive a sequent that does not have an atom on the right. If we wish to allow only uniform proofs, then this does not restrict what is provable by the logic of Figure~\ref{fig:hohhform}. The set of uniform proofs using rules in Figure~\ref{fig:hohhform} in a restricted manner is sound and complete with respect to the set of proofs that can be built using the same rules without this restriction.

Uniform proofs can be generalized to the notion of \emph{focusing} as presented in~\cite{LM:TCS09} and~\cite{Chaudhuri:LNCS08}, where the logic presented above is viewed as the negative fragment of intuitionistic logic.

\section{Focusing}
\label{sec:focusing}
\index{focusing}

A proof search strategy that reduces nondeterminism will make it easier to add automation to proof search. Here we describe a strategy that divides proof search into two stages by augmenting the inference system in a way that reduces the number of rule choices available at each step.

The idea is, in a bottom-up proof, we apply the appropriate right-rule for the top-level connective of the consequent of the sequent until the consequent is atomic. At this point we will transition to using left rules on a formula from the context; we choose a single formula from the context and apply left rules on this formula until this ``focused'' formula is atomic. If it matches the atom on the right of the sequent, the branch is complete, otherwise the proof fails and we return to the point where a formula from the context was focused.

{
\renewcommand{\arraystretch}{3.5}
\newcommand{\hhinit}{\infer[\rl{init}]{\seq{A}}{A \in \dyncon{}}}
\newcommand{\hhtop}{\infer[\rl{$\top_R$}]{\seq{\top}}{}}
\newcommand{\hhfocus}{\infer[\rl{focus}]{\seq{A}}{\seq[\dyncon{} ; {[D]}]{A} & D \in \dyncon{}}}
\newcommand{\hhmatch}{\infer[\rl{match}]{\seq[\dyncon{} ; {[A]}]{A}}}
\newcommand{\hhandl}{\infer[\rl{$\&_{L_i}$}]{\seq[\dyncon{} , {[D_1 \& D_2]}]{A}}{\seq[\dyncon{} , {[D_i]}]{A}}}
\newcommand{\hhandr}{\infer[\rl{$\&_R$}]{\seq{G_1 \& G_2}}{\seq{G_1} & \seq{G_2}}}
\newcommand{\hhorl}{\infer[\rl{$\vee_L$}]{\seq[\dyncon{} , {[D_1 \vee D_2]}]{A}}{\seq[\dyncon{} , {[D_1]}]{A} & \seq[\dyncon{} , {[D_2]}]{A}}}
\newcommand{\hhorr}{\infer[\rl{$\vee_{R_i}$}]{\seq{G_1 \vee G_2}}{\seq{G_i}}}
\newcommand{\hhimpl}{\infer[\rl{$\longrightarrow_L$}]{\seq[\dyncon{} , {[G \longrightarrow D]}]{A}}{\seq{G} & \seq[\dyncon{} , {[D]}]{A}}}
\newcommand{\hhimpr}{\infer[\rl{$\longrightarrow_R$}]{\seq{D \longrightarrow G}}{\seq[\dyncon{} , D]{G}}}
\newcommand{\hhalll}{\infer[\rl{$\forall_L$}]{\seq[\dyncon{} , {[\forall_\tau x, D]}]{A}}{\seq[\dyncon{} , {[D\{ x / t\}]}]{A}}}
\newcommand{\hhallr}{\infer[\rl{$\forall_R$}]{\seq{\forall_\tau x , G}}{\seq{G \{ x / c \}}}}
\newcommand{\hhexl}{\infer[\rl{$\exists_L$}]{\seq[\dyncon{} , {[\exists_\tau x , D]}]{A}}{\seq[\dyncon{} , {[D \{ x / c \}]}]{A}}}
\newcommand{\hhexr}{\infer[\rl{$\exists_R$}]{\seq{\exists_\tau x , G}}{\seq{G \{ x / t \}}}}

\begin{figure}
\begin{center}
\begin{tabular}{c c}
\hhfocus{} & \hhinit{} \\
\hhmatch{} & \hhtop{} \\
\hhandl{} & \hhandr{} \\
\hhorl{} & \hhorr{} \\
\hhimpl{} & \hhimpr{} \\
\hhalll{} & \hhallr{} \\
\hhexl{} & \hhexr{}
\end{tabular}
\end{center}
\caption{The logic of higher-order hereditary Harrop formulas with focusing \label{fig:hohhfoc}}
\end{figure}
}

To accomplish this, the logic of Figure~\ref{fig:hohhform} is extended to that in Figure~\ref{fig:hohhfoc}. A sequent $\seq[\dyncon{} ; {[D]}]{A}$ is called a \emph{focused sequent} and we call rules with a focused sequent conclusion \emph{focused rules}. The interpretation is that $D$ is a formula under focus. Here we use the same sequent arrow $\vdash$ for both kinds of sequents. Notice that the consequent of the sequent for all focused rules is an atom. This is because, as stated above, we first apply the right-rules until the right side of the sequent is atomic. Missing from the above description was a method to ensure the left rules are applied to a single element of the context until it too is atomic. This is accomplished with the \rl{hhfocus} rule, which acts as a gateway from the right rules to the focused rules. The focused rules appear to be similar to the left rules in Figure~\ref{fig:hohhform}, but now we are focused on a particular formula and are not able to alternate between the elements in the context that we apply left rules to.

An additional optimization is achieved by requiring the right branch of the \rl{$\longrightarrow_R$} rule (a focused sequent) to be fully explored before the left (unfocused sequent). This way, once we are applying left rules, we continue to do so until we have reached a leaf with sequent \seq[\dyncon{} ; {[A']}]{A}. If $A = A'$, then the branch is completed with \rl{match} and we work on the left (unfocused sequent) branch of the last application of the \rl{$\longrightarrow_R$} rule. Otherwise, the branch cannot be completed and a different choice of formula must be focused at the start of the current sequence of focused rule applications. We will illustrate this by showing how we can write a focused derivation of the sequent in example~\ref{expl:hh}.

\begin{expl}[Hereditary Harrop Focused Derivation]
This proof is very similar to the second derivation of example~\ref{expl:hh}, but now we use focused rules where there we used left rules and make use of the \rl{focus} and \rl{match} rules. See below for the derivation tree of \seq[\forall x , (A_1 \; x) \& (A_2 \; x)]{(A_1 \; t) \& (A_2 \; t)} using the focused sequent calculus of Figure~\ref{fig:hohhfoc}.

{\scriptsize
$$
\infer[\rl{$\&_R$}]{\seq[\forall_\tau x , (A_1 \; x) \& (A_2 \; x)]{(A_1 \; t) \& (A_2 \; t)}}{
	\infer[\rl{focus}]{\seq[\forall_\tau x , (A_1 \; x) \& (A_2 \; x)]{A_1 \; t}}{
		\infer[\rl{$\forall_L$}]{\seq[\forall_\tau x , (A_1 \; x) \& (A_2 \; x) ; {[\forall_\tau x , (A_1 \; x) \& (A_2 \; x)]}]{A_1 \; t}}{
			\infer[\rl{$\&_{L_1}$}]{\seq[\forall_\tau x , (A_1 \; x) \& (A_2 \; x) ; {[(A_1 \; t) \& (A_2 \; t)]}]{A_1 \; t}}{
				\infer[\rl{match}]{\seq[\forall_\tau x , (A_1 \; x) \& (A_2 \; x) ; {[A_1 \; t]}]{A_1 \; t}}{}
			}
		}
	}
	&
	\infer[\rl{focus}]{\seq[\forall_\tau x , (A_1 \; x) \& (A_2 \; x)]{A_2 \; t}}{
		\infer[\rl{$\forall_L$}]{\seq[\forall_\tau x , (A_1 \; x) \& (A_2 \; x) ; {[\forall_\tau x , (A_1 \; x) \& (A_2 \; x)]}]{A_2 \; t}}{
			\infer[\rl{$\&_{L_2}$}]{\seq[\forall_\tau x , (A_1 \; x) \& (A_2 \; x) ; {[(A_1 \; t) \& (A_2 \; t)]}]{A_2 \; t}}{
				\infer[\rl{match}]{\seq[\forall_\tau x , (A_1 \; x) \& (A_2 \; x) ; {[A_2 \; t]}]{A_2 \; t}}{}
			}
		}
	}
}
$$
}

Notice that even though we use the \rl{focus} rule to select a formula from the context to focus and used the focused rules to manipulate it, we still retain the original formula in the context.
\end{expl}

Another important observation is that proofs using this focused sequent calculus are forced to be uniform proofs, because we cannot freely choose between applying left or right rules; the search strategy forces us to use the rule introducing the top-level connective of the principal formula of the sequent. This logic is also sound and complete with respect to the logic of Figure~\ref{fig:hohhform}. 

In the next chapter we will present a specification logic that is a slight modification of the sequent calculus of Figure~\ref{fig:hohhfoc}. We modify this logic is because there are rules that are unnecessary for our application of hereditary Harrop formulas and focusing. There are also some implementation details built in to the rules presented later, as will be explained in Chapter~\ref{ch:sl}.

%The negative connectives are those that have invertible right rules in Figure~\ref{fig:hohhform}. These connectives are $\&, \longrightarrow$, and $\top$. The positive connectives have invertible left rules, 



%Maybe some general text about removing redundancies like in Frank's section 2.


%- Then present the rules of Frank's section 3, possibly with modifications to fit your syntax and discuss the rules.


%- Then present the rules of your logic and note the differences. And say where they came from (a modified version of the logic in the Abella paper, with a restriction on the types that can be quantified over, with extensions to include the full set of hereditary Harrop formulas, which include disjunction and existential quantification).




\cleardoublepage



\part{Contributions}

%\notachapter{Notations}

\allowdisplaybreaks[0]

%%%%%%%%%%%%%%%%%%%%%%%%%%%%%%%%%%%%%%%%%%%%%%%%%%
% Specification Logic
%%%%%%%%%%%%%%%%%%%%%%%%%%%%%%%%%%%%%%%%%%%%%%%%%%

\chapter{Specification Logic}
\label{ch:sl}


%The main contribution to Hybrid presented here has been to add a new specification logic (SL) and prove the necessary structural rules. This SL is an inference system based on higher-order hereditary Harrop formulas. (*motivation, names of people whose research this builds on, etc)

%We follow the presentation in~\cite{WCGN:PPDP13} to define this SL in a similar manner as the SL for Abella that is described there. That is, we distinguish between goal-reduction rules and backchaining rules (see figures~\ref{fig:grseq} and~\ref{fig:bcseq}). Goal-reduction rules are the right-introduction rules and reduce a formula to atomic (bottom-up). Backchaining rules are the left-introduction rules and allow backchaining over a formula focused from the context of assumptions at this reasoning level (bottom-up).

%The rules of this logic are encoded in inductive types for goal-reduction and backchaining sequents. Goal-reduction sequents have signature \sltm{grseq} $:$ \sltm{context} $\rightarrow$ \sltm{oo} $\rightarrow$ \coqtm{Prop} and we write \seqsl{\beta} as notation for \sltm{grseq} $\dyncon{}$ $\beta$. Backchaining sequents have signature \sltm{bcseq} $:$ \sltm{context} $\rightarrow$ \sltm{oo} $\rightarrow$ \sltm{atm} $\rightarrow$ \coqtm{Prop} and we write \bchsl{\beta}{\alpha} as notation for \sltm{bcseq} $\dyncon{}$ $\beta$ $\alpha$, understanding $\beta$ to be a formula from \dyncon{} that we focus.

%Something has been ignored in our classification of the rules so far. The rules \rlnmsbc{} and \rlnmsinit{} are presented with the goal-reduction rules in figure~\ref{fig:grseq}, even though they are not used to reduce a goal any further (the conclusion is atomic in these cases). Also, the rule \rlnmbmatch{} is considered a backchaining rule in figure~\ref{fig:bcseq}. The reason for this comes from how these rules are defined in Coq; a rule whose conclusion is a goal-reduction sequent must be defined in \sltm{grseq} and a rule whose conclusion is a backchaining sequent must be defined in \sltm{bcseq}.

%Since \sltm{grseq} references \sltm{bcseq} in the rule \rlnmsinit{} and \sltm{bcseq} references \sltm{grseq} in the rule \rlnmbimp{}, these are defined as mutually inductive types. Both types have a constructor for each rule that has conclusion of their type. The identifier (*right word?) of the constructor is the rule name, and the premises and conclusion of the rule are premises and conclusion of the constructor's type, respectively (with quantification over the necessary parameters in the type). For example, the definition of \sltm{grseq} has a constructor \rlnmsinit{} of type $\forall (L : \sltm{context}) (G : \sltm{oo}), G \in L \rightarrow \bchsl[L]{G}{A} \rightarrow \seqsl[L]{\atom{A}}$.

%The types \seqsl{\beta} and \bchsl{\beta}{\alpha} are dependent types, where $\dyncon{} : \sltm{context}$, $\beta : \sltm{oo}$, and $\alpha : \sltm{atm}$. Before considering the rules of the logic in detail, the types \sltm{atm}, \sltm{oo}, and \sltm{context} defined and used by the SL need to be explained.


The first stage of the contributions outlined in this thesis is defining a specification logic to increase the reasoning power of Hybrid. The new specification logic (SL) for Hybrid is based on hereditary Harrop formulas using an intuitionistic logic with focusing as described in Chapter~\ref{ch:hh}. We adopt a modified version of the rules very close to the style of the rules of the specification logic used in the higher-order version of Abella~\cite{WCGN:PPDP13}. We do not include any rules for disjunction here because they have not been necessary for object logics in case studies of interest. The SL could easily be extended to add these rules and the proofs of SL metatheory would have the same structure, as will be seen in Chapter~\ref{ch:gsl}.

%t the specification level, the terms of the language of hereditary Harrop formulas are the terms of the simply-typed $\lambda$-calculus. 
% When we refer to this logic as higher-order, we mean the
% implicational complexity.
We note that unlike in all previous SLs for Hybrid there is no restriction on the implicational complexity (see~\cite{FeltyMomigliano:JAR10}), because $G$-formulas in higher-order hereditary Harrop language allow $D$-formulas as the antecedent of implication as was seen in Section~\ref{sec:hohh}. In all previous SLs, only atomic formulas were allowed in place of the more general D-formulas allowed here.
%

The SL presented in this chapter is a sequent calculus implemented as an inductive type in Coq. Section~\ref{sec:context} describes how contexts are defined for this SL. Section~\ref{sec:hhsl} presents the Coq implementation of the SL based on hereditary Harrop formulas and we see how to prove properties of this SL by structural induction in Section~\ref{sec:induction}.

In Appendix~\ref{ch:notations}, we list notations that will be used in the rest of the thesis.

\section{Contexts in Coq}
\label{sec:context}

The type \sltm{context} represents contexts of assumptions in sequents and is defined using the Coq \coqtm{ensemble} library as \coqtm{ensemble} \sltm{oo} since we want contexts to behave as sets with elements of type \sltm{oo}. In proofs of some context lemmas stated below we use the \coqtm{ensemble} extensional equality axiom:
$$
\coqtm{Extensionality\_Ensembles} : \forall (E_1 \; E_2 : \coqtm{ensemble}),(\coqtm{Same\_set} \; E_1 \; E_2) \rightarrow E_1 = E_2
$$
where \coqtm{Same\_set} is defined in the \coqtm{Ensemble} library. We use $o \in c$ as notation for $\coqtm{elem} \; o \; c$ which means formula $o$ is an element of context $c$. Context subset, written $\dyncon{}_1 \subseteq \dyncon{}_2$, is defined as $\forall (o : \sltm{oo}), o \in \dyncon{}_1 \rightarrow o \in \dyncon{}_2$.

We write ($\dyncon{}, \beta$) as notation for ($\sltm{context\_cons} \; \dyncon{} \; \beta$). We write write $c$ or $\dyncon{}$ to denote contexts when discussing formalized proofs.

The context lemmas below are proven as part of this work and are used in later proofs in this thesis. See the accompanying source code for the proofs. Note that all variables are externally quantified and each occurrence of $\beta$ and $\dyncon{}$, possibly with subscripts, has type \sltm{oo} and \sltm{context}, respectively.
\begin{lemma}[\sltm{elem\_inv}] %$\forall (c : \mathtt{context}) (o_1 \; o_2 : \sltm{oo}), o_1 \in (c , o_2) \rightarrow (o_1 \in c \vee o_1 = o_2)$
\label{lem:elem_inv}
$$
\vcenter{\infer{(\beta_1 \in \dyncon{}) \vee (\beta_1 = \beta_2)}{\beta_1 \in (\dyncon{}, \beta_2)}}
$$
\end{lemma}

\begin{lemma}[\sltm{elem\_sub}] %$\forall (c : \mathtt{context}) (o_1 \; o_2 : \sltm{oo}),  \rightarrow $
$$
\vcenter{\infer{\beta_1 \in (\dyncon{} , \beta_2)}{\beta_1 \in \dyncon{}}}
$$
\end{lemma}

\begin{lemma}[\sltm{elem\_self}] %$\forall (c : \mathtt{context}) (o : \sltm{oo}), o \in (c , o)$
\label{lem:elem_self}
$$
\vcenter{\infer{\beta \in (\dyncon{} , \beta)}{}}
$$
\end{lemma}

\begin{lemma}[\sltm{elem\_rep}] %$\forall (c : \mathtt{context}) (o_1 \; o_2 : \sltm{oo}),  \rightarrow $
$$
\vcenter{\infer{\beta_1 \in (\dyncon{} , \beta_2)}{\beta_1 \in (\dyncon{} , \beta_2 , \beta_2)}}
$$
\end{lemma}

\begin{lemma}[\sltm{context\_swap}] %$\forall (c : \mathtt{context}) (o_1 \; o_2 : \sltm{oo}), (c , o_1 , o_2) = (c , o_2 , o_1)$
$$
\vcenter{\infer{(\dyncon{} , \beta_1 , \beta_2) = (\dyncon{} , \beta_2 , \beta_1)}{}}
$$
\end{lemma}

\begin{lemma}[\sltm{context\_sub\_sup}] %$\forall (c_1 \; c_2 : \mathtt{context}) (o : \mathtt{context}), c_1 \subseteq c_2 \rightarrow (c_1 , o) \subseteq (c_2 , o)$
\label{lem:context_sub_sup}
$$
\vcenter{\infer{(\dyncon{}_1 , \beta) \subseteq (\dyncon{}_2 , \beta)}{\dyncon{}_1 \subseteq \dyncon{}_2}}
$$
\end{lemma}

\section{Hereditary Harrop Specification Logic in Coq}
\label{sec:hhsl}

%We follow the description in~\cite{WCGN:PPDP13} of a specification logic for Abella based on hereditary Harrop formulas and as described in Chapter~\ref{ch:hh}.
The inference rules of the SL are implemented using two sequent judgments that distinguish between \emph{goal-reduction rules}\index{goal-reduction rule} and \emph{backchaining rules}\index{backchaining rule} which correspond to the right rules and left focused rules, respectively, of Figure~\ref{fig:hohhfoc} in Section~\ref{sec:focusing}. %To differentiate sequents of this implemented inference system, we use $\triangleright$ as the sequent arrow rather than $\vdash$. Goal-reduction sequents have the form \seqsl{G} and backchaining sequents are written \bchsl{D}{A}, where the latter is a left focusing judgment with $D$ the formula under (left) focus.
Figures~\ref{fig:grseq} and~\ref{fig:bcseq} implement the inference rules of the SL (except for the disjunction rules, as mentioned in the introduction to this chapter). They are encoded in Coq as two mutually inductive types, one each for goal-reduction and backchaining sequents. The syntax used in the figures is a pretty-printed version of the Coq inductive types \sltm{grseq} and \sltm{bcseq}. Goal-reduction sequents have type $\sltm{grseq} : \sltm{context} \rightarrow \sltm{oo} \rightarrow \coqtm{Prop}$, and we write $\seqsl{\beta}$ as notation for $\sltm{grseq}~\dyncon{}~ \beta$. Backchaining sequents have type $\sltm{bcseq} : \sltm{context} \rightarrow \sltm{oo} \rightarrow \sltm{atm} \rightarrow \coqtm{Prop}$ and we write \bchsl{\beta}{\alpha} as notation for $\sltm{bcseq}~ \dyncon{}~ \beta~ \alpha$, understanding $\beta$ to be the focused formula from \dyncon{}. The symbol $\forall$ is used for universal quantification in Coq, rather than universal quantification in SL formulas. When we see $\forall$ in the premises of rules, this is to make it clear that it is only over the premise of the rule. 

The rule names in the figures are the constructor names in the inductive definitions in the Coq files. The premises and conclusion of a rule are the argument types and the target type, respectively, of one clause in the definition. Quantification at the outer level is implicit and, as noted, inner quantification is written explicitly in the premises. For example, the linear format of the \rlnmsinit{} rule from Figure~\ref{fig:grseq} with all quantifiers explicit is
$$
\forall (\Gamma : \sltm{context}) (D : \sltm{oo}) (A :\sltm{atm}),D \in \Gamma \rightarrow \bchsl[\Gamma]{D}{A} \rightarrow \seqsl[\Gamma]{\atom{A}}
$$
This is the type of the \rlnmsinit{} constructor in the inductive definition of $\sltm{grseq}$ (see the definition of $\sltm{grseq}$ in the Coq files).

%A number of implementation details can be seen in the rules. 
The notation \atom{A} is to say that $A : \coqtm{atm}$ is coerced to have type \coqtm{oo}, where \coqtm{oo} is the implemented type of formulas (see Figure~\ref{fig:oofig}), referred to as $o$ in Chapter~\ref{ch:hh}. The constants \coqtm{Some} and \coqtm{All} are used for existential and universal quantification in SL formulas, respectively, over the type \coqtm{expr con} which is the type for OL expressions. \coqtm{Allx} is a constant for universal quantification over a type \coqtm{X} of type \coqtm{Set}. We juxtapose appropriate terms to denote application since Coq will reduce the expression, rather than explicitly writing the substitution (for example, compare rule \rl{$\forall_R$} in Figure~\ref{fig:hohhfoc} to rule \rlnmsalls{} in Figure~\ref{fig:grseq}). A final implementation byproduct is the predicate \coqtm{proper} appearing in the premise of some rules. This is used in the Hybrid library to rule out terms of type \coqtm{expr} that have dangling indices (see~\cite{FeltyMomigliano:JAR10}).

In the sequents for this SL there is also an implicit fixed context $\Delta$, called the \emph{static program clauses}, containing closed clauses ($D$-formulas) of the form
$$
\forall_{\tau_1}\ldots\forall_{\tau_n}.G\longrightarrow A
$$
with $n\ge0$. Any set of $D$-formulas can be transformed to an equivalent one that all have this form. These clauses represent the inference rules of an OL. We do not explicitly mention $\Delta$ in the rules for this SL because no rules modify it.

%Our encoding of the formulas of the SL in Coq is shown in Figure~\ref{fig:oofig}. Since \sltm{oo} is an inductive type, Coq will automatically generate the induction principle shown in Figure (*TODO) as discussed in Section~\ref{sec:coqinduction}.

%
%\begin{figure}
%\begin{lstlisting}
%Inductive oo : Type :=
%| atom : atm -> oo
%| T : oo
%| Conj : oo -> oo -> oo
%| Imp : oo -> oo -> oo
%| All : (expr con -> oo) -> oo
%| Allx : (X -> oo) -> oo
%| Some : (expr con -> oo) -> oo.
%\end{lstlisting}
%\centering{\caption{Type of SL Formulas \label{fig:oofig}}}
%\end{figure}
%
%In this implementation, the type \sltm{atm} is a parameter to the definition of \sltm{oo} and is used to define the predicates needed for reasoning about a particular OL. For instance, our above example might include a predicate $\oltm{hodb} : (\mltm{expr}~ \mltm{con}) \rightarrow \coqtm{nat} \rightarrow (\mltm{expr}~ \mltm{con})$ relating the higher-order and de Bruijn encodings at a given depth. The constant \sltm{atom} coerces an atomic formula (a predicate applied to its arguments) to an SL formula. Also, note that in this implementation, we restrict the type of universal quantification to two types, (\mltm{expr}~ \mltm{con}) and \mltm{X}, where \mltm{X} is a parameter that can be instantiated with any primitive type; in our running example, \mltm{X} would become \coqtm{nat} for the depth of binding in a de Bruijn term.  We also leave out disjunction. It is not difficult to extend our implementation to include disjunction and quantification (universal or existential) over other primitive types, but these have not been needed in reasoning about OLs.

%We write \atom{\alpha}, ($\beta_1$ \& $\beta_2$), and ($\beta_1 \longrightarrow \beta_2$) as notation for (\sltm{atom} $\alpha$), (\sltm{Conj} $\beta_1$ $\beta_2$), and (\sltm{Imp} $\beta_1$ $\beta_2$), respectively. Note that we write $\beta$ or $\delta$ for formulas (type \sltm{oo}), and $\alpha$ for elements of type \sltm{atm}, possibly with subscripts. When discussing proofs, we also write $o$ for formulas and $a$ for atoms. When we want to make explicit when a formula is a goal or clause, we write $G$ or $D$, respectively. Formulas quantified by \sltm{All} are written $(\sltm{All}~ \beta)$ or $(\sltm{All}~ \lambda (x:\mltm{expr}~ \mltm{con}) \; . \; \beta x)$. The latter is the $\eta$-long form with types included explicitly. The other quantifiers are treated similarly.

{
\renewcommand{\arraystretch}{3.5}
\newcommand{\GRrlsbc}{\inferH[\rlnmsbc{}]{\seqsl[\Gamma]{\atom{A}}}{\prog{A}{G} & \seqsl[\Gamma]{G}}}
\newcommand{\GRrlsinit}{\inferH[\rlnmsinit{}]{\seqsl[\Gamma]{\atom{A}}}{D \in \Gamma & \bchsl[\Gamma]{D}{A}}}
\newcommand{\GRrlst}{\inferH[\rlnmst{}]{\seqsl[\Gamma]{\sltm{T}}}{}}
\newcommand{\GRrlsand}{\inferH[\rlnmsand{}]{\seqsl[\Gamma]{G_1 \, \& \, G_2}}{\seqsl[\Gamma]{G_1} & \seqsl[\Gamma]{G_2}}}
\newcommand{\GRrlsimp}{\inferH[\rlnmsimp{}]{\seqsl[\Gamma]{D \longrightarrow G}}{\seqsl[\Gamma \, , \, D]{G}}}
\newcommand{\GRrlsall}{\inferH[\rlnmsall{}]{\seqsl[\Gamma]{\sltm{All} \; G}}{\forall (E : \hybridtm{expr con}), (\sltm{proper} \; E \rightarrow \seqsl[\Gamma]{G \, E})}}
\newcommand{\GRrlsalls}{\inferH[\rlnmsalls{}]{\seqsl[\Gamma]{\sltm{Allx} \; G}}{\forall (E : \sltm{X}), (\seqsl[\Gamma]{G \, E})}}
\newcommand{\GRrlssome}{\inferH[\rlnmssome{}]{\seqsl[\Gamma]{\sltm{Some} \; G}}{\sltm{proper} \; E & \seqsl[\Gamma]{G \, E}}}

\begin{figure}%[h]
$$
\begin{tabular}{c c c}
\GRrlsbc{}
&
\GRrlsinit{}
&
\GRrlst{} \\
%
\GRrlsand{}
&
\GRrlsimp{}
&
\GRrlssome{} \\
%
\multicolumn{3}{c}{
\GRrlsall{} \;\;\; \GRrlsalls{}
}
\end{tabular}
$$
\centering{\caption{Goal-Reduction Rules, $\sltm{grseq} : \sltm{context} \rightarrow \sltm{oo} \rightarrow \coqtm{Prop}$ \label{fig:grseq}}}

\end{figure}
}
%
{
\renewcommand{\arraystretch}{3.5}
\newcommand{\BCrlbmatch}{\inferH[\rlnmbmatch{}]{\bchsl[\Gamma]{\atom{A}}{A}}{}}
\newcommand{\BCrlbanda}{\inferH[\rlnmbanda{}]{\bchsl[\Gamma]{D_1 \, \& \, D_2}{A}}{\bchsl[\Gamma]{D_1}{A}}}
\newcommand{\BCrlbandb}{\inferH[\rlnmbandb{}]{\bchsl[\Gamma]{D_1 \, \& \, D_2}{A}}{\bchsl[\Gamma]{D_2}{A}}}
\newcommand{\BCrlbimp}{\inferH[\rlnmbimp{}]{\bchsl[\Gamma]{G \longrightarrow D}{A}}{\seqsl[\Gamma]{G} & \bchsl[\Gamma]{D}{A}}}
\newcommand{\BCrlball}{\inferH[\rlnmball{}]{\bchsl[\Gamma]{\sltm{All} \; D}{A}}{\sltm{proper} \; E & \bchsl[\Gamma]{D \, E}{A}}}
\newcommand{\BCrlballs}{\inferH[\rlnmballs{}]{\bchsl[\Gamma]{\sltm{Allx} \; D}{A}}{\bchsl[\Gamma]{D \, E}{A}}}
\newcommand{\BCrlbsome}{\inferH[\rlnmbsome{}]{\bchsl[\Gamma]{\sltm{Some} \; D}{A}}{\forall (E : \hybridtm{expr con}), (\sltm{proper} \; E \rightarrow \bchsl[\Gamma]{D \, E}{A})}}

\begin{figure}%[h]

$$
\begin{tabular}{c c c}
\BCrlbmatch{}
&
\BCrlbanda{}
&
\BCrlbandb{} \\
%
\BCrlbimp{}
&
\BCrlball{}
&
\BCrlballs{} \\
\multicolumn{3}{c}{
\BCrlbsome{}
}
\end{tabular}
$$
\centering{\caption{Backchaining Rules, $\sltm{bcseq} : \sltm{context} \rightarrow \sltm{oo} \rightarrow \sltm{atm} \rightarrow \coqtm{Prop}$ \label{fig:bcseq}}}

\end{figure}
}
%
The goal-reduction rules of Figure~\ref{fig:grseq} are implemented version of the right introduction rules of this sequent calculus as seen in Figure~\ref{fig:hohhfoc}. The rules \rlnmsbc{} and \rlnmsinit{} are the only goal-reduction rules with an atomic principal formula.

The rule \rlnmsbc{} is used to backchain over the static program clauses $\Delta$, which are defined for each new OL as an inductive type called \sltm{prog} of type $\sltm{atm} \rightarrow \sltm{oo} \rightarrow \coqtm{Prop}$, and represent the inference rules of the OL (this is discussed further in Section~\ref{sec:hybridsl}). The rule \rlnmsbc{} is the interface between the SL and OL layers and we say that the SL is parametric in OL provability. We write \prog{A}{G} for $(\sltm{prog} \; A \; G)$ to suggest backward implication. Recall that clauses in $\Delta$ may have outermost universal quantification. The premise \prog{A}{G} actually represents an instance of a clause in $\Delta$.

The rule \rlnmsinit{} allows backchaining over dynamic assumptions (i.e. a formula from \dyncon{}) and is the implemented version of the \rl{focus} rule of Figure~\ref{fig:hohhfoc}. To use this rule to prove \seqsl{\atom{A}}, we need to show $D \in \dyncon{}$ and \bchsl{D}{A}. Formula $D$ is chosen from, or shown to be in, the dynamic context \dyncon{} and we use the backchaining rules of Figure~\ref{fig:bcseq} to show \bchsl{D}{A} (where $D$ is the focused formula).

The backchaining rules of Figure~\ref{fig:bcseq} are the implemented version of the standard focused left rules for conjunction, implication, and universal and existential quantification seen in Figure~\ref{fig:hohhfoc}. Considered bottom up, they provide backchaining over the focused formula. In using the backchaining rules, each branch is either completed by \rlnmbmatch{} where the focused formula is an atomic formula identical to the goal of the sequent, or \rlnmbimp{} is used resulting in one branch switching back to using goal-reduction rules.

%We mention several Coq tactics when presenting proofs. The main one is the \coqtm{constructor} tactic, which applies a clause of an inductive definition in a backward direction (a step of meta-level backchaining), determining automatically which clause to apply.
%The \coqtm{apply} tactic also does a step of backchaining and takes as
%argument the name of a definition clause, hypothesis, or lemma.  The
%\coqtm{assumption} tactic is the ``base case'' for \coqtm{apply},
%closing the proof when there are no further subgoals.


%(*cut?) Note that the rules \rlnmsbc{} and \rlnmsinit{} are presented with the goal-reduction rules in Figure~\ref{fig:grseq}, even though they are not used to reduce a goal any further (the conclusion is atomic in these cases). Also, the rule \rlnmbmatch{} is considered a backchaining rule in Figure~\ref{fig:bcseq}. The reason for this comes from how these rules are defined in Coq; a rule whose conclusion is a goal-reduction sequent must be defined in \sltm{grseq} and a rule whose conclusion is a backchaining sequent must be defined in \sltm{bcseq}.


%Recall that Coq's dependent products are written $\forall(x_1:t_1)\cdots(x_n:t_n),M$, where $n\ge0$ and for $i=1,\ldots, n$, $x_i$ may appear free in $x_{i+1},\ldots,x_n,M$.  If it doesn't, implication can be used as an abbreviation, e.g., the premise of the \rlnmsall{} rule is an abbreviation for $\forall (E : \hybridtm{expr con})(H:\sltm{proper} \; E), (\seqsl[\Gamma]{G \, E})$.

\section{Mutual Structural Induction}
\label{sec:induction}

%
%This means that there will be a subcase for each rule/constructor, and every rule with a goal-reduction premise will have an induction hypothesis and every rule with a backchaining premise will have an induction hypothesis (two when both are present). \\
%
%Suppose the goal is to prove
Our theorem statements will often have the form
\begin{align*}
(\forall \; (c : \sltm{context}) & \; (o : \sltm{oo}), (\seqsl[c]{o}) \rightarrow (P_1 \; c \; o)) \;\; \wedge \\
(\forall \; (c : \sltm{context}) & \; (o : \sltm{oo}) \; (a : \sltm{atm}), (\bchsl[c]{o}{a}) \rightarrow (P_2 \; c \; o \; a))
\end{align*}
where we extract predicates $P_1 :$ \sltm{context} $\rightarrow$ \sltm{oo} $\rightarrow$ \coqtm{Prop} and $P_2 :$ \sltm{context} $\rightarrow$ \sltm{oo} $\rightarrow$ \sltm{atm} $\rightarrow$ \coqtm{Prop} from the statement to be proven. We can generate an induction principle over the mutually inductive sequent types to allow proof by mutual structural induction. This is done using the Coq \coqtm{Scheme} command.

To prove a statement of the above form by mutual structural induction over \seqsl[c]{o} and \bchsl[c]{o}{a}, 15 subcases must be proven, one corresponding to each inference rule of the SL. The proof state of each subcase of this induction is constructed from an inference rule of the system.
%as follows (where there is fresh quantification over all parameters):
%
We can see a snippet of the sequent mutual induction principle in Figure~\ref{fig:seqind}, where each antecedent (clause of the induction principle defining the cases) corresponds to a rule of the SL and a subcase for an induction using this technique. 
After applying the induction principle, the subcases are generated and
externally quantified variables in each antecedent are introduced to the context of assumptions of the proof state and are then considered \emph{signature variables}.
\begin{figure}%[h]
%\vspace{-20pt}
\begin{align*}
& \sltm{seq\_mutind} : \forall (P_1 : \sltm{context} \rightarrow \sltm{oo} \rightarrow \coqtm{Prop}) \\
& \qquad\qquad\qquad (P_2 : \sltm{context} \rightarrow \sltm{oo} \rightarrow \sltm{atm} \rightarrow \coqtm{Prop}), \\
% g_dyn:
& (* \rlnmsinit{} *) \quad (\forall (c : \sltm{context}) (o : \sltm{oo}) (a : \sltm{atm}), \\
& \qquad\qquad\qquad o \in c \rightarrow \bchsl[c]{o}{a} \rightarrow P_2 \; c \; o \; a \rightarrow P_1 \; c \; \atom{a}) \rightarrow \\
% g_all:
& (* \rlnmsall{} *) \quad (\forall (c : \sltm{context}) (o : \sltm{expr con} \rightarrow \sltm{oo}), \\
& \qquad\qquad\qquad (\forall (e : \sltm{expr con}), \hybridtm{proper} \; e \rightarrow \seqsl[c]{o \; e} \rightarrow \\
& \qquad\qquad\qquad (\forall (e : \sltm{expr con}), \hybridtm{proper} \; e \rightarrow P_1 \; c \; (o \; e) \rightarrow \\
& \qquad\qquad\qquad P_1 \; c \; (\sltm{All} \; o)) \rightarrow \\
% b_imp:
& (* \rlnmbimp{} *) \quad (\forall (c : \sltm{context}) (o_1 \; o_2 : \sltm{oo}) (a : \sltm{atm}), \\
& \qquad\qquad\qquad \seqsl[c]{o_1} \rightarrow P_1 \; c \; o_1 \rightarrow \bchsl[c]{o_2}{a} \rightarrow P_2 \; c \; o_2 \; a \rightarrow \\
& \qquad\qquad\qquad P_2 \; c \; (o_1 \longrightarrow o_2) \; a)   \rightarrow \\
& \quad \cdots\\
& \quad (\forall (c : \; \sltm{context}) (o : \sltm{oo}), \seqsl[c]{o} \rightarrow P_1 \; c \; o) \; \wedge \\
& \quad (\forall (c : \; \sltm{context}) (o : \sltm{oo}) (a : \sltm{atm}), \bchsl[c]{o}{a} \rightarrow P_2 \; c \; o \; a)
\end{align*}
\centering{\caption{Sequent Mutual Induction Principle Snippet \label{fig:seqind}}}
%\vspace{-20pt}
\end{figure}

This induction principle is automatically generated following the description shown below, with examples from the figure given in each point.
\begin{itemize}
 \item Non-sequent premises are assumptions of the induction subcase (e.g. $o \in c$ from the \rlnmsinit{} rule).
 \item For every rule premise that is a goal-reduction sequent (with possible local quantifiers) of the form $\forall (x_1 : T_1) \cdots (x_n : T_n), \seqsl{\beta}$ where $n \geq 0$, the induction subcase has assumptions ($\forall (x_1 : T_1) \cdots (x_n : T_n), \seqsl{\beta}$) and ($\forall (x_1 : T_1) \cdots (x_n : T_n), P_1 \; \dyncon{} \; \beta$) (e.g. $\forall (e : \sltm{expr con}), \sltm{proper} \; e \rightarrow \seqsl[c]{o \; e}$ and $\forall (e : \sltm{expr con}), \sltm{proper} \; e \rightarrow P_1 \; c \; (o \; e)$ from the \rlnmsall{} rule with $n = 2$ and unabbreviated prefix $\forall (e : \sltm{expr con}) (H : \sltm{proper} \; e)$).
 \item For every rule premise that is a backchaining sequent (with possible local quantifiers) of the form $\forall (x_1 : T_1) \cdots (x_n : T_n), \bchsl{\beta}{\alpha}$ where $n \geq 0$, the induction subcase has assumptions ($\forall (x_1 : T_1) \cdots (x_n : T_n), \bchsl{\beta}{\alpha}$) and ($\forall (x_1 : T_1) \cdots (x_n : T_n), P_2 \; \dyncon{} \; \beta \; \alpha$) (e.g. $\bchsl[c]{o_2}{a}$ and $(P_2 \; c \; o_2 \; a)$ from the \rlnmbimp{} rule).
 \item If the rule conclusion is a goal-reduction sequent of the form \seqsl{\beta}, then the subcase goal is $P_1 \; \dyncon{} \; \beta$ (e.g. $(P_1 \; c \; \atom{a})$ from the \rlnmsinit{} rule). 
 \item If the rule conclusion is a backchaining sequent of the form \bchsl{\beta}{\alpha}, then the subcase goal is $P_2 \; \dyncon{} \; \beta \; \alpha$ (e.g. $(P_2 \; c \; (o_1 \longrightarrow o_2) \; a)$ from the \rlnmbimp{} rule).
\end{itemize}
Implicit in these last two points is the possible introduction of more assumptions, in the case when $P_1$ and $P_2$ are dependent products themselves (i.e. 
contain quantification and/or implication).
We will refer to assumptions introduced this way as \emph{induction assumptions} in future proofs, since they are from a predicate that is used to construct induction hypotheses. That is, assumptions of the form $(P_1 \; \dyncon{} \; \beta)$ or $(P_2 \; \dyncon{} \; \beta \; \alpha)$ are induction hypotheses for any proof subcase for a rule with premises \seqsl{\beta} or \bchsl{\beta}{\alpha}. In this SL, exactly two cases of this induction principle have more than one induction hypothesis (\rlnmbimp{} and \rlnmsand{}).

%Given specific $P_1$ and $P_2$, if we are trying to prove $(\forall (c
%: \sltm{context}) (o : \sltm{oo}), \seqsl[c]{o} \rightarrow P_1 \; c
%\; o) \; \wedge \; (\forall (c : \sltm{context}) (o : \sltm{oo}) (a :
%\sltm{atm}), \bchsl[c]{o}{a} \rightarrow P_2 \; c \; o \; a)$,
In describing proofs, we will follow the Coq style and write the proof state in a vertical format with the assumptions above a horizontal line and the goal below it. For example, the \rlnmsinit{} subcase will have the following form:
\begin{align*}
H_1 &: o \in c \\
H_2 &: \bchsl[c]{o}{a} \\
\mathit{IH} &: P_2 \; c \; o \; a \\[\pfshift{}]
\cline{1-2}
&P_1 \; c \; \atom{a}
\end{align*}
As in Coq, we provide hypothesis names (so that we can refer to them as needed). Also, we often omit the type declarations of signature variables, in this case $c : \sltm{context}, o:\sltm{oo},$ and $a : \sltm{atm}$, when they can be easily inferred from context. Unlike in Coq, when we have multiple subcases to prove with the same context of assumptions we will write them all under the horizontal line in the same proof state, separated by commas.


%At a higher-level of abstraction, we note that all rules of the SL
%have one of the two following forms:


\cleardoublepage

%%%%%%%%%%%%%%%%%%%%%%%%%%%%%%%%%%%%%%%%%%%%%%%%%%
% Proofs of Properties of SL
%%%%%%%%%%%%%%%%%%%%%%%%%%%%%%%%%%%%%%%%%%%%%%%%%%

\chapter{Specification Logic Metatheory}
\label{ch:slind}

%{


Proving admissibility of structural rules of a specification logic (SL) frees us from defining them as axiomatic and having to make external justifications for such axioms. We prove admissibility of the structural rules of contraction, weakening, exchange, and cut for both goal-reduction and backchaining sequents. Once proven at the specification level, they can be reused for any OL using this SL. Cut admissibility is particularly useful and considerably more challenging to prove than the other structural rules. It establishes consistency and also provides justification for substituting a formula for an assumption in a context of assumptions. It can greatly simplify reasoning about OLs in systems that provide HOAS.

We can prove properties of this logic using the mutual structural induction principle over the rules of the SL from Figure~\ref{fig:seqind} when the theorem (or goal statement) is the same form as the conclusion of the induction principle. Backchaining over the induction principle, we will have fifteen subcases; one subcase corresponding to each rule of the SL. Many of these cases have similar proofs. We will look at a few cases that are interesting for the following reasons:
\begin{description}
 \item[\rlnmsinit{}] ~\\
  This rule has a goal-reduction sequent conclusion, a non-sequent premise depending on the context of the conclusion and a backchaining sequent premise.
 \item[\rlnmsimp{}] ~\\
  This rule has a goal-reduction sequent conclusion and a sequent premise with a context different from that of the conclusion.
 \item[\rlnmbimp{}] ~\\
  This rule has a backchaining sequent conclusion and both a goal-reduction and backchaining sequent premise.
\end{description}

\section{Structural Rules}
\label{sec:structsl}
\index{structural rules}

For our SL we prove the standard structural rules of weakening, contraction, and exchange for both goal-reduction and backchaining sequents: \index{weakening}\index{contraction}\index{exchange}
\begin{theorem}[\sltm{gr\_weakening}]
\label{thm:gr_weakening}
$$
\vcenter{\infer{\seqsl[\dyncon{} \, , \beta_1]{\beta_2}}{\seqsl{\beta_2}}}
$$
\end{theorem}

\smallskip

\begin{theorem}[\sltm{bc\_weakening}]
\label{thm:bc_weakening}
$$
\vcenter{\infer{\bchsl[\dyncon{} \, , \beta_1]{\beta_2}{\alpha}}{\bchsl{\beta_2}{\alpha}}}
$$
\end{theorem}

\smallskip

\begin{theorem}[\sltm{gr\_contraction}]
$$
\vcenter{\infer{\seqsl[\dyncon{} \, , \beta_1]{\beta_2}}{\seqsl[\dyncon{} \, , \, \beta_1 \, , \beta_1]{\beta_2}}}
$$
\end{theorem}

\smallskip

\begin{theorem}[\sltm{bc\_contraction}]
$$
\vcenter{\infer{\bchsl[\dyncon{} \, , \, \beta_1]{\beta_2}{\alpha}}{\bchsl[\dyncon{} \, , \, \beta_1 \, , \, \beta_1]{\beta_2}{\alpha}}}
$$
\end{theorem}

\smallskip

\begin{theorem}[\sltm{gr\_exchange}]
$$
\vcenter{\infer{\seqsl[\dyncon{} \, , \beta_1 \, , \, \beta_2]{\beta_3}}{\seqsl[\dyncon{} \, , \, \beta_2 \, , \, \beta_1]{\beta_3}}}
$$
\end{theorem}

\smallskip

\begin{theorem}[\sltm{bc\_exchange}]
$$
\vcenter{\infer{\bchsl[\dyncon{} \, , \, \beta_1 \, , \, \beta_2]{\beta_3}{\alpha}}{\bchsl[\dyncon{} \, , \, \beta_2 \, , \, \beta_1]{\beta_3}{\alpha}}}
$$
\end{theorem}

\bigskip

\noindent These are all corollaries of a general theorem:

\begin{theorem}[\sltm{monotone}]
$$
\vcenter{\infer{\seqsl[\inddyncon{}]{\beta}}{\dyncon{} \subseteq \inddyncon{} & \seqsl[\dyncon{}]{\beta}}} \;\; \mathrm{and} \;\; \vcenter{\infer{\bchsl[\inddyncon{}]{\beta}{\alpha}}{\dyncon{} \subseteq \inddyncon{} & \bchsl[\dyncon{}]{\beta}{\alpha}}}
$$
\label{thm:monotone}
\end{theorem}

\begin{proof}

Theorem~\ref{thm:monotone} is proven by mutual structural induction over the premises \seqsl{\beta} and \bchsl{\beta}{\alpha}. Defining $P_1$ and $P_2$ as
\begin{align*}
P_1 :=& \lambda \; (c : \sltm{context}) (o : \sltm{oo}) \; . \\
& \qquad \forall \; (\inddyncon{} : \sltm{context}), c \subseteq \inddyncon{} \rightarrow \seqsl[\inddyncon{}]{o} \\
P_2 :=& \lambda \; (c : \sltm{context}) (o : \sltm{oo}) (a : \sltm{atm}) \; . \\
& \qquad \forall \; (\inddyncon{} : \sltm{context}), c \subseteq \inddyncon{} \rightarrow \bchsl[\inddyncon{}]{o}{a}
\end{align*}
we are proving
\begin{align*}
(\forall \; (c : \sltm{context}) & \; (o : \sltm{oo}), \\
& (\seqsl[c]{o}) \rightarrow (P_1 \; c \; o)) \;\; \wedge \\
(\forall \; (c : \sltm{context}) & \; (o : \sltm{oo}) \; (a : \sltm{atm}), \\
& (\bchsl[c]{o}{a}) \rightarrow (P_2 \; c \; o \; a))
\end{align*}
which has the form discussed in Section~\ref{sec:induction}, so the mutual structural induction principle may be used. Here we will show the cases for the rules \rlnmsinit{}, \rlnmsimp{}, and \rlnmbimp{}. The antecedent of the induction principle for each subcase gives the initial subgoals.\\

\paragraph{Case $\vcenter{\rlsinit{}}$:} ~\\

\smallskip

This rule has one non-sequent premise and one backchaining sequent premise. So there will be one induction hypothesis from the backchaining sequent premise. From the induction principle in Figure~\ref{fig:seqind} we need to prove
\begin{align*}
\forall (c : \sltm{context}) & (o : \sltm{oo}) (a : \sltm{atm}), \\
& o \in c \rightarrow \bchsl[c]{o}{a} \rightarrow P_2 \; c \; o \; a \rightarrow P_1 \; c \; \atom{a}
\end{align*}
After introductions the proof state is
\begin{align*}
H_1 &: o \in c \\
\mathit{Hb}_1 &: \bchsl[c]{o}{a} \\
\mathit{IHb}_1 &: P_2 \; c \; o \; a \\[\pfshift{}]
\cline{1-2}
& P_1 \; c \; \atom{a}
\end{align*}
Unfolding $P_1$ and $P_2$ as defined for this theorem, we have
\begin{align*}
H_1 &: o \in c \\
\mathit{Hb}_1 &: \bchsl[c]{o}{a} \\
\mathit{IHb}_1 &: \forall \; (\inddyncon{} : \sltm{context}), c \subseteq \inddyncon{} \rightarrow \bchsl[\inddyncon{}]{o}{a} \\[\pfshift{}]
\cline{1-2}
& \forall \; (\inddyncon{} : \sltm{context}), c \subseteq \inddyncon{} \rightarrow \seqsl[\inddyncon{}]{\atom{a}}
\end{align*}
Next we make introductions from the goal.
\begin{align*}
H_1 &: o \in c \\
\mathit{Hb}_1 &: \bchsl[c]{o}{a} \\
\mathit{IHb}_1 &: \forall \; (\inddyncon{} : \sltm{context}), c \subseteq \inddyncon{} \rightarrow \bchsl[\inddyncon{}]{o}{a} \\
\inddyncon{} &: \sltm{context} \\
P_1 &: c \subseteq \inddyncon{} \\[\pfshift{}]
\cline{1-2}
& \seqsl[\inddyncon{}]{\atom{a}}
\end{align*}
Now the goal is a goal-reduction sequent with an atomic formula. We can backchain with the rule \rlnmsinit{} and will get two new subgoals from the premises of this rule.
\begin{align*}
H_1 &: o \in c \\
\mathit{Hb}_1 &: \bchsl[c]{o}{a} \\
\mathit{IHb}_1 &: \forall \; (\inddyncon{} : \sltm{context}), c \subseteq \inddyncon{} \rightarrow \bchsl[\inddyncon{}]{o}{a} \\
\inddyncon{} &: \sltm{context} \\
P_1 &: c \subseteq \inddyncon{} \\[\pfshift{}]
\cline{1-2}
& (o \in \inddyncon{}), (\bchsl[\inddyncon{}]{o}{a})
\end{align*}
To prove the second subgoal we use induction hypothesis $\mathit{IHb}_1$ to get the new subgoal $c \subseteq \inddyncon{}$ which is provable by induction assumption $P_1$. To prove the first, we need to unfold the definition of subset in $P_1$.
\begin{align*}
H_1 &: o \in c \\
\mathit{Hb}_1 &: \bchsl[c]{o}{a} \\
\mathit{IHb}_1 &: \forall \; (\inddyncon{} : \sltm{context}), c \subseteq \inddyncon{} \rightarrow \bchsl[\inddyncon{}]{o}{a} \\
\inddyncon{} &: \sltm{context} \\
P_1 &: \forall (o : \sltm{oo}), o \in c \rightarrow o \in \inddyncon{} \\[\pfshift{}]
\cline{1-2}
& o \in \inddyncon{}
\end{align*}
Backchaining over $P_1$ we get the new subgoal $o \in c$ which is provable by assumption $H_1$. The proof for this case is complete.

\paragraph{Case $\vcenter{\rlsimp{}}$:} ~\\

This rule has one goal-reduction sequent premise which gives one induction hypothesis. From the induction principle the goal is
$$
\forall (c : \sltm{context}) (o_1 \; o_2 : \sltm{oo}), \seqsl[c,o_1]{o_2} \rightarrow P_1 \; (c,o_1) \; o_2 \rightarrow P_1 \; c \; (o_1 \longrightarrow o_2)
$$
After introductions we are proving
\begin{align*}
\mathit{Hg}_1 &: \seqsl[c,o_1]{o_2} \\
\mathit{IHg}_1 &: P_1 \; (c,o_1) \; o_2 \\[\pfshift{}]
\cline{1-2}
& P_1 \; c \; (o_1 \longrightarrow o_2)
\end{align*}
Unfolding $P_1$ as defined for this theorem, we have
\begin{align*}
\mathit{Hg}_1 &: \seqsl[c,o_1]{o_2} \\
\mathit{IHg}_1 &: \forall (\inddyncon{} : \sltm{context}), (c,o_1) \subseteq \inddyncon{} \rightarrow \seqsl[\inddyncon{}]{o_2} \\[\pfshift{}]
\cline{1-2}
& \forall (\inddyncon{} : \sltm{context}), c \subseteq \inddyncon{} \rightarrow \seqsl[\inddyncon{}]{o_1 \longrightarrow o_2}
\end{align*}
Next we make introductions from the goal.
\begin{align*}
\mathit{Hg}_1 &: \seqsl[c,o_1]{o_2} \\
\mathit{IHg}_1 &: \forall (\inddyncon{} : \sltm{context}), (c,o_1) \subseteq \inddyncon{} \rightarrow \seqsl[\inddyncon{}]{o_2} \\
\inddyncon{} &: \sltm{context} \\
P_1 &: c \subseteq \inddyncon{} \\[\pfshift{}]
\cline{1-2}
& \seqsl[\inddyncon{}]{o_1 \longrightarrow o_2}
\end{align*}
The rule \rlnmsimp{} is the only rule of the SL that we can backchain over with the current goal.
\begin{align*}
\mathit{Hg}_1 &: \seqsl[c,o_1]{o_2} \\
\mathit{IHg}_1 &: \forall (\inddyncon{} : \sltm{context}), (c,o_1) \subseteq \inddyncon{} \rightarrow \seqsl[\inddyncon{}]{o_2} \\
\inddyncon{} &: \sltm{context} \\
P_1 &: c \subseteq \inddyncon{} \\[\pfshift{}]
\cline{1-2}
& \seqsl[\inddyncon{} , o_1]{o_2}
\end{align*}
Now we use the induction hypothesis $\mathit{IHg}_1$. This step of backward reasoning gives the new subgoal $c , o_1 \subseteq \inddyncon{} , o_1$. Next backchain with the context lemma \nameref{lem:context_sub_sup} (Lemma~\ref{lem:context_sub_sup}) and we have to prove $c \subseteq \inddyncon{}$ which is provable by the induction assumption $P_1$. The proof for this case is complete.

\paragraph{Case $\vcenter{\rlbimp{}}$:} ~\\

\medskip

This rule has one goal-reduction sequent premise and one backchaining sequent premise. So there will be one induction hypothesis from each sequent premise. From the induction principle we need to prove 
\begin{align*}
& \forall (c : \sltm{context}) (o_1 \; o_2 : \sltm{oo}) (a : \sltm{atm}), \\
& \qquad \seqsl[c]{o_1} \rightarrow P_1 \; c \; o_1 \rightarrow \bchsl[c]{o_2}{a} \rightarrow P_2 \; c \; o_2 \; a \rightarrow P_2 \; c \; (o_1 \longrightarrow o_2) \; a
\end{align*}
After introductions the proof state is
\begin{align*}
\mathit{Hg}_1 &: \seqsl[c]{o_1} \\
\mathit{IHg}_1 &: P_1 \; c \; o_1 \\
\mathit{Hb}_1 &: \bchsl[c]{o_2}{a} \\
\mathit{IHb}_1 &: P_2 \; c \; o_2 \; a \\[\pfshift{}]
\cline{1-2}
& P_2 \; c \; (o_1 \longrightarrow o_2) \; a
\end{align*}
We unfold uses of $P_1$ and $P_2$.
\begin{align*}
\mathit{Hg}_1 &: \seqsl[c]{o_1} \\
\mathit{IHg}_1 &: \forall (\inddyncon{} : \sltm{context}), c \subseteq \inddyncon{} \rightarrow \seqsl[\inddyncon{}]{o_1} \\
\mathit{Hb}_1 &: \bchsl[c]{o_2}{a} \\
\mathit{IHb}_1 &: \forall (\inddyncon{} : \sltm{context}), c \subseteq \inddyncon{} \rightarrow \bchsl[\inddyncon{}]{o_2}{a} \\[\pfshift{}]
\cline{1-2}
& \forall (\inddyncon{} : \sltm{context}), c \subseteq \inddyncon{} \rightarrow \bchsl[c]{o_1 \longrightarrow o_2}{a}
\end{align*}
Next we can make introductions from the goal.
\begin{align*}
\mathit{Hg}_1 &: \seqsl[c]{o_1} \\
\mathit{IHg}_1 &: \forall (\inddyncon{} : \sltm{context}), c \subseteq \inddyncon{} \rightarrow \seqsl[\inddyncon{}]{o_1} \\
\mathit{Hb}_1 &: \bchsl[c]{o_2}{a} \\
\mathit{IHb}_1 &: \forall (\inddyncon{} : \sltm{context}), c \subseteq \inddyncon{} \rightarrow \bchsl[\inddyncon{}]{o_2}{a} \\
\mathit{IP}_1 &: c \subseteq \inddyncon{} \\[\pfshift{}]
\cline{1-2}
& \bchsl[\inddyncon{}]{o_1 \longrightarrow o_2}{a}
\end{align*}
The only SL rule whose conclusion matches the goal is \rlnmbimp{} so we backchain with this rule to get two new subgoals.

\begin{align*}
\mathit{Hg}_1 &: \seqsl[c]{o_1} \\
\mathit{IHg}_1 &: \forall (\inddyncon{} : \sltm{context}), c \subseteq \inddyncon{} \rightarrow \seqsl[\inddyncon{}]{o_1} \\
\mathit{Hb}_1 &: \bchsl[c]{o_2}{a} \\
\mathit{IHb}_1 &: \forall (\inddyncon{} : \sltm{context}), c \subseteq \inddyncon{} \rightarrow \bchsl[\inddyncon{}]{o_2}{a} \\
\mathit{IP}_1 &: c \subseteq \inddyncon{} \\[\pfshift{}]
\cline{1-2}
& (\seqsl[\inddyncon{}]{o_1}), (\bchsl[\inddyncon{}]{o_2}{a})
\end{align*}
We backchain over the appropriate induction hypothesis for each of these subgoals, and in both cases get the subgoal $c \subseteq \inddyncon{}$, provable by induction assumption $\mathit{IP}_1$. The proof of this subcase is complete.


\section{Cut Admissibility}
\label{sec:cutadmisssl}
\index{cut admissibility}

The cut rule is shown to be admissible in this specification logic by proving the following:

\begin{theorem}[\sltm{cut\_admissible}]
\label{thm:cut_admissible}
$$
\vcenter{\infer{\seqsl[\dyncon{}]{\beta}}{\seqsl[\dyncon{} , \delta]{\beta} & \seqsl[\dyncon{}]{\delta}}}
\;\; \mathrm{and} \;\;
\vcenter{\infer{\bchsl[\dyncon{}]{\beta}{\alpha}}{\bchsl[\dyncon{} , \delta]{\beta}{\alpha} & \seqsl[\dyncon{}]{\delta}}}
$$
\end{theorem}
\noindent Since our specification logic makes use of two kinds of sequents, we prove two cut rules. These correspond to the two conjuncts above, where the first is for goal-reduction sequents and the second is for backchaining sequents. \\

\begin{proof}[\textbf{Outline}]

This proof will be a nested induction, first over the cut formula $\delta$, then over the sequent premises with $\delta$ in their contexts. Since there are seven rules for constructing formulas and 15 SL rules, this will result in 105 subcases. These can be partitioned into five classes with the same proof structure, four of which we briefly illustrate presently.

%Technical details based on the particular statement to be proven will be seen in the main proof where we again consider the generalized form of SL rule and also see what the proof state will look like for specific subcases.



The cases for the axioms \rlnmst{} and \rlnmbmatch{} are proven by one use of \coqtm{constructor} (7 formulas * 2 rules = 14 subcases).

{\small
$$
\infer[\coqtm{constructor}]{\fbox{goal sequent}}{}
$$
}

Cases for rules with only sequent premises, including those with inner quantification, with the same context as the conclusion have the same proof structure. Note that by \emph{same context}, we include rules modifying the focused formula. The rules in this class are \rlnmsand{}, \rlnmsall{}, \rlnmsalls{}, \rlnmbanda{}, \rlnmbandb{}, \rlnmbimp{}, \rlnmballs{}, and \rlnmbsome{} (7 formulas * 8 rules = 56 subcases). We apply \coqtm{constructor} to the goal sequent which, after any introductions, will give a sequent subgoal for each sequent premise of the rule. To each of the new subgoals we apply the appropriate induction hypothesis, giving new subgoals for each antecedent of each induction hypothesis used. Now all goals can be proven by assumption (hypotheses from the induction principle and induction assumptions).

{\small
$$
\infer[\coqtm{constructor}]{\fbox{goal sequent}}{
	\infer[\coqtm{apply} \; IH]{\Big\{ \fbox{rule sequent premise(s)} \Big\}}{
	    \infer[\coqtm{assumption}]{\Big\{ \fbox{IH antecedents} \Big\}}{}
	}
}
$$
}

Only one rule modifies the context of the sequent, \rlnmsimp{} (7 formulas * 1 rule = 7 subcases). The proof of the subcase for this rule is similar to above, but requires the use of another structural rule, \nameref{thm:gr_weakening} (Theorem \ref{thm:gr_weakening}), before the sequent subgoal will match the sequent assumption introduced from the goal.

The remaining four rules have both a non-sequent premise and a sequent premise. Of these, the subcases for \rlnmsbc{}, \rlnmssome{}, and \rlnmball{} have a similar proof structure; apply \coqtm{constructor} to the goal so that the non-sequent premise is provable by assumption, then prove the branch for the sequent premise as above (7 formulas * 3 rules = 21 subcases).


{\small
$$
\infer[\coqtm{constructor}]{\fbox{goal sequent}}{
	\infer[\coqtm{assumption}]{\fbox{non-sequent premise}}{}
	&
	\infer[\coqtm{apply} \; IH]{\fbox{sequent premise}}{
	    \infer[\coqtm{assumption}]{\fbox{IH antecedents}}{}
	}
}
$$
}

The proof of the subcase for \rlnmsinit{} is more complicated due to the form of the non-sequent premise, $D \in \dyncon{}$, which depends on the context in the goal sequent, \seqsl{\atom{A}}. We need more details to analyse the subcases for this rule further.

So 98 of 105 subcases are proven following this outline.

\hfill (end outline)

\hfill \end{proof}

The cut admissibility theorem stated above is a simple corollary of the following theorem (with explicit quantification):
\begin{align*}
\forall (\delta : \sltm{oo}), \; & (\forall (c : \sltm{context}) (o : \sltm{oo}), \; \seqsl[c]{o} \rightarrow \\
& \qquad \forall (\inddyncon{} : \sltm{context}), c = \inddyncon{}, \delta \rightarrow \seqsl[\inddyncon{}]{\delta} \rightarrow \seqsl[\inddyncon{}]{o}) \; \wedge \\
& (\forall (c : \sltm{context}) (o : \sltm{oo}) (a : \hybridtm{atm}), \; \bchsl[c]{o}{a} \rightarrow \\
& \qquad \forall (\inddyncon : \sltm{context}), c = \inddyncon{}, \delta \rightarrow \seqsl[\inddyncon{}]{\delta} \rightarrow \bchsl[\inddyncon{}]{o}{a})
\end{align*}
%We are explicit with quantification and slightly modify the cut rule to allow the necessary inductions. In particular, we need the form 
%\begin{multline}
%(\forall \; (\dyncon{}_0 : \sltm{context})  (\beta : \sltm{oo}),
%\seqsl[\dyncon{}_0]{\beta} \rightarrow P_1 \; \dyncon{}_0 \; \beta) \; \wedge \\
%(\forall \; (\dyncon{}_0 : \sltm{context}) (\beta : \sltm{oo}) (\alpha : \sltm{atm}), \\
%\bchsl[\dyncon{}_0]{\beta}{\alpha} \rightarrow P_2 \; \dyncon{}_0 \; \beta \; \alpha)
%\label{cutseqbod}
%\end{multline}
%with appropriately defined $P_1$ and $P_2$ to perform mutual induction over sequents of the SL.

%Throughout the explanation the proof state will be shown in a manner similar to what is displayed in Coq (*or only CoqIDE?). Unlike Coq, for simplicity we will ignore variables in the context of assumptions at the level of the ambient logic. Other deviations from the formal proof for the purpose of streamlining the presentation will be mentioned as necessary.

\begin{proof}

We begin with an induction over $\delta$, so we are proving $\forall (\delta : \sltm{oo}), P \; \delta$ with $P$ defined as
\begin{align*}
& P : \sltm{oo} \rightarrow \hybridtm{Prop} := \lambda (\delta : \sltm{oo}) \; . \\
& \qquad\qquad (\forall (c : \sltm{context}) (o : \sltm{oo}), \\
& \qquad\qquad\qquad\qquad\qquad\quad \seqsl[c]{o} \rightarrow P_1 \; c \; o) \; \wedge \\
& \qquad\qquad (\forall (c : \sltm{context}) (o : \sltm{oo}) (a : \hybridtm{atm}), \\
& \qquad\qquad\qquad\qquad\qquad\quad \bchsl[c]{o}{a} \rightarrow P_2 \; c \; o \; a)
\end{align*}
where
\begin{align*}
P_1 &: \sltm{context} \rightarrow \sltm{oo} \rightarrow \coqtm{Prop} := \lambda (c : \sltm{context}) (o : \sltm{oo}) \; . \\
& \qquad\qquad \forall (\inddyncon{} : \sltm{context}), c = (\inddyncon{}, \delta) \rightarrow \seqsl[\inddyncon{}]{\delta} \rightarrow \seqsl[\inddyncon{}]{o} \\
P_2 &: \sltm{context} \rightarrow \sltm{oo} \rightarrow \sltm{atm} \rightarrow \coqtm{Prop} := \lambda (c : \sltm{context}) (o : \sltm{oo}) (a : \sltm{atm}) \; . \\
& \qquad\qquad \forall (\inddyncon{} : \sltm{context}), c = (\inddyncon{}, \delta) \rightarrow \seqsl[\inddyncon{}]{\delta} \rightarrow \bchsl[\inddyncon{}]{o}{a}
\end{align*}
$P$, $P_1$, and $P_2$ will provide the induction hypotheses used in this proof. Next is a nested induction, which is a mutual structural induction over \seqsl[c]{o} and \bchsl[c]{o}{a} using $P_1$ and $P_2$ as above.

In the proof presentation here we will only look at cases for the rule \rlnmsinit{}. Later we will see a generalization of the SL and a proof that captures the remaining 98 cases, as well as the proof of \nameref{thm:monotone} (Theorem~\ref{thm:monotone}) seen above. Since in the proof of \nameref{thm:monotone} we have already seen how to prove a few concrete cases in detail using the mutual structural induction principle, it would be tedious to continue to work through more subcases in the same way.

\subsection{Subcase for \rlnmsinit{}: Alternate Proof Attempt}

Before proving this subcase for the nested induction, suppose that rather than an outer induction over the cut formula $\delta$ we had simply introduced this variable into the context of the proof state and begun the proof as a mutual structural induction over the sequent premises with $\delta$ in their context. Then we can wait until it is necessary to have an induction over the cut formula.

The subcase of the induction principle for \rlnmsinit{} from Figure~\ref{fig:seqind} requires a proof of
\begin{align*}
\forall (c : \sltm{context}) & (o : \sltm{oo}) (a : \sltm{atm}), \\
& o \in c \rightarrow \bchsl[c]{o}{a} \rightarrow P_2 \; c \; o \; a \rightarrow P_1 \; c \; \atom{a}
\end{align*}
After introductions and unfolding $P_1$ and $P_2$ as defined for this theorem, the proof state is
\begin{align*}
H_1 &: o \in c \\
\mathit{Hb}_1 &: \bchsl[c]{o}{a} \\
\mathit{IHb}_1 &: \forall (\inddyncon{} : \sltm{context}), c = (\inddyncon{}, \delta) \rightarrow \seqsl[\inddyncon{}]{\delta} \rightarrow \bchsl[\inddyncon{}]{o}{a} \\[\pfshift{}]
\cline{1-2}
& \forall (\inddyncon{} : \sltm{context}), c = (\inddyncon{}, \delta) \rightarrow \seqsl[\inddyncon{}]{\delta} \rightarrow \seqsl[\inddyncon{}]{\atom{a}}
\end{align*}
Next we make introductions from the goal.
\begin{align*}
H_1 &: o \in c \\
\mathit{Hb}_1 &: \bchsl[c]{o}{a} \\
\mathit{IHb}_1 &: \forall (\inddyncon{} : \sltm{context}), c = (\inddyncon{}, \delta) \rightarrow \seqsl[\inddyncon{}]{\delta} \rightarrow \bchsl[\inddyncon{}]{o}{a} \\
\inddyncon{} &: \sltm{context} \\
\mathit{IP}_1 &: c = (\inddyncon{}, \delta) \\
\mathit{IP}_2 &: \seqsl[\inddyncon{}]{\delta} \\[\pfshift{}]
\cline{1-2}
& \seqsl[\inddyncon{}]{\atom{a}}
\end{align*}
Next we substitute $(\inddyncon , \delta)$ for $c$ using $\mathit{IP}_1$ and rename $\inddyncon{}$ to $\dyncon{}_0$ in $\mathit{IHb}_1$ to distinguish the bound variable from the free variable $\inddyncon{}$. Now ignore $\mathit{IP}_1$.
\begin{align*}
H_1 &: o \in \inddyncon{} , \delta \\
\mathit{Hb}_1 &: \bchsl[\inddyncon{} , \delta]{o}{a} \\
\mathit{IHb}_1 &: \forall ({\dyncon{}_0} : \sltm{context}), (\inddyncon{} , \delta) = (\dyncon{}_0 , \delta) \rightarrow \seqsl[\dyncon{}_0]{\delta} \rightarrow \bchsl[\dyncon{}_0]{o}{a} \\
\inddyncon{} &: \sltm{context} \\
\mathit{IP}_2 &: \seqsl[\inddyncon{}]{\delta} \\[\pfshift{}]
\cline{1-2}
& \seqsl[\inddyncon{}]{\atom{a}}
\end{align*}
We can get a new premise $P_3 : \bchsl[\inddyncon{}]{o}{a}$ by specializing $\mathit{IHb}_1$ with $\inddyncon{}$, a reflexivity lemma and $\mathit{IP}_2$. Now ignore $\mathit{IHb}_1$ which is no longer needed and $\mathit{Hb}_1$ which we can get from~\nameref{thm:bc_weakening} (Theorem~\ref{thm:bc_weakening}) and $P_3$.

\begin{align*}
H_1 &: o \in \inddyncon{} , \delta \\
\inddyncon{} &: \sltm{context} \\
\mathit{IP}_2 &: \seqsl[\inddyncon{}]{\delta} \\
P_3 &: \bchsl[\inddyncon{}]{o}{a} \\[\pfshift{}]
\cline{1-2}
& \seqsl[\inddyncon{}]{\atom{a}}
\end{align*}
We can apply the context lemma \sltm{elem\_inv} to $H_1$ to get the premise $(o \in \inddyncon{}) \vee (o = \delta)$. Applying \coqtm{inversion} to this, we have two new subgoals with diverging sets of assumptions. In the second we substitute $\delta$ for $o$ by $H_1$ in that proof state.
\begin{align*}
H_1 &: o \in \inddyncon{}  &H_1 &: o = \delta \\
\inddyncon{} &: \sltm{context} &\inddyncon{} &: \sltm{context} \\
\mathit{IP}_2 &: \seqsl[\inddyncon{}]{\delta} &\mathit{IP}_2 &: \seqsl[\inddyncon{}]{\delta} \\
P_3 &: \bchsl[\inddyncon{}]{o}{a} &P_3 &: \bchsl[\inddyncon{}]{\delta}{a} \\[\pfshift{}]
\cline{1-5}
& \seqsl[\inddyncon{}]{\atom{a}}
&& \seqsl[\inddyncon{}]{\atom{a}}
\end{align*}
The left subgoal is provable by first applying \rlnmsinit{} to get subgoals $o \in \inddyncon{}$ and \bchsl[\inddyncon{}]{o}{a}, both proven by assumption.

%Notice that $\mathit{IHb}_1$ is the induction hypothesis corresponding to the portion of the cut rule for backchaining sequents. We get this because of the backchaining sequent premise of the \rlnmsinit{} rule. If we had a hypothesis about the goal-reduction portion of this rule, then we could finish this proof as in Figure \ref{fig:hyppf}.

%\begin{figure}
%$$
%\infer[\mathit{gr \; cut \; IH}]{\seqsl[\inddyncon{}]{\atom{a_1}}}{
%	\infer[\coqtm{apply \rlnmsinit{}}]{\seqsl[\inddyncon{} , \delta]{\atom{a_1}}}{
%		\infer[\coqtm{apply elem\_self}]{\delta \in \inddyncon{} , \delta}{}
%		&
%		\infer[\coqtm{assumption}]{\bchsl[\inddyncon{} , \delta]{\delta}{a_1}}{}
%	}
%	&
%	\infer[\coqtm{assumption}]{\seqsl[\inddyncon{}]{\delta}}{}
%}
%$$
%\centering{\caption{Cut Admissibility \rlnmsinit{} Branch with Goal-Reduction Hypothesis} \label{fig:hyppf}}
%\end{figure}

The proof on the right will be continued with an induction over $\delta$. The property to prove is
\begin{align*}
P_0 &: \sltm{oo} \rightarrow \coqtm{Prop} := \lambda (\delta : \sltm{oo}) \; . \\
& \qquad \forall (\inddyncon{} : \sltm{context}) (a : \sltm{atm}), \\
& \qquad\qquad \seqsl[\inddyncon{}]{\delta} \rightarrow \bchsl[\inddyncon{}]{\delta}{a} \rightarrow \seqsl[\inddyncon{}]{\atom{a}}
\end{align*}
We will now look at a specific subcase of this induction. \\

\paragraph{Subcase $\delta = o_1 \longrightarrow o_2$ :} ~\\
%\noindent\textbf{Subcase} $\delta = o_1 \longrightarrow o_2$ \textbf{:} ~\\

In this case we prove the appropriate antecedent of the induction principle for induction over $\delta$ (see Figure~\ref{fig:ooip}), shown below.
$$
\forall (o_1 \; o_2 : \sltm{oo}), P_0 \; o_1 \rightarrow P_0 \; o_2 \rightarrow P_0 \; (o_1 \longrightarrow o_2)
$$
The expanded proof state after premise introductions is:\\

\begin{align*}
\mathit{IH}_1 &: \forall (\inddyncon{} : \sltm{context}) (a : \sltm{atm}), \seqsl[\inddyncon{}]{o_1} \rightarrow \bchsl[\inddyncon{}]{o_1}{a} \rightarrow \seqsl[\inddyncon{}]{\atom{a}} \\
\mathit{IH}_2 &: \forall (\inddyncon{} : \sltm{context}) (a : \sltm{atm}), \seqsl[\inddyncon{}]{o_2} \rightarrow \bchsl[\inddyncon{}]{o_2}{a} \rightarrow \seqsl[\inddyncon{}]{\atom{a}} \\
\inddyncon{} &: \sltm{context} \\
\mathit{IP}_2 &: \seqsl[\inddyncon{}]{(o_1 \longrightarrow o_2)} \\
P_3 &: \bchsl[\inddyncon{}]{o_1 \longrightarrow o_2}{a} \\[\pfshift{}]
\cline{1-2}
& \seqsl[\inddyncon{}]{\atom{a}}
\end{align*}

We can apply \coqtm{inversion} to the premises $\mathit{IP}_2$ and $P_3$ to get new assumptions in the context.
\begin{align*}
\mathit{IH}_1 &: \forall (\inddyncon{} : \sltm{context}) (a : \sltm{atm}), \seqsl[\inddyncon{}]{o_1} \rightarrow \bchsl[\inddyncon{}]{o_1}{a} \rightarrow \seqsl[\inddyncon{}]{\atom{a}} \\
\mathit{IH}_2 &: \forall (\inddyncon{} : \sltm{context}) (a : \sltm{atm}), \seqsl[\inddyncon{}]{o_2} \rightarrow \bchsl[\inddyncon{}]{o_2}{a} \rightarrow \seqsl[\inddyncon{}]{\atom{a}} \\
\inddyncon{} &: \sltm{context} \\
\mathit{IP}_2 &: \seqsl[\inddyncon{}, o_1]{o_2} \\
P_{3_1} &: \bchsl[\inddyncon{}]{o_2}{a} \\
P_{3_2} &: \seqsl[\inddyncon{}]{o_1} \\[\pfshift{}]
\cline{1-2}
& \seqsl[\inddyncon{}]{\atom{a}}
\end{align*}
$\mathit{IH}_1$ is not useful here, since we have no way to prove sequents with $o_1$ focused. Applying $\mathit{IH}_2$ and ignoring induction hypotheses, we have:
\begin{align*}
\inddyncon{} &: \sltm{context} \\
\mathit{IP}_2 &: \seqsl[\inddyncon{}, o_1]{o_2} \\
P_{3_1} &: \bchsl[\inddyncon{}]{o_2}{a} \\
P_{3_2} &: \seqsl[\inddyncon{}]{o_1} \\[\pfshift{}]
\cline{1-2}
& (\bchsl[\inddyncon{}, o_2]{o_2}{a}), (\seqsl[\inddyncon{}]{o_2}), (\bchsl[\inddyncon{}]{o_2}{a}) 
\end{align*}
The first subgoal is proven using~\nameref{thm:bc_weakening} (Theorem~\ref{thm:bc_weakening}) and assumption $P_{3_1}$, and the third subgoal by $P_{3_1}$.

On trying to prove the second subgoal, we should reflect on two things. First, proving \seqsl[\inddyncon{}]{o_2} from the assumptions $\mathit{IP}_2$ and $P_{3_2}$ would be a use of the goal-reduction cut rule. Second, we are proving the subcase corresponding to the \rlnmsinit{} rule and the only sequent premise of this rule is a backchaining sequent; we only get the backchaining part of the cut rule in the induction hypothesis. To illustrate this, recall that for this subcase we have $\bchsl[c]{o}{a}$ and the induction hypothesis $P_2 \; c \; o \; a$ in the context of assumptions. The induction hypothesis expands to
$$
\forall (\inddyncon{} : \sltm{context}), c = (\inddyncon{}, \delta) \rightarrow \seqsl[\inddyncon{}]{\delta} \rightarrow \bchsl[\inddyncon{}]{o}{a}
$$
Combining these assumptions we have
$$
\bchsl[\inddyncon{} , \delta]{o}{a} \rightarrow \seqsl[\inddyncon{}]{\delta} \rightarrow \bchsl[\inddyncon{}]{o}{a}
$$
which is the conjunct of the cut rule for backchaining sequents. Combining the above observations, we see that this branch cannot be continued any further. \\



\subsection{Subcase for \rlnmsinit{}: Original Proof Structure}
\label{subsec:cutadmissnonseq}

%Recall that before the induction over sequent premises, we had induction over the cut formula \delta. To finish this proof we need to consider the subcases corresponding to the \rlnmsinit{} rule for each form of \delta.
Convinced of the necessity of our original proof structure, now we will move on with our proof of the cut rule by nested inductions, first on the cut formula $\delta$ then over the sequent premises with $\delta$ in the context.
Below is a proof of the \rlnmsinit{} subcase where $\delta = o_1 \longrightarrow o_2$. The \rlnmsinit{} subcases for other formula constructions follow similarly.
%
\paragraph{Case $\delta = o_1 \; \longrightarrow \; o_2$ :} ~\\

From Figure~\ref{fig:ooip}, the antecedent of the \sltm{oo} induction principle for this case is
$$
\forall (o_1 \; o_2 : \sltm{oo}), P \; o_1 \rightarrow P \; o_2 \rightarrow P \; (o_1 \longrightarrow o_2)
$$
where $P \; o_1$ and $P \; o_2$ are induction hypotheses and $P$ is as defined at the start of this proof. Expanding the goal (we will wait to expand the premises), the proof state is
\begin{align*}
\mathit{IH}_1 &: P \; o_1 \\
\mathit{IH}_2 &: P \; o_2 \\[\pfshift{}]
\cline{1-2}
(\forall & (c : \sltm{context}) (o : \sltm{oo}), \seqsl[c]{o} \rightarrow \forall (\inddyncon{} : \sltm{context}), \\
& \qquad c = (\inddyncon{}, (o_1 \longrightarrow o_2)) \rightarrow \seqsl[\inddyncon{}]{(o_1 \longrightarrow o_2)} \rightarrow \seqsl[\inddyncon{}]{o}) \; \wedge \\
(\forall & (c : \sltm{context}) (o : \sltm{oo}) (a : \sltm{atm}), \bchsl[c]{o}{a} \rightarrow \forall (\inddyncon{} : \sltm{context}), \\
& \qquad c = (\inddyncon{}, (o_1 \longrightarrow o_2)) \rightarrow \seqsl[\inddyncon{}]{(o_1 \longrightarrow o_2)} \rightarrow \bchsl[\inddyncon{}]{o}{a})
\end{align*}
Next we have the mutual induction over sequents. As stated above, we will only show the subcase for the \rlnmsinit{} rule.

\paragraph{Subcase $\vcenter{\rlsinit{}}$ :} ~\\

\bigskip

The goal for this subcase is
$$
\forall (c : \sltm{context}) (o : \sltm{oo}) (a : \sltm{atm}), o \in c \rightarrow \bchsl[c]{o}{a} \rightarrow P_2 \; c \; o \; a \rightarrow P_1 \; c \; \atom{a}
$$
After introductions, the proof state is
\begin{align*}
\mathit{IH}_1 &: P \; o_1 \\
\mathit{IH}_2 &: P \; o_2 \\
H_1 &: o \in c \\
\mathit{Hb}_1 &: \bchsl[c]{o}{a} \\
\mathit{IHb}_1 &: \forall (\inddyncon{} : \sltm{context}), c = (\inddyncon{} , o_1 \longrightarrow o_2) \rightarrow \seqsl[\inddyncon{}]{(o_1 \longrightarrow o_2)} \rightarrow \bchsl[\inddyncon{}]{o}{a} \\
\inddyncon{} &: \sltm{context} \\
\mathit{IP}_1 &: c = \inddyncon{} , o_1 \longrightarrow o_2 \\
\mathit{IP}_2 &: \seqsl[\inddyncon{}]{o_1 \longrightarrow o_2} \\[\pfshift{}]
\cline{1-2}
& \seqsl[\inddyncon{}]{\atom{a}}
\end{align*}
Next substitute $(\inddyncon{} , o_1 \longrightarrow o_2)$ for $c$ using $\mathit{IP}_1$ and rename $\inddyncon{}$ to $\dyncon{}_0$ in $\mathit{IHb}_1$ to distinguish the bound variable from the free variable $\inddyncon{}$. Now ignore $\mathit{IP}_1$.
\begin{align*}
\mathit{IH}_1 &: P \; o_1 \\
\mathit{IH}_2 &: P \; o_2 \\
H_1 &: o \in (\inddyncon{} , o_1 \longrightarrow o_2) \\
\mathit{Hb}_1 &: \bchsl[\inddyncon{} , o_1 \longrightarrow o_2]{o}{a} \\
\mathit{IHb}_1 &: \forall (\dyncon{}_0 : \sltm{context}), \\
& \qquad (\inddyncon{} , o_1 \longrightarrow o_2) = (\dyncon{}_0 , o_1 \longrightarrow o_2) \rightarrow \seqsl[\dyncon{}_0]{(o_1 \longrightarrow o_2)} \rightarrow \bchsl[\dyncon{}_0]{o}{a} \\
\inddyncon{} &: \sltm{context} \\
\mathit{IP}_2 &: \seqsl[\inddyncon{}]{o_1 \longrightarrow o_2} \\[\pfshift{}]
\cline{1-2}
& \seqsl[\inddyncon{}]{\atom{a}}
\end{align*}
We can specialize $\mathit{IHb}_1$ with $\inddyncon{}$, a reflexivity lemma and $\mathit{IP}_2$ to get the new premise $P_3 : \bchsl[\inddyncon{}]{o}{a}$ and apply~\nameref{lem:elem_inv} (Lemma~\ref{lem:elem_inv}) to $H_1$ to get $(o \in \inddyncon{}) \vee (o = o_1 \longrightarrow o_2)$. Now ignore $\mathit{IHb}_1$ and $\mathit{Hb}_1$ (we can get the latter from assumption $P_3$ and~\nameref{thm:bc_weakening}, Theorem~\ref{thm:bc_weakening}).

\newpage

\vspace{-20pt}

\begin{align*}
\mathit{IH}_1 &: P \; o_1 \\
\mathit{IH}_2 &: P \; o_2 \\
H_1 &: (o \in \inddyncon{}) \vee (o = o_1 \longrightarrow o_2) \\
%\inddyncon{} &: \sltm{context} \\
\mathit{IP}_2 &: \seqsl[\inddyncon{}]{o_1 \longrightarrow o_2} \\
P_3 &: \bchsl[\inddyncon{}]{o}{a} \\[\pfshift{}]
\cline{1-2}
& \seqsl[\inddyncon{}]{\atom{a}}
\end{align*}
Inverting $H_1$, we get two new subgoals with different sets of assumptions. In the second we substitute $o_1 \longrightarrow o_2$ for $o$ using $H_1$ in that proof state.
\begin{align*}
\mathit{IH}_1 &: P \; o_1  & IH_1 &: P \; o_1 \\
\mathit{IH}_2 &: P \; o_2  & IH_2 &: P \; o_2 \\
H_1 &: o \in \inddyncon{}  & H_1 &: o = o_1 \longrightarrow o_2 \\
\mathit{IP}_2 &: \seqsl[\inddyncon{}]{o_1 \longrightarrow o_2}  &\mathit{IP}_2 &: \seqsl[\inddyncon{}]{o_1 \longrightarrow o_2} \\
P_3 &: \bchsl[\inddyncon{}]{o}{a}  & P_3 &: \bchsl[\inddyncon{}]{o_1 \longrightarrow o_2}{a} \\[\pfshift{}]
\cline{1-4}
& \seqsl[\inddyncon{}]{\atom{a}} &&\seqsl[\inddyncon{}]{\atom{a}}
\end{align*}
To prove the first, we apply \rlnmsinit{} to the goal, then need to prove $o \in \inddyncon{}$ and \bchsl[\inddyncon{}]{o}{a} which are both provable by assumption.

For the second (right) subgoal, it will be necessary to apply \coqtm{inversion} to some assumptions to get structurally simpler assumptions, before being able to apply the induction hypotheses $\mathit{IH}_1$ and $\mathit{IH}_2$.
% Also, applying either of $\mathit{IH}_1$ or $\mathit{IH}_2$ to the goal will give two subgoals. So it will simplify the proof to do all inversions on the structure of assumptions before using induction hypotheses.
Inverting $\mathit{IP}_2$ and $P_3$, and unfolding $P$, we have:
\begin{align*}
\mathit{IH}_1 &: (\forall (c : \sltm{context}) (o : \sltm{oo}), \seqsl[c]{o} \rightarrow \\
& \qquad\qquad \forall (\inddyncon{} : \sltm{context}), c = (\inddyncon{}, o_1) \rightarrow \seqsl[\inddyncon{}]{o_1} \rightarrow \seqsl[\inddyncon{}]{o}) \; \wedge \\
& \;\;\; (\forall (c : \sltm{context}) (o : \sltm{oo}) (a : \sltm{atm}), \bchsl[c]{o}{a} \rightarrow \\
& \qquad\qquad \forall (\inddyncon{} : \sltm{context}), c = (\inddyncon{}, o_1) \rightarrow \seqsl[\inddyncon{}]{o_1} \rightarrow \bchsl[\inddyncon{}]{o}{a}) \\
\mathit{IH}_2 &: (\forall (c : \sltm{context}) (o : \sltm{oo}), \seqsl[c]{o} \rightarrow \\
& \qquad\qquad \forall (\inddyncon{} : \sltm{context}), c = (\inddyncon{}, o_2) \rightarrow \seqsl[\inddyncon{}]{o_2} \rightarrow \seqsl[\inddyncon{}]{o}) \; \wedge \\
& \;\;\; (\forall (c : \sltm{context}) (o : \sltm{oo}) (a : \sltm{atm}), \bchsl[c]{o}{a} \rightarrow \\
& \qquad\qquad \forall (\inddyncon{} : \sltm{context}), c = (\inddyncon{}, o_2) \rightarrow \seqsl[\inddyncon{}]{o_2} \rightarrow \bchsl[\inddyncon{}]{o}{a}) \\
\mathit{IP}_2 &: \seqsl[\inddyncon{} , o_1]{o_2} \\
P_{3_1} &: \seqsl[\inddyncon{}]{o_1} \\
P_{3_2} &: \bchsl[\inddyncon{}]{o_2}{a} \\[\pfshift{}]
\cline{1-2}
& \seqsl[\inddyncon{}]{\atom{a}}
\end{align*}
Backchaining on the first conjunct of $\mathit{IH}_2$, instantiating $c$ with $(\inddyncon{} , o_2)$, gives three new subgoals.
\begin{align*}
\mathit{IH}_1 &: (\forall (c : \sltm{context}) (o : \sltm{oo}), \seqsl[c]{o} \rightarrow \\
& \qquad\qquad \forall (\inddyncon{} : \sltm{context}), c = (\inddyncon{}, o_1) \rightarrow \seqsl[\inddyncon{}]{o_1} \rightarrow \seqsl[\inddyncon{}]{o}) \; \wedge \\
& \;\;\; (\forall (c : \sltm{context}) (o : \sltm{oo}) (a : \sltm{atm}), \bchsl[c]{o}{a} \rightarrow \\
& \qquad\qquad \forall (\inddyncon{} : \sltm{context}), c = (\inddyncon{}, o_1) \rightarrow \seqsl[\inddyncon{}]{o_1} \rightarrow \bchsl[\inddyncon{}]{o}{a}) \\
\mathit{IH}_2 &: (\forall (c : \sltm{context}) (o : \sltm{oo}), \seqsl[c]{o} \rightarrow \\
& \qquad\qquad \forall (\inddyncon{} : \sltm{context}), c = (\inddyncon{}, o_2) \rightarrow \seqsl[\inddyncon{}]{o_2} \rightarrow \seqsl[\inddyncon{}]{o}) \; \wedge \\
& \;\;\; (\forall (c : \sltm{context}) (o : \sltm{oo}) (a : \sltm{atm}), \bchsl[c]{o}{a} \rightarrow \\
& \qquad\qquad \forall (\inddyncon{} : \sltm{context}), c = (\inddyncon{}, o_2) \rightarrow \seqsl[\inddyncon{}]{o_2} \rightarrow \bchsl[\inddyncon{}]{o}{a}) \\
P_{3_1} &: \seqsl[\inddyncon{}]{o_1} \\
P_{3_2} &: \bchsl[\inddyncon{}]{o_2}{a} \\
\mathit{IP}_2 &: \seqsl[\inddyncon{} , o_1]{o_2} \\[\pfshift{}]
\cline{1-2}
& (\seqsl[\inddyncon{} , o_2]{\atom{a}}), (\inddyncon{} , o_2 = \inddyncon{} , o_2), (\seqsl[\inddyncon{}]{o_2})
\end{align*}
For the first, apply \rlnmsinit{}, then we need to prove $o_2 \in (\inddyncon{} , o_2)$ (proven by~\nameref{lem:elem_self}, Lemma~\ref{lem:elem_self}) and \bchsl[\inddyncon{} , o_2]{o_2}{a} (proven by~\nameref{thm:bc_weakening}, Theorem~\ref{thm:bc_weakening}, and assumption $P_{3_2}$). The second is proven by \coqtm{reflexivity}. For the third, we backchain on the first conjunct of $\mathit{IH}_1$, instantiating $c$ with $(\inddyncon{} , o_1)$, and get three new subgoals.
\begin{align*}
\mathit{IH}_1 &: (\forall (c : \sltm{context}) (o : \sltm{oo}), \seqsl[c]{o} \rightarrow \\
& \qquad\qquad \forall (\inddyncon{} : \sltm{context}), c = (\inddyncon{}, o_1) \rightarrow \seqsl[\inddyncon{}]{o_1} \rightarrow \seqsl[\inddyncon{}]{o}) \; \wedge \\
& \;\;\; (\forall (c : \sltm{context}) (o : \sltm{oo}) (a : \sltm{atm}), \bchsl[c]{o}{a} \rightarrow \\
& \qquad\qquad \forall (\inddyncon{} : \sltm{context}), c = (\inddyncon{}, o_1) \rightarrow \seqsl[\inddyncon{}]{o_1} \rightarrow \bchsl[\inddyncon{}]{o}{a}) \\
\mathit{IH}_2 &: (\forall (c : \sltm{context}) (o : \sltm{oo}), \seqsl[c]{o} \rightarrow \\
& \qquad\qquad \forall (\inddyncon{} : \sltm{context}), c = (\inddyncon{}, o_2) \rightarrow \seqsl[\inddyncon{}]{o_2} \rightarrow \seqsl[\inddyncon{}]{o}) \; \wedge \\
& \;\;\; (\forall (c : \sltm{context}) (o : \sltm{oo}) (a : \sltm{atm}), \bchsl[c]{o}{a} \rightarrow \\
& \qquad\qquad \forall (\inddyncon{} : \sltm{context}), c = (\inddyncon{}, o_2) \rightarrow \seqsl[\inddyncon{}]{o_2} \rightarrow \bchsl[\inddyncon{}]{o}{a}) \\
P_{3_1} &: \seqsl[\inddyncon{}]{o_1} \\
P_{3_2} &: \bchsl[\inddyncon{}]{o_2}{a} \\
\mathit{IP}_2 &: \seqsl[\inddyncon{} , o_1]{o_2} \\[\pfshift{}]
\cline{1-2}
& (\seqsl[\inddyncon{} , o_1]{o_2}), (\inddyncon{} , o_1 = \inddyncon{} , o_1), (\seqsl[\inddyncon{}]{o_1})
\end{align*}
The sequent subgoals are proven by \coqtm{assumption} and the other by \coqtm{reflexivity}. \\

The \rlnmsinit{} subcases for the remaining six constructors of \sltm{oo} follow a similar argument requiring \sltm{inversion} on hypotheses and induction hypothesis specialization. \\

\bigskip

From this presentation we can see that working through the details for every case can be a tedious and repetitive task. We later see a generalization that helps us to understand what subcases have the same structure and separate out the challenging cases. This understanding leads us to a condensed automated Coq proof for~\nameref{thm:monotone} (Theorem~\ref{thm:monotone}, see Figure~\ref{fig:coqpfmonotone}) and proofs of 98 of 105 subcases in the proof of~\nameref{thm:cut_admissible} (Theorem~\ref{thm:cut_admissible}, see Figure~\ref{fig:coqpfcutadmiss} where \sltm{delta} is the cut formula in the implementation, in place of \delta).
\begin{figure}
\begin{lstlisting}
Proof.
Hint Resolve context_sub_sup.
eapply seq_mutind; intros;
try (econstructor; eauto; eassumption).
Qed.
\end{lstlisting}
\centering{\caption{Coq proof of~\nameref{thm:monotone} (Theorem~\ref{thm:monotone}) \label{fig:coqpfmonotone}}}
\end{figure}

\begin{figure}
\begin{lstlisting}
Proof.
Hint Resolve gr_weakening context_swap.
induction delta; eapply seq_mutind; intros;
subst; try (econstructor; eauto; eassumption).
...
\end{lstlisting}
\centering{\caption{Coq proof of 98/105 cases of~\nameref{thm:cut_admissible} (Theorem~\ref{thm:cut_admissible}) \label{fig:coqpfcutadmiss}}}
\end{figure}
%}

\cleardoublepage

%%%%%%%%%%%%%%%%%%%%%%%%%%%%%%%%%%%%%%%%%%%%%%%%%%
% Generalized Specification Logic
%%%%%%%%%%%%%%%%%%%%%%%%%%%%%%%%%%%%%%%%%%%%%%%%%%

\chapter{Generalized Specification Logic}
\label{ch:gsl}

%{
%\allowdisplaybreaks[0]

%\section{Proofs by Mutual Structural Induction over Sequents}

%We have seen that the SL is a collection of rules for proving goal-reduction and backchaining sequents. These rules may have premises that are either kind of sequent, or not a sequent at all. To use this mutual induction technique to prove something about goal-reduction and backchaining sequents, we have to prove 15 subcases; one for each rule of this SL. There are a few approaches that could be taken to present such a proof:

%1. state the higher-level induction structure and leave it to the reader to work out the details from the code

%2. present the reasoning for all subcases

%3. present the reasoning for a select few subcases to illustrate the reasoning

\section{Generalized SL Part I: Abstract Rules}
\label{sec:gsl}

Here we present generalized specification logic rules to reduce the number of induction cases and allow us to partition cases of the original SL based on rule structure. Our goal is to gain insight into the high-level structure of such inductive proofs, providing the proof writer and reader with the ability to understand where the difficult cases are and how similar cases can be handled in a general way.

All rules of the SL have some number of premises that are either non-sequent predicates, goal-reduction sequents, or backchaining sequents. Also, all rule conclusions are sequents; this is necessary to encode these rules in inductive types \sltm{grseq} and \sltm{bcseq}. With this observation, we can generalize the rules of the SL inference system and say that all rules have one of the following forms:

\begin{prooftree}
\Axiom$\fCenter \ol{Q_m} \args{c , o}$
\noLine
  \UnaryInf$\forall \ol{(x_{n,s_n} : R_{n,s_n})}, \fCenter (\seqsl[c \cup \ol{\gamma_n} \args{o}]{\ol{F_n} \args{o , \ol{x_{n, s_n}}}})$
  \noLine
  \UnaryInf$\forall \ol{(y_{p,t_p} : S_{p,t_p})}, \fCenter (\bchsl[c \cup \ol{\gamma'_p} \args{o}]{\ol{F'_p} \args{o , \ol{y_{p,t_p}}}}{\ol{a_p}})$
    \RightLabel{\rl{gr\_rule}}
    \UnaryInf$\fCenter \seqsl[c]{o}$
\end{prooftree}
\begin{prooftree}
\Axiom$\fCenter \ol{Q_m} \args{c , o}$
\noLine
  \UnaryInf$\forall \ol{(x_{n,s_n} : R_{n,s_n})}, \fCenter (\seqsl[c \cup \ol{\gamma_n} \args{o}]{\ol{F_n} \args{o , \ol{x_{n, s_n}}}})$
  \noLine
  \UnaryInf$\forall \ol{(y_{p,t_p} : S_{p,t_p})}, \fCenter (\bchsl[c \cup \ol{\gamma'_p} \args{o}]{\ol{F'_p} \args{o , \ol{y_{p,t_p}}}}{\ol{a_p}})$
    \RightLabel{\rl{bc\_rule}}
    \UnaryInf$\fCenter \bchsl[c]{o}{a}$
\end{prooftree}
%$$
%\inferH[\rl{gr\_rule}]{\seqsl[c]{o}}{\ol{Q_m} \args{c , o} & \forall \ol{(x_{n,s_n} : R_{n,s_n})}, (\seqsl[c \cup \ol{\gamma_n} \args{o}]{\ol{G_n} \args{o , \ol{x_{n, s_n}}}}) & \forall \ol{(y_{p,t_p} : S_{p,t_p})}, (\bchsl[c \cup \ol{\gamma'_p} \args{o}]{\ol{D_p} \args{o , \ol{y_{p,t_p}}}}{\ol{a_p}})}
%$$
%or
%$$
%\inferH[\rl{bc\_rule}]{\bchsl[c]{o}{a}}{\ol{Q_m} \args{c , o} & \forall \ol{(x_{n,s_n} : R_{n,s_n})}, (\seqsl[c \cup \ol{\gamma_n} \args{o}]{\ol{G_n} \args{o , \ol{x_{n, s_n}}}}) & \forall \ol{(y_{p,t_p} : S_{p,t_p})}, (\bchsl[c \cup \ol{\gamma'_p} \args{o}]{\ol{D_p} \args{o , \ol{y_{p,t_p}}}}{\ol{a_p}})}
%$$
where $m, n, p$ represent the (possibly zero) number of non-sequent
premises, goal-reduction sequent premises, and backchaining sequent
premises, respectively. Note that for all rules in our implemented SL,
$0\le m\le 1$, $0\le n\le 2$, and $0\le p\le 1$.

We call this collection of inference rules consisting of \rl{gr\_rule} and \rl{bc\_rule} the generalized specification logic (GSL). This is \emph{not} implemented in Coq as the previously described SL is; but rather all rules of the SL can be instantiated from the two rules of the GSL (see Subsection \ref{subsec:sltogsl}). The GSL allows us to investigate the SL without needing to consider each of the 15 rules of the SL separately. This makes it possible to more efficiently study and explain the metatheory of the SL.

Much of the notation used in these rules requires further explanation. A horizontal bar above an element with some subscript index, say $z$, means we have a collection of such items indexed from 1 to $z$. For example, the ``premise'' $\ol{Q_m} \args{c,o}$ represents the $m$ premises $Q_1 \args{c,o} , \ldots , Q_m \args{c,o}$. The premises with sequents can possibly have local quantification. For $i=1 , \ldots , n$, $\ol{(x_{i,s_i} : R_{i, s_i})}$ represents the prefix $(x_{i,1} : R_{i,1})\cdots(x_{i,s_i} : R_{i,s_i})$.

The notation $\args{\cdot}$ is used to list arguments from the conclusion that may be used by a function or predicate. We wish to show how elements of the rule conclusion propagate through a proof.

Given types $T_0, T_1 , \ldots , T_z$, when we write $F \args{a_1 : T_1 , \ldots , a_z : T_z} : T_0$, we mean a term of type $T_0$ that may contain any (sub)terms appearing in conclusion terms $a_1, \ldots , a_z$. For example, given $\gamma_1 \args{D \longrightarrow G : \sltm{oo}} : \sltm{context}$, we may ``instantiate'' this expression to $\{ D \}$. We often omit types and use definitional notation, e.g., in this case we may write $\gamma_1 \args{D \longrightarrow G} \coloneqq \{ D \}$.

We infer the following typing judgments from the GSL rules:
\begin{itemize}
 \item For $i = 1 ,\ldots , m$, the definition of $Q_i$ may use the context and formula of the conclusion, so with full typing information, $Q_i \args{c : \sltm{context} , o : \sltm{oo}} : \coqtm{Prop}$
 \item For $j = 1 , \ldots , n$, SL context $\gamma_j$ may use the formula of the conclusion and SL formula $F_j$ may use the formula of the conclusion and locally quantified variables. So with full typing information, $\gamma_j \args{o : \sltm{oo}} : \sltm{context}$ and $F_j \args{o : \sltm{oo} , x_{j,1} : R_{j,1} , \ldots , x_{j,s_j} : R_{j,s_j}} : \sltm{oo}$
 \item For $k = 1 , \ldots , p$, SL context $\gamma'_k$ may use the formula of the conclusion and SL formula $F'_k$ may use the formula of the conclusion and locally quantified variables. So with full typing information $\gamma'_k \args{o : \sltm{oo}} : \sltm{context}$ and $F'_k \args{o : \sltm{oo} , y_{k,1} : S_{k,1} , \ldots , y_{k,t_k} : S_{k,t_k}} : \sltm{oo}$
\end{itemize}

%In the GSL we have made the rules general enough to capture the rules of the SL, but it could be generalized further to explore other specification logics that do not fit the restrictions here.


\subsection{SL Rules from GSL Rules}
\label{subsec:sltogsl}

The rules of the GSL can be instantiated to obtain the SL by
specifying the values of the variables in the GSL rules. We first fill
in $m$, $n$, and $p$.  Then for $i = 1 , \ldots , m$, we specify
$Q_i$.  For $j = 1 , \ldots , n$, we specify $s_j, \gamma_j$, $F_j$,
$x_{j,s_j}$, and $R_{j,s_j}$. For $k = 1 , \ldots , p$, we specify
$\gamma'_k$, $F'_k$, $y_{k,t_k}$, and $S_{k,t_k}$. Below are examples
for SL rules \rlnmsinit{}, \rlnmsalls{} and \rlnmbimp{}.

\noindent
\begin{tabular}{c c c c c c}
\\
\hline
Rule & $m$ & $n$ & $p$ & $c$ & $o$ \\
\hline \hline \noalign{\smallskip}
$\vcenter{\rlsinit{}}$ & 1 & 0 & 1 & \dyncon{} & $\atom{A}$ \\
\noalign{\smallskip} \hline \noalign{\smallskip}
\multicolumn{6}{c}{$t_1 \coloneqq 0$} \\
\multicolumn{6}{c}{$Q_1 \args{\dyncon{} , \atom{A}} \coloneqq D \in \dyncon{} \;\;\;\; \gamma'_1 \args{\atom{A}} \coloneqq \emptyset \;\;\;\; F'_1 \args{\atom{A}} \coloneqq D$} \\
\noalign{\smallskip} \hline \hline \noalign{\smallskip}

$\vcenter{\rlsalls{}}$ & 0 & 1 & 0 & \dyncon{} & $\sltm{Allx} \; G$ \\
\noalign{\smallskip} \hline \noalign{\smallskip}
\multicolumn{6}{c}{$s_1 \coloneqq 1 \;\;\;\; x_{1,1} \coloneqq E \;\;\;\; R_{1,1} \coloneqq \sltm{X}$} \\
\multicolumn{6}{c}{$\gamma_1 \args{\sltm{Allx} \; G} \coloneqq \emptyset \;\;\;\; F_1 \args{\sltm{Allx} \; G , E} \coloneqq G \; E$} \\
\noalign{\smallskip} \hline \hline \noalign{\smallskip}

$\vcenter{\rlbimp{}}$ & 0 & 1 & 1 & \dyncon{} & $G \longrightarrow D$ \\
\noalign{\smallskip} \hline
\multicolumn{6}{c}{$s_1 \coloneqq 0 \;\;\;\; t_1 \coloneqq 0$} \\
\multicolumn{6}{c}{$\gamma_1 \args{G \longrightarrow D} \coloneqq \emptyset \;\;\;\; F_1 \args{G \longrightarrow D} \coloneqq G$} \\
\multicolumn{6}{c}{$\gamma'_1 \args{G \longrightarrow D} \coloneqq \emptyset \;\;\;\; F'_1 \args{G \longrightarrow D} \coloneqq D$} \\
\hline
\\
\end{tabular}

\noindent
%Notice that for the \rlnmsinit{} rule, $D$ is used in the definition
%of $Q_i$, even though it is not in the argument list from the rule
%conclusion. In instantiations from the GSL rules, we allow signature
%variables that have been introduced in the course of a proof to be
%used in these definitions.
Notice that for the \rlnmsinit{} rule, $D$ appears in $Q_1$, even
though it is not in the argument list of $Q_1$.  The notation
$\args{\cdot}$ only specifies arguments from the rule conclusion.  Any
variables that only appear in the premises of a rule of the SL are
also permitted to appear in the propositions, formulas, and contexts
when specializing the premises of a GSL rule to obtain the premises of
a specific SL rule.

%\paragraph{\rlsbc{}} ~\\

%Uses partial function $\mathit{extr} : \sltm{oo} \rightarrow \sltm{atm}$ to extract the atom from atomic formulas and we refer to some externally quantified $G : \sltm{oo}$. We have $m = 1, n = 1, p = 0, s_n = 0$ so we define $Q_i \args{c , o} \coloneqq \prog{(\mathit{extr} \; o)}{G}$, $\gamma_1 \args{o} \coloneqq \emptyset$, and $G_1 \args{o} \coloneqq G$. \\

%Then we can write \rlnmsbc{} as $\vcenter{\infer[\rlnmsbc{}]{\seqsl{\atom{A}}}{Q_1 \args{\dyncon{} , \atom{A}} & \seqsl[\dyncon{} \cup \gamma_1 \args{\atom{A}}]{G_1 \args{\atom{A}}}}}$.

%\paragraph{\rlsand{}} ~\\

%Uses partial functions $\mathit{fst} : \sltm{oo} \rightarrow \sltm{oo}$ and $\mathit{snd} : \sltm{oo} \rightarrow \sltm{oo}$ to extract the first and second conjunts, respectively, of a formula that is a conjunction. We have $m = 0, n = 2, p = 0, s_n = 0$ so we define $\gamma_1 \args{o} \coloneqq \emptyset$, $\gamma_2 \args{o} \coloneqq \emptyset$, $G_1 \args{o} \coloneqq (\mathit{fst} \; o)$, and $G_2 \args{o} \coloneqq (\mathit{snd} \; o)$. \\

%...eeks... rule naming issue where formulas $G_1$ and $G_2$ conflict with these in generalize rule. will need to rename something. Use $\ol{F_n}$ and $\ol{F'_p}$ in generalized rule? \\

%Then we can write \rlnmsand{} as $\vcenter{\infer[\rlnmsand{}]{\seqsl{o_1 \& o_2}}{\seqsl[\dyncon{} \cup \gamma_1 \args{o_1 \& o_2}]{G_1 \args{o_1 \& o_2}} & \seqsl[\dyncon{} \cup \gamma_2 \args{o_1 \& o_2}]{G_2 \args{o_1 \& o_2}}}}$.


%where $i$ (possibly zero) is the number of non-sequent premises, and
%$\overline{Q_i}$ denotes the $i$ hypotheses $Q_1,\ldots,Q_i$.
%Similarly for $j$ and $k$ and the corresponding goal-reduction sequent
%premises and backchaining sequent premises, respectively.  Note that
%for all the rules in our SL, $0\le i,k\le 1$ and $0\le j\le 2$.

%With these rule forms, we can begin to explore how proofs using mutual structural induction over sequents may be constructed without appealing to individual rules of the SL. We let the details of the statement to be proven dictate the constraints on the rule. The motivation for this approach is to encapsulate multiple proof subcases into (at most) two arguments so that such a proof explanation can be both comprehensive and brief. A benefit of this approach is that we build the proof gradually from these rules, using the minimum number of constraints on them to prove the desired metatheorems about this logic. We can then apply these proofs to any specification logic that satisfies the same constraints.

%Suppose the non-sequent rule premise depends on the contexts, formulas, and atoms that occur elsewhere in the rule. For brevity we will simply write $Q_i$ until we need further consideration of its use in hypotheses or goals, at which time we will write $Q_i$ followed by a list of arguments that may be part of its definition. This list will vary according to the rule constraints necessary for each proof.


%The only rule variables that cannot be instantiated as we choose are those occuring in the goal. So we will allow $Q_i$ to depend on the context and formula from the conclusion of the rule, writing hypothesis $H_i$ as $Q_i \; c \; o$. One of the current subgoals of the proof is $Q_i \; \dyncon{} \; \beta$. To prove this subgoal, we rely on some induction assumptions introduced after unfolding $P_1$ to allow us to use $H_i$


%}

\cleardoublepage

%%%%%%%%%%%%%%%%%%%%%%%%%%%%%%%%%%%%%%%%%%%%%%%%%%
% Proof By Induction over the Generalized Rules
%%%%%%%%%%%%%%%%%%%%%%%%%%%%%%%%%%%%%%%%%%%%%%%%%%

\chapter{Generalized Specification Logic Metatheory}
\label{ch:gslind}

%{
%\allowdisplaybreaks[0]

\section{Proof by Induction over the Generalized Rules}
\label{sec:pfgsl}

The induction subcase corresponding to \rl{gr\_rule}
(resp. \rl{bc\_rule}) requires a proof of:
\pagebreak[0]
\begin{align*}
\ol{H_m} &: \ol{Q_m} \args{c , o} \\
\ol{\mathit{Hg}_n} &: \forall \ol{(x_{n,s_n} : R_{n,s_n})}, (\seqsl[c \cup \ol{\gamma_n} \args{o}]{\ol{F_n} \args{o , \ol{x_{n,s_n}}})} \\
\ol{\mathit{IHg}_n} &: \forall \ol{(x_{n,s_n} : R_{n,s_n})}, P_1 \; (c \cup \ol{\gamma_n} \args{o}) \; (\ol{F_n} \args{o , \ol{x_{n,s_n}}}) \\
\ol{\mathit{Hb}_p} &: \forall \ol{(y_{p,t_p} : S_{p,t_p})}, (\bchsl[c \cup \ol{\gamma'_p} \args{o}]{\ol{F'_p} \args{o , \ol{y_{p,t_p}}}}{\ol{a_p}}) \\
\ol{\mathit{IHb}_p} &: \forall \ol{(y_{p,t_p} : S_{p,t_p})}, P_2 \; (c \cup \ol{\gamma'_p} \args{o}) \; (\ol{F'_p} \args{o , \ol{y_{p,t_p}}}) \; \ol{a_p} \\[\pfshift{}]
\cline{1-2}
& P_1 \; c \; o \; (\mathit{resp.} \; P_2 \; c \; o \; a)
\end{align*}

Given specific $P_1$ and $P_2$, we could unfold uses of these predicates and continue the proof.
%We will consider a restricted version of this abstraction that is sufficient for the SL and its metatheory presented here. For $j = 1 \ldots n$ and $k = 1 \ldots p$, we restrict $C_j \args{c,o}$ and $C'_k \args{c,o}$ to have the form $c \cup (\gamma_j \args{c,o})$ and $c \cup (\gamma'_k \args{c,o})$, respectively. So we require the conclusion context to be a subset of the premise contexts. This is satisfied by the rules of our implemented SL. In fact, we will have $\gamma_j \args{c,o} = \gamma'_k \args{c,o} = \emptyset$ for all rules other than \rlnmsimp{} where $j = 1$ and $\gamma_1 \args{\dyncon{} , (D \longrightarrow G)} = \{ D \}$. (*idea: build this restriction into the GSL rules, comment that could be generalized further, remove this paragraph? OR might just remove this restriction since it doesn't add much here)
Suppose
\begin{align*}
P_1 &:= \lambda c \; o . \forall (\inddyncon{} : \sltm{context}), \\
& \mathit{IA}_1 \args{c,o,\inddyncon{}} \rightarrow \dots \rightarrow \mathit{IA}_w \args{c,o,\inddyncon{}} \rightarrow \underline{\seqsl[\inddyncon{}]{o}} \qquad \mathrm{and}\\
P_2 &:= \lambda c \; o \; a . \forall (\inddyncon{} : \sltm{context}), \\
& \mathit{IA}_1 \args{c,o,\inddyncon{}} \rightarrow \dots \rightarrow \mathit{IA}_w \args{c,o,\inddyncon{}} \rightarrow \underline{\bchsl[\inddyncon{}]{o}{a}}
\end{align*}
The underlining of sequents in the definitions of $P_1$ and $P_2$ is
to highlight that these are the sequents we apply the generalized
rules to (following introductions). In particular, we unfold uses of
$P_1$ and $P_2$ in the proof state and introduce the variables and
induction assumptions.  Then the goal is either
\seqsl[\inddyncon{}]{o} or \bchsl[\inddyncon{}]{o}{a}. Apply
\rl{gr\_rule} or \rl{bc\_rule} as appropriate, and either will give
($m + n + p$) new subgoals which come from the three premise forms in
these rules, with appropriate instantiations for the externally
quantified variables. Now the proof state is
\begin{align*}
\ol{H_m} &: \ol{Q_m} \args{c , o} \\
\ol{\mathit{Hg}_n} &: \forall \ol{(x_{n,s_n} : R_{n,s_n})}, (\seqsl[c \cup \ol{\gamma_n} \args{o}]{\ol{F_n} \args{o , \ol{x_{n,s_n}}})} \\
\ol{\mathit{IHg}_n} &: \forall \ol{(x_{n,s_n} : R_{n,s_n})} (\inddyncon{} : \sltm{context}), \\
& \mathit{IA}_1 \args{c \cup \ol{\gamma_n} \args{o} , \ol{F_n} \args{o , \ol{x_{n,s_n}}} , \inddyncon{}} \rightarrow \dots \rightarrow \\
& \mathit{IA}_w \args{c \cup \ol{\gamma_n} \args{o} , \ol{F_n} \args{o , \ol{x_{n,s_n}}} , \inddyncon{}} \rightarrow \seqsl[\inddyncon{}]{\ol{F_n} \args{o , \ol{x_{n,s_n}}}} \\
\ol{\mathit{Hb}_p} &: \forall \ol{(y_{p,t_p} : S_{p,t_p})}, (\bchsl[c \cup \ol{\gamma'_p} \args{o}]{\ol{F'_p} \args{o , \ol{y_{p,t_p}}}}{\ol{a_p}}) \\
\ol{\mathit{IHb}_p} &: \forall \ol{(y_{p,t_p} : S_{p,t_p})} (\inddyncon{} : \sltm{context}), \\
& \mathit{IA}_1 \args{c \cup \ol{\gamma'_p} \args{o} , \ol{F'_p} \args{o , \ol{y_{p,t_p}}} , \inddyncon{}} \rightarrow \dots \rightarrow \\
& \mathit{IA}_w \args{c \cup \ol{\gamma'_p} \args{o} , \ol{F'_p} \args{o , \ol{y_{p,t_p}}} , \inddyncon{}} \rightarrow \bchsl[\inddyncon{}]{\ol{F'_p} \args{o , \ol{y_{p,t_p}}}}{\ol{a_p}} \\
\ol{\mathit{IP}_w} &: \ol{\mathit{IA}_w} \args{c , o , \inddyncon{}} \\[\pfshift{}]
\cline{1-2}
& \ol{Q_m} \args{\inddyncon{} , o}, \\
& \forall \ol{(x_{n,s_n} : R_{n,s_n})}, (\seqsl[\inddyncon{} \cup \ol{\gamma_n} \args{o}]{\ol{F_n} \args{o , \ol{x_{n,s_n}}}}), \\
& \forall \ol{(y_{p,t_p} : S_{p,t_p})}, (\bchsl[\inddyncon{} \cup \ol{\gamma'_p} \args{o}]{\ol{F'_p} \args{o , \ol{y_{p,t_p}}}}{\ol{a_p}})
\end{align*}
where $\inddyncon{}$ is a new signature variable.

\subsection{Subproofs for Sequent Premises}

\begin{figure}
\begin{align*}
\ol{H_m} &: \ol{Q_m} \args{c , o} \\
\ol{\mathit{Hg}_n} &: \forall \ol{(x_{n,s_n} : R_{n,s_n})}, (\seqsl[c \cup \ol{\gamma_n} \args{o}]{\ol{F_n} \args{o , \ol{x_{n,s_n}}})} \\
\ol{\mathit{IHg}_n} &: \forall \ol{(x_{n,s_n} : R_{n,s_n})} (\inddyncon{} : \sltm{context}), \\
& \mathit{IA}_1 \args{c \cup \ol{\gamma_n} \args{o} , \ol{F_n} \args{o , \ol{x_{n,s_n}}} , \inddyncon{}} \rightarrow \dots \rightarrow \\
& \mathit{IA}_w \args{c \cup \ol{\gamma_n} \args{o} , \ol{F_n} \args{o , \ol{x_{n,s_n}}} , \inddyncon{}} \rightarrow \seqsl[\inddyncon{}]{\ol{F_n} \args{o , \ol{x_{n,s_n}}}} \\
\ol{\mathit{Hb}_p} &: \forall \ol{(y_{p,t_p} : S_{p,t_p})}, (\bchsl[c \cup \ol{\gamma'_p} \args{o}]{\ol{F'_p} \args{o , \ol{y_{p,t_p}}}}{\ol{a_p}}) \\
\ol{\mathit{IHb}_p} &: \forall \ol{(y_{p,t_p} : S_{p,t_p})} (\inddyncon{} : \sltm{context}), \\
& \mathit{IA}_1 \args{c \cup \ol{\gamma'_p} \args{o} , \ol{F'_p} \args{o , \ol{y_{p,t_p}}} , \inddyncon{}} \rightarrow \dots \rightarrow \\
& \mathit{IA}_w \args{c \cup \ol{\gamma'_p} \args{o} , \ol{F'_p} \args{o , \ol{y_{p,t_p}}} , \inddyncon{}} \rightarrow \bchsl[\inddyncon{}]{\ol{F'_p} \args{o , \ol{y_{p,t_p}}}}{\ol{a_p}} \\
\ol{\mathit{IP}_w} &: \ol{\mathit{IA}_w} \args{c , o , \inddyncon{}} \\[\pfshift{}]
\cline{1-2}
& \ol{\mathit{IA}_w} \args{c \cup \ol{\gamma_n} \args{o} , \ol{F_n} \args{o , \ol{x_{n,s_n}}} , \inddyncon{} \cup \ol{\gamma_n} \args{o}} \\
& (\mathit{resp.} \; \ol{\mathit{IA}_w} \args{c \cup \ol{\gamma'_p} \args{o} , \ol{F'_p} \args{o , \ol{y_{p,t_p}}} , \inddyncon{} \cup \ol{\gamma'_p} \args{o}})
\end{align*}
\centering{\caption{Incomplete proof branches for sequent premises \label{fig:premgrseq}}}
\end{figure}

To prove the last ($n + p$) subgoals (the ``second'' and ``third''
subgoals above) we first introduce any locally quantified variables as
signature variables. For the goal-reduction (resp. backchaining)
subgoals, for $j = 1 , \ldots , n$ (resp. $k = 1 , \ldots , p$), we
apply induction hypothesis $\mathit{IHg}_j$ (resp. $\mathit{IHb}_k$),
instantiating $\inddyncon{}$ in the induction hypothesis with
$\inddyncon{} \cup \gamma_j \args{o}$ (resp. $\inddyncon{} \cup
\gamma'_k \args{o}$). This yields the proof state in Figure
\ref{fig:premgrseq} for goal-reduction premises (resp. backchaining
premises).


\subsection{Subproofs for Non-Sequent Premises}
\label{subsec:subpfnonseq}

The proof of the first $m$ subgoals depends on the definition of $Q_i$ for $i = 1 \ldots m$. If the first argument (a \sltm{context}) is not used in its definition, then $Q_i \args{\inddyncon{} , o}$ is provable by assumption $H_i$, since we will have $Q_i \args{\inddyncon{} , o} = Q_i \args{c , o}$. Any other dependencies on signature variables can be ignored since we can assign the variables as we choose when applying the generalized rule. We will illustrate this by considering each rule with non-sequent premises, starting from the second proof state in Section \ref{sec:pfgsl} and, for $(i = 1 , \ldots , m)$, $(j = 1 , \ldots , n)$, $(k = 1 , \ldots , p)$, show how to define $Q_i$, $\gamma_j$, $F_j$, $\gamma'_k$, and $F'_k$ and finish the subproofs where possible.

\paragraph{Case \rlnmsbc{} :} This rule has one non-sequent premise and one goal-reduction sequent premise with no local quantification, so $m = n = 1$, $p = 0$, $o = \atom{A}$, and $c = \dyncon{}$. Define $Q_1 \args{\dyncon{} , \atom{A}} \coloneqq \prog{A}{G}$, $\gamma_1 \args{\atom{A}} \coloneqq \emptyset$, and $F_1 \args{\atom{A}} \coloneqq G$, where $G : \sltm{oo}$ is a signature variable. Then we are proving
the following:
\begin{align*}
H_1 &: \prog{A}{G} \\
\mathit{Hg}_1 &: \seqsl{G} \\
\mathit{IHg}_1 &: \forall (\inddyncon{} : \sltm{context}), \mathit{IA}_1 \args{\dyncon{} , G , \inddyncon{}} \rightarrow \dots \rightarrow \\
& \qquad \mathit{IA}_w \args{\dyncon{} , G , \inddyncon{}} \rightarrow \seqsl[\inddyncon{}]{G} \\
\ol{\mathit{IP}_w} &: \ol{\mathit{IA}_w} \args{\dyncon{} , \atom{A} , \inddyncon{}} \\[\pfshift{}]
\cline{1-2}
& \prog{A}{G}
\end{align*}
which is completed by assumption $H_1$.

\paragraph{Case \rlnmsinit{} :} This rule has one non-sequent premise and one backchaining sequent premise with no local quantification, so $m = p = 1$, $n = 0$, $c = \dyncon{}$, and $o = \atom{A}$. Define $Q_1 \args{\dyncon{} , \atom{A}} \coloneqq D \in \dyncon{}$, $\gamma'_1 \args{\atom{A}} \coloneqq \emptyset$, and $F'_1 \args{\atom{A}} \coloneqq D$, where $D : \sltm{oo}$ is a signature variable. Then we need to prove what is displayed in Figure \ref{fig:incpfdyn}.
\begin{figure}
\begin{align*}
H_1 &: D \in \dyncon{} \\
\mathit{Hb}_1 &: \bchsl{D}{a_1} \\
\mathit{IHb}_1 &: \forall (\inddyncon{} : \sltm{context}), \mathit{IA}_1 \args{\dyncon{} , D , \inddyncon{}} \rightarrow \dots \rightarrow \\
& \qquad \mathit{IA}_w \args{\dyncon{} , D , \inddyncon{}} \rightarrow \bchsl[\inddyncon{}]{D}{a_1} \\
\ol{\mathit{IP}_w} &: \ol{\mathit{IA}_w} \args{\dyncon{} , \atom{A} , \inddyncon{}} \\[\pfshift{}]
\cline{1-2}
& D \in \inddyncon{}
\end{align*}
\centering{\caption{Incomplete proof branch (\rlnmsinit{} case) \label{fig:incpfdyn}}}
\end{figure}
Here we do not have enough information to finish this branch of the proof. An induction assumption may be of use, but we will need specific $P_1$ and $P_2$.

\paragraph{Case \rlnmssome{} :} This rule has one non-sequent premise and one goal-reduction sequent premise with no local quantification, so $m = n = 1$, $p = 0$, $c = \dyncon{}$, and $o = \sltm{Some} \; G$. Define $Q_1 \args{\dyncon{} , \sltm{Some} \; G} \coloneqq \sltm{proper} \; E$, $\gamma_1 \args{\sltm{Some} \; G} \coloneqq \emptyset$, and $F_1 \args{\sltm{Some} \; G} \coloneqq G \; E$ where $E : \sltm{expr con}$ is a signature variable. Then we are proving
the following:
\begin{align*}
H_1 &: \sltm{proper} \; E \\
\mathit{Hg}_1 &: \seqsl{G \; E} \\
\mathit{IHg}_1 &: \forall (\inddyncon{} : \sltm{context}), \mathit{IA}_1 \args{\dyncon{} , G \; E , \inddyncon{}} \rightarrow \dots \rightarrow \\
& \qquad \mathit{IA}_w \args{\dyncon{} , G \; E , \inddyncon{}} \rightarrow \seqsl[\inddyncon{}]{G \; E} \\
\ol{\mathit{IP}_w} &: \ol{\mathit{IA}_w} \args{\dyncon{} , \sltm{Some} \; G , \inddyncon{}} \\[\pfshift{}]
\cline{1-2}
& \sltm{proper} \; E
\end{align*}
which is completed by assumption $H_1$.

\paragraph{Case \rlnmball{} :} This case is proven as above but with $m = p = 1$, $n = 0$, $c = \dyncon{}$, and $o = \sltm{All} \; D$. Define $Q_1 \args{\dyncon{} , \sltm{All} \; D} \coloneqq \sltm{proper} \; E$, $\gamma'_1 \args{\sltm{All} \; D} \coloneqq \emptyset$, and $F'_1 \args{\sltm{All} \; D} \coloneqq D \; E$ where $E : \sltm{expr con}$ is a signature variable. The goal $\sltm{proper} \; E$ is provable by the assumption of the same form
as in the previous case.
%from the definition of the rule.

In the next two sections we will return to this idea of proofs about a specification logic from a generalized form of SL rule to prove properties of the SL once we have fully defined $P_1$ and $P_2$. The proof states in Figures \ref{fig:premgrseq} and \ref{fig:incpfdyn} (the incomplete branches) will be roots of these explanations.


\section{GSL Induction Part II: The Structural Rules Hold}
\label{sec:structrules}

%\begin{align*}
%P_1 :=& \lambda \; (\dyncon{}_1 : \sltm{context}) \; . \; \lambda \; (\beta : \sltm{oo}) \; . \\
%& \forall \; (\dyncon{}_2 : \sltm{context}), \dyncon{}_1 \subseteq \dyncon{}_2 \rightarrow \seqsl[\dyncon{}_2]{\beta} \\
%P_2 :=& \lambda \; (\dyncon{}_1 : \sltm{context}) \; . \; \lambda \; (\beta : \sltm{oo}) \; . \; \lambda \; (\alpha : \sltm{atm}) \; . \\
%& \forall \; (\dyncon{}_2 : \sltm{context}), \dyncon{}_1 \subseteq \dyncon{}_2 \rightarrow \bchsl[\dyncon{}_2]{\beta}{\alpha}
%\end{align*}
%we are proving
%\begin{align*}
%&(\forall \; (\dyncon{}_1 : \sltm{context}) \; (\beta : \sltm{oo}), \\
%&\;\;\;\;\; (\seqsl[\dyncon{}_1]{\beta}) \rightarrow (P_1 \; \dyncon{}_1 \; \beta)) \\
%\wedge \; & (\forall \; (\dyncon{}_1 : \sltm{context}) \; (\beta : \sltm{oo}) \; (\alpha : \sltm{atm}), \\
%&\;\;\;\;\; (\bchsl[\dyncon{_1}]{\beta}{\alpha}) \rightarrow (P_2 \; \dyncon{} \; \beta \; \alpha))
%\end{align*}

% section started here...

Recall from Section \ref{sec:structsl} we prove the standard rules of weakening\index{weakening}, contraction\index{contraction} and exchange\index{exchange} for both the goal-reduction and backchaining sequents as corollaries of~\nameref{thm:monotone} (Theorem~\ref{thm:monotone}) which states
\begin{align*}
(\forall \; (c : \sltm{context}) & \; (o : \sltm{oo}), \\
& (\seqsl[c]{o}) \rightarrow (P_1 \; c \; o)) \;\; \wedge \\
(\forall \; (c : \sltm{context}) & \; (o : \sltm{oo}) \; (a : \sltm{atm}), \\
& (\bchsl[c]{o}{a}) \rightarrow (P_2 \; c \; o \; a))
\end{align*}
where $P_1$ and $P_2$ are defined as
\begin{align*}
P_1 :=& \lambda \; (c : \sltm{context}) (o : \sltm{oo}) \; . \\
& \qquad \forall \; (\inddyncon{} : \sltm{context}), c \subseteq \inddyncon{} \rightarrow \seqsl[\inddyncon{}]{o} \\
P_2 :=& \lambda \; (c : \sltm{context}) (o : \sltm{oo}) (a : \sltm{atm}) \; . \\
& \qquad \forall \; (\inddyncon{} : \sltm{context}), c \subseteq \inddyncon{} \rightarrow \bchsl[\inddyncon{}]{o}{a}
\end{align*}

We build on the inductive proof in Section \ref{sec:gsl} over the GSL to prove~\nameref{thm:monotone} for this new logic. Recall that when we took the proof as far as we could we had three remaining groups of branches to finish ($m + n + p$ subgoals), one group for rules with non-sequent premises depending on the context of the rule conclusion, and one for each kind of sequent premise (see Figures \ref{fig:premgrseq} and \ref{fig:incpfdyn}). We will continue this effort below, using the $P_1$ and $P_2$ defined for this theorem. This means we will have one induction assumption (i.e., $w = 1$) which is $\mathit{IA}_1 \args{c , \inddyncon{}} \coloneqq c \subseteq \inddyncon{}$.

\subsection{Sequent Subgoals}

First we will prove the subgoals coming from the sequent premises, building on Figure \ref{fig:premgrseq} and using $\mathit{IA}_1$ as defined above. The proof state for goal-reduction (resp. backchaining) premises is

\begin{align*}
\ol{H_m} &: \ol{Q_m} \args{c , o} \\
\ol{\mathit{Hg}_n} &: \forall \ol{(x_{n,s_n} : R_{n,s_n})}, (\seqsl[c \cup \ol{\gamma_n} \args{o}]{\ol{F_n} \args{o , \ol{x_{n,s_n}}})} \\
\ol{\mathit{IHg}_n} &: \forall \ol{(x_{n,s_n} : R_{n,s_n})} (\inddyncon{} : \sltm{context}), (c \cup \ol{\gamma_n} \args{o}) \subseteq \inddyncon{} \rightarrow \seqsl[\inddyncon{}]{\ol{F_n} \args{o , \ol{x_{n,s_n}}}} \\
\ol{\mathit{Hb}_p} &: \forall \ol{(y_{p,t_p} : S_{p,t_p})}, (\bchsl[c \cup \ol{\gamma'_p} \args{o}]{\ol{F'_p} \args{o , \ol{y_{p,t_p}}}}{\ol{a_p}}) \\
\ol{\mathit{IHb}_p} &: \forall \ol{(y_{p,t_p} : S_{p,t_p})} (\inddyncon{} : \sltm{context}), (c \cup \ol{\gamma'_p} \args{o}) \subseteq \inddyncon{} \rightarrow \bchsl[\inddyncon{}]{\ol{F'_p} \args{o , \ol{y_{p,t_p}}}}{\ol{a_p}} \\
\inddyncon{} &: \sltm{context} \\
\mathit{IP}_1 &: c \subseteq \inddyncon{} \\
\ol{x_{n,s_n}} &: \ol{R_{n,s_n}} \; (\mathit{resp.} \; \ol{y_{p,t_p}} : \ol{S_{p,t_p}}) \\[\pfshift{}]
\cline{1-2}
& (c \cup \ol{\gamma_n} \args{o}) \subseteq (\inddyncon{} \cup \ol{\gamma_n} \args{o}) \; (\mathit{resp.} \; (c \cup \ol{\gamma'_p} \args{o}) \subseteq (\inddyncon{} \cup \ol{\gamma'_p} \args{o}))
\end{align*}
The goal is provable by~\nameref{lem:context_sub_sup} (Lemma~\ref{lem:context_sub_sup}) and assumption $\mathit{IP}_1$.


\subsection{Non-Sequent Subgoals}

Still to be proven are the subgoals for non-sequent premises. As seen in Section \ref{subsec:subpfnonseq}, the only rule of the SL whose corresponding subcase still needs to be proven is \rlnmsinit{}. From Figure \ref{fig:incpfdyn} and using $P_1$ and $P_2$ as defined here, we are proving
\begin{align*}
H_1 &: D \in \dyncon{} \\
\mathit{Hb}_1 &: \bchsl{D}{a_1} \\
\mathit{IHb}_1 &: \forall (\inddyncon{} : \sltm{context}), \dyncon{} \subseteq \inddyncon{} \rightarrow \bchsl[\inddyncon{}]{D}{a_1} \\
\inddyncon{} &: \sltm{context} \\
\mathit{IP}_1 &: \dyncon{} \subseteq \inddyncon{} \\[\pfshift{}]
\cline{1-2}
& D \in \inddyncon{}
\end{align*}
Unfolding the definition of context subset in $\mathit{IP}_1$ it becomes $\forall (o : \sltm{oo}), o \in \dyncon{} \rightarrow o \in \inddyncon{}$. 
%Applying this to the goal
Backchaining on this form of the goal gives subgoal $D \in \dyncon{}$, provable by assumption $H_1$.


In Section \ref{sec:gsl}, we explored how to prove statements about the GSL for a restricted form of theorem statement. There were three classes of incomplete proof branches that had a final form shown in Figures \ref{fig:premgrseq} and \ref{fig:incpfdyn}. In Section \ref{sec:sltogsl} we saw how to derive the SL from the GSL. So here we have proven a structural theorem for the rules of the GSL in a general way that can be followed for any SL rule.
\end{proof}

\section{GSL Induction Part III: Cut Rule Proven Admissible}
\label{sec:cutadmiss}
\index{cut admissibility}

% section started here...

Recall from Section \ref{sec:cutadmisssl} we are proving $\forall (\delta : \sltm{oo}), P \; \delta$ with $P$ defined as
\begin{align*}
& P : \sltm{oo} \rightarrow \hybridtm{Prop} := \lambda (\delta : \sltm{oo}) \; . \\
& \qquad\qquad (\forall (c : \sltm{context}) (o : \sltm{oo}), \\
& \qquad\qquad\qquad\qquad\qquad\quad \seqsl[c]{o} \rightarrow P_1 \; c \; o) \; \wedge \\
& \qquad\qquad (\forall (c : \sltm{context}) (o : \sltm{oo}) (a : \hybridtm{atm}), \\
& \qquad\qquad\qquad\qquad\qquad\quad \bchsl[c]{o}{a} \rightarrow P_2 \; c \; o \; a),
\end{align*}
where
\begin{align*}
P_1 &: \sltm{context} \rightarrow \sltm{oo} \rightarrow \coqtm{Prop} := \\
& \qquad \lambda (c : \sltm{context}) (o : \sltm{oo}) \; . \\
& \qquad\qquad \forall (\inddyncon{} : \sltm{context}), c = (\inddyncon{}, \delta) \rightarrow \seqsl[\inddyncon{}]{\delta} \rightarrow \underline{\seqsl[\inddyncon{}]{o}} \\
P_2 &: \sltm{context} \rightarrow \sltm{oo} \rightarrow \sltm{atm} \rightarrow \coqtm{Prop} := \\
& \qquad \lambda (c : \sltm{context}) (o : \sltm{oo}) (a : \sltm{atm}) \; . \\
& \qquad\qquad \forall (\inddyncon{} : \sltm{context}), c = (\inddyncon{}, \delta) \rightarrow \seqsl[\inddyncon{}]{\delta} \rightarrow \underline{\bchsl[\inddyncon{}]{o}{a}}
\end{align*}

As in the GSL proof of~\nameref{thm:monotone} (Theorem~\ref{thm:monotone}), we build on the inductive proof in Chapter \ref{ch:gslind}, unfolding $P_1$ and $P_2$ as defined here.
% Redundant:
%We will also make use of \sltm{weakening}, a corollary of the structural rule \sltm{monotone}.
Recall that we have now introduced assumptions and applied the appropriate generalized SL rule to the underlined sequents in the definition of $P_1$ and $P_2$. For the proof of cut admissibility, there are two induction assumptions from $P_1$ and $P_2$ (so $w = 2$). Define $\mathit{IA}_1 \args{c , \inddyncon{}} \coloneqq (c = (\inddyncon{} , \delta))$ and $\mathit{IA}_2 \args{c , \inddyncon{}} \coloneqq \seqsl[\inddyncon{}]{\delta}$, where $\delta$ is the cut formula in the cut rule.

\subsection{Sequent Subgoals}
\label{subsec:cutpfseqprem}

First we will prove the subgoals coming from the sequent premises, building on Figure \ref{fig:premgrseq} and using $\mathit{IA}_1$ and $\mathit{IA}_2$ as defined above. For a moment we will ignore the outer induction over the cut formula $\delta$. By ignore we mean let $\delta \coloneqq \eta$ where $\eta : \sltm{oo}$, and we will not display the induction hypothesis for this induction. The proof state for goal-reduction premises (resp. backchaining premises) is

\newpage

\begin{align*}
\ol{H_m} &: \ol{Q_m} \args{c , o} \\
\ol{\mathit{Hg}_n} &: \forall \ol{(x_{n,s_n} : R_{n,s_n})}, (\seqsl[c \cup \ol{\gamma_n} \args{o}]{\ol{F_n} \args{o , \ol{x_{n,s_n}}})} \\
\ol{\mathit{IHg}_n} &: \forall \ol{(x_{n,s_n} : R_{n,s_n})} (\inddyncon{} : \sltm{context}), \\
& \; (c \cup \ol{\gamma_n} \args{o}) = (\inddyncon{} , \eta) \rightarrow \seqsl[\inddyncon{}]{\eta} \rightarrow \seqsl[\inddyncon{}]{\ol{F_n} \args{o , \ol{x_{n,s_n}}}} \\
\ol{\mathit{Hb}_p} &: \forall \ol{(y_{p,t_p} : S_{p,t_p})}, (\bchsl[c \cup \ol{\gamma'_p} \args{o}]{\ol{F'_p} \args{o , \ol{y_{p,t_p}}}}{\ol{a_p}}) \\
\ol{\mathit{IHb}_p} &: \forall \ol{(y_{p,t_p} : S_{p,t_p})} (\inddyncon{} : \sltm{context}), \\
& \; (c \cup \ol{\gamma'_p} \args{o}) = (\inddyncon{} , \eta) \rightarrow \seqsl[\inddyncon{}]{\eta} \rightarrow \bchsl[\inddyncon{}]{\ol{F'_p} \args{o , \ol{y_{p,t_p}}}}{\ol{a_p}} \\
\inddyncon{} &: \sltm{context} \\
\mathit{IP}_1 &: c = (\inddyncon{} , \eta) \\
\mathit{IP}_2 &: \seqsl[\inddyncon{}]{\eta} \\
\ol{x_{n,s_n}} &: \ol{R_{n,s_n}} \; (\mathit{resp.} \; \ol{y_{p,t_p}} : \ol{S_{p,t_p}}) \\[\pfshift{}]
\cline{1-2}
& (c \cup \ol{\gamma_n} \args{o} = ((\inddyncon{} \cup \ol{\gamma_n} \args{o}) , \eta)) , (\seqsl[\inddyncon{} \cup \ol{\gamma_n} \args{o}]{\eta}) \\
& (\mathit{resp.} \; (c \cup \ol{\gamma'_p} \args{o} = ((\inddyncon{} \cup \ol{\gamma'_p} \args{o}) , \eta)) , (\seqsl[\inddyncon{} \cup \ol{\gamma'_p} \args{o}]{\eta}))
\end{align*}

To prove the sequent subgoal \seqsl[\inddyncon{} \cup \ol{\gamma_n} \args{o}]{\eta} (resp. \seqsl[\inddyncon{} \cup \ol{\gamma'_p} \args{o}]{\eta}), first apply weakening and the new subgoal is \seqsl[\inddyncon{}]{\eta} (resp. \seqsl[\inddyncon{}]{\eta}), provable by assumption $\mathit{IP}_2$.

The subgoals concerning context equality are proven by context lemmas and assumption $\mathit{IP}_1$. That is, we rewrite $((\inddyncon{} \cup \ol{\gamma_n} \args{o}) , \eta)$ to $(\inddyncon{} , \eta) \cup \ol{\gamma_n} \args{o}$ (resp. $(\inddyncon{} \cup \ol{\gamma'_p} \args{o}) , \eta$ to $(\inddyncon{} , \eta) \cup \ol{\gamma'_p} \args{o}$). The new subgoal is $c \cup \ol{\gamma_n} \args{o} = (\inddyncon{} , \eta) \cup \ol{\gamma_n} \args{o}$ (resp. $(c \cup \ol{\gamma'_p} \args{o} = (\inddyncon{} , \eta) \cup \ol{\gamma'_p} \args{o}$). Apply~\nameref{lem:context_sub_sup} (Lemma~\ref{lem:context_sub_sup}) to get assumption $\mathit{IP}_1$.

%(*maybe move?) A few comments are in order to review what is proven so far. We have considered a generalized form of rules of the SL and attempted to prove cut admissibility. We made constraints on the rule that allow us to work through the proof as described in the proof outline and the diagrams presented there. With these constraints we were able to prove the subcases of the sequent mutual induction for all rules other than \rlnmsinit{}. Since induction was first over the cut formula $\delta$ and there are seven formula construction rules and 14 sequent rule subcases just proven, the above argument can be applied to 98 of the 105 subcases (see Figure~\ref{fig:cutpf} for Coq code to automate proofs of these 98 subcases).


\subsection{Non-Sequent Subgoals}%

In Section \ref{subsec:subpfnonseq} we saw that the only rule of the SL whose corresponding subcase still needs to be proven is \rlnmsinit{}. For the non-sequent subgoals we were able to complete the proof while the cut formula $\delta$ was represented as a parameter (and thus could have any formula structure). In the remaining non-sequent proof branch we need to make use of the nested structure of this induction. The proof of this subcase is shown in detail in Section~\ref{subsec:cutadmissnonseq}.

\bigskip

In summary, the outer induction over $\delta$ gave seven cases for seven \sltm{oo} constructors. For each of these, an inner induction over sequents gave 15 new subgoals for 15 rules. We saw that for 14 of 15 rules, each rule has the same proof structure for every form of $\delta$. The remaining subgoals were all for the rule \rlnmsinit{} and were more challenging due to the presence of a non-sequent premise that depends on the context of the conclusion.

\end{proof}

\bigskip

Using the generalized proof presented in this chapter and instantiating the GSL to the SL as in Section~\ref{sec:sltogsl}, we have found condensed proofs of~\nameref{thm:monotone} (Theorem~\ref{thm:monotone}) and~\nameref{thm:cut_admissible} (Theorem~\ref{thm:cut_admissible}).
%}

\cleardoublepage

%%%%%%%%%%%%%%%%%%%%%%%%%%%%%%%%%%%%%%%%%%%%%%%%%%
% Conclusion (including related and future work)
%%%%%%%%%%%%%%%%%%%%%%%%%%%%%%%%%%%%%%%%%%%%%%%%%%

\chapter{Conclusion}
\label{ch:concl}

In this thesis we have seen how the Coq implementation of Hybrid has been extended by the addition of a new specification logic (SL) based on hereditary Harrop formulas. This extension increases the class of object logics that Hybrid can reason about efficiently. The metatheory of this SL is formalized in Coq with proofs by mutual structural induction over the structure of sequent types. We saw the proofs of some specific subcases and the later insight that many of the cases are proven in a similar way. This led to the development of a generalized SL and form of metatheory statement that we could use to better understand the proofs of the SL metatheory.

\section{Related Work}

Throughout this thesis we have seen some mention of related work. Hybrid is a system implementing HOAS and as seen in Section~\ref{sec:hybridcompare} there are other systems with the same goal that also use this technique. As previously discussed, Hybrid is the only known system implementing HOAS in an existing trusted general-purpose theorem prover. See~\cite{FMP:CoRR15} and~\cite{FMP:JAR15} for a more in-depth comparison of these systems on benchmarks defined there.

Although this work is contributing to the area of mechanizing programming language metatheory, the majority of the research presented here is applicable to the more general field of proof theory. We have seen proofs of the admissibility of structural rules of a specific sequent calculus, as well as a generalized sequent calculus which we tried to make only as general as necessary to encapsulate the specification logic presented earlier. Typically these kinds of proofs are by an induction on the height of derivations, but here we have proofs by mutual structural induction over dependent sequent types; the structural proofs in this thesis follow the style of Pfenning in~\cite{Pfenning:IC00}. The sequents in our logic do not have a natural number to represent the height of the derivation. So our presentation of this sequent calculus is perhaps more ``pure'' in some sense, but we may have lost a way to reason about some object logics. It is not yet clear if building proof height into the definition sequents is necessary for studying some object logics. Overall, a better understanding of the relationship between proofs of the metatheory of sequent calculi by induction on the height of derivations versus over the structure of sequents is desirable.

\section{Future Work}

The highest priority future task is to show the utility of the new specification logic in Hybrid. This will be done by presenting an object logic that makes use of the higher-order nature (in the sense of unrestricted implicational complexity) of the new specification logic. Object logics that we plan to represent include:
\begin{itemize}
 \item correspondence between HOAS and de Bruijn encodings of untyped $\lambda$-terms; this is our example OL of Chapter~\ref{ch:hybrid} but we have not yet proven Theorems~\ref{thm:hodb_det1} and~\ref{thm:hodb_det3} of Section~\ref{sec:hybridol} (see~\cite{WCGN:PPDP13})
 \item structural characterization of reductions on untyped $\lambda$-terms (see~\cite{WCGN:PPDP13})
 \item algorithmic specification of bounded subtype polymorphism in System F (see~\cite{Pientka:TPHOLs07}); this comes from the \poplmark{} challenge~\cite{Aydemir05TPHOLs}
\end{itemize}
We would also like to add automation to proofs containing object logic judgments so that the user of Hybrid will not need to be an expert user of proof assistants to be able to use the system.

The encoding of the new Hybrid SL follows the development of the specification logic of Abella as presented in~\cite{WCGN:PPDP13}, but it seems that the proofs of the admissibility of the structural rules differ between these systems. These proofs in Abella are not fully explained in~\cite{WCGN:PPDP13} so some work will need to be done to compare the different proofs. Also, the proof of cut admissibility for this specification logic in Abella requires a third conjunct that we did not need for our proof:
$$
\forall (c : \sltm{context}) (o : \sltm{oo}) (a: \hybridtm{atm}),\\
\seqsl[c]{o} \rightarrow \bchsl[c]{o}{a} \rightarrow \seqsl[c]{\atom{a}}
$$
Our understanding so far is that these proofs in Abella are over the height of derivations, which is an implicit parameter; it is not by structural induction over sequents in the fashion of the proofs founding this thesis.

\cleardoublepage

%%%%%%%%%%%%%%%%%%%%%%%%%%%%%%%%%%%%%%%%%%%%%%%%%%
% The End
%%%%%%%%%%%%%%%%%%%%%%%%%%%%%%%%%%%%%%%%%%%%%%%%%%

%\topskip0pt

\vspace*{\fill}

\centering{\fontsize{50}{50}{\flobstertwo{The End.}}}

\vspace*{\fill}

\cleardoublepage


%%%%%%%%%%%%%%%%%%%%%%%%%%%%%%%%%%%%%%%%%%%%%%%%%%%%%%%%%%%%%%%%%%%%%%
% If the following line of code is uncomment, then
% the following chapters will be numbered A, B, ...
% 
% If there are no section in your appendix, your theorems, ..., and
% equations will be numbered Theorem A.0.1, ...  To eliminate the
% extra 0 in the numbering, you should modify the numbering as
% follows.  Add
%
% \newtheorem{Atheo}{Theorem}[chapter]
% \newtheorem{Alem}[Atheo]{Lemma}
% \newtheorem{Adefn}[Atheo]{Definition}
% \newtheorem{Acor}[Atheo]{Corollary}
% \newtheorem{Aprop}[Atheo]{Proposition}
%
% after \documentclass[12pt]{UOthesis}
%
% and use
%
% \begin{Atheo}
% This is a theorem.
% \end{Atheo}
%
% \begin{Alem}
% This is a emma.
% \end{Alem}
%
% etc.
%
% in the appendix.
%
% You should also add
%
% \numberwithin{equation}{chapter}
%
% at the beginning of the appendix to get the right numbering for the
% equations.
%%%%%%%%%%%%%%%%%%%%%%%%%%%%%%%%%%%%%%%%%%%%%%%%%%%%%%%%%%%%%%%%%%%%%%
\appendix

\include{appendix_A}

\chapter{Notations}
\label{ch:notations}

\begin{flushleft}
Many symbols are used to denote values of different types. This allows us to impose some structure that is useful in understanding the work presented later, but is not actually built in to the system. We summarize the meta-variables used in the contributions chapters of this thesis. All symbols described here may also be seen with subscripts or the prime notation (i.e.$'$) when we need to talk about more than one term of a given type.
\end{flushleft}

$$
\begin{tabular}{c c p{10cm}}
Symbol & Type & Description \\ \hline
$\alpha$ & \sltm{atm} & atom representing OL formula in pretty-printed inference rule notation for Coq statements \\
$a$ & \sltm{atm} & atom representing OL formula in linear forms of Coq statements \\
$A$ & \sltm{atm} & atom representing OL formula in SL inference rules \\ \hline
$\beta$ & \sltm{oo} & SL formula in pretty-printed inference rule notation for Coq statements \\
$o$ & \sltm{oo} & SL formula in linear form of Coq statements \\
$G$ & \sltm{oo} & SL formula representing a goal in SL inference rules \\
$D$ & \sltm{oo} & SL formula representing a clause in SL inference rules \\ \hline
$\dyncon{}$ & \sltm{context} & context of assumptions in pretty-printed inference rule notation for Coq statements and SL inference rules \\
$c$ & \sltm{context} & context of assumptions in linear form of theorem statements \\
\end{tabular}
$$

\begin{flushleft}
We collect here the following additional notations seen in this thesis:
\end{flushleft}

$$
\begin{tabular}{l l c p{6cm}}
Definition & Type & Notation & Notes \\ \hline
$\sltm{atom} \; a$ & $\sltm{atm} \rightarrow \sltm{oo}$ & \atom{a} & coerces an atom to a formula \\
$\sltm{Imp} \; o_1 \; o_2$ & $\sltm{oo} \rightarrow \sltm{oo} \rightarrow \sltm{oo}$ & $o_1 \longrightarrow o_2$ & implication in SL formula (right associative) \\
$\sltm{Conj} \; o_1 \; o_2$ & $\sltm{oo} \rightarrow \sltm{oo} \rightarrow \sltm{oo}$ & $o_1 \& o_2$ & conjunction in SL formula \\
$\sltm{prog} \; a \; o$ & $\sltm{atm} \rightarrow \sltm{oo} \rightarrow Prop$ & $\prog{a}{o}$ & parameter of SL representing OL inference rules \\
\end{tabular}
$$

\cleardoublepage

\include{appendix_B}
\cleardoublepage

%%%%%%%%%%%%%%%%%%%%%%%%%%%%%%%%%%%%%%%%%%%%%%%%%%%%%%%%%%%%%%%%%%%%%%
% We provide two methods to introduce your bibliography.
%
% The hand made bibliography:
% The basic method used the file biblio.tex.  It makes used of
% the standard LaTeX environment \begin{thebibliography}{} and
% \end{thebibliography}.
%
% The bibliography made with BibTeX:
% The second method used the file biblio.bib.  It makes used of
% BibTeX with the commands \bibliography{} and \bibliographystyle{}
% \bibliographystyle{} is defined in the preamble.
%
% For examples on how to use the command  \cite[]{} in the text
% to refer to items of the bibliography, look at the end of the
% section on the "Logistic Equation" in the source file
% qualitative.tex .  The results are dsiplayed at the end of
% Section 2.1 after compilation.
%
% Instead of \cite[]{}, one can use \citet[]{}  and  \citep[]{}.
% We have not illustrated how to use these commands but they are
% used like \cite[]{}.
%
%%%%%%%%%%%%%%%%%%%%%%%%%%%%%%%%%%%%%%%%%%%%%%%%%%%%%%%%%%%%%%%%%%%%%%

% Hand made bibliography
% \begin{BasicBibliography}
% \input{biblio.tex}
% \end{BasicBibliography}

% Default style for BibTeX
% The bibliography made with this command will include only references
% in the bib file which are cited in the text.
\bibTexBibliography{Thesis}

% The bibliography made with this command will include all references
% in the bib file even if they are not cited in the text.
% \bibTexBibliography*{biblio}

%%%%%%%%%%%%%%%%%%%%%%%%%%%%%%%%%%%%%%%%%%%%%%%%%%%%%%%%%%%%%%%%%%%%%%
% Finally, we print the index
%
% To mark item "name_of_the_item" for inclusion in the index, you have
% to insert the command \index{name_of_the_item} after you introduce
% the item for the first time in the text.
%
% If the item is part of the group "name_of_the_group", you may use
% the command \index{name_of_the_group!name_of_the_index}
% The item will then appear under the group name.
%
% If you don't want to have an index, comment out the following line
% and don't run makeindex template.idx .
%%%%%%%%%%%%%%%%%%%%%%%%%%%%%%%%%%%%%%%%%%%%%%%%%%%%%%%%%%%%%%%%%%%%%%
\PrintIndex

\end{document}

%%% Local Variables: 
%%% mode: latex
%%% TeX-master: t
%%% End: 

