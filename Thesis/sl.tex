
%The main contribution to Hybrid presented here has been to add a new specification logic (SL) and prove the necessary structural rules. This SL is an inference system based on higher-order hereditary Harrop formulas. (*motivation, names of people whose research this builds on, etc)

%We follow the presentation in~\cite{WCGN:PPDP13} to define this SL in a similar manner as the SL for Abella that is described there. That is, we distinguish between goal-reduction rules and backchaining rules (see figures~\ref{fig:grseq} and~\ref{fig:bcseq}). Goal-reduction rules are the right-introduction rules and reduce a formula to atomic (bottom-up). Backchaining rules are the left-introduction rules and allow backchaining over a formula focused from the context of assumptions at this reasoning level (bottom-up).

%The rules of this logic are encoded in inductive types for goal-reduction and backchaining sequents. Goal-reduction sequents have signature \sltm{grseq} $:$ \sltm{context} $\rightarrow$ \sltm{oo} $\rightarrow$ \coqtm{Prop} and we write \seqsl{\beta} as notation for \sltm{grseq} $\dyncon{}$ $\beta$. Backchaining sequents have signature \sltm{bcseq} $:$ \sltm{context} $\rightarrow$ \sltm{oo} $\rightarrow$ \sltm{atm} $\rightarrow$ \coqtm{Prop} and we write \bchsl{\beta}{\alpha} as notation for \sltm{bcseq} $\dyncon{}$ $\beta$ $\alpha$, understanding $\beta$ to be a formula from \dyncon{} that we focus.

%Something has been ignored in our classification of the rules so far. The rules \rlnmsbc{} and \rlnmsinit{} are presented with the goal-reduction rules in figure~\ref{fig:grseq}, even though they are not used to reduce a goal any further (the conclusion is atomic in these cases). Also, the rule \rlnmbmatch{} is considered a backchaining rule in figure~\ref{fig:bcseq}. The reason for this comes from how these rules are defined in Coq; a rule whose conclusion is a goal-reduction sequent must be defined in \sltm{grseq} and a rule whose conclusion is a backchaining sequent must be defined in \sltm{bcseq}.

%Since \sltm{grseq} references \sltm{bcseq} in the rule \rlnmsinit{} and \sltm{bcseq} references \sltm{grseq} in the rule \rlnmbimp{}, these are defined as mutually inductive types. Both types have a constructor for each rule that has conclusion of their type. The identifier (*right word?) of the constructor is the rule name, and the premises and conclusion of the rule are premises and conclusion of the constructor's type, respectively (with quantification over the necessary parameters in the type). For example, the definition of \sltm{grseq} has a constructor \rlnmsinit{} of type $\forall (L : \sltm{context}) (G : \sltm{oo}), G \in L \rightarrow \bchsl[L]{G}{A} \rightarrow \seqsl[L]{\atom{A}}$.

%The types \seqsl{\beta} and \bchsl{\beta}{\alpha} are dependent types, where $\dyncon{} : \sltm{context}$, $\beta : \sltm{oo}$, and $\alpha : \sltm{atm}$. Before considering the rules of the logic in detail, the types \sltm{atm}, \sltm{oo}, and \sltm{context} defined and used by the SL need to be explained.


The first stage of the contributions outlined in this thesis is defining a specification logic to increase the reasoning power of Hybrid. The new specification logic (SL) for Hybrid is based on hereditary Harrop formulas using an intuitionistic logic with focusing as described in Chapter~\ref{ch:hh}. We adopt a modified version of the rules very close to the style of the rules of the specification logic used in the higher-order version of Abella~\cite{WCGN:PPDP13}. We do not include any rules for disjunction here because they have not been necessary for object logics in case studies of interest. The SL could easily be extended to add these rules and the proofs of SL metatheory would have the same structure, as will be seen in Chapter~\ref{ch:gsl}.

%t the specification level, the terms of the language of hereditary Harrop formulas are the terms of the simply-typed $\lambda$-calculus. 
% When we refer to this logic as higher-order, we mean the
% implicational complexity.
We note that unlike in all previous SLs for Hybrid there is no restriction on the implicational complexity (see~\cite{FeltyMomigliano:JAR10}), because $G$-formulas in higher-order hereditary Harrop language allow $D$-formulas as the antecedent of implication as was seen in Section~\ref{sec:hohh}. In all previous SLs, only atomic formulas were allowed in place of the more general D-formulas allowed here.
%

The SL presented in this chapter is a sequent calculus implemented as an inductive type in Coq. Section~\ref{sec:context} describes how contexts are defined for this SL. Section~\ref{sec:hhsl} presents the Coq implementation of the SL based on hereditary Harrop formulas and we see how to prove properties of this SL by structural induction in Section~\ref{sec:induction}.

In Appendix~\ref{ch:notations}, we list notations that will be used in the rest of the thesis.

\section{Contexts in Coq}
\label{sec:context}

The type \sltm{context} represents contexts of assumptions in sequents and is defined using the Coq \coqtm{ensemble} library as \coqtm{ensemble} \sltm{oo} since we want contexts to behave as sets with elements of type \sltm{oo}. In proofs of some context lemmas stated below we use the \coqtm{ensemble} extensional equality axiom:
$$
\coqtm{Extensionality\_Ensembles} : \forall (E_1 \; E_2 : \coqtm{ensemble}),(\coqtm{Same\_set} \; E_1 \; E_2) \rightarrow E_1 = E_2
$$
where \coqtm{Same\_set} is defined in the \coqtm{Ensemble} library. We use $o \in c$ as notation for $\coqtm{elem} \; o \; c$ which means formula $o$ is an element of context $c$. Context subset, written $\dyncon{}_1 \subseteq \dyncon{}_2$, is defined as $\forall (o : \sltm{oo}), o \in \dyncon{}_1 \rightarrow o \in \dyncon{}_2$.

We write ($\dyncon{}, \beta$) as notation for ($\sltm{context\_cons} \; \dyncon{} \; \beta$). We write write $c$ or $\dyncon{}$ to denote contexts when discussing formalized proofs.

The context lemmas below are proven as part of this work and are used in later proofs in this thesis. See the accompanying source code for the proofs. Note that all variables are externally quantified and each occurrence of $\beta$ and $\dyncon{}$, possibly with subscripts, has type \sltm{oo} and \sltm{context}, respectively.
\begin{lemma}[\sltm{elem\_inv}] %$\forall (c : \mathtt{context}) (o_1 \; o_2 : \sltm{oo}), o_1 \in (c , o_2) \rightarrow (o_1 \in c \vee o_1 = o_2)$
\label{lem:elem_inv}
$$
\vcenter{\infer{(\beta_1 \in \dyncon{}) \vee (\beta_1 = \beta_2)}{\beta_1 \in (\dyncon{}, \beta_2)}}
$$
\end{lemma}

\begin{lemma}[\sltm{elem\_sub}] %$\forall (c : \mathtt{context}) (o_1 \; o_2 : \sltm{oo}),  \rightarrow $
$$
\vcenter{\infer{\beta_1 \in (\dyncon{} , \beta_2)}{\beta_1 \in \dyncon{}}}
$$
\end{lemma}

\begin{lemma}[\sltm{elem\_self}] %$\forall (c : \mathtt{context}) (o : \sltm{oo}), o \in (c , o)$
\label{lem:elem_self}
$$
\vcenter{\infer{\beta \in (\dyncon{} , \beta)}{}}
$$
\end{lemma}

\begin{lemma}[\sltm{elem\_rep}] %$\forall (c : \mathtt{context}) (o_1 \; o_2 : \sltm{oo}),  \rightarrow $
$$
\vcenter{\infer{\beta_1 \in (\dyncon{} , \beta_2)}{\beta_1 \in (\dyncon{} , \beta_2 , \beta_2)}}
$$
\end{lemma}

\begin{lemma}[\sltm{context\_swap}] %$\forall (c : \mathtt{context}) (o_1 \; o_2 : \sltm{oo}), (c , o_1 , o_2) = (c , o_2 , o_1)$
$$
\vcenter{\infer{(\dyncon{} , \beta_1 , \beta_2) = (\dyncon{} , \beta_2 , \beta_1)}{}}
$$
\end{lemma}

\begin{lemma}[\sltm{context\_sub\_sup}] %$\forall (c_1 \; c_2 : \mathtt{context}) (o : \mathtt{context}), c_1 \subseteq c_2 \rightarrow (c_1 , o) \subseteq (c_2 , o)$
\label{lem:context_sub_sup}
$$
\vcenter{\infer{(\dyncon{}_1 , \beta) \subseteq (\dyncon{}_2 , \beta)}{\dyncon{}_1 \subseteq \dyncon{}_2}}
$$
\end{lemma}

\section{Hereditary Harrop Specification Logic in Coq}
\label{sec:hhsl}

%We follow the description in~\cite{WCGN:PPDP13} of a specification logic for Abella based on hereditary Harrop formulas and as described in Chapter~\ref{ch:hh}.
The inference rules of the SL are implemented using two sequent judgments that distinguish between \emph{goal-reduction rules}\index{goal-reduction rule} and \emph{backchaining rules}\index{backchaining rule} which correspond to the right rules and left focused rules, respectively, of Figure~\ref{fig:hohhfoc} in Section~\ref{sec:focusing}. %To differentiate sequents of this implemented inference system, we use $\triangleright$ as the sequent arrow rather than $\vdash$. Goal-reduction sequents have the form \seqsl{G} and backchaining sequents are written \bchsl{D}{A}, where the latter is a left focusing judgment with $D$ the formula under (left) focus.
Figures~\ref{fig:grseq} and~\ref{fig:bcseq} implement the inference rules of the SL (except for the disjunction rules, as mentioned in the introduction to this chapter). They are encoded in Coq as two mutually inductive types, one each for goal-reduction and backchaining sequents. The syntax used in the figures is a pretty-printed version of the Coq inductive types \sltm{grseq} and \sltm{bcseq}. Goal-reduction sequents have type $\sltm{grseq} : \sltm{context} \rightarrow \sltm{oo} \rightarrow \coqtm{Prop}$, and we write $\seqsl{\beta}$ as notation for $\sltm{grseq}~\dyncon{}~ \beta$. Backchaining sequents have type $\sltm{bcseq} : \sltm{context} \rightarrow \sltm{oo} \rightarrow \sltm{atm} \rightarrow \coqtm{Prop}$ and we write \bchsl{\beta}{\alpha} as notation for $\sltm{bcseq}~ \dyncon{}~ \beta~ \alpha$, understanding $\beta$ to be the focused formula from \dyncon{}. The symbol $\forall$ is used for universal quantification in Coq, rather than universal quantification in SL formulas. When we see $\forall$ in the premises of rules, this is to make it clear that it is only over the premise of the rule. 

The rule names in the figures are the constructor names in the inductive definitions in the Coq files. The premises and conclusion of a rule are the argument types and the target type, respectively, of one clause in the definition. Quantification at the outer level is implicit and, as noted, inner quantification is written explicitly in the premises. For example, the linear format of the \rlnmsinit{} rule from Figure~\ref{fig:grseq} with all quantifiers explicit is
$$
\forall (\Gamma : \sltm{context}) (D : \sltm{oo}) (A :\sltm{atm}),D \in \Gamma \rightarrow \bchsl[\Gamma]{D}{A} \rightarrow \seqsl[\Gamma]{\atom{A}}
$$
This is the type of the \rlnmsinit{} constructor in the inductive definition of $\sltm{grseq}$ (see the definition of $\sltm{grseq}$ in the Coq files).

%A number of implementation details can be seen in the rules. 
The notation \atom{A} is to say that $A : \coqtm{atm}$ is coerced to have type \coqtm{oo}, where \coqtm{oo} is the implemented type of formulas (see Figure~\ref{fig:oofig}), referred to as $o$ in Chapter~\ref{ch:hh}. The constants \coqtm{Some} and \coqtm{All} are used for existential and universal quantification in SL formulas, respectively, over the type \coqtm{expr con} which is the type for OL expressions. \coqtm{Allx} is a constant for universal quantification over a type \coqtm{X} of type \coqtm{Set}. We juxtapose appropriate terms to denote application since Coq will reduce the expression, rather than explicitly writing the substitution (for example, compare rule \rl{$\forall_R$} in Figure~\ref{fig:hohhfoc} to rule \rlnmsalls{} in Figure~\ref{fig:grseq}). A final implementation byproduct is the predicate \coqtm{proper} appearing in the premise of some rules. This is used in the Hybrid library to rule out terms of type \coqtm{expr} that have dangling indices (see~\cite{FeltyMomigliano:JAR10}).

In the sequents for this SL there is also an implicit fixed context $\Delta$, called the \emph{static program clauses}, containing closed clauses ($D$-formulas) of the form
$$
\forall_{\tau_1}\ldots\forall_{\tau_n}.G\longrightarrow A
$$
with $n\ge0$. Any set of $D$-formulas can be transformed to an equivalent one that all have this form. These clauses represent the inference rules of an OL. We do not explicitly mention $\Delta$ in the rules for this SL because no rules modify it.

%Our encoding of the formulas of the SL in Coq is shown in Figure~\ref{fig:oofig}. Since \sltm{oo} is an inductive type, Coq will automatically generate the induction principle shown in Figure (*TODO) as discussed in Section~\ref{sec:coqinduction}.

%
%\begin{figure}
%\begin{lstlisting}
%Inductive oo : Type :=
%| atom : atm -> oo
%| T : oo
%| Conj : oo -> oo -> oo
%| Imp : oo -> oo -> oo
%| All : (expr con -> oo) -> oo
%| Allx : (X -> oo) -> oo
%| Some : (expr con -> oo) -> oo.
%\end{lstlisting}
%\centering{\caption{Type of SL Formulas \label{fig:oofig}}}
%\end{figure}
%
%In this implementation, the type \sltm{atm} is a parameter to the definition of \sltm{oo} and is used to define the predicates needed for reasoning about a particular OL. For instance, our above example might include a predicate $\oltm{hodb} : (\mltm{expr}~ \mltm{con}) \rightarrow \coqtm{nat} \rightarrow (\mltm{expr}~ \mltm{con})$ relating the higher-order and de Bruijn encodings at a given depth. The constant \sltm{atom} coerces an atomic formula (a predicate applied to its arguments) to an SL formula. Also, note that in this implementation, we restrict the type of universal quantification to two types, (\mltm{expr}~ \mltm{con}) and \mltm{X}, where \mltm{X} is a parameter that can be instantiated with any primitive type; in our running example, \mltm{X} would become \coqtm{nat} for the depth of binding in a de Bruijn term.  We also leave out disjunction. It is not difficult to extend our implementation to include disjunction and quantification (universal or existential) over other primitive types, but these have not been needed in reasoning about OLs.

%We write \atom{\alpha}, ($\beta_1$ \& $\beta_2$), and ($\beta_1 \longrightarrow \beta_2$) as notation for (\sltm{atom} $\alpha$), (\sltm{Conj} $\beta_1$ $\beta_2$), and (\sltm{Imp} $\beta_1$ $\beta_2$), respectively. Note that we write $\beta$ or $\delta$ for formulas (type \sltm{oo}), and $\alpha$ for elements of type \sltm{atm}, possibly with subscripts. When discussing proofs, we also write $o$ for formulas and $a$ for atoms. When we want to make explicit when a formula is a goal or clause, we write $G$ or $D$, respectively. Formulas quantified by \sltm{All} are written $(\sltm{All}~ \beta)$ or $(\sltm{All}~ \lambda (x:\mltm{expr}~ \mltm{con}) \; . \; \beta x)$. The latter is the $\eta$-long form with types included explicitly. The other quantifiers are treated similarly.

{
\renewcommand{\arraystretch}{3.5}
\newcommand{\GRrlsbc}{\inferH[\rlnmsbc{}]{\seqsl[\Gamma]{\atom{A}}}{\prog{A}{G} & \seqsl[\Gamma]{G}}}
\newcommand{\GRrlsinit}{\inferH[\rlnmsinit{}]{\seqsl[\Gamma]{\atom{A}}}{D \in \Gamma & \bchsl[\Gamma]{D}{A}}}
\newcommand{\GRrlst}{\inferH[\rlnmst{}]{\seqsl[\Gamma]{\sltm{T}}}{}}
\newcommand{\GRrlsand}{\inferH[\rlnmsand{}]{\seqsl[\Gamma]{G_1 \, \& \, G_2}}{\seqsl[\Gamma]{G_1} & \seqsl[\Gamma]{G_2}}}
\newcommand{\GRrlsimp}{\inferH[\rlnmsimp{}]{\seqsl[\Gamma]{D \longrightarrow G}}{\seqsl[\Gamma \, , \, D]{G}}}
\newcommand{\GRrlsall}{\inferH[\rlnmsall{}]{\seqsl[\Gamma]{\sltm{All} \; G}}{\forall (E : \hybridtm{expr con}), (\sltm{proper} \; E \rightarrow \seqsl[\Gamma]{G \, E})}}
\newcommand{\GRrlsalls}{\inferH[\rlnmsalls{}]{\seqsl[\Gamma]{\sltm{Allx} \; G}}{\forall (E : \sltm{X}), (\seqsl[\Gamma]{G \, E})}}
\newcommand{\GRrlssome}{\inferH[\rlnmssome{}]{\seqsl[\Gamma]{\sltm{Some} \; G}}{\sltm{proper} \; E & \seqsl[\Gamma]{G \, E}}}

\begin{figure}%[h]
$$
\begin{tabular}{c c c}
\GRrlsbc{}
&
\GRrlsinit{}
&
\GRrlst{} \\
%
\GRrlsand{}
&
\GRrlsimp{}
&
\GRrlssome{} \\
%
\multicolumn{3}{c}{
\GRrlsall{} \;\;\; \GRrlsalls{}
}
\end{tabular}
$$
\centering{\caption{Goal-Reduction Rules, $\sltm{grseq} : \sltm{context} \rightarrow \sltm{oo} \rightarrow \coqtm{Prop}$ \label{fig:grseq}}}

\end{figure}
}
%
{
\renewcommand{\arraystretch}{3.5}
\newcommand{\BCrlbmatch}{\inferH[\rlnmbmatch{}]{\bchsl[\Gamma]{\atom{A}}{A}}{}}
\newcommand{\BCrlbanda}{\inferH[\rlnmbanda{}]{\bchsl[\Gamma]{D_1 \, \& \, D_2}{A}}{\bchsl[\Gamma]{D_1}{A}}}
\newcommand{\BCrlbandb}{\inferH[\rlnmbandb{}]{\bchsl[\Gamma]{D_1 \, \& \, D_2}{A}}{\bchsl[\Gamma]{D_2}{A}}}
\newcommand{\BCrlbimp}{\inferH[\rlnmbimp{}]{\bchsl[\Gamma]{G \longrightarrow D}{A}}{\seqsl[\Gamma]{G} & \bchsl[\Gamma]{D}{A}}}
\newcommand{\BCrlball}{\inferH[\rlnmball{}]{\bchsl[\Gamma]{\sltm{All} \; D}{A}}{\sltm{proper} \; E & \bchsl[\Gamma]{D \, E}{A}}}
\newcommand{\BCrlballs}{\inferH[\rlnmballs{}]{\bchsl[\Gamma]{\sltm{Allx} \; D}{A}}{\bchsl[\Gamma]{D \, E}{A}}}
\newcommand{\BCrlbsome}{\inferH[\rlnmbsome{}]{\bchsl[\Gamma]{\sltm{Some} \; D}{A}}{\forall (E : \hybridtm{expr con}), (\sltm{proper} \; E \rightarrow \bchsl[\Gamma]{D \, E}{A})}}

\begin{figure}%[h]

$$
\begin{tabular}{c c c}
\BCrlbmatch{}
&
\BCrlbanda{}
&
\BCrlbandb{} \\
%
\BCrlbimp{}
&
\BCrlball{}
&
\BCrlballs{} \\
\multicolumn{3}{c}{
\BCrlbsome{}
}
\end{tabular}
$$
\centering{\caption{Backchaining Rules, $\sltm{bcseq} : \sltm{context} \rightarrow \sltm{oo} \rightarrow \sltm{atm} \rightarrow \coqtm{Prop}$ \label{fig:bcseq}}}

\end{figure}
}
%
The goal-reduction rules of Figure~\ref{fig:grseq} are implemented version of the right introduction rules of this sequent calculus as seen in Figure~\ref{fig:hohhfoc}. The rules \rlnmsbc{} and \rlnmsinit{} are the only goal-reduction rules with an atomic principal formula.

The rule \rlnmsbc{} is used to backchain over the static program clauses $\Delta$, which are defined for each new OL as an inductive type called \sltm{prog} of type $\sltm{atm} \rightarrow \sltm{oo} \rightarrow \coqtm{Prop}$, and represent the inference rules of the OL (this is discussed further in Section~\ref{sec:hybridsl}). The rule \rlnmsbc{} is the interface between the SL and OL layers and we say that the SL is parametric in OL provability. We write \prog{A}{G} for $(\sltm{prog} \; A \; G)$ to suggest backward implication. Recall that clauses in $\Delta$ may have outermost universal quantification. The premise \prog{A}{G} actually represents an instance of a clause in $\Delta$.

The rule \rlnmsinit{} allows backchaining over dynamic assumptions (i.e. a formula from \dyncon{}) and is the implemented version of the \rl{focus} rule of Figure~\ref{fig:hohhfoc}. To use this rule to prove \seqsl{\atom{A}}, we need to show $D \in \dyncon{}$ and \bchsl{D}{A}. Formula $D$ is chosen from, or shown to be in, the dynamic context \dyncon{} and we use the backchaining rules of Figure~\ref{fig:bcseq} to show \bchsl{D}{A} (where $D$ is the focused formula).

The backchaining rules of Figure~\ref{fig:bcseq} are the implemented version of the standard focused left rules for conjunction, implication, and universal and existential quantification seen in Figure~\ref{fig:hohhfoc}. Considered bottom up, they provide backchaining over the focused formula. In using the backchaining rules, each branch is either completed by \rlnmbmatch{} where the focused formula is an atomic formula identical to the goal of the sequent, or \rlnmbimp{} is used resulting in one branch switching back to using goal-reduction rules.

%We mention several Coq tactics when presenting proofs. The main one is the \coqtm{constructor} tactic, which applies a clause of an inductive definition in a backward direction (a step of meta-level backchaining), determining automatically which clause to apply.
%The \coqtm{apply} tactic also does a step of backchaining and takes as
%argument the name of a definition clause, hypothesis, or lemma.  The
%\coqtm{assumption} tactic is the ``base case'' for \coqtm{apply},
%closing the proof when there are no further subgoals.


%(*cut?) Note that the rules \rlnmsbc{} and \rlnmsinit{} are presented with the goal-reduction rules in Figure~\ref{fig:grseq}, even though they are not used to reduce a goal any further (the conclusion is atomic in these cases). Also, the rule \rlnmbmatch{} is considered a backchaining rule in Figure~\ref{fig:bcseq}. The reason for this comes from how these rules are defined in Coq; a rule whose conclusion is a goal-reduction sequent must be defined in \sltm{grseq} and a rule whose conclusion is a backchaining sequent must be defined in \sltm{bcseq}.


%Recall that Coq's dependent products are written $\forall(x_1:t_1)\cdots(x_n:t_n),M$, where $n\ge0$ and for $i=1,\ldots, n$, $x_i$ may appear free in $x_{i+1},\ldots,x_n,M$.  If it doesn't, implication can be used as an abbreviation, e.g., the premise of the \rlnmsall{} rule is an abbreviation for $\forall (E : \hybridtm{expr con})(H:\sltm{proper} \; E), (\seqsl[\Gamma]{G \, E})$.